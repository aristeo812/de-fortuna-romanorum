\chapter*{Обзор использованной литературы}
\addcontentsline{toc}{chapter}{Обзор использованной литературы}


Литература, посвящённая Фортуне, в целом, довольно обширна\footnote{Большой список библиографии см.: Arya,~D.~A. Goddess Fortuna in Imperial Rome: Cult, Art, Text. Austin, 2002. P. 370--408.}. Связано это с тем, что Фортуна, взятая как предмет исследования, представляется сложным и многогранным явлением и в трудах исследователей Нового времени рассматривается в разных своих аспектах:
Фортуна как богиня и культ, с ней связанный; фортуна как отвлечённое понятие; фортуна как действующая сила в истории, согласно представлениям античных авторов; фортуна как философская категория и предмет специального познавательного исследования древних мыслителей. Предметом настоящего исследования является богиня Фортуна со всем своим окружением в виде культовой практики и легенд, поэтому мы используем, в первую очередь, труды, посвящённые этим религиозным вопросам.

% Написать про то, какие вопросы поднимаются на страницах научной литературы: история становления культа Фортуны; попытка охватить взглядом разнообразие проявлений культа Фортуны; проблема возникновения культа Фортуны в Риме: происхождение этой богини, представление о ней в архаические времена, возможная её связь с общеиталийскими и даже общеиндоевропейскими божествами.

Исследователи римской религии, приступающие к изучению богини Фортуны, в первую очередь сталкиваются с задачей создания как можно более полной и целостной картины её культа и представления о ней. Задача эта довольно трудна как в силу отрывочности, разрозненности и разнотипности источников, к которым приходится прибегать, так и в силу многогранности и противоречивости образа самой богини, что проявляется во множестве разнообразных форм культовой практики. Некоторые авторы касаются лишь отдельных аспектов обозначенной предметной области, не пытаясь обрисовать картину в целом.

Особый круг вопросов связан со становлением культа Фортуны в Риме и происхождением самой богини: здесь остаётся много <<белых пятен>> и спорных моментов. Логическим продолжением этой темы становится проблема дальнейшей эволюции образа богини и её культа. Большое значение имеет изучение эллинистического влияния на культ Фортуны, а также взаимосвязи её с другими божествами "--- этрусской Норцией, римской Felicitas, эллинистическими Тюхе и Исидой. Отдельному рассмотрению подвергаются важные святилища Фортуны в Лации, за пределами Рима: в Пренесте и Анции.

% Warde Fowler: его труд представляет собою описание религиозных праздников по месяцам и дням. Ценность его работы: обзор различных мнений касательно этимологии имени Фортуны; обращение ко множеству эпиграфических источников, т.к. именно фасты дают нам информацию о религиозных праздненствах; 

Самой ранней из используемых нами работ, но, тем не менее, одной из наиболее ценных и до сих пор актуальных, является монография У.~Фаулера, посвящённая римским праздненствам эпохи Республики\footnote{Warde Fowler, W. The Roman festivals of the period of the Republic. London, 1899.}. Автор организовал изложение материала по аналогии с римскими фастами, по месяцам года и дням, когда проводились праздненства, посвящённые определённым божествам. В связи с таким построением своей работы, У. Фаулер обращается, в первую очередь, к истории важнейших храмов Фортуны, с которыми были связаны праздненства, внесённые в фасты: Фортуны на Бычьем форуме\footnote{Ibid. P. 156--157.}, Мужской Фортуны\footnote{Ibid. P. 68}, Fortis Fortunae\footnote{Ibid. P. 161--163.}, Фортуны Примигении в Пренесте\footnote{Ibid. P. 72, см. тж. Ibid. P. 165--168.}. Конкретная дата, связанная с каким-либо божеством, служит для У.~Фаулера поводом к рассмотрению проблемы в более широком ключе. Так, он затрагивает важные темы происхождения богини Фортуны и, в частности, этимологии её имени\footnote{Ibid. P. 163--171.}. Согласно У. Фаулеру, общепринятой гипотезой являлось сведение имени Фортуны к глаголу \textit{ferre}, что позволяет толковать функцию богини как \textit{dea, quae fert}\footnote{Ibid. P. 163--164.} (т.е. богиня, приносящая различные блага, в том числе помогающая деторождению). Такая точка зрения до сих пор остаётся актуальной, несмотря на то, что это всего лишь предположение\footnote{Arya,~D.~A. Op. cit. P. 39--40.}.

Одной из самых фундаментальных работ по римской религии является монография Г.~Виссовы <<Religion und Kultus der R\"{o}mer>>. Отдельная глава в ней посвящена исследованию культа Фортуны. Виссова делит римских божеств на di indigetes (тех, почитание которых было установлено царём Нумой, и которые, поэтому, могут считаться исконными римскими богами) и di novensidenses, тех, чьи культы были заимствованы позднее. Указывая, что почитание Фортуны официально учредил Сервий Туллий, Г.~Виссова относит эту богиню ко второй категории\footnote{Wissowa, G. Religion und Kultus der R\"{o}mer. M\"{u}nchen, 1902. S. 206.}. Он даёт обстоятельный обзор истории культа Фортуны начиная со времён Сервия Туллия и до эпохи Империи. Г. Виссова подчёркивает первоначальную связь Фортуны с такими божествами судьбы как как италийские парки и греческие мойры\footnote{Ibid. S. 213.}, таким образом, он не разделяет точку зрения, что Фортуна изначально была исключительно богиней плодородия. Интересно его сравнение представления древних о Фортуне с представлением о fatum или fata\footnote{Ibid. S. 213--214.}.

В энциклопедическом словаре Брокгауза и Ефрона в статье о Фортуне отмечается, что её культ был одним из самых древних заимстствованных италийских культов\footnote{<<Энциклопедический словарь>>. Т. XXXVI. СПб, 1902. С. 320--321.}. Указывается вероятный путь заимствования: из Лация, где находились древние святилища в Анции и Пренесте. В статье обрисовывается история становления культа, отмечается, что <<вскоре он дифференцировался на сотни и тысячи отдельных культов>>\footnote{Там же. С. 320.}, характеризуются разные стороны характера Фортуны: покровительство определённым лицам, группам, учреждениям, дням, временам года и т.п. Не остаются без внимания также изобразительные атрибуты, которыми древние наделяли богиню счастья: рог изобилия, шар, руль. Подчёркивается слияние Фортуны с другими божествами, в частности, с эллинистической Исидой.

Наиболее подробное исследование культа Фортуны мы находим в энциклопедии Pauly-Wissowa\footnote{Paulys Real-Encyclop\"{a}die der classischen Altertumwissenschaft. Hb 13. Stuttgart. 1910. S. 12--42.}. Статья разделена на две части: в первой рассматриваются общие проблемы изучения культа Фортуны, во второй "--- собственно культ в разнообразных своих проявлениях, поклонение Фортуне под различными именами (Beinamen). Примечательно, что автор статьи подчёркивает характер случайности, присущий Фортуне изначально\footnote{Ibid. S. 13.}. Рассмотрению подвергаются различные стороны многогранного и противоречивого характера богини: её связь с эллинистической \graeca{T'uqh} и греческой мифологией, матрональный характер её культа, общеиталийские аспекты поклонения Фортуне, её роль как богини-предстказательницы\footnote{Ibid. S. 14--15.}.

С.~Бейли в монографии, посвящённой этапам развития римской религии, указывает, что Фортуна первоначально была божеством плодородия и, в частности, деторождения, а как олицетворение случайности была переосмыслена под влиянием культа Тюхе в городах Великой Греции\footnote{Bailey, C. Phases in the religion of ancient Rome. California, 1932. P. 137.}. Он отмечает связь Фортуны с эллинистической Исидой, каковая прослеживается во времена Империи\footnote{Ibid. P. 255, 258.}. Интересно, что Бейли соотносит Фортуну и \graeca{T'uqh} с таким понятием как fatum\footnote{Ibid. P. 234, 255.}.

Ф.~Альтхайм в обширной монографии, посвящённой римской религии\footnote{Altheim,~F. A history of Roman religion. L., 1938.}, касается лишь некоторых аспектов культа Фортуны. Так, он рассматривает влияние эллинистических представлений о богине Тюхе на культ Фортуны\footnote{Ibid. P. 190.}, рассматривает частные случаи почитания Fortunae Huiusque Diei\footnote{Ibid.} и Fortunae Reducis\footnote{Ibid. P. 268.}.

Х.~Розе полагает, что изначально культ Фортуны имел сельскохозяйственный характер (<<there are some fairly definite indications that she was originally an agricultural deity>>\footnote{Rose,~H.~J. Greek and Roman religion. NY, 1959. P. 238.}, пишет он). С осторожностью он указывает на связь культов Фортуны и Матери Матуты. По его мнению, характер богини случая и удачи она приобрела позднее, в силу того, что в сельском хозяйстве многое зависит от обстоятельств, над которыми земледелец не властен\footnote{Ibid.}. Далее, полагает Х.~Розе, Фортуна со временем приобрела множественные проявления (Мужская Фортуна, Женская и т.д.). Также он отмечает существование знаменитых лацийских святилищ: оракула в Пренесте и храма в Анции\footnote{Ibid. P. 238--239.}.

% Латте. Что он нам даёт? Описание различных храмов Фортуны, перечисление их
В 1960 г. К.~Латте издал фундаментальный труд по римской религии "--- в нём, по утверждению самого автора, на основании новых данных должны были быть пересмотрены положения Г.~Виссовы, со времени публикации работы которого прошло уже более полувека\footnote{Latte,~K. R\"{o}mische Religionsgeschichte. M\"{u}nchen, 1960. S. VII.}. Отдельная глава в этой монографии посвящена богине Фортуне. Автор указывает на общелатинское происхождение культа Фортуны\footnote{Ibid. S. 178.}. Привлекая широкий фактологический материал, он описывает историю развития культа Фортуны в Риме от времён Сервия Туллия до эпохи Империи. Отметим, что в отношении богини Фортуны труд К.~Латте носит менее концептуальный характер по сравнению с работой Г.~Виссовы. Сосредоточиваясь на подробном изложении фактов, К.~Латте оставляет в стороне те проблемы интерпретации Фортуны, которые затрагивает в своём труде Г.~Виссова.

Работа А.~И.~Немировского <<Идеология и культура раннего Рима>>\footnote{Немировский~А.~И. Идеология и культура раннего Рима. Воронеж, 1964.} посвящена первоначальным этапам становления римской религии. Автор обращается к истокам культа Фортуны, относящимся ко времени царя Сервия Туллия. Он полагает Фортуну изначально материнским божеством\footnote{Там же. С. 72.}, подробно останавливаясь на её качествах женской богини. Но он считает, что Фортуна одновременно являлась и божеством судьбы\footnote{Там же. С. 74.}, указывая на существование оракулов, связанных с этой богиней. В круг внимания автора попадают святилища Фортуны, основанные в царское время и в эпоху ранней Республики: святилище на Бычьем форуме, храм Женской Фортуны и Фортуны Примигении в Пренесте. А.~И.~Немировский подробно останавливается на рассмотрении богатого мифологического материала, относящегося к интересующей его ранней эпохе\footnote{Там же. С. 73--74.}. Обращается он и к вопросу об этимологии имени Фортуны и, со ссылкой на М.~Маркони, сводит её к слову \textit{fortis [for c tis]} в значении <<сильная>>, <<богатая>>, <<плодовитая>>, а также <<беременная>>\footnote{Там же. С. 72.}.


% Штаерман: богатый фактический материал с опорой на многочисленные источники. К сожалению, то, что касается культа Фортуны, она не рассматривает концепции, имеющие место быть в историографии.


К истории культа Фортуны обращался и знаменитый религиовед Ж.~Дюмезиль\label{DumezilFort}. Вопросу возникновения этого культа в Риме посвящена его статья <<Сервий Туллий и Фортуна>>\footnote{Dum\'{e}zil,~G. Servius et la fortune: essai sur la fonctíon sociale de louange et de bl\^{a}me et sur les \'{e}l\'{e}ments indo-europ\'{e}ens du cens romain. 1943.}. Воззрения, изложеннные в этой статье, он повторял и на страницах своей работы <<Архаическая римская религия>>\footnote{Dum\'{e}zil,~G. La Religion romaine archa\"\i{}que, avec un appendice sur la religion des \'{E}trusques. 1966.}. К сожалению, эти работы недоступны для нас, поэтому мы сообщаем их содержание в пересказе А.~И.~Немировского\footnote{Немировский~А.~И. Указ. соч. С. 85--86.}. Дюмезиль исходил из своей концепции первоначального деления индоевропейского общества на три общественных группы (жрецов, воинов и земледельцев), у каждой из которых были свои боги-покровители\footnote{См. Дюмезиль,~Ж. Верховные боги индоевропейцев. М., 1986.}. Экстраполируя это положение на римскую историческую действительность, он указывает в качестве такой троицы богов-покровителей Юпитера, Марса и Квирина. Устранение этого архаического деления на три группы античная традиция приписывает Сервию Туллию, который также сыграл решающую роль в утверждении культа Фортуны в Риме. Дюмезиль полагает, что мероприятия Сервия в сакральной области полностью соответствуют его социальным преобразованиям. Фортуна "--- не просто материнское божество, но богиня, олицетворяющая богатство как основной принцип огранизации общества.

В своей работе, посвящённой индоевропейской религии в целом, Дюмезиль сопоставляет Фортуну с ведийским божеством удачи и случая Бхаги\footnote{Дюмезиль~Ж. Верховные боги индоевропейцев. С. 75, 80, 131--132.}. 

Отметим, что особенности исследовательского подхода Дюмезиля заключаются в том, что, используя методы сравнительной мифологии, он рассматривает в римской религии, в первую очередь, общее, что связывает её с индоевропейской культурой и религией в целом. Он задаётся целью выявить концепции, лежащие в основе общеиндоевропейских представлений о религии, поэтому к интересущей нас проблеме он подходит с противоположной стороны, ибо нас интересует par excellence то особенное, что составляет отличительные черты культа Фортуны в Риме. А.~И.~Немировский критиковал подход Ж.~Дюмезиля, указывая, в частности, что <<в древнем Риме не существовало особой жреческой касты, которую Ж.~Дюмезиль постулирует у всех индоевропейских народов>>\footnote{Немировский~А.~И. Указ. соч. С. 177.}.

В работе Дж. Фергюсона о религиях Римской империи отдельная глава посвящена обзору богини Тюхе. Таковой выбор темы не случаен: в эпоху Империи, выходя за пределы Рима и даже Италии, мы наблюдаем, как размываются границы представления о божествах, и Фортуна начинает идентифицироваться с Тюхе, так, что сами древние авторы зачастую не разделяют их. Дж.~Фергюссон указывает, что если Плутарх пишет о Тюхе, подразумевая римскую Фортуну, то Плиний Старший "--- наоборот, повествуя о Фортуне, подразумевает общеимперский культ Тюхе\footnote{Ferguson,~J. The religions of the Roman empire. Southampton, 1970. P. 79.}. В его работе большое внимание уделяется представлению древних авторов об этих двух соотносящихся божествах. Опираясь на множество нарративных источников, в том числе на философские и художественные произведения, Фергюсон рисует образ противоречивой, ненадёжной и непостоянной богини случая и удачи, отношение к которой в эпоху античности было двойственным. Отметим, что в то время как многие другие авторы уделяли большое внимание ранним этапам становления культа Фортуны в Риме, Фергюсон показывает, что и в эпоху поздней Империи, Домината, роль культа Фортуны-Тюхе была высока\footnote{Ibid. P. 86--87.}. 

В <<Encyclop\ae{}dia Britannica>> в статье, посвящённой римской религии\footnote{Encyclop\ae{}dia Britannica. Macrop\ae{}dia. Vol. 15. Chicago etc. 1978. P. 1063.}, сообщается, что Фортуна, чей культ был основан в Риме в конце царского периода, была первоначально <<a farming deity>>, божеством плодородия, впоследствии же стала идентифицироваться с эллинистической богиней Тюхе, покровительницей городов и олицетворением удачи.

В своём труде, посвящённом социальным аспектам римской религии, Е.~М.~Штаерман обращается к богатому фактическому материалу, связанному с культом Фортуны на протяжении всей римской истории "--- от царских времён до эпохи Империи\footnote{Е.~М.~Штаерман. Социальные основы римской религии. М., 1987.}. В связи с особенностями построения этой работы Е.~М.~Штаерман, в ней не выделяются отдельные главы, посвящённые конкретным божествам, в том числе и Фортуне. Это не позволило автору обратиться к каким-либо общим концепциям, каковые сложились в современной науке по поводу представления об интересующей нас богине или об истории её культа, тем не менее, ценность этой работы заключается в том, что Е.~М.~Штаерман вписала историю культа Фортуны в общую картину развития религиозных институтов римского общества.

Д.~Арья создал довольно пространную диссертацию, посвящённую богине Фортуне в эпоху Римской империи. Большое внимание в ней он уделяет привлечению изобразительного материала, описанию и анализу памятников искусства. Довольно широко, хотя, на наш взгляд, несколько бессистемно, он привлекает и материал нарративных источников. Хотя темой его исследования заявлен культ Фортуны в эпоху Империи, хронологические рамки работы несколько шире, и Д.~Арья захватывает в своём обзоре также и период поздней Республики. Говоря об актуальности выбранной темы, автор подчёркивает, что в эпоху Империи Фортуна приобретает новую значимость (новую иконографию, культовые ассоциации и разновидности культа, связанные с императорской властью)\footnote{Arya,~D.~A. Op. cit. P. V, VIII--IX.}. Несмотря на заявленную тему, Д.~Арья не рассматривает в своей работе период империи целиком, сосредоточивась на изучении эпохи Августа и сделав в конце работы некоторый экскурс вплоть до времени Севера "--- в то время как работа Дж.~Фергюсона, о которой шла речь несколько выше, показывает, что эпоха Домината также предоставляет весьма любопытный материал для интересующей нас темы.

В работе Дж.~Шайда о римской религии Фортуне уделено крайне мало внимания. Можно отметить, что автор включает культ Женской Фортуны в число матрональных культов\footnote{Шайд~Дж. Религия римлян. С. 137.} и отмечает, что участвовать в поклонении богине могли также и рабы\footnote{Там же. С. 160.}.

В энциклопедии <<Brill's New Pauly>> богине Фортуне посвящена отдельная статья\footnote{Brill's New Pauly. Ed. by H.~Cancik and H.~Schneider. Vol. 5. Leiden-Boston, 2004. P. 505--509.}. В ней даётся краткий обзор истории культа в Риме и за его пределами, в Лации. Отмечается, что крайне трудно выдвинуть единую версию происхождения этой богини и её функций\footnote{Ibid. P. 505.}. Отдельный параграф посвящён представлению о богине Фортуне как в письменных источниках, так и в изобразительном искусстве\footnote{Ibid. P. 508.}.

% Написать про то, что....

