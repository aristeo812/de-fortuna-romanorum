\chapter*{Заключение}
\addcontentsline{toc}{chapter}{Заключение}

%Подводя общие итоги исследования, отметим? что каждый из трёх периодов, на которые мы разбили рассматриваемый нами хронологический отрезок, обладает своими особенностями не только в плане темы исследования, но также и в плане полноты и достоверности источников, проливающих свет на .

%Культ Фортуны в Риме возникает в эпоху Сервия Туллия. Первые храмы Фортуны были созданы на берегу Тибра.

Первые храмы Фортуны возникают в Риме в правление царя Сервия Туллия: это храм Фортуны и Матери Матуты на Бычьем форуме и храм Форс Фортуны на берегу Тибра. Хотя археологические данные свидетельствуют о том, что святилище на Бычьем форуме существовало и в более отдалённую эпоху, мы не можем сказать ничего определённого о характере отправлявшегося там культа. Свидетельства письменных источников о культе Фортуны в царскую эпоху весьма многочисленны, однако представляют собой, по большей части, легенды и сказания об отношениях этой богини и царя Сервия. Фактические данные, предоставленные нам археологией, позволяют утверждать, что античная традиция, возводящая основание культа Фортуны в Риме к эпохе Сервия Туллия, в целом, достоверна. К сожалению, точных сведений о том времени мы имеем крайне мало, что не позволяет нам со всей определённостью назвать пути проникновения культа Фортуны в Рим и судить о его первоначальном характере.

Недостаток фактических данных о самом раннем этапе становления культа Фортуны в Риме заставляет современных исследователей выдвигать различные гипотезы о его происхождении и наиболее архаических представлениях римлян о Фортуне. Нам следует признать несостоятельными предположения о том, что первоначально Фортуна воспринималась как богиня просто (доброй) удачи или судьбы: такие гипотезы не соответствуют фактам, свидетельствующим о том, что Фортуна в самую раннюю эпоху уже предстаёт могущественной богиней, чей образ достаточно сложен. Бесспорна изначальная связь Фортуны с материнскими божествами, о чём говорят культы этой богини на Бычьем Форуме и Женской Фортуны (возникающий в раннереспубликанскую эпоху). Также Фортуна соотносилась с идеей человеческой судьбы, что раскрывается в популярных сказаниях, сложенных о Сервии Туллии и в том, что со святилищем Фортуны в Пренесте был связан знаменитый оракул "--- впрочем, мы не можем точно датировать возникновение пренестинского культа. В Пренесте Фортуна почиталась также как богиня-мать.

Топографическое положение первых храмов Фортуны в Риме на берегу Тибра, а также популярность этой богини par excellence в плебейской среде (о чём также свидетельствуют мифы о царе Сервии), возможно, являются указаниями на пути проникновения этого культа в Рим, а также на его связь с лацийскими или даже общеиталийскими религиозными представлениями. Такой характер генезиса культа Фортуны предопределил его <<периферийное>> положение в Риме, которое постепенно преодолевалось в период Республики.

% Эволюция в эпоху Р-ки. Новые данные источников

Для республиканского периода, так же, как и для царского, основной массив фактических данных мы черпаем из письменных источников. Археологические свидетельства крайне скудны, однако для царской эпохи они имеют решающее значение, потому что являются подтверждением достоверности античной традиции; для эпохи Республики значение собственно археологических данных, в целом, падает (по сравнению с другими типами источников), однако многочисленные эпиграфические памятники, а также изобразительные свидетельства и монеты позволяют не только подтвердить, но и дополнить сообщения древних авторов.

Эпоха Республики предоставляет нам более многочисленные и точные свидетельства источников, хотя о многих формах культа мы не знаем ничего, кроме когноменов. Тем не менее, мы можем точно датировать возникновение крупных храмов Фортуны: Женской (486 г. до н.э.), Форс Фортуны (293 г. до н.э.), Фортуны Примигении Римского Народа (194 г. до н.э.), Всаднической (173 г. до н.э.). Эти сведения мы можем считать вполне достоверными.

% Обетование храмов на полях битвы. Это закономерный этап развития
Для республиканской эпохи мы выделяем три основные тенденции развития культа Фортуны. Во-первых, это распостранение его <<вширь>>, т.е. появление множества отдельных, конкретных форм культа. Во-вторых, распространение культа <<по-вертикали>> связано с тем, что Фортуна становится покровительницей различных общественных и социальных групп: женщин, девиц, мужчин, всадников, в конце концов, всего римского народа. И, в-третьих, Фортуна начинает покровительствовать военному успеху римлян: первые признаки этого мы можем видеть в обстоятельствах создания храма Женской Фортуны, построенном в честь того, что женщины отвратили от Рима угрозу со стороны Кориолана. Однако переломный момент в развитии культа приходится на рубеж III--II вв. до н.э., когда сначала обетуется на поле битвы, а потом строится храм Фортуны Примигении Римского Народа: в результате этого возникает <<государственная>> форма культа Фортуны. Образ Фортуны Римского народа впоследствии используется в качестве инструмента государственной пропаганды "--- именно этот когномен единственный появляется на монетах позднереспубликанского периода.

Все эти тенденции находятся в русле развития римской религии. Более того, рубеж III и II вв. до н.э. служит одной из важнейших вех римской истории: именно после победы над Ганнибалом Рим становится сильнейшей средиземноморской державой, и перед ним открываются перспективы дальнейшего расширения своего могущества. То, что именно в этот период происходит значительная трансформация культа Фортуны, показывает неразрывную связь его истории с историей Римского государства в целом, указывает, что, несмотря на первоначальную популярность, в основном, в плебейской среде, Фортуна играла далеко не последнюю роль в жизни всей римской civitas.

Именно во 2-й пол. III--II вв. до н.э. начинается интенсивный процесс восприятия римлянами греческой культуры и образования. Греческие представления о богине Тихе, ассоциированной с Фортуной, привели к возникновению в письменной традиции образа, называемого нами \graeca{T'uqh}-\textit{Fortuna}, представляющего собой олицетворение силы слепого случая, переменчивости обстоятельств, божества, стоящего вне сферы нравственности, покровительницы как достойных, так и недостойных. Этот образ впервые мы находим у Пакувия, и в дальнейшем у некоторых других античных авторов, как римских, так и греческих. Образ этот является плодом влияния эллинистической культуры на культуру римскую, и мы не видим подтверждений тому, чтобы он оказал определяющее влияние на римский культ или перевернул типично римские представления о Фортуне. У некоторых латинских авторов (Плавта, Горация) литературный образ Фортуны несёт в себе черты исконно римских или италийских представлений об этой богине.

Яркий антропоморфный образ Фортуны, который мы наблюдаем с момента её появления на исторической сцене на протяжении всего рассматриваемого нами периода, свидетельствует о том, что перед нами <<полноценная>> богиня, и мы не можем причислить её к категории обожествлённых абстракций, хотя отдельные аспекты её образа можно признать таковыми. Рассматривая историю развития культа Фортуны и разнообразие представлений о ней, мы, всё же, можем выделить то ядро, вокруг которого эти представления строятся, выражением чего являются: мы можем назвать Фортуну богиней счастья и неразрывно связанного с ним несчастья.

К началу императорской эпохи мы видим необычайное разнообразие форм поклонения Фортуне в Риме, которые обладают не только впечатляющей количественной характеристикой "--- огромной популярностью, но также и качественной, ибо свои Фортуны были у различных общественных групп римской civitas "--- у фамилий, матрон, всадников и всего римского народа, свои Фортуны покровительствовали определённым местам (баням, житница, просто <<этому месту>>, huius loci). Таким образом, отдельные аспекты культа Фортуны играют важную роль в совершенно различных областях жизни римской civitas, что говорит о том необыкновенном почитании, которым эта богиня пользовалась у всех римлян.

Культ Фортуны не пережил языческого Рима, и теперь типично римское представление об этой богине необходимо воссоздавать научными методами, но яркий образ олицетворённой Фортуны, живущей на страницах сочинений писателей, дожил до наших дней и является частью нашей культуры, хотя, насколько мы можем судить, он имеет мало общего с той богиней, которой в VI в. до н.э. были посвящены два храма на берегах Тибра в тогда ещё мало кому известном городе.



