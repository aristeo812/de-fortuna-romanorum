\chapter{История культа Фортуны в Риме}

\section{Культ Фортуны в царскую эпоху}

Античные авторы почти единодушно приписывают сооружение первых храмов Фортуны в Риме Сервию Туллию. Единственное исключение "--- сообщение Плутарха о том, что царь Анк Марций построил храм <<Мужской Фортуны>> (\graeca{T'uqh >andre'ia}, Plut. De Fort. Rom. 5, Mor. 318 E--F). Сообщения Плутарха о храмах Фортуны, относящихся к царской эпохе, будут разобраны нами отдельно, в этом же разделе мы поместили рассмотрение тех культов, возникновение которых, по данным различных источников, мы можем однозначно отнести к эпохе Сервия Туллия.

\subsection{Храм Фортуны на Бычьем форуме}\label{FortunaInForoBoario}

Дионисий Галикарнасский сообщает, что Сервий Туллий построил в Риме два храма (\graeca{nao~i}) Фортуны, один из которых "--- храм\footnote{\textit{Templum} согласно Liv. XXIV.47.15, Ovid. Fast. VI.625, \textit{aedes} согласно Liv. V.19.6, XXV.7.5, Val. Max. I.8.11.} на Бычьем форуме\footnote{Dion. Hal. IV.27.7: \graecafn{nao`us d'uo kataskeuas'amenos T'uqhs \ldots{} t`on m`en >en >agor~a| t~h| kaloum'enh| Boar'ia|}.}. Этот храм был посвящён двум божествам: Фортуне и Матери Матуте. Ливий нередко упоминает двух божеств вместе, говоря об этом храме, когда ему нужно указать конкретное местоположение на карте древнего Рима (см. Liv. XXIV.47.15, XXV.7.6, XXXIII.27.4). 

% Связь с культом Матери Матуты, двойной храм
% Праздник матралий

Храм был освящён Сервием Туллием 11 июня (Ovid. Fast. VI.569 и далее). В этот день римляне отмечали праздник Матралий в честь Матери Матуты, чей культ был широко распространён в Италии\footcites[Pp. 154--157]{Fowler1899}[Pp. 150--151]{Scullard1981}. Считается, что её имя происходит от слова \textit{maturus}, <<зрелый>>\footcite[S. 88]{Latte1960}. Плутарх отождествляет её с Левкофеей, или Ино\footnote{См. \cites[P. 154]{Fowler1899}[P. 154]{Latte1960}. А.~И.~Немировский также указывает, что такое отождествление характерно для римских антикваров конца Республики. См. \cite[С. 75--76]{Nemirovsky1964}.}, (Plut. Camill. 5). Таким образом, мы можем нарисовать себе образ довольно архаического женского божества, связанного с материнством и деторождением, покровительницы матрон\footcites[P. 155]{Fowler1899}[P. 150]{Scullard1981}. Соотнесение культа Фортуны с культом Матери Матуты позволяет пролить больше света на представление древних римлян о первой богине.

% Археологические свидетельства

\label{Archaeologia}Обратимся к археологическим свидетельствам. Остатки этого храма были открыты в ходе раскопок района церкви Сан-Омобоно в 1937 г. Культурный слой в районе храма прослеживается вплоть до раннего железного века, когда этот участок ещё не был священным. В VII "--- нач. VI вв. до н.э. на этом месте существовал алтарь, ориентированный на восток, но храма ещё не было. Ко второй четверти VI в. до н.э. относятся остатки фундамента т.н. <<архаического храма>> (Рис. \ref{pic:ArchaicTemple1}), создание которого приписывается Сервию Туллию\footcite[P. 246]{Richardson1992}. Храм был ориентирован по линии север-юг, выходил фасадом на север, на Vicus Iugarius, а тыльной своей стороной был обращён в сторону Бычьего форума. Обратим внимание на план архаического святилища: мы видим, что храм, посвящённый двум божествам, имел единое внутреннее святилище (cella). Подобная планировка "--- ещё одно свидетельство культовой близости Фортуны и Матери Матуты. Храм несколько раз перестраивался. В 396 г. до н.э. его перестроил Камилл после победы над Вейями (Liv. V.19.6). Во время пожара 213 г. до н.э. храм сгорел (Liv. XXIV.47.15) и был восстановлен в следующем году (Liv. XXV.7.6), получив новую планировку и став по-настоящему <<двойным храмом>>\footcite[P. 35, 246]{Richardson1992} (Рис. \ref{pic:TwinTemples}).

% Свидетельства фастов

% История статуи Фортуны в этом храме

В этом храме находилась позолоченная деревянная статуя Фортуны, или, по некоторым сообщениям, Сервия\footnote{Е.~М.~Штаерман видит в этом факте <<туманный намёк на обожествление Сервия Туллия>>. См.: \cite[С. 47--48]{Shtaerman1987}.}. Дионисий Галикарнасский пишет о статуе Сервия (Dion. Hal. IV.40.7), Овидий тоже говорит о Сервии и добавляет, что голова статуи была покрыта тогой (Ovid. Fast. VI.570--572, 623--624). Впрочем, Плиний Старший сообщает, что в храме стояло изображение не самого Сервия, а всё же Фортуны, посвящённое богине царём и покрытое царской претекстой, которая сохранялась до смерти Сеяна, будучи на протяжении 560 лет удивительным образом не подвержена ни тлению, ни вреду от моли\footnote{Plin. NH VIII.197: \textit{Servi Tulli praetextae quibus signum Fortunae ab eo dicatae coopertum erat, duravere ad Seiani exitum, mirumque fuit neque diffluxisse eas neque teredinum iniurias sensisse annis quingentis sexaginta}}. Надо думать, именно об этой тоге чуть ранее говорит Плиний, ссылаясь на Варрона, что-де её, тогу, изготовила Танаквиль\footnote{Ibid. 194: \textit{\ldots{}~durasse prodente se auctor est M.~Varro, factamque ab ea [Tanaquile "--- В.~Ж.] togam regiam undulatam in aede Fortunae, qua Ser. Tullius fuerat usus}.}. Это покрывало было настолько священным, что матроны, отправляющие службу в храме, не могли к нему прикасаться и должны были молиться на расстоянии\footnote{Ovid. Fast. VI.621--622: \textit{parcite, matronae, vetitas attingere vestes / (sollemni satis est voce movere preces)}.}. 

Дион Кассий пишет о том, что в храме стояло, всё же, изображение Фортуны, созданное в царствование Сервия, и что его забрал в свой дом Сеян, окружив большим почитанием, а во время жертвоприношения статуя богини повернулась к нему спиной\footnote{Dio Cass. LVIII.7.2--3: \graecafn{T'uqhs t'e ti >'agalma, <`o >egeg'onei m'en, <'ws fasi, Toull'iou to~u basile'usant'os pote >en t~h| <R'wmh|, t'ote d`e <o Se"ian'os o>'ikoi te e>~iqe ka`i meg'alws >'hgallen, a>ut'os te j'uwn e>~iden apostref'omenon}.}. Данное сообщение Диона Кассия проливает некоторый свет на слова Плиния о том, что тога, покрывавшая изображение, сохранялась до смерти Сеяна.
%  τύχης τέ τι ἄγαλμα, ὃ ἐγεγόνει μέν, ὥς φασι, Τουλλίου τοῦ βασιλεύσαντός ποτε ἐν τῇ Ῥώμῃ, τότε δὲ ὁ Σεϊανὸς οἴκοι τε εἶχε καὶ μεγάλως ἤγαλλεν, αὐτός τε θύων εἶδεν ἀποστρεφόμενον ... 1 καὶ μετὰ τοῦθ᾽ ἕτεροι συνεξιόντες σφίσιν

Это изображение чудесным образом уцелело во время пожара 213 г. до н.э. (Ovid. Fast. VI.625, Val. Max. 1.8.11). Дионисий, который должен был видеть эту статую собственными глазами, подчёркивает, что хотя сам восстановленный храм и \textit{<<принадлежит искусству новых мастеров>>}\footnote{Dion. Hal. IV.40.7: \graecafn{t~hs kain~hs >esti t'eqnhs}.}, \textit{<<статуя, как и прежде, убеждает в том, что она является древним произведением>>}\footnote{Ibid: \graecafn{<h d> >eik'wn, <'oia pr'oteron >'hn, >arqaik`h t`hn kataskeu'hn}.}. Он сообщает далее, что это изображение и в его время пользуется поклонением у римлян.

%Противоречивые мнения древних авторов о том, чьё именно изображение стояло в храме, объясняются, быть может, тем, что за давностью лет покрытая тогой статуя Фортуны стала восприниматься как статуя Сервия, хотя трудно найти неопровержимые аргументы как за, так и против этой гипотезы\footcite[С. 72]{Nemirovsky1964}. Впрочем, об отношениях Сервия Туллия и Фортуны среди римлян ходило много сказаний, и нет ничего удивительного в том, что про это изображение рассказывали то одно, то другое: обе эти фигуры были близки в мифологическом сознании римлян. Вероятно, если бы могли оказаться на улицах позднереспубликанского Рима и расспросить самих горожан, мы вряд ли бы внесли ясность в наши представления; скорее всего, мы бы услышали такие же противоречивые мнения. Достоверно мы можем сказать только, что это было древнее, почитаемое изображение, что Дионисий Галикарнасский видел его своими глазами, и что после смерти Сеяна пропала, как минимум, священная тога, покрывавшая его, а может быть, и сама статуя.

Противоречивые мнения древних авторов о том, чьё именно изображение стояло в храме, объясняются, быть может, тем, что за давностью лет покрытая тогой статуя Фортуны стала восприниматься как статуя Сервия, хотя А.~И.~Немировский считает, что трудно найти неопровержимые аргументы как за, так и против этой гипотезы\footcite[С. 72]{Nemirovsky1964}. Стоит, однако, обратить внимание на то, что о статуе Сервия говорят Дионисий Галикарнасский и Овидий, а о статуе Фортуны "--- Плиний и Дион Кассий. Скорее всего, сообщения всех этих писателей восходят к одному и тому же источнику: народным сказаниям о храме и стоявшей в нём статуе. Однако два последних автора писали уже после смерти Сеяна, поэтому можно предположить, что когда тот забрал статую в свой дом, она лишилась своего священного покрывала, и тогда всем стало ясно, что это именно Фортуна.

Обратим внимание на тот срок, который, согласно Плинию, изображение стояло в храме нетронутым: 560 лет вплоть до смерти Сеяна (Plin. NH VIII.197). Отняв от даты смерти Сеяна 560 лет, получим примерно 530 г. до н.э., т.е. царствование Сервия Туллия. Нам, однако, известно, что царь Нума запретил римлянам изображать богов в виде людей или животных, и в течение 170 лет этот запрет сохранялся (Plut. Numa 8.7). Отняв от вычисленной нами даты ещё 170 лет, мы получим примерно 700 г. до н.э., т.е. царствование Нумы. На основании этого мы можем считать изображение Фортуны в храме на Бычьем форуме если не первым, то одним из первых культовых изображений у римлян, что свидетельствует о той важной роли, которую в истории римской религии сыграло основание культа этой богини.

Интересно, что Фортуна на Бычьем форуме почиталась без эпиклезы "--- или, по крайней мере, мы должны признать, что она нам неизвестна. То, что многие исследователи отождествляют эту Фортуну с Fortuna Virgo, является недостаточно обоснованным и, скорее всего, ошибочным (подробнее об этом см. ниже п. \ref{FortunaVirgo}).

\subsection{Храм Форс Фортуны}\label{FortisFortunae1}

% Про храм. То, где он находится

Согласно античной традиции, Сервий Туллий основал также храм (\textit{fanum} у Варрона, \textit{aedes} у Тита Ливия) богини Счастливого Случая, Форс Фортуны, на берегу Тибра. Варрон пишет: \textit{<<День [богини] Счастливого Случая провозглашён царём Сервием Туллием, который в месяце июне посвятил ей храм у Тибра за городом>>}\footnote{Varr. De ling. Lat. VI.17: \textit{Dies Fortis Fortunae appellatus ab Servio Tullio rege, quod is fanum Fortis Fortunae secundum Tiberim extra urbem Romam dedicavit Iunio mense}.}. Тит Ливий также указывает, что Сервий Туллий освятил храм этой богини (согласно Ливию, консул 238 г. до н.э. Карвилий вложил деньги из военной добычи в строительство храма Форс Фортуны рядом с храмом той же богини, освящённым Сервием Туллием\footnote{Liv. X.46.14: \textit{aere aedem Fortis Fortunae de manubiis faciendam locavit prope aedem eius deae ab rege Ser. Tullio dedicatam}.}). Овидий также сообщает, что сервиев храм Форс Фортуны располагался на берегу Тибра\footnote{Ovid. Fast. VI.775--776: \textit{Ite, deam laeti Fortem celebrate, Quirites: / in Tiberis ripa munera regis habet}.}.

% Дионисий Галикарнасский и Плутарх

Дионисий Галикарнасский, правда, указывает, что Сервий Туллий построил на берегу Тибра храм (\graeca{na'os}) <<Мужской Фортуны>> (\graeca{T'uqh >andre'ia})\footnote{Dion. Hal. IV.27.7: \graecafn{nao`us d'uo kataskeuas'amenos T'uqhs \ldots{} t'on d' <'eteron >ep`i ta~is >hi'osi to~u Teb'erios, <`hn >andre'ian proshg'oreusen, <ws ka`i n~un <up`o <Rwma'iwn kale~itai}.}. Плутарх же приписывает сооружение этого храма (\graeca{T'uqhs andre'ias <ier'on}) царю Анку\footnote{Plut. De Fort. Rom. 5, Mor. 318 E--F: \graecafn{pr~wtos m`en g`ar <idr'usato T'uqhs <ier`on M'arkios >'Agkos \ldots{} t~h| t'uqh| t`hn >andre'ian parwn'omasen}.}. \label{LapsusGraecorum}Однако среди современных исследователей считается общепризнанным, что в данном случае греческие авторы спутали латинское название Fors Fortuna с Fortuna Fortis и неправильно его перевели\footcites[S. 479]{Becker1843I}[S. 180]{Preller1883}[Pp. 510--511]{Swain1989}. Таким образом, оба греческих автора в данном случае говорят о храме Форс Фортуны, а Плутарх спутал ещё и личность основателя храма (остальные "--- и более информированные "--- античные авторы вполне единодушны во мнении, что первый храм Форс Фортуны построил Сервий, так что у нас нет оснований доверять в этом случае Плутарху).

%  ναοὺς δύο κατασκευασάμενος Τύχης ...  τὸν δ᾽ ἕτερον ἐπὶ ταῖς ἠιόσι τοῦ Τεβέριος, ἣν ἀνδρείαν προσηγόρευσεν, ὡς καὶ νῦν ὑπὸ Ῥωμαίων καλεῖται (Dion. Hal. IV.27.7)
% πρῶτος μὲν γὰρ ἱδρύσατο Τύχης ἱερὸν Μάρκιος Ἄγκος (Plut. Mor. 218 E) ... τῇ τύχῃ τὴν ἀνδρείαν παρωνόμασεν (Mor. 218 E)

% Свидетельства фастов и надписей

Dies Fortis Fortunae, о котором пишет Варрон, приходится, согласно Овидию, на 24 июня (Fast. VI.773--784). Подтверждение этому мы находим и в эпиграфических источниках. Согласно Амитернинским фастам, на 24 июня приходился праздник \textit{\label{Fastae}Forti Fortunae trans Tiber(im) | ad milliar(ium) prim(um) et sex(tum)}\footnote{CIL I\textsuperscript{2}, p. 243.}; согласно Эсквилинским фастам: \textit{Fort(i) Fort(unae) t(rans) | T(iberim) ad mil(iarium) | I et [VI]}\footnote{Ibid., p. 211.}; согласно Венузинским фастам: \textit{Fortis Fortunae}\footnote{Ibid., p. 221.}; согласно календарю Филокала: \textit{Fortis Fortunae Solstitium}\footnote{Ibid., p. 266.}.

В самом Риме были найдены посвятительные надписи Форс Фортуне, оставленные ремесленными коллегиями купальщиков\footnote{CIL VI.167 = ILLRP 97: \textit{[F]orte For[tunai] | donum dant | conlegi(um) lani(es) | Piscinenses | Magistreis | coiraverunt | A. Cassi(us) C.l. | T.Cornelius C(orneliae) l.}}, мясников\footnote{CIL VI.168 = ILLRP 98: \textit{Forti For[tunai] | lanies. Ma[gistreis] | L. Maeci(us) M.l. S[- - -], Teupilos Q. Iuni Sal[vi s(ervos)]}.}, цветочников\footnote{CIL VI.169 = ILS 3682 = ILLRP 99: \textit{Forte Fo[rtunai] | violaries, | rosaries, | coronaries. | [M]ag[istreis] | co[iravere]}.}, а также эрариев\footnote{CIL VI.36771 = ILLRP 96: \textit{Conlegia aerarior(um) | Forte Fortunae | donu(m) dant. Mag(istri) | C.Carvilius M.l., L.Munius L.l. [-]acius; | minis(tri) T.Mari Carvil.m. | [- -]stimi, D.Quinctius}.}. Обратим внимание на то, что это коллегии ремесленников почти из самых низов общества, а их магистры, подписавшиеся под посвящениями, почти все вольноотпущенники.

% Праздник, связанный с храмом. Овидий и Цицерон

В день праздника Форс Фортуны в Риме проходили народные гуляния, сопровождавшиеся попойкой, которые в восторженных тонах описывает Овидий: \textit{<<Кто пешком, кто даже в лодках, быстро сбирайтесь / Чтобы не стыдиться хмельными после вернуться домой. / Несите пиршества юношей, о украшенные цветами лодки, / И многие вина выпиваются на середине реки>>}\footnote{Ovid. Fast. VI.777--780: \textit{pars pede, pars etiam celeri decurrite cumba, / nec pudeat potos inde redire domum. / ferte coronatae iuvenum convivia, lintres, / multaque per medias vina bibantur aquas}.}. Возможно, \textit{<<плавание вниз по Тибру в тот праздничный день>>}\footnote{Cic. De fin. V.70: \textit{Tiberina descensio festo illo die}.}, упомянутое Цицероном, также указывает на этот праздник.

% Какие черты культа связаны с городской жизнью

Итак, Форс Фортуна в Риме пользовалась популярностью у весьма широких слоёв населения (Ovid. Fast. VI.781: \textit{plebs colit hanc}), в основном, городских низов. Её почитали ремесленные коллегии, которые мы могли бы назвать непривилегированными, вольноотпущенники и даже рабы (Ibid. 783: \textit{convenit et servis}). Овидий объясняет это тем, что Сервий, создатель храма, сам происходил из народа (Ibid. 781: \textit{qui posuit de plebe fuisse}) и вообще был рождён рабыней (Ibid. 783: \textit{serva quia Tullius ortus}). Возможно, однако, что причиной такой популярности был особый характер праздника Форс Фортуны, на котором мы остановимся подробнее.

% Обряды и поверья, связанные с растениями: 21, 26, 46, 49, 59, 70, 73, 96, 97, 108, 121, 126, 145, 155, 166, 175, 197, 208--209, 229--231, 249, 250, 272,

% Обряды, связанные с водой: 10, 25, 46, 97, 121, 127, 146, 155--156, 175, 186, 209, 229, 272--273,

\subsection{Праздник Форс Фортуны и проблема происхождения богини}

% Толкование праздника. Связь ритуалов с теми, которые в Европе

Праздненство Форс Фортуны приходилось на знаменательную дату, день летнего солнцестояния, известный теперь как день Ивана Купалы. На этот факт обращали внимание ещё Л. Преллер\footcite[S. 180]{Preller1883}, У. Фаулер\footcite[P. 163]{Fowler1899} и Дж. Фрэзер, который называл этот праздник <<a sort of Midsummer Saturnalia>>\footcite[P. 272]{Frazer1911II}. Г. Скаллард, отмечая совпадение дат, указывал, что общность деталей ритуала не убедительна\footcite[P. 156]{Scullard1981}. Ж.~Шампо, считая праздник Ивана Купалы <<солярным культом>>, не находила такового у римлян и делала упор на связь праздничных обрядов Форс Фортуны с водой\footcite[Pp. 212--216]{Champeaux1982}. В целом, исследователи-антиковеды занимают весьма осторожную позицию по отождествлению праздника Форс Фортуны с днём Ивана Купалы.

Специалисты в области этнологии более смелы в своих интерпретациях. Связь праздника летнего солцестояния с культом Фортуны признаётся не только для собствено итальянцев\footcite[С. 5]{Calendar1978}, но также и для немцев. По мнению А. Шпамера, обычай купаться в реке на день св. Иоанна Крестителя, зафиксированный для средневекового Кёльна, пришёл в Германию через Галлию из Рима\footnote{\cite[S. 130]{Spamer1935}, цит. по: \cite[С. 127]{Calendar1978}.}.

Общеизвестно, что праздник Ивана Купалы в традиционной Европе (и России) связан, в первую очередь, с обрядами вокруг огня: костры, горящие колёса и т.п. Ничего подобного в римском празднике Форс Фортуны мы не видим. Однако многочисленные этнографические наблюдения показывают, что для обрядов дня св. Иоанна практически у всех народов Европы характерны ритуалы, связанные не только с огнём, но также с водой (купание, омовение, вообще особые свойства воды в этот день)\footnote{Об обрядах и поверьях, связанных с водой в день Ивана Купалы у различных народов, см: \cite[С. 10, 25, 46, 97, 121, 127, 146, 155--156, 175, 186, 209, 229, 272--273]{Calendar1978}.} и растениями (сбор растений и цветов, плетение венков, украшение жилищ)\footnote{Об обрядах и поверьях, связанных с растениями, см: \cite[С. 21, 26, 46, 49, 59, 70, 73, 96, 97, 108, 121, 126, 145, 155, 166, 175, 197, 208--209, 229--231, 249, 250, 272]{Calendar1978}.}. 

Эти две черты явственно проступают и в известном нам ритуале праздненства Форс Фортуны. Связь с водой "--- это, во-первых, само по себе положение храма на берегу Тибра, во-вторых, обычай плавать на лодках в праздник. Добавим сюда же посвятительную надпись, оставленную коллегией piscenenses\footnote{А. Деграсси связывал их название с общественным прудом, piscina publica. См.: \textit{Degrassi}, ILLRP I, p. 77.} (CIL VI.167 = ILLRP 97). На значимость воды в связи с культом Фортуны указывала Ж. Шампо (см. выше), однако, как кажется, никто из исследователей-антиковедов не заострял внимание на той роли, которую играли в праздненстве Форс Фортуны растительные обряды. Тем не менее, украшение лодок венками (Ovid. Fast. VI.779: \textit{corotanate lintres}) и надпись, оставленная коллегиями цветочников (CIL VI.169 = ILS 3682 = ILLRP 99), свидетельствуют о том, что римляне в этот праздник придавали растениям большое значение, как и все европейские народы в день Ивана Купалы. Наряду с водой, растения выступают в обрядности как символы плодородия, носители животворящих сил природы\footnote{\cite[С. 88]{Pokrovskaya1983}. О роли растительности в мифологическом сознании см. тж.: \cite[С. 251--305]{Eliade1999}.}.

Далее, необходимо подчеркнуть, что праздник Ивана Купалы в первую очередь "--- сельскохозяйственный, земледельческий праздник. Он связан с солярным календарём постольку, поскольку солнечный цикл определяет рост и развитие растительности и, таким образом, оказывается одним из решающих факторов в жизни земледельца\footnote{См., напр., этнографические наблюдения над купальскими обрядами жителей Британских островов: <<Все~\ldots{} обряды и обычаи дня св. Иоанна наиболее полно соблюдались в тех местностях, где господствовало земледельческогое направление сельского хозяйства. В тех же областях, где преобладало скотоводство (особенно овцеводство), день летнего солнцестояния имел в народном календаре меньшее значение и с ним было связано меньше старых обрядов>> (\cite[С. 74]{Calendar1978}).}. Тесная связь с земледелием указывает на древность этого праздника; также праздники солнцеворотов признаются исследователями древнейшими из календарных\footcite[С. 46]{Serov1983}.

% Колумелла

Обращаясь к античным свидетельствам, мы можем видеть, что Форс Фортуна почиталась не только в городе Риме, но и среди земледельцев. Колумелла в X книге советует пахарям, принимающимся за летние работы, произносить незамысловатую молитву этой богине: \textit{<<Но когда нива зазолотится спелыми колосьями, / и в день, когда Солнце выступит из созвездия Близнецов / И охватит сияньем клешни Лернейского Рака, / тогда чеснок с луком, мак с укропом / соединяйте, а когда зазеленеют, несите, связав в пучки, / и многие Форс Фортуне произносите хвалы / и, продав товар, радостными возвращайтесь в имения>>}\footnote{Colum. X.311--317: \textit{Sed cum maturis flavebit messis aristis / atque diem gemino Titan extenderit astro, / hauserit et flammis Lernaei brachia Cancri, / alia tunc caepis, Cereale papaver anetho / iungite, dumque virent, nexos deferte maniplos / et celebres Fortis Fortunae dicite laudes / mercibus exactis hilarisque recurrite in hortos}.}. Работы, которые описывает здесь Колумелла, относятся ко времени вхождения Солнца в созвездие Рака, т.е. к месяцу июню\footnote{Согласно Венузинским фастам, вхождение Солнца в созвездие Рака приходится на 19 июня (CIL I\textsuperscript{2}, p. 221), согласно календарю Филокала "--- на 15 июня (Ibid., p. 266).}. Кроме того, июньский праздник Fortis Fortunae также отмечен в Menologia Rustica (календаре сельских праздников) Colontianum и Valense IV в. н.э.\footnote{\mancite{Ibid., p. 280.}} Приведённые факты напрямую свидетельствуют об аграрном характере культа Форс Фортуны, правда, для периода первых веков н.э.

% Очевидно, что аграрный праздник, тем более, приходящийся на солнцеворот, будет древнее любого, связанного с гражданской жизнью римской общины; таким образом, невозможно предположить, что изначально римский праздник распространился по Лацию и Италии, превратившись в аграрный. Наоборот, следует считать, что изначально аграрное праздненство и земледельческий культ, укоренившись на римской почве, приобрели определённые черты, связанные с городской жизнью.

Итак, выделенные нами общие черты в обряде и несомненная связь культа Форс Фортуны с земледелием позволяют нам утверждать, что ритуал этого римского праздненства напрямую соотносится с ритуалом известного сегодня дня св. Иоанна и имеет, таким образом, сельскохозяйственное происхождение, несмотря на то, что в историческое время, освещённое нашими письменными источниками, это праздник самых широких слоёв городского населения, в основном, бедноты. Однако означает ли это, что Форс Фортуна "--- изначально аграрная богиня, божество плодородия?

Само имя богини указывает на то, что перед нами божество, олицетворяющее случай. Цицерон пишет о богине Форс, \textit{<<которая более выражает неопределённый случай>>}\footnote{\mancite{Cic. De leg. II.28}: \textit{Fors in quo incerti casus significantur magis}.} (чем Фортуны Примигения, Сегодняшнего Дня и т.д.). Ноний Марцелл указывает, что Форс и Фортуна различаются тем, что первая есть преходящий случай (\textit{casus temporalis}), вторая же "--- собственно богиня (\textit{dea ipsa})\footnote{\mancite{Non. 425, 5}: \textit{Fors et Fortuna hoc distant. Fors est casus temporalis, Fortuna est dea ipsa}.}. Таким образом, богиня Форс представляет собой \textit{обожествлённое понятие}.

Но как объяснить появление этой богини в земледельческом культе плодородия? Г.~Экстелл справедливо указывает, что для сельских менологий вообще характерны божества, изначально никак не связанные с аграрным культом (напр. Меркурий, Геркулес, Спес, Салюс) "--- таким образом, земледельцы просто подбирали наиболее значимые календарные праздненства для соответствующих месяцев\footcite[P. 10]{Axtell1907}. Заметим, что Иоанн Креститель также изначально не был каким-то божеством плодородия, не может ли быть то же самое и с Форс Фортуной? Возможно, Сервий Туллий \textit{непреднамеренно} посвятил ей храм именно 24 июня, в день солнцестояния? Тогда мы должны прийти к выводу, что божество случая Форс стало ассоциироваться с древним земледельческим праздником \textit{случайно}. Если принять эту точку зрения, то необходимо заключить, что ритуал праздника Форс Фортуны связан с представлениями о богинях Форс и Фортуне точно так же, как традиционные европейские купальские обряды связаны с евангельской легендой об Иоанне Предтече, т.е. никак. Сам праздник летнего солнцеворота должен быть древнее самого Рима, и он, очевидно, никак не связан с чем-то конкретно римским, с историей римской цивитас; когда Сервий Туллий основывал храм Форс Фортуны, этот праздник уже должен был как-то отмечаться.

Итак, можно предположить, что связь Форс Фортуны с аграрными культами образовалась из-за непреднамеренного совпадения римского праздника этой богини с днём летнего солнцеворота. С ростом влияния Рима праздник Форс Фортуны стал известен в округе, и день летнего солнцестояния стал ассоциироваться именно с ней. Но мы не можем говорить о том, что подобное влияние распространялось далее. Никто не видел гипотетических романизированных галлов, принесших на берега Рейна обряд, исполнявшийся на Тибре.

%Свидетельства Колумеллы и менологий "--- поздние, и они ничего не говорят об архаических представлениях о богине Форс. Хотя мы и видим прямую связь Форс Фортуны с аграрным культом, но не видим связи изначальной.

Связь богини, олицетворяющей Счастливый Случай, с земледельческим праздником ставит больше вопросов, чем даёт ответов. Неизвестно, насколько эта выявленная нами связь может характеризовать архаическое представление римлян о Форс Фортуне, ведь имеющиеся у нас свидетельства Колумеллы и менологий "--- поздние по времени. Всё зависит от того, преднамеренно или нет дата посвящения храма этой богини совпадает с праздником летнего солнцестояния, а это нам неизвестно.



\section{Культовые сооружения, приписываемые Плутархом Сервию Туллию}\label{PlutarchiCulti}

Плутарх описывает храмы и алтари Фортуны в Риме в двух произведениях: <<Об удаче римлян>> (De Fort. Rom., 10, Mor. 322--323) и <<Римских вопросах>> (Quae. Rom. 74, Mor. 281 D--E). В общей сложности, он приписывает Сервию Туллию создание 9 храмов Фортуны; о некоторых из этих культов нам известно по другим источникам, о некоторых "--- нет. В этом разделе мы рассмотрим те когномены Фортуны и связанные с ними культы, основание которых Плутарх относит к царскому времени.

\subsection{Фортуна Дева (Fortuna Virgo)}\label{FortunaVirgo}

В трактате <<Об удаче римлян>> Плутарх сообщает, что храм Фортуны-Девы (\graeca{Parj'enou T'uqhs <ier'on}) находился \textit{<<неподалёку от источника, известного под названием Мускоса>>}\footnote{Plut. De Fort. Rom. 10, Mor. 322~F "--- 323~A: \graecafn{par`a d`e t`hn Mousk~wsan kaloum'enhn kr'hnhn >'eti Parj'enou T'uqhs <ier'on >esti}} (где это "--- неизвестно\footcite[Pp. 153, 158]{Richardson1992}). В <<Римских вопросах>> Плутарх приписывает создание этого храма Сервию Туллию\footnote{Plut. Quae. Rom. 74, Mor. 281 E: \graecafn{kateske'uasen \ldots{} T'uqhs <ier'on \ldots{} >'allo parj'enou}.}. Арнобий, христианский автор III--IV вв. н.э. пишет, что девушки приносили Девственной Фортуне свои одежды\footnote{Arnob. II.67.3: \textit{puellarum togulas Fortunam defertis ad Virginalem}.}. Следует полагать, что такой обряд выполнялся перед выходом замуж\footcite[P. 507]{FortunaBrill2004}. Ноний Марцелл, цитируя трактат Варрона De vita populi Romani, приводит пассаж о Фортуне Деве, облачённой в две \textit{togae undulatae}: \textit{<<Et a quibusdam dicitur esse Virginis Fortunae, ab eo quod duabus undulatis togis est opertum, proinde ut olim reges nostri et undulatas et praetextas togas soliti sint habere>>} (Non. 189, 18). Эти четыре небольшие отрывка "--- всё, что нам известно о Fortuna Virgo или Virginalis.

%  κατεσκεύασεν

Связь обряда Фортуны Девы с \textit{toga undulata}, на что указывает Варрон, позволила многим исследователям отождествить эту богиню с Фортуной, почитавшейся в храме на Бычьем форуме, статуя которой была покрыта тогой, согласно Плинию (Plin. NH VIII.194, 197), вышитой самой Танаквилью\footcites[S. 182]{Preller1883}[Sp. 1510]{Peter1890Fortuna}[S. 207]{Wissowa1902}[Sp. 19]{FortunaOtto1910}[С. 73]{Nemirovsky1964}[P. 503]{Kajanto1981}. Однако согласно вышеприведённым свидетельствам, никто из античных авторов прямо не отождествляет Фортуну на Бычьем форуме с Фортуной Девой; более того, Плутарх указывает, что храм Фортуны Девы находился рядом с Fons Muscosa, а не на Бычьем форуме; неясно, впрочем, насколько точен здесь греческий автор. Эти факты делают более обоснованным предположение, что перед нами два различных культа. Об общности двух описанных обрядов (Фортуны Девы и той, что на Бычьем форуме) мы можем сказать только, что в обоих случах там фигурирует тога или одеяние. Этого явно не достаточно для достоверного отождествления. Таким образом, мы вынуждены заключить, что Фортуна Дева и Фортуна на Бычьем форуме "--- два различных культа, и в последнем случае богиня почиталась, скорее всего, без когномена. Этой точки зрения придерживаются А. Энман\footcite[Комм. к с. 226]{Enman1896} и Ж. Шампо\footcite[P. 270]{Champeaux1982}.

Ясно, что Fortuna Virgo рассматривалась римлянами как покровительница девиц; однако была ли она их покровительницей <<вообще>> или же <<заведовала>> только одной стороной их жизни, переходом из девичества в замужество, мы, располагая имеющимися сведениями, точно сказать не можем.

% Однако ни у кого из исследователей мы не нашли точный перевод этой цитаты, а К. Латте честно признаётся, что не знает, как истолковать выражение \textit{toga undulata}\footcite[S. 180]{Latte1960}. Но здесь мы сталкиваемся с ещё большими трудностями. Действительно, смысл пассажа Варрона, приведённого Нонием, туманный, поскольку в нём отсутствует логическое подлежащее в ablativus absolutus при логическом сказуемом esse. Объяснить это можно тем, что цитата вырвана из контекста (предложение начинается с \textit{Et}), и логическое подлежащее подразумевается, вероятно, исходя из предыдущего повествования Варрона. Очевидно, что \textit{quod} есть quod explicativum, и, возможно, оно относится к логическому подлежащему: этот объект должен быть покрыт двумя тогами. Непонятно также, к чему относится местоимение \textit{eo}, очевидно только, что речь идёт о слове мужского или среднего рода. Быть может, в тексте идёт речь о \textit{жрице}\footnote{Очевидно, отправлять священнодействия Фортуне Деве должны были особы женского пола.} Фортуны Девы, и тогда можно предположить, что \textit{eo} = \textit{simulacro (Fortunae Virginis)}, и \textit{ab eo} означает конкретную жрицу, совершающую обряд перед изображением богини\footnote{Нас не должно удивлять, что жрица могла быть облачена в тогу в ритуальных целях. Л.~Преллер указывает: <<Togen tragen in alter Zeit nicht blos die M\"{a}nner, sondern auch die Frauen>> (\cite[S. 182]{Preller1883}).} (\textit{[sacerdos] ab eo [=simulacro] Virginis Fortunae}). Возможно, однако, что \textit{quod} относится к самой Fortuna Virgo (или её подразумеваемому изображению). Мы должны признать, что истолковать эту цитату, вырванную из контекста, мы можем только приблизительно.


\subsection{Фортуна Примигения на Капитолии (Fortuna Primigenia in Capitolio)}\label{FortunaPrimigeniaInCapitolio}

% Про Культ Примигении в Пренесте
Считается, что культ Фортуны Примигении был перенесён в Рим из Пренесты, где находилось древнее святилище и оракул этой богини (Cic. De div. II.85--86, Liv. XXIII.19.18). Рассмотрение пренестинского храма, находившегося за пределами Рима, выходит за рамки нашего исследования, однако мы ненадолго задержим на нём своё внимание. Несмотря на то, что точная дата основания храма в Пренесте неизвестна, исследователи сходятся в том, что это был древний культ, в котором, возможно, видны следы раннего влияния греческого культа Тихи\footcites[S. 211]{Wissowa1902}[P. 171]{Rose1959}[S. 176]{Latte1960}. Пренестинская Фортуна была изображена в виде матери, держащей на руках младенцев Юпитера и Юнону (Cic. De div. II.85), что говорит о ней как о материнском божестве\footcites[S. 209--211]{Wissowa1902}[P. 165--166, 223--224]{Fowler1899}[P. 268]{Altheim1938}[С. 73]{Nemirovsky1964}. Е.~М.~Штаерман полагает, что появлению культа Фортуны Примигении в Риме способствовала популярность гадания по жребиям в Пренесте\footcite[С. 128]{Shtaerman1987}. %Пренестинский оракул действительно пользовался широкой популярностью среди всех слоёв населения, так, Светоний указывает, что к нему ежегодно обращался даже император Домициан (Suet. Dom. 15.2).

% Различные толкования, согласно античным авторам, термина Primigenia
%Ещё в античности писатели пытались давать толкования названию Primigenia, что должно переводиться на русский как Перворожденная. Цицерон объясняет это так, что Фортуна Примигения сопутствует человеку с его рождения\footnote{Cic. De leg. II.28: \textit{[Fortuna] Primigenia a gignendo comes}.}. Плутарх (Quae. Rom. 106) более многословен и даёт целых три толкования, ни одно из которых не перекликается с цицероновым. Согласно первому, римляне поклоняются Фортуне Примигении в память о Сервии Туллии, рождённом от рабыни и сделавшимся, благодаря заступничеству богини, царём Рима (Plut. Mor. 289~C); этого, продолжает Плутарх, придерживаются многие римляне\footnote{Plut. Quae. Rom. 106, Mor. 289~C: \graecafn{o<'utw g'ar o<i pollo`i <Rwma'iwn <upeil'hfasin}.}. Согласно второму толкованию, Фортуна дала начало и рождение Риму\footnote{Ibid.: \graecafn{t~hs <R'wmhs <h t'uqh par'esqe t`hn >arq`hn ka`i t`hn g'enesin}}. Третье объяснение Плутарха таково: <<Фортуна — начало всех вещей, и благодаря ей природа возникает из случайных соединений>>\footnote{Ibid.: \graecafn{t`hn t'uqhn p'antwn o>'usan >arq`hn ka`i t`hn f'usin >ek to~u kat`a t'uqhn sunistam'enhn}}. Отметим, что первое из этих объяснений есть, действительно, передача сказаний, имевших хождение среди римлян и, скорее всего, par excellence в плебейской среде, последнее же толкование больше похоже на плод отвлечённых философских размышлений.

В трактате <<Об удаче римлян>> Плутарх приписывает Сервию Туллию, помимо прочего, создание храма (\graeca{<ier'on}) Фортуны Примигении на Капитолии\footnote{Plut. De Fort. Rom. 10, Mor. 322 F: \graecafn{<idr'usato d> o>~un T'uqhs <ier`on >en m`en Kapetwl'iw| t`o t~hs Primigene'ias legom'enhs, <`o prwtog'onou tis >a`n <hrmhne'useie}.}. Из других источников, впрочем, нам известно, что храм Фортуны Примигении был построен на Квиринале и посвящён в 194 г. до н.э. (см. п. \ref{FortunaPublicaPrimigenia}). О храме, построенном Сервием на Капитолии, мы больше ничего не знаем.

Известна, однако, надпись, в которой автор, обращаясь к Фортуне, называет её \textit{<<живущей на Тарпейской скале>>} и \textit{<<соседкой Громовержцу>>}: \textit{<<tu, quae Tarpeio coleris vicina Tonanti, / votorum vindex semper, Fortuna \ldots{}>>} (ILS 3696). Отметим, что этот поэтический образ вполне может указывать и на Фортуну из храма на Бычьем форуме: этот храм располагался у подножия Тарпейской скалы, и храм Юпитера Капитолийского находился недалеко от него. В Анцийском календаре под 13 ноября имеется надпись: \textit{[Fer]on(iae), Fort(unae) Pr(imigeniae), [Pie?]tati}\footnote{\textit{Degrassi}, ILLRP 9, Nov. 13.}. Какая именно Фортуна Примигения имеется в виду, неизвестно\footnote{Римский праздник Фортуны Примигении отмечался 25 мая, в Пренесте же, согласно Пренестинским фастам "--- 11 апреля (CIL I\textsuperscript{2}, p. 235).}, возможно, это был местный, анцийский храм. Полагать, что обе эти надписи указывают непременно на храм Фортуны Примигении на Капитолии, основанный Сервием Туллием, о котором пишет Плутарх "--- значит делать слишком большие натяжки. Нам остаётся только признать, что никаких данных, достоверно подтверждающих это сообщение Плутарха, мы не имеем.


\subsection{Мужская Фортуна (Fortuna Virilis)}\label{FortunaVirilis}

1 апреля римляне отмечали праздненство Венералий в честь Венеры Вертикордии (Обратительницы Сердец)\footcites[Pp. 67--69]{Fowler1899}[Pp. 96--97]{Scullard1981}. На этот же день приходился праздник Мужской Фортуны (Fortuna Virilis). Как ни парадоксально, поклонялись ей именно женщины. Овидий так описывает этот ритуал: \textit{Знайте теперь, почему Мужской Фортуне ладан / Подн\'{о}сите там, где тёплая влага течёт. / Принимает это место всех, сбросивших одежды, / И каждый изъян обнажённого тела видит. / Чтобы скрыть это от мужей, Фортуна Virilis, / умоленная при помощи кусочка ладана, это исправляет}\footnote{Ovid. Fast. IV.145--150: \textit{ discite nunc, quare Fortunae tura Virili / 
detis eo, calida qui locus umet aqua. / 
accipit ille locus posito velamine cunctas / 
et vitium nudi corporis omne videt; / 
ut tegat hoc celetque viros, Fortuna Virilis / 
praestat et hoc parvo ture rogata facit.}}. О том, что женщины в этот день мылись в мужских банях, мы можем прочесть в Пренестинских фастах: \textit{Frequenter mulieres supplicant Fortunae Virili, humiliores etiam in balineis, quod in iis ea parte corpor[is] utique viri nudantur qua feminarum gratia desideratur}\footnote{CIL I\textsuperscript{2}, p. 235.} (\textit{<<Многие женщины молятся Мужской Фортуне, незнатные также в банях, ибо в них мужчины всегда обнажаются той частью тела, которая желает женской прелести>>}). Плутарх в <<Римских вопросах>> приписывает Сервию Туллию создание храма <<Фортуны Мужей>> (\graeca{T'uqhs \ldots >'arrenos \ldots <ier'on}, Quae. Rom. 74, Mor. 281~E). Скорее всего, здесь Плутарх говорит именно о Fortuna Virilis, по крайней мере, других вариантов идентификации этого храма мы назвать не можем. Это всё, что мы знаем о Мужской Фортуне; никакие другие источники не могут подтвердить, что её храм был создан при Сервии Туллии.

И. Кайянто отмечает, что обряд этого праздника мог сложиться не ранее II в. до н.э., когда в Риме появились общественные бани\footcite[P. 504]{Kajanto1981}. Однако пассаж \textit{etiam in balineis} из пренестинской надписи намекает на то, что мытьё в банях было не единственным способом поклонения Мужской Фортуне в этот день. В конце концов, здравый смысл подсказывает, что мыться можно не только в банях.

Совпадение дня Мужской Фортуны с праздником Венеры Вертикордии и неприкрытый эротический подтекст обряда указывают на то, что представления об этой богине были связаны с деторождением и c плодородием вообще. Немаловажное значение имеет и ритуал мытья: на связь воды с идеей плодородия и рождения указывает М. Элиаде\footcite[С. 183 и далее]{Eliade1999}. Эти детали ритуала свидетельствуют об архаичности представлений, которые находят выражение в культе Мужской Фортуны. Правда, за неимением иных сведений, подтверждающих или опровергающих сообщение Плутарха, мы не можем уверенно отнести возникновение этого культа к царской эпохе вообще и ко времени Сервия Туллия в частности.

\subsection{Фортуна Милостивая (Fortuna Obsequens)}\label{FortunaObsequens}

В сочинении <<Об удаче римлян>> Плутарх приписывает Сервию Туллию создание храма Фортуны Послушной (\graeca{peij'hnia}) или Милостивой (\graeca{meil'iqia}), приводя латинскую транскрипцию этой эпиклезы: Obsequentis\footnote{Plut. De Fort. Rom. 10, Mor. 322 F: \graecafn{ka`i t`o t~hs >Oyekou'entis, <`hn o<i m`en peij'hnion, o<i d`e meil'iqion e>'inai nom'izousi}.}. В <<Римских вопросах>> он также говорит о том, что Сервий создал храм Фортуны Послушной (\graeca{meil'iqia})\footnote{Plut. Quae. Rom. 74, Mor. 281~D.}. Упоминание этого когномена мы находим у Плавта: \textit{Fortunam, atque Obsequentem} (Asin. 716). Также известны некоторые надписи, посвящённые ей: CIL VI.191 (\textit{Fortunae Obsequenti |  L. Rufinus | v(otum) s(olvit)}), ILS 3708 = ILLRP 111 (\textit{Fortunae opse[q.] | P. Pelius L.f., C. Calvius P.f. | cens.}). В императорское время, начиная со II в. н.э., когномен Fortuna Obsequens появляется на легендах римских монет. Об истории этого культа нам больше ничего не известно, нет у нас также и толкования, которое могли бы дать античные авторы этому когномену.

\subsection{Фортуна Оборачивающаяся (Fortuna Rescipiens)}\label{FortunaRescipiens}

В <<Римских вопросах>> Плутарх сообщает, что Сервий создал храм Фортуны Оборачивающейся (\graeca{T'uqhs epistrefom'enhs <ier'on}, Quae. Rom. 74, Mor. 281~E). В трактате <<Об удаче римлян>> Плутарх пишет, что этот храм находился на Эсквилине\footnote{Plut. De Fort. Rom. 10, Mor. 323~A: \graecafn{T'uqhs <ier'on >esti ka`i >en A>iskul'iais >epistrefom'enhs}.}. Очевидно, здесь идёт речь о Fortuna Rescipiens, которую упоминает Цицерон, объясняя такой её когномен тем, что она оборачивается, чтобы оказать помощь: \textit{Rescipiens ad opem ferendam} (Cic. De leg. II.28). Каких-либо других сведений об этом культе мы не имеем.

\subsection{Прочие культы Фортуны, упоминаемые Плутархом}\label{FortunaeVariaePlutarchi}

Помимо вышеперечисленных, Плутарх в <<Римских вопросах>> приписывает Сервию Туллию создание ещё нескольких храмов (\graeca{<ier'a}) Фортуны. Так, согласно греческому автору, шестой римский царь создал храм Fortunae Brevis\footnote{Plut. Quae. Rom. 74, Mor. 281 D: \graecafn{mikr~as T'uqhs <ier`on <idr'usato Sero'uios To'ullios <o basile`us <`hn <br'ebem> kalo~usi}.}, храм Фортуны Подательницы Надежд (\graeca{T'uqhs e>u'elpidos})\footnote{Ibid. 281 E.}, Отвратительницы Бед (\graeca{>apotropa'iou})\footnote{Ibid.} "--- вероятно, Fortuna Averrunica по-латыни\footcite[P. 155]{Richardson1992}, "--- а также Фортуны Собственной (\graeca{>id'ias T'uqhs <ier'on})\footnote{Plut. Mor. 281~E.}, что по-латыни, видимо, должно звучать как Fortuna Privata.

В трактате о римской удаче Плутарх указывает, что храм Фортуны Собственной находился на Палатине\footnote{Plut. De Fort. Rom. 10, Mor. 322~F: \graecafn{>id'ias T'uqhs <ier'on >estin >en Palat'iw|}.}. Там же он упоминает не храм, а \textit{алтарь} Фортуны Подательницы Надежд (\graeca{T'uqhs bwm`os E>u'elpidos}), указывая, что он находился в Длинном переулке (\graeca{>en t~w| makr~w| stenwp~w|}) (Plut. De Fort. Rom. 10, Mor. 322~F). Здесь, видимо, идёт речь о Vicus Longus на Квиринале, а латинский когномен звучал, скорее всего, как Fortuna Bonae Spei\footfullcite[P. 155]{Richardson1992}.
%  ἰδίας Τύχης ἱερόν ἐστιν ἐν Παλατίῳ
%  ἰδίας Τύχης ἱερόν

Из других источников нам ничего не известно об этих культовых сооружениях. Их существование также не подтверждено археологически\footcite[Pp. 155--158]{Richardson1992}.

% μικρᾶς Τύχης ἱερὸν ἱδρύσατο Σερούιος Τούλλιος ὁ βασιλεὺς ἣν ' βρέβεμ ' καλοῦσι

\subsection{Критика сообщений Плутарха}\label{PlutarchiCritica}

У. Фаулер отмечал, что в Риме существовала устойчивая тенденция приписывать создание храмов Фортуны Сервию Туллию\footcite[P. 68]{Fowler1899}. Мы, однако, видим, что эта <<тенденция>> восходит только к одному Плутарху, и у нас есть ряд причин, чтобы поставить его сообщения под сомнение. Во-первых, он единственный приписывает основание первого храма Фортуны царю Анку; во-вторых, согласно Плутарху, Сервий Туллий создал какое-то невероятно большое число храмов и алтарей Фортуны, при этом те храмы, про которые нам достоверно известно, что их основал Сервий (на Бычьем форуме и Fortis Fortunae), Плутарх не упоминает; в-третьих, мы не можем подтвердить данными других источников его подробные топографические указания. Именно поэтому мы должны относиться с большой осторожностью к тем фактам, которые сообщает только он один.

Скорее всего, при создании трактата <<Об удаче римлян>> и написании 74-го <<вопроса>> Плутарх обращался к одному источнику или кругу источников, откуда он и почерпнул эти сведения. Что это за источники, мы не знаем. Очевидно, они восходят к какому-то латинскому тексту, потому что Плутарх приводит транслитерацию некоторых латинских названий храмов и переводит их на греческий язык, давая своё толкование. Обращался ли он напрямую к латинскому тексту, или же взял перевод и толкование из какого-то греческого источника, мы сказать не можем. Очевидно также, что в основе этих источников лежит достоверная фактическая информация, но в изложении Плутарха она передана более или менее искажённо, что мы можем видеть при сравнении с другими источниками.

Оценивая низкую точность фактической информации, которую приводит Плутарх, мы приходим к выводу, что подобное изложение ориентировано скорее на читателя или слушателя, не знакомого с римскими реалиями, у которого эти сведения вызовут такое же удивление, как и у самого Плутарха. Фактические неточности, видимо, проистекают от недостаточно критического отношения Плутарха к своим источникам.

Подводя итог, мы вынуждены заключить, что не можем достоверно отнести к царской эпохе те культы Фортуны, создание которых Плутарх приписывает Сервию Туллию. Однако за недостатком сведений из других источников, ничего более конкретного о происхождении этих культов сказать нельзя.

\section{Культ Фортуны в эпоху Республики и начала империи}\label{RespublicaeCulti}

\subsection{Женская Фортуна (Fortuna Muliebris)}\label{FortunaMuliebris}

Первым крупным храмом Фортуны, построенным после изгнания царей, был храм (\textit{templum} у Ливия и Аврелия Виктора, \textit{aedes} у Валерия Максима) Фортуны Женской (Fortunae Muliebris). Согласно римской легенде, это было сделано, чтобы вознаградить женщин, которые отговорили Гая Марция Кориолана вести вольсков на Рим: \textit{<<в память о случившемся,} "--- пишет Тит Ливий, "--- \textit{был воздвигнут и освящён храм Женской Фортуны>>}\footnote{ Liv. II.40.11--12: \textit{monumento quoque quod esset, templum Fortunae muliebri aedificatum dedicatumque est}.}. То, что \textit{templum} (святилище) в изложении Тита Ливия было \textit{aedificatum}, указывает, скорее всего, на существование именно здания храма (\textit{aedes}). О \textit{templum} Женской Фортуны, основанном на месте, где женское посольство встретилось с Кориоланом, пишет Аврелий Виктор\footnote{\mancite{Aur. Vict. De vir. ill. XIX.5}: \textit{Ibi templum Fortunae muliebri constitutum erat}.}.

% Рассказ Дионисия Галикарнасского об этом событии
Дионисий Галикарнасский приводит гораздо более подробный рассказ об этом событии. Он пишет, что женщины стали \textit{<<просить, чтобы сенат дозволил им возвести храм Женской Фортуны на том месте, где они обратились с просьбами за свой город, и ежегодно всем вместе приносить ей жертвы в тот день, когда они прекратили войну>>}\footnote{Dion. Hal. VIII.55.3: \graecafn{>axio~un, d> >epitr'eyai sf'isi t'hn boul`hn >ep`i t'uqhs gunaik~wn <idr'usasjai <ier'on, >en <~w| t`as per`i t~hs p'olews >epoi'hsanto lit`as qwr'iw|, jus'ias te kaj> <'ekaston >'etos a>ut~h| sunio~usas >epitele~in >en <~h| p'olemon >'elusan <hm'era|}.}. И далее: \textit{<<сенат и народ постановили посвятить богине участок, купленный на общественные средства, и из них же соорудить на нём храм и алтарь, как укажут жрецы, и жертвенных животных доставлять за общественный счёт, а начинать священные обряды той, кого сами женщины назначать исполнять эти обряды>>}\footnote{Ibid.: \graecafn{<h boul`h ka`i <o d~hmos >ap`o t~wn koin~wn >eyef'isanto qrhm'atwn t'emen'os t> >wnhj`en kajeirwj~hnai t~h| je~w|, ka`i >en a>ut~w| ne`wn ka`i bwm'on, <ws >`an o<i <ieromn'hmones >exhg~wntai, suntelesj~hnai, jus'ias te pros'agesjai dhmotele~is katarxom'enhs t~wn <ier~wn gunaik'os, <`hn >`an >apode'ixwsin a>uta`i leitourg`on <ier~wn}.}.

% Что пишет Плутарх об этом?
В сочинении <<Об удаче римлян>> Плутарх приводит такой же вариант легенды: \textit{<<Храм Женской Удачи они построили ещё до Камилла, после того как благодаря женщинам повернули прочь Марция Кориолана, который вёл на город вольсков>>}\footnote{Plut. De Fort. Rom. 5, Mor. 318~F: \graecafn{t`o d`e t~hs Gunaike'ias T'uqhs kateskeu'asanto pr`o Kam'illou <'ote M'arkion Kori'olanon >ep'agonta t~h| p'olei O>uolo'uskous >apestr'eyanto di`a t~wn gunaik~wn}.}.
% τὸ δὲ τῆς Γυναικείας Τύχης κατεσκευάσαντο πρὸ Καμίλλου ὅτε Μάρκιον Κοριόλανον ἐπάγοντα τῇ πόλει Οὐολούσκους ἀπεστρέψαντο διὰ τῶν γυναικῶν
% Биография Кориолана
Рассказывая о том же событии в биографии Гая Марция Кориолана, Плутарх пишет, что женщины \textit{<<попросили [у сената "--- В.~Ж.] только дозволения соорудить храм Женской Удачи с тем, чтобы средства на постройку собрали они сами, а расходы по совершению обрядов и всех прочих действий, каких требует культ богов, приняло на себя государство>>} (%*вставить оригинальную цитату*
Plut. Coriol. 37). 

Дионисий Галикарнасский в своём повествовании далее указывает, что \textit{<<впервые тогда женщинами была избрана жрица>>}\footnote{Dion. Hal. VIII.55.4: \graecafn{<i'ereia <up`o t~wn gunaik~wn >apede'iqjh t'ote pr~wton}.}, а также называет её имя: это была Валерия (\graeca{O>ualer'ia}), та самая, которая, согласно тому же Дионисию, сподвигла мать и жену Кориолана отправиться к нему с посольством (см. Dion. Hal. VIII.39--44). Первые жертвоприношения, сообщает далее античный историк, были принесены женщинами за народ, и \textit{<<Валерия начала священные обряды на алтаре, установленном на священном участке, ещё до того как были воздвигнуты храм и статуя>>}\footnote{Ibid. VIII.55.4: \graecafn{jus'ian d`e pr'wthn a<i guna~ikes >'equsan <up`er to~u d'hmou katarqum'enhs t~wn <ier~wn t~hs O>ualer'ias >ep`i to~u kataskeuasj'entos >en t~w| tem'enei bwmo~u, pr`in >`h t`on ne`wn ka`i t`o x'oanon >anastaj~hnai}.}. Скрупулёзность Дионисия в изложении фактов доходит до того, что он сообщает нам дату этого первого жертвоприношения: его совершили в декабрьские календы года, следовавшего после похода Кориолана на Рим\footnote{Ibid.: \graecafn{mhn`i Dekembr'iw| to~u kat'opin >eniauto~u, t~h| n'ea| sel'hnh|, <`hn \ldots <Rwma~ioi kal'andas kalo~usin}.} (т.е. в 487 г. до н.э.).

Автор <<Римских древностей>> не обходит стороной и, пожалуй, более важную дату окончания постройки и освящения храма Женской Фортуны, указывая, что произошло это событие \textit{<<приблизительно в седьмой день, ведя счёт по луне, месяца Квинтилия: этот день у римлян является предшествующим дню квинтильских нон>>}\footnote{Ibid. 4--5: \graecafn{ne`ws sunetel'esjh te kai kajier'wjh Kointil'ion mh'os <ebd'omh| m'alista kat`a sel'hnhn: a<'uth kat`a <Rwma'ious >est`in <o prohgoum'enh t~wn Kointil'iwn nwn~wn <hm'era}.}. В то время шёл год консульства Прокула Вергиния, который и освятил храм\footnote{Ibid. 5: \graecafn{<o d`e kajier'wsas a>ut`on >~hn Pr'oklos O<uerginios, <'ateros t~wn <up'atwn}.}. Таким образом, сведения, передаваемые Дионисием, позволяют точно датировать основание храма Fortunae Muliebris: согласно Ливию (Liv. II.41.1), Прокул Вергиний был консулом вместе со Спурием Кассием в 486 г. до н.э., т.е. через два года после похода Кориолана на Рим (Liv. II.39.1 и далее), что соответствует и рассказу Дионисия (на основании вышеизложенного, через год после войны были совершены первые жертвоприношения, а ещё через год "--- уже освящён храм).

% καταρχομένης τῶν ἱερῶν γυναικός, ἣν ἂν ἀποδείξωσιν αὐταὶ λειτουργὸν τῶν ἱερῶν. (Dion. Hal. VIII.55.3)
% ἱέρεια μὲν ὑπὸ τῶν γυναικῶν ἀπεδείχθη τότε πρῶτον (Dion. Hal. VIII.55.4)
% θυσίαν δὲ πρώτην αἱ γυναῖκες ἔθυσαν ὑπὲρ τοῦ δήμου καταρχομένης τῶν ἱερῶν τῆς Οὐαλερίας ἐπὶ τοῦ κατασκευασθέντος ἐν τῷ τεμένει βωμοῦ, πρὶν ἢ τὸν νεὼν καὶ τὸ ξόανον ἀνασταθῆναι (Dion. Hal. VIII.55.4)
% μηνὶ Δεκεμβρίῳ τοῦ κατόπιν ἐνιαυτοῦ, τῇ νέᾳ σελήνῃ, ἣν Ἕλληνες μὲν νουμηνίαν, Ῥωμαῖοι [p. 209] δὲ καλάνδας καλοῦσιν: αὕτη γὰρ ἦν ἡ λύσασα τὸν πόλεμον ἡμέρα (Dion. Hal. VIII.55.4)
%  νεὼς συνετελέσθη τε καὶ καθιερώθη Κοιντιλίου μηνὸς ἑβδόμῃ μάλιστα κατὰ σελήνην: (5) αὕτη δὲ κατὰ Ῥωμαίους ἐστὶν ὁ προηγουμένη τῶν Κοιντιλίων νωνῶν ἡμέρα. ὁ δὲ καθιερώσας αὐτὸν ἦν Πρόκλος Οὑεργίνιος ἅτερος τῶν ὑπάτων.

% Разобрать рассказ о чуде, произошедшем со статуей


% Написать про то, что сообщение о храме Женской Фортуны хорошо задокументировано

Итак, храм Женской Фортуны был освящён 6 июля 486 г. до н.э., и нам повезло, что мы можем с такой высокой точностью датировать столь отдалённое от нас событие. В данном случае нас выручает подробный и обстоятельный рассказ Дионисия Галикарнасского. Наличие множества деталей строгого фактического характера (имена консула и жрицы, детали отправления обрядов и обстоятельства юридического характера, вроде тех, за чей счёт возводился храм и ставилась статуя (см. Dion. Hal. VIII.56.2), а также точные до дня даты событий) говорит о том, что здесь Дионисий прибегал к документальным источникам. Чуть далее в своём повествовании о чуде, произошедшем в храме Женской Фортуны, он, действительно, ссылается на записи верховных жрецов (\graeca{<ws a<i <ierofant~wn peri'equosi grafa'i}, Ibid. 56.1). Таким образом, степень достоверности событий в пересказе Дионисия определяется степенью достоверности его источников, которыми, судя по всему, являются в данном случае tabulae pontificum. Отсылка же к материалам подобного рода может только прибавить весомости сообщению нашего историка; нам представляется уместным присоединиться к мнению П.~Уолша, считавшего традицию, опирающуюся на официальные документы римских коллегий, вполне достоверной (в смысле правильной передачи первоначальных сведений) вследствие той сакральности, каковую придавали ей сами римляне\footcite[P. 34]{Walsh1974}.

% ὡς αἱ τῶν ἱεροφαντῶν περιέχουσι γραφαί (Dion. Hal. VIII.56.1)

% Подумать, куда ткнуть этот абзац
% Отметить, что Плутарх сообщает некоторые подробности, в частности, то, о чём именно просили женщины, каковых нет у Дионисия. Это, видимо, связано с тем, какими именно источниками пользовался Плутарх
Отметим, что, несмотря на то, что Дионисий Галикарнасский в своём изложении, как мы видим, скрупулёзно приводит даже самые мелкие факты, сведений о том, за чей счёт женщины просили построить храм и проводить жертвоприношения, о чём пишет Плутарх в биографии Кориолана (Plut. Coriol. 37), у него нет. Вряд ли Плутарх мог выдумать такого рода подробность; нам представляется более вероятным, что эти сведения также имеют документальное происхождение. Это говорит в пользу того, что Плутарх, скорее всего, опирался либо на те же архивы, что и Дионисий (нам известно, что греческий философ бывал в Риме, и, быть может, имел возможность работать с документами жреческих коллегий), либо на некоторый не дошедший до нас источник, в котором использовались те же материалы. Но сравнивая сообщения двух авторов, мы вынуждены признать, что Дионисий передаёт нам гораздо более ценные факты, чем Плутарх.

% ‘θεοφιλεῖ με θεσμῷ γυναῖκες δεδώκατε (Plut. Coriol. 37.3)

%Далее он описывает случившееся со всеми возможными подробностями: <<когда сенат постановил оплатить все расходы на храм и статую за общественный счёт, а женщины соорудили другую статую на те средства, которые собрали, и обе они одновременно были принесены в дар в первый день освящения храма, одно из изваяний "--- то, которое установили женщины, "--- в присутствии многих произнесло отчётливо и громко фразу \ldots ``По священному закону города вы, жены, принесли меня в дар''>> (\graeca{}, Ibid. 2).

% τὸ δηλῶσαι τὴν γενομένην ἐπιφάνειαν τῆς θεοῦ κατ᾽ ἐκεῖνον τὸν χρόνον οὐχ ἅπαξ, ἀλλὰ καὶ δίς (Dion. Hal. VIII.56.1)

% ὅτι τῆς βουλῆς ψηφισαμένης ἐκ τοῦ δημοσίου πάσας ἐπιχορηγηθῆναι τὰς εἰς τὸν νεών τε καὶ τὸ ξόανον δαπάνας, ἕτερον δ᾽ ἄγαλμα κατασκευασαμένων τῶν γυναικῶν ἀφ᾽ ὧν αὐταὶ συνήνεγκαν [p. 210] χρημάτων, ἀνατεθέντων τ᾽ αὐτῶν ἀμφοτέρων ἅμα ἐν τῇ πρώτῃ τῆς ἀνιερώσεως ἡμέρᾳ, θάτερον τῶν ἀφιδρυμάτων, ὃ κατεσκευάσανθ᾽ αἱ γυναῖκες, ἐφθέγξατο πολλῶν παρουσῶν γλώττῃ Λατίνῃ φωνὴν εὐσύνετόν τε καὶ γεγωνόν: ἧς ἐστι φωνῆς ἐξερμηνευόμενος ὁ νοῦς εἰς τὴν Ἑλλάδα διάλεκτον τοιόσδε: ὁσίῳ (3)  πόλεως νόμῳ γυναῖκες γαμεταὶ δεδώκατέ με (Dion. Hal. VIII.56.2--3)

% Что пишет о говорящей статуе Валерий Максим
%Об этом же событии мы можем прочесть и у Валерия Максима: он сообщает точное местоположение храма и священного изображения "--- у четвёртого милевого камня Латинской дороги\footnote{\textit{Fortunae etiam Muliebris simulacrum, quod est Latina via ad quartum miliarium, eo tempore cum aede sua consecratum}, Val. Max. 1.8.4.} (Val. Max. 1.8.4). Согласно Валерию Максиму, богиня дважды (\textit{non semel sed bis}, Ibid.) произнесла следующие слова: \textit{rite me, matronae, dedistis riteque dedicastis} (Ibid.) (<<Правильно, матроны, вы доверились мне и правильно посвятили себя мне>>). Отметим, что Плутарх, в отличие от Дионисия и Валерия Максима, считает неправдоподобным то обстоятельство, что статуя могла заговорить дважды: <<Утверждают, "--- пишет он, "--- будто эти слова раздались и во второй раз: нас хотят убедить в том, что похоже на небылицу и звучит весьма неубедительно>>\footnote{\graecafn{ta'uthn ka`i d`is gen'esjai t`hn fwn`hn mujologo~usin, >agen'htois <'onoia ka`i qalep`a peisj~hnai pe'ijontes <hm~as}, Plut. Coriol. 38.1.} (Plut. Coriol. 38.1). Таким образом, все авторы единодушно утверждают, что богиня, согласно традиции, обратилась к женщинам именно дважды.

Валерий Максим также пишет об этом храме и культовом изображении Женской Фортуны, указывая его точное местоположение: у четвёртого милевого камня Латинской дороги\footnote{Val. Max. 1.8.4: \textit{Fortunae etiam Muliebris simulacrum, quod est Latina via ad quartum miliarium, eo tempore cum aede sua consecratum}.}. Об этой же статуе богини рассказывает и Фест, прибавляя, что прикасаться к нему могла лишь та женщина, которая была единожды в браке\footnote{Fest. 242~M: \textit{via Latina ad milliarium illi (?) Fortunae muliebris, nefas est attingi, nisi ab ea, quae semel nupsit}.}. Это подтверждается сообщением Тертуллиана, который в трактате <<О единобрачии>> отмечает, что \textit{<<увенчивать венком Женскую Фортуну, также как и Матерь Матуту, может лишь та, кто была лишь один раз замужем>>}\footnote{Tert. De Monogam. XVII.4: \textit{Fortunae Muliebri coronam non imponit nisi univiris sicut Matri Matutae}.}. Эти отрывочные свидетельства представляют для нас высокую ценность, потому что доказывают, что поклонение Женской Фортуне продолжалось и в I в. до н.э., во времена Веррия Флакка, и во II в. н.э., во время жизни Тертуллиана, причём с сохранением традиционного обряда. Обратим также внимание на близость ритуалов Женской Фортуны и Матери Матуты, которую подчёркивает христианский автор. Итак, если из сочинений Дионисия Галикарнасского, Плутарха и Валерия Максима мы узнаём, что ещё первоначально Женской Фортуне поклонялись матроны, то эти поздние свидетельства открывают для нас также и детали ритуала, который, вследствие крайней консервативности римлян в религиозной области\footnote{Х.~Розе таким образом отмечает эту черту римской религиозности: <<It follows that in dealing with Roman religion, the department in which that conservative people were most conservative, we can quite easily find, almost on the surface as it were, remnants of a very early simple type of thought>>. \cite[P. 158]{Rose1959}.}, мы можем полагать сохранявшимся в неизменности с раннереспубликанских времён.

Уже в императорское время, при Марке Аврелии, был выпущен денарий с изображением Женской Фортуны, посвящённый Фаустине Младшей (RIC III, Marc. Aur. 683, см. рис. \ref{pic:FortunaMuliebris}). На монете отчеканен один из традиционных типов Фортуны для нумизматических памятников: сидящая женщина с корабельным рулём, стоящем на шаре, в правой руке и с рогом изобилия в левой. Нет оснований считать, что это изображение воспроизводит статую богини из храма. Это единственное известное нам появление когномена Женской Фортуны на легендах монет за всю римскую историю. Данная монета принадлежит к ряду монетных выпусков, посвящённых Фаустине Младшей и связанных с различными женскими божествами, как представляющими собой <<отвлечённые понятия>>, так и <<настоящих богинь>> "--- с Церерой (RIC III Marc. Aur. 668), Дианой (Ibid. 673), Юноной (Ibid. 687, 691), Aeternitas (Ibid. 667), Fecunditas (Ibid. 665, 667) и т.д. Здесь Fortuna Muliebris выступает в качестве богини-покровительницы женщин и женственности и помещается в один ряд с другими женскими божествами. Надо полагать, что этот монетный выпуск отражает представление о Женской Фортуне, восходящее к эпохе ранней Республики, когда её культ и был основан.

\subsection{Храм Форс Фортуны (Aedes Fortis Fortunae)}\label{FortisFortunae2}

В 293 г. до н.э., согласно сообщению Тита Ливия, консул Карвилий после победы над самнитами и этрусками (см. Liv. X.39 и далее) употребил военную добычу на возведение храма богини Счастливого Случая (Fortis Fortunae) рядом с храмом той же богини, освящённым Сервием Туллием\footnote{Liv. X.46.14: \textit{reliquo aere aedem Fortis Fortunae de manubiis faciendam locavit prope aedem eius deae ab rege Ser. Tullio dedicatam}.}. Об истории этого храма мы больше ничего не знаем. Вероятно, в фастах (приведены на с. \pageref{Fastae}) этот храм обозначался как \textit{<<у шестого милевого камня>>} (\textit{ad miliarium sextum}, Кампанской дороги), в то время как храм, построенный Сервием, находился \textit{<<у первого милевого камня>>} (\textit{ad miliarium primum}). Согласно упоминанию Тита Ливия (продигии 209 г. до н.э.), в одном из храмов Форс Фортуны в Риме стояло изображение богини, украшенное венком\footnote{Liv. XXVII.11.3: \textit{Romae intus in cella aedis Fortis Fortunae de capite signum quod in corona erat in manum sponte sua prolapsum}.}.

\subsection{Фортуна Примигения Римского Народа (Aedes Fortunae Primigeniae Publicae Populi Romani in Colle)}\label{FortunaPublicaPrimigenia}

В 204 г. до н.э. консул Публий Семпроний в Кротонской области столкнулся Ганнибалом, и в результате завязавшейся схватки римляне вынуждены были отступить, потеряв около 1200 человек (Liv. XXIX.36.4--6). На следующий день, вступив в битву с Ганнибалом, консул \textit{<<дал обет построить храм Фортуне Примигении, если он в этот день разобьет врага>>}\footnote{Liv. XXIX.36.8: \textit{consul principio pugnae aedem Fortunae Primigeniae vovit si eo die hostes fudisset}.}. Обет консула осуществился, пишет дальше римский историк\footnote{Ibid.: \textit{composque eius voti fuit}}. Храм был построен и освящён спустя 10 лет, в 194 г. до н.э.: \textit{<<храм Фортуне Примигении на Квиринальском холме посвятил Квинт Марций Ралла, дуумвир, нарочно для того назначенный; а обет дал консул Публий Семпроний Соф за десять лет пред тем, во время Пунической войны; он же, будучи уже цензором, сдал подряд на строительство этого храма>>}\footnote{Liv. XXXIV.53.5--6: \textit{aedem Fortunae Primigeniae in colle Quirinali dedicavit Q. Marcius Ralla, duumvir ad id ipsum creatus: voverat eam decem annis ante Punico bello P. Sempronius Sophus consul, locaverat idem censor}.}.

Праздник этой богини отмечался римлянами 25 мая. В Эсквилинских фастах праздник обозначается как \textit{Fort(unae) Public(ae) P(opuli) R(omani) in Colle}\footnote{CIL I\textsuperscript{2}, p. 211.}; в Церетанских фастах: \textit{Fortunae P(rimigeniae) P(ublicae) P(opuli) R(omani) Q(uiritum) in Colle Quirin(ale)}\footnote{Ibid., p. 213.}; в Венузинских фастах: \textit{Fortun(ae) Prim(igeniae) in Coll(e)}\footnote{Ibid., p. 221.}. Овидий в <<Фастах>> говорит о \textit{templum Fortunae Publicae Populi}\footnote{Ovid. Fast. V.729--731: \textit{Nec te praetereo, populi Fortuna potentis publica, cui templum luce sequente datum est}.}; очевидно, \textit{templum} у Овидия "--- это поэтический образ, и мы с достоверностью можем говорить о существовании храма (\textit{aedes}) на Квиринале.

Любопытно, что Фортуна почиталась в этом храме под двумя эпиклезами: Fortuna Publica Populi Romani и Fortuna Primigenia. Нет оснований считать, что здесь имеются в виду два разных храма, стоявших в одном месте и почитавшихся в один день. Ж. Шампо полагает, что две эпиклезы храма несли разные ритуальные функции: одна выявляла роль этого святилища в жизни римской общины, другая указывала на происхождение культа, который Рим унаследовал или даже захватил\footcite[P. 9--10]{Champeaux1987}.

Отметим, что, согласно сообщению Тита Ливия, консул Семпроний дал обет именно Фортуне Примигении. Однако римский культ этой богини нимало не походил на пренестинский: с последним, как было показано выше, был связан знаменитый оракул, про него ходили древние легенды, известные в изложении Цицерона (Cic. De div. II.85--86). %Культ пренестинской Фортуны Примигении был весьма популярен в Лации, поэтому нам известно довольно много посвятительных надписей, оставленных ей частными лицами и коллегиями; правда, надписи эти происходят, в основном, из Пренесте (напр. ILS 3683, 6254--6256, \textit{Degrassi}, ILLRP 103--107). Они не могут характеризовать римский культ этой богини. Собственно римские надписи (CIL VI.193--195, 3681=30875), оставленные частными лицами, судя по именам посвятителей, относятся к эпохе империи (напр. CIL VI.195: \textit{M. Ulpius Strato}); к республиканской эпохе, возможно, относится только надпись CIL VI.193(=30711): \textit{A. Gabinius A. l. | Narcissus | Fortunae Prim. | Votum solvit l. m.}. Однако она мало что может дать нам для понимания культа Фортуны в храме на Квиринальском холме.
Однако \textit{существенно римской} следует считать ту <<ипостась>> богини, которая выражалась когноменом \textit{Publica Populi Romani (Quiritum)}. Поэтому для раскрытия сути культа нам следует обратить внимание, в первую очередь, на официальный, публичный его характер, начиная с обстоятельств обетования храма "--- консул Семпроний дал обет богине в последнем серьёзном сражении Второй Пунической войны на территории Италии, проиграв которое, Ганнибал уже не мог вести наступательные действия, а вскоре был вынужден вернуться в Африку ввиду высадки там войска Сципиона.

Античные писатели упоминают <<фортуну римского народа>> и вне культового контекста. Тит Ливий, рассказывая о событиях, произошедших после отражения угрозы со стороны Кориолана, пишет, что из-за междоусобиц вольски и эквы так и не смогли объединиться для похода на Рим. \textit{<<Фортуна римского народа истребила оба вражеских войска>>}\footnote{\mancite{Liv. II.40.13}: \textit{Ibi fortuna populi Romani duos hostium exercitus \ldots{} confecit}.}, "--- говорит историк. В другом месте он вкладывает в уста военного трибуна, предлагающего консулу рискованный военный манёвр, такие слова: \textit{<<А нас потом выручат или фортуна римского народа, или наша собственная доблесть>>}\footnote{\mancite{Liv. VII.34.6}: \textit{Nos deinde aut fortuna populi Romani aut nostra virtus expediet}.}. Тацит, рассказывая в <<Истории>> о том, что войска консула Муциана оказались в нужное время в нужном месте, чтобы отразить вторжение даков, пишет, что римлянам, как и во многих других случаях, помогла фортуна римского народа\footnote{\mancite{Tac. Hist. III.46}: \textit{adfuit, ut saepe alias, fortuna populi Romani}.}. Таким образом, в представлении римских историков I в. до н.э. "--- I в. н.э. фортуна римского народа (не как богиня, которой был посвящён культ, а как понятие) соотносилась с военным счастьем и военной удачей римлян.

%. Битва при Кротоне "--- последнее серьёзное сражение Второй Пунической войны на территории Италии. После понесённого поражения Ганнибал отступил в Бруттий и уже не мог вести наступательные действия в Италии, а вскоре был вынужден вернуться в Африку ввиду высадки там войска Сципиона. Храм, обетованный Фортуне Примигении, был завершён 10 лет спустя, когда был известен не только итог тяжёлой войны с Ганнибалом, но уже стало понятно значение этой победы для римского народа. Таким образом, Фортуна Римского Народа оказывается связанной с идеей победы, военной удачи или, вернее, военного счастья римлян. Значение же военного успеха в римской истории трудно приуменьшить, поэтому неудивительно, что именно с ним связывается Фортуна Римского Народа.

Мы располагаем также нумизматическими свидетельствами. Из различных эпиклез Фортуны, которые мы можем прочесть на легендах римских монет, впервые появляется именно когномен Fortuna Populi Romani. Это первая и единственная эпиклеза Фортуны на монетах республиканского периода. Известно два подобных выпуска. Во-первых, это денарий 49 г. до н.э. (RRC 440/1, рис. \ref{pic:RRC440}): на аверсе монеты изображена голова Фортуны в диадеме и надпись \textit{Fort(una) P(opuli) R(omani)}, на реверсе "--- лавровый венок, пальмовая ветвь (символ победы) и кадуцей (вероятно, офицерский жезл). Второй монетный выпуск "--- ауреус 41 г. до н.э. (RRC 513/1, рис. \ref{pic:RRC513}). На аверсе монеты находится также изображение головы Фортуны с надписью \textit{F(ortuna) P(opuli) R(omani)}, на реверсе "--- лавровый венок, копьё (hasta) и фалера. Отметим, что подобные монетные типы, как изображение головы Фортуны в диадеме на аверсе, так и соединение этого образа с военной симоликой на реверсе, более никогда не появляются.

В эпоху гражданских войн конца Республики, как отмечает М. Г. Абрамзон\footcite[С. 9]{Abramson1995}, монеты становятся наиболее эффективным средством политической пропаганды. Военную символику на реверсах монет, а также изображение Фортуны Римского Народа следует рассматривать именно в контексте пропаганды; здесь мы видим безусловную связь культа Fortunae Populi Romani с идеей военной славы Рима. К сожалению, нам неизвестно, кто из политических деятелей той беспокойной эпохи чеканил эти монеты. Вероятно, они предназначались для выплаты жалования легионерам (в первую очередь, это можно сказать о денариях).

Итак, обстоятельства обетования храма Фортуны Римского Народа, употребление её имени у римских историков, связь её когномена с политической военной пропагандой на монетах свидетельствуют о том, что эта богиня считалась, в первую очередь, покровительницей военной удачи или, вернее, военного счастья Рима.

\subsection{Всадническая Фортуна (Fortuna Equestris)}\label{FortunaEquestris}

В 180 г. до н.э. проконсул Испании Квинт Фульвий Флакк во время битвы с кельтиберами в Манлиевом урочище (см. Liv. XL.39 и далее) дал обет Фортуне "--- правда, в отличие от Публия Семпрония, обратившегося к высшим силам перед сражением, сделал он это ближе к концу битвы, когда исход её был уже ясен: \textit{<<\ldots{}все кельтиберы обратились в бегство, а римский полководец, глядя на опрокинутого врага, обетовал храм Всаднической Фортуне и игры Юпитеру Всеблагому Величайшему>>}\footnote{Liv. XL.40.10: \textit{Celtiberi omnes in fugam effunduntur, et imperator Romanus aversos hostes contemplatus aedem Fortunae equestri Iovique optimo maximo ludos vovit}.}. Римский историк отмечает далее, что \textit{<<деньги на это были собраны для него [консула Флакка "--- В.~Ж.] испанцами>>}\footnote{Ibid. 44.9: \textit{vovisse \ldots{} aedem equestri Fortunae sese facturum: in eam rem sibi pecuniam collatam esse ab Hispanis}.}, к тому же, \textit{<<было решено избрать дуумвиров, чтобы они сдали подряд на строительство храма>>}\footnote{Ibid. 10: \textit{ut duumviri ad aedem locandam crearentur}.}.

% Что нехорошего сделал Квинт Фульвий Флакк:
% Версия Ливия
Античная традиция передаёт нам, что при строительстве этого храма (\textit{aedes}) Фульвий Флакк совершил святотатство, разорив храм Юноны Лацинийской в Бруттии: он снял мраморные плиты, покрывавшие крышу того храма, чтобы украсить ими свою постройку (Liv. XLII.3.1--4). Ливий описывает, что такое деяние вызвало возмущение сената, и Флакку воспретили украшать свой храм этими плитами, которые уже были доставлены в Рим (Ibid. 5--11). Валерий Максим передаёт ту же историю несколько иначе и с нравоучительным оттенком; согласно его версии, Флакк был наказан высшими силами: \textit{<<когда он узнал, что из двух его сыновей, сражавшихся в Иллирии, один погиб, а другой был тяжело ранен, он в отчаянии испустил дух>>}\footnote{Val. Max. 1.1.20: \textit{per summam aegritudinem animi expiravit, cum ex duobus filiis in Illyrico militantibus alterum decessisse, alterum graviter audisset adfectum}.}. Сенат, согласно Валерию, потрясённый этим событием, распорядился \textit{<<отправить плиты обратно в Локры>>}\footnote{Ibid.: \textit{senatus tegulas Locros reportandas curavit}.}. Валерий почему-то отправляет мраморные плиты в Локры, тогда как, согласно Ливию, храм Юноны Лацинийской находился в шести милях от Кротона (Liv. XXIV.3.3) "--- оба города, действительно, находятся в регионе под названием Бруттий. Ливий более точен, чем Валерий, он указывает, что именно Флакк освятил обетованный им храм Всаднической Фортуны в 173 г. до н.э.\footnote{Liv. XLII.10.5: \textit{Fulvius aedem Fortunae equestris, quam proconsul in Hispania dimicans cum Celtiberorum legionibus voverat, annis sex post, quam voverat, dedicavit}.}. Сообщение Валерия о том, что Флакк умер, не пережив известий о несчастии со своими сыновьями, подтверждается и Ливием. Римский историк пишет, что тому \textit{<<сообщили, что из двух сыновей его, служивших в Иллирии, один умер, а другой опасно и тяжело болен>>}\footnote{Liv. XLII.28.11: \textit{ex duobus filiis eius, qui tum in Illyrico militabant, nuntiatum alterum <mortuum, alterum> gravi et periculoso morbo aegrum esse}.}. После этого \textit{<<рабы, вошедшие утром в спальню хозяина, нашли его висящим в петле>>}\footnote{Ibid. 12: \textit{mane ingressi cubiculum servi laqueo dependentem invenere}.}. Далее, согласно Ливию, в народе возник слух, что так Флакка постигла кара богини Юноны за осквернение храма\footnote{Ibid. 13: \textit{vulgo Iunonis Laciniae iram ob spoliatum templum alienasse mentem ferebant}.}. Ливий, таким образом, указывает и источник той легенды, которую в качестве нравоучительного примера приводит в своём сборнике Валерий Максим.

% Версия Валерия Максима

% Сообщение про отсутствие храма Всаднической Фортуны в Риме за авторством Тацита

Дальнейшая история храма Всаднической Фортуны также ставит перед нами любопытные вопросы. Тацит сообщает нам, что уже в 22 г. н.э. в Риме не могли найти этот храм: когда римские всадники захотели принести Всаднической Фортуне дар за выздоровление Юлии Августы (о её болезни см. Tac. Ann. III.64), они не смогли отыскать в Риме святилище с таким названием, хотя храмов Фортуны в Городе насчитывалось множество\footnote{Tac. Ann. III.71: \textit{Incessit dein religio quonam in templo locandum foret donum quod pro valetudine Augustae equites Romani voverant equestri Fortunae: nam etsi delubra eius deae multa in urbe, nullum tamen tali cognomento erat}.}. В конце концов, выяснилось, что храм с таким именем находится в Анции\footnote{Существование храма Всаднической Фортуны в Анции подтверждается эпиграфическими данными. Согласно Анцийскому календарю, празднование этой богини приходилось на 13 августа (\textit{Degrassi}, ILLRP 9 Sext. 13).}, туда-то и направили дар (Tac. Ann. III.71).

% И сообщение об этом храме Витрувия!
Получается, что храма, основанного в Риме в 1-й половине II в. до н.э. уже не существовало в 1-й половине I в. н.э.? Этому, однако, противоречит сообщение Витрувия, который явно свидетельствует о том, что в Риме в его время был храм Всаднической Фортуны: говоря о \textit{систилосе}, т.е. способе построения колоннады, при котором расстояние между колоннами в два раза больше их диаметра, он приводит в пример именно такой храм, указывая, что находится он рядом с Каменным театром: \textit{<<Как, например, (храм) Фортуны Всаднической у Каменного театра и другие, построенные таким же образом>>}\footnote{Vitruv. III.2.2: \textit{quemadmodum est Fortunae Equestris ad theatrum lapideum reliquaeque, quae eisdem rationibus sunt conpositae}.}.

% Исследовать, почему Тацит противоречит Витрувию, и кому из них доверять стоит больше
Сравнивая сообщения двух авторов, отметим в первую очередь, что Тацит пишет о событии, отстоявшем от него почти на столетие (учитывая время написания <<Анналов>>), таким образом, он в своём изложении мог опираться только на документальные (или иного рода) источники, каковые были в его распоряжении. Витрувий, с другой стороны, приводит здание храма Всаднической Фортуны в качестве иллюстрации к своему изложению, иллюстрации, которая, судя по всему, была хорошо понятна и знакома как ему, так и его читательской аудитории. Тем не менее, нельзя сбрасывать со счетов вероятность того, что Витрувий мог ошибиться, приняв за храм Fortunae Equestris некий другой храм Фортуны или даже какой-нибудь другой богини. Но существует также возможность и того, что сообщения обоих авторов истинны, а ошибку совершили те, кто в 22 г. н.э. искали и не смогли найти реально существовавший храм Fortunae Equestris в Риме. Решить однозначно, кому из авторов стоит доверять больше, на основании только письменных источников невозможно. В разрешении этого противоречия могло бы помочь привлечение археологических данных, но, к сожалению, нам неизвестно, чтобы поблизости от Каменного театра или где-нибудь ещё в Риме был найден храм Всаднической Фортуны\footcite[P. 200]{Arya2002}. Впрочем, Л. Ричардсон выдвигает предположение, что этот храм мог быть уничтожен в результате пожара, произошедшего в 21 г. н.э. в театре Помпея\footcite[P. 155]{Richardson1992} (Tac. Ann. III.72, VI.45).

% В этом храме служат матроны

Примечательно, что Витрувий, критикуя далее недостатки описываемого им стиля \textit{systilos}, пишет: \textit{<<Ведь матери семейств, когда поднимаются по ступеням, чтобы сделать приношение, не могут пройти между колоннами все вместе, но только по очереди>>}\footnote{Vitruv. III.3.3: \textit{Matres enim familiarum cum ad supplicationem gradibus ascendunt, non possunt per intercolumnia amplexae adire, nisi ordines fecerint}.}. Это говорит нам о том, что в храме Fortunae Equestris (или, быть может, другой Фортуны) религиозные обряды отправляли \textit{матроны}, также как и в храме Fortunae Muliebris. Нас не должен удивлять тот факт, что женскому божеству служат именно матроны\footnote{Впрочем, W. Warde Fowler принимает эту черту множества культов Фортуны (он называет Fortuna Equestris, Muliebris, Virilis) за <<perhaps, the most striking fact>>. См.: \cite[Pp. 167--168]{Fowler1899}.}, однако важно, что мы приходим к такому выводу не на основании догадок или косвенных данных, а располагаем на то прямым указанием источника.

\subsection{Фортуна Сегодняшнего Дня (Fortuna Huiusce Diei)}\label{FortunaHuiusceDiei}

Фортуну Сегодняшнего Дня (Huiusce Diei) упоминает Цицерон, говоря, что она \textit{<<имеет силу во все дни>>}\footnote{Cic. De leg. II.28: \textit{Fortunaque sit [consecrata "--- В.~Ж.] vel Huiusce diei — nam valet in omnis dies}.}. Плутарх в биографии Мария сообщает, что во время битвы с кимврами при Верцеллах 30 июля 101 г. до н.э. проконсул Катул, воздев руки, молился и давал обеты Фортуне Сегодняшнего Дня (\graeca{t`hn t'uqhn t~hs <hm'eras >eke'inhs})\footnote{Plut. Marius 26.2: \graecafn{e>'uxato d`e ka`i K'atlos <omo'iws >anasq'wn t`as qe~iras kajier'wsein t`hn t'uqhn t~hs <hm'eras >eke'inhs}.}. Плиний Старший упоминает храм (\textit{aedes}) Фортуны Сегодняшнего Дня, указывая, что рядом с ним стояли семь обнажённых статуй и одна статуя старика, созданные скульптором Пифагором Самосским и окружённые почитанием\footnote{Plin. NH XXXIV.60: \textit{Fuit et alius Pythagoras Samius, initio pictor, cuius signa ad aedem Fortunae Huiusce Diei septem nuda et senis unum laudata sunt}.}. Храм (\textit{aedes}) Катула упоминает в \textit{De re rustica} Варрон: \textit{<<Это ротонда с колоннадой, как в храме Катула, только с колоннами вместо стен>>}\footnote{Varr. De re rust. III.5.12: \textit{\ldots{}rutundus columnatus, ut est in aede Catuli, si pro parietibus feceris columnas}.}. На основании этого свидетельства считается, что храму Катула соответствует т.н. \textit{Храм B} из Area Sacra di Largo Argentina\footcite[P. 156]{Richardson1992}, представляющий собой ротонду, украшенную по периметру 18 колоннами коринфского ордера (см. прил. \ref{appendix:FortunaHD}).

Плиний Старший, однако, указывает также, что Павел Эмилий посвятил храм Минерве рядом с храмом Фортуны Сегодняшнего Дня\footnote{Plin. NH XXXIV.54: \textit{\ldots{}Minervam, quam Romae Paulus Aemilius ad aedem fortunae Huiusce Diei dicavit}.}. Если имеется в виду Л. Эмилий Павел Македонский, тогда нам придётся предположить, что этот храм существовал в Риме уже ко времени битвы при Пидне. Возможно также, что Плиний, указывая местоположение \textit{ad aedem fortunae Huiusce Diei}, описывал современное ему положение вещей и не учёл, что храма, обетованного Катулом, во времена Эмилия Павла ещё не существовало.

В целом, мы можем достоверно утверждать о существовании храма Катула, обетованного им на поле битвы при Верцеллах. Существовал ли в Риме ещё какой-то храм Фортуны Сегодняшнего Дня, в частности, в 1-й пол. II в. до н.э., мы не знаем.

У. Фаулер предполагает, что aedes Fortunae Huiusce Diei был основан не ранее битвы при Пидне и что 30 июня, в день битвы при Верцеллах, там совершались жертвоприношения\footcite[P. 165]{Fowler1899}.  Е.~М.~Штаерман относит этот храм к числу тех, которые были построены в довольно широкий промежуток времени <<II "--- нач. I в. до н.э.>>\footcite[С. 112]{Shtaerman1987}.

\subsection{Народная Фортуна (Fortuna Publica Citerior in Colle)}\label{FortunaPublicaCiteriori}

Овидий <<Фастах>> указывает, что 5 апреля отмечалось освящение храма Fortunae Publicae: \textit{<<Кто скажет: ``некогда освящена на Квиринальском холме / в этот день Фортуна Народная'', будет прав>>}\footnote{\mancite{Ovid. Fast. IV.375--376}: \textit{Qui dicet 'quondam sacrata est colle Quirini / hac Fortuna die Publica', verus erit}.}. В Пренестинских фастах указано, что 5 апреля совершались \textit{Ludi Fortunae Publicae Citerio[ri] in Colle}\footnote{\mancite{CIL I\textsuperscript{2}}, p. 235.}. Об истории этого святилища более ничего не известно, однако когномен Publica позволяет однозначно отнести возникновение этого культа к эпохе Республики. Когномен Citerior, добавленный к эпиклезе богини, должен был обозначать топографическое положение храма, чтобы отличать его от расположенного рядом храма Фортуны Publicae, носившей также название Примигении\footcite[P. 9]{Champeaux1987} (п.~\ref{FortunaPublicaPrimigenia}). Таким образом, данный храм должен был располагаться ближе к центру города.

\subsection{Храм Форс Фортуны в садах Цезаря}\label{FortisFortunae3}

Плутарх в трактате <<Об удаче Римлян>> пишет о неком храме Фортуны Сильной (\graeca{>isqur'a}), Доблестной (\graeca{>aristeutik'a}) или Мужественной (\graeca{>andre'ia}): \textit{<<Удачу, [которая почитается] у реки, называют словом FORTIN, что означает сильная, доблестная или мужественная, как имеющую силу побеждать всё[, что угодно]. Её храм построили в садах, завещанных Цезарем народу>>}\footnote{\mancite{Plut. De Fort. Rom. 5, Mor. 319~A--B}: \graecafn{t`hn d`e pr`os t~w| potam~w| T'uqhn ``f'ortin'' in kalo~usin <'oper >est`in >isqur`an >`h >aristeutik`hn >`h >andre'ian, <ws t`o nikhtik`on <ap'antwn kr'atos >'eqousan. ka`i t'on ge na`on a>ut~hs >en to~is <up`o Ka'isaros t~w| d'hmw| kataliefje~isi k'hpois >w|kod'omhsan}.}. Однако из <<Анналов>> Тацита нам известно, что в 16 г. н.э. в этом месте был освящён храм \textit{Форс Фортуны}: \textit{<<на берегу Тибра, в садах, завещанных народу диктатором Цезарем\footnote{О садах, завещанных Цезарем народу, см.: Suet. Div. Jul. 83.2, Nic. Dam. 15, Plut. Brut. 20, Dio Cass. XLIV.35.3, App. B.C. II.143.}, был также освящен храм Форс Фортуны>>}\footnote{\mancite{Tac. Ann. II.41}: \textit{aedes Fortis Fortunae Tiberim iuxta in hortis, quos Caesar dictator populo Romano legaverat \ldots dicantur}.}. Очевидно, сообщение Тацита мы должны признать более точным. Плутарх, как мы указывали выше (см. с.~\pageref{LapsusGraecorum}), перепутал латинские словосочетания Fors Fortuna и Fortuna Fortis и дал неправильный перевод. Таким образом, в нач. I в. н.э. в Риме был возведён третий храм Форс Фортуны в дополнение к первым двум, построенным при Сервии Туллии (п.~\ref{FortisFortunae1}) и в консульство Карвилия в 293 г. до н.э. (п.~\ref{FortisFortunae2}).

%Однако из сообщения Тацита мы знаем, что в 16 г. н.э. был освящён храм \textit{Форс Фортуны} на берегу Тибра, в садах, завещанных Цезарем народу.

\subsection{Три Фортуны (Tres Fortunae)}\label{TresFortunae}

Витрувий пишет, что у Коллинских ворот находились сразу три храма Фортуны, поэтому само место носило название <<Tres Fortunae>>: \textit{<<Храм (aedes) называется ``с пилястрами'' (in antis), когда у него пилястры (antae) на стенах, окружающих внутреннее святилище (cella), а посередине между пилястрами "--- две колонны, и сверху фронтон \ldots{} Пример такого [храма] у трёх Фортун, из трёх [храмов] тот, что ближе к Коллинским воротам>>}\footnote{\mancite{Vitruv. III.2.2: \textit{In antis erit aedis, cum habebit in fronte antas parietum qui cellam circumcludunt, et inter antas in medio columnas duas, supraque fastigium \ldots{} conlocatum \ldots{} Huius autem exemplar erit ad tres Fortunas ex tribus quod est proxime portam Collinam}.}}. Два сооружения из трёх отождествляются с храмами Fortunae Primigeniae Publicae Populi Romani in Colle (п. \ref{FortunaPublicaPrimigenia}) и Fortunae Publicae Citerioris in Colle (п. \ref{FortunaPublicaCiteriori}); третий храм нам неизвестен\footcite[P. 9--10]{Champeaux1987}.

К какому же именно храму относится это описание? То, что один из них, находясь на Квиринальском холме, носил когномен Citerior, т.е. располагался ближе к центру города, позволяет заключить, что Витрувий говорит не о нём. Вполне вероятно, что перед нами описание храма Фортуны Примигении Римского Народа. На основании материала археологических раскопок, проводившихся в этом районе, затруднительно сделать какие-либо определённые выводы\footcite[P. 158]{Richardson1992}.

\subsection{Фортуна-Птицеловка (Fortuna Viscata)}\label{FortunaViscata}

Существовал в Риме и храм (\graeca{<ier'on}) Фортуны-<<Птицеловки>> (греч. \graeca{>ixeutr'ia}), название которой Плутарх толкует следующим образом: \textit{<<Есть на Палатине \ldots храм Фортуны-``Птицеловки'' "--- название хотя и смешное, но иносказанием наводящее на размышление о природе судьбы: она как бы издали заманивает и цепко удерживает всё, что к ней прикоснулось>>}\footnote{Plut. De Fort. Rom. 10, Mor. 322~F: \graecafn{\ldots{}>estin >en Palat'iw|, ka`i t`o t~hs >ixeutr'ias [T'uqhs <ier'on}\textit{ "--- В.~Ж.}\graecafn{], e>i ka`i gelo~ion, >all> >'eqon >ek metafor~as >anaje'wrhsin, o<'ion <elko'ushs t`a p'orrw ka`i krato'ushs sumprosisq'omena}.}. В <<Римских вопросах>> он также говорит об этом храме, упоминая его латинское название (Viscata) и давая схожее объяснение такой эпиклезе: \textit{<<\ldots{}есть даже святилище Фортуны-``Птицеловки'' (Viscata — как называют ее [римляне]); эта Фортуна как бы издали уловляет нас и крепко держит в путах обстоятельств>>}\footnote{Plut. Quae. Rom. 74, Mor. 281~E: \graecafn{T'uqhs >ixeutr'ias <ier'on >estin, <`hn bisk~atan >onom'azousin, <ws p'orrwjen <hm~wn <aliskom'enwn <up> a>ut~hs ka`i prosisqom'enwn to~is pr'agmasin}.}.

Упоминание Плутархом этого храма и толкование, данное им, помогает нам лучше понять оксюморон Сенеки из его VIII письма Луцилию; там он, ведя речь о дарах фортуны (\textit{munera fortunae}, Sen. Ep. VIII.3), предлагает за лучшее считать их всё же её кознями (\textit{insidiae}, Ibid.): \textit{<<Кто из вас хочет прожить наиболее безопасную жизнь, пусть, насколько возможно, избегает этих вымазанных птичьим клеем благодеяний>>}\footnote{Sen. Ep. VIII.3: \textit{Quisquis vestrum tutam agere vitam volet, quantum plurimum potest ista viscata beneficia devitet}.}. Сенека в <<письмах>> рассматривает фортуну как отвлечённое понятие, связанное с внешними по отношению к человеку обстоятельствами, однако его \textit{viscata beneficia} есть, скорее всего, аллюзия на действительно существовавший культ Фортуны-<<Птицеловки>>, аллюзия, хорошо понятная как ему, так и его адресату.%, риторический приём, вполне достойный его мастерства.

\subsection{Добрая Фортуна (Fortuna Bona)}\label{FortunaBona}

Культ Доброй Фортуны в Риме известен нам из надписей: CIL VI.183: \textit{Fortunae | Bon(ae) | Q. Lucilius | Felix | v. s.}, CIL VI.184: \textit{Fortunae | Bonae | Salutari}. Её упоминает Плавт (\textit{<<si Bona Fortuna veniat>>}, Plaut. Aulul. 100), а также бл. Августин (\textit{<<ergo dea Fortuna aliquando bona est, aliquando mala>>}, Aug. C.D. IV.18).

Цицерон в части четвёртой Второй речи против Верреса обвиняет его также и в том, что тот взял у жителя Мессаны Гая Гея из божницы все статуи, оставив одно очень старое изображение, которое оратор называет Fortuna Bona\footnote{\mancite{Cic. Verr. IV.4.7}: \textit{nullum, inquam, horum reliquit neque aliud ullum tamen praeter unum pervetus ligneum, Bonam Fortunam, ut opinor}.}. Очевидно, эта <<очень старая>> (\textit{pervetus}) статуя была греческой работы и изображала \graeca{>Agaj`h T'uqh}\footcite[Sp. 1512]{Peter1890Fortuna}, с которой Цицерон соотносил римскую богиню. При той дотошности, с которой Цицерон вёл следствие по делу Верреса, можно не сомневаться, что он видел изображение своими глазами.

\subsection{Злая Фортуна (Fortuna Mala)}\label{FortunaMala}

Противоположностью Доброй Фортуне была <<Фортуна Злая>>, которой был посвящён алтарь (\textit{ara}) на Эсквилине, как о том пишет Цицерон: \textit{<<Araque~\ldots Esquiliis Malae Fortunae>>} (Cic. De leg. II.28), \textit{<<Febris enim fanum in Palatio et Orbonae ad aedem Larum et aram Malae Fortunae Exquiliis consecratam videmus>>} (Cic. De nat. deor. III.63). Плиний Старший также упоминает алтарь <<Злой Фортуны>>, видимо, пересказывая последний процитированный нами отрывок из Цицерона: \textit{<<Ideoque etiam publice Febris fanum in Palatio dicatum est, Orbonae ad aedem Larum, ara et Malae Fortunae Esquiliis>>} (Plin. NH II.16).

Злая Фортуна упоминается у Плавта (\textit{<<Malam fortunam in aedis te adduxi meas>>}, Plaut. Rud. 501). О том, что Фортуна может быть недоброй, пишет и бл. Августин (\textit{<<Quia fortuna potest esse et mala>>}, Aug. C.D. IV.18). Эти упоминания, впрочем, вряд ли относятся собственно к культу Злой Фортуны, но показывают, на каких представлениях римлян этот культ мог основываться.

\subsection{Фортуна Оздоравливающая (Fortuna Salutaris)}\label{FortunaSalutaris}

Культ Fortunae Salutaris известен нам только по надписям (\mancite{CIL VI.184}: \textit{Fortunae | Bonae | Salutari}, CIL VI.201: \textit{Fortunae | Salutari | sacrum}, CIL VI.202: \textit{Fortunae | Salutari | C. Val(erius) Tertius | fisci curator v. [s.]}). Из надписи CIL VI.184 ясно, что эта Фортуна сближалась с Fortuna Bona, в смысле своего положительного аспекта, однако понималась в более конкретном смысле. Каким именно образом римляне представляли себе Фортуну полезной для здоровья, мы точнее сказать не можем.

\subsection{Фортуна Защитница и Помощница (Fortuna Tutela et Adiutrix)}\label{FortunaTutela}

Fortuna Tutela известна нам только из посвятительных надписей (CIL VI.177: \textit{Fort[unae et] | Tutela[e huius loci] | P. Aelius | p. p. | aedem cu[m porticum?] | asolo r[estituit]}, CIL VI.178 = ILS 3722: \textit{deae Fortunae Tutelae | L. Baburius Iuvenis}). Надпись CIL VI.179 посвящена Фортуне Помощнице и Защитнице: \textit{Fortunae Adiutri | ci et Tutelae Val. | Florentinus | v. l. s}. Интересна надпись CIL VI.30718, посвящённая Гению и Фортуне Защитнице Huius Loci, правда, однозначно относящаяся уже к имперскому времени: \textit{Genio et  Fortunae | Tutelaeque Huius Loci cohortium | praetoriarum | piarum vindicarum | [?] | Aeterni Augusti}. 

Про храм (aedes) этой Фортуны, упомянутый в надписи CIL VI.177, нам ничего более неизвестно. Судя по когномену, она почиталась как защитница и охранительница, об этом же свидетельствует её упоминание вместе с Фортуной Помощницей и Гением. Близость Фортуны и Гения находит отражение и в письменных источниках, так, в <<Метаморфозах>> Апулея герой призывает Фортуны и Гениев\footnote{\mancite{Apul. Met. VIII.20}: \textit{Per Fortunas vestrosque Genios}.}. Арнобий, ведя речь о Пенатах, обращается к трудам Цезия Басса, где Пенатами считаются Фортуна, Церера, Палес и Гений Юпитера\footnote{\mancite{Arnob. III.40.2}: \textit{id sequens Fortunam arbitratur et Cererem, Genium Iovialem ac Palem}; \mancite{Ibid. 43.2}: \textit{Ceres Pales Fortuna, Iovialis aut Genius}.}.

% Дописать про Арнобия ещё?

% И про картинку

Обратим внимание на рис.~\ref{pic:Cavemalum}, фреску из Помпей, где изображена Фортуна со своими традиционными атрибутами (рогом изобилия и кормилом), фигура человека, по двум сторонам от которого находятся змеи, и приведена несколько ироничная надпись \textit{Cacator, cave malu[m]}, очевидно, предостерегающая от отправления естественных потребностей в неположенном месте. Фигура человека, принявшего известную позу "--- это, скорее всего, и есть \textit{cacator}, и опасаться, согласно надписи, ему следует именно гнева изображённой рядом Фортуны. Таким образом, Фортуна выступает здесь в качестве защитницы и охранительницы места, причём защитницы такого рода, которая может доставить неприятности злоумышленнику. Змеи "--- распространённая тема для изображений на ларариях, и обычно они располагаются по сторонам от алтаря\footcite[Pp. 99--100]{ArtePompei1989}, а на этой фреске вместо алтаря "--- человеческая фигура.

Безусловно, у данного изображения может быть и другое, более серьёзное и торжественное толкование, а надпись, возможно, была нацарапана на фреске в шутку. Но сам факт подобной надписи свидетельствует о том, что приведённое нами ироническое толкование картины вполне допускалось римлянами, а вероятная двусмысленность могла только добавить сарказма.

% Общий вывод

Таким образом, письменные источники, надписи и изобразительные свидетельства говорят о близости Фортуны и таких божеств-покровителей отдельных мест как Гении, Лары и Пенаты. Подобное сближение совершилось, скорее всего, не ранее эпохи поздней Республики "--- начала Империи.

\subsection{Фортуна Житниц (Fortuna Horreorum)}\label{FortunaHorreorum}

<<Фортуна Житниц>> (Horreorum) известна нам по одной посвятительной римской надписи CIL VI.188 = ILS 3721: \textit{L. Dunnius Apella, | C. Annius Tyrannus | mag. prim. | Fort(unae) horr(eorum) d.d.} Скорее всего, эта Фортуна воспринималась в апотропаическом смысле и была близка к Fortuna Tutela Huius Loci.

\subsection{Фортуна Бань (Fortuna Balnearum)}\label{FortunaBalnearum}

Фронтон упоминает <<Фортуну Бань>> в письме императору Антонину: \textit{<<Omnis ibi Fortunas~\ldots balnearum~\ldots reperias>>} (Fronto De orat. P. 157 Nab.). О её римском культе известно из одной надписи CIL VI.182: \textit{Fortunab(us) | Bal(nei) Verul(ani) | C.Hostilius | A.Gathopus | d. d.} Существовал ли этот культ в эпоху Республики, или же он был основан уже в императорское время, нам неизвестно. Возможно, здесь, как и в случае с Fortuna Horreorum, можно говорить о Фортуне в смысле покровительницы Huius Loci. Неясно, соотносился ли культ <<Фортуны Бань>> каким-либо образом с культом Мужской Фортуны и ритуалом мытья в банях, описанным выше.

\subsection{Фортуна Бородатая (Fortuna Barbata)}\label{FortunaBarbata}

Тертуллиан в сочинении <<Против язычников>> пишет, что \textit{<<Ювента "--- богиня юношей, надевающих тогу, а Фортуна Барбата "--- богиня мужчин>>}\footnote{\mancite{Tert. Ad nat. II.2.11}: \textit{Est et Iuventa novorum togatorum, virorum iam Fortuna barbata}.}. Бл. Августин говорит о \textit{<<Фортуне Барбате, которая покрывает бородой взрослых>>}\footnote{\mancite{Aug. C.D. VI.11}: \textit{\ldots{}sit et Fortuna barbata, quae adultos barba induat}.}. Вероятно, её следует соотносить с Fortuna Muliebris, которая была покровительницей женщин, эта же Фортуна выступала в роли покровительницы мужчин. О её культе в Риме ничего не известно. Немногочисленные посвятительные надписи Фортуне Барбате происходят не из Рима\footcite[Sp. 1519.]{Peter1890Fortuna}.

\subsection{Фортуны отдельных фамилий}\label{FortunaeFamiliarum}

Судя по когноменам, известным нам, впрочем, только из надписей, римляне почитали также Фортун "--- покровительниц отдельных фамилий. Таковы Fortuna Flavia (CIL VI.187: \textit{Fortunae | Flaviae | Callitatorio}), Fortuna Iuveniana (CIL VI.189 = ILS 3715: \textit{Fortuna | Iuveniana | Lampadia | na}), Fortuna Torquatiana (CIL VI.204: \textit{Fortunae | Torquatianae | Q. Caecilius | Narcissus | d. d.}), Fortuna Cancesis (CIL VI.185: \textit{[Co]rnelius Anton[ianus?] | F[o]rtune Cancesi d. d.}). Последняя надпись, возможно, относится к периоду Империи. Известен также ara Fortunae Crassianae (CIL VI.186 = ILS 3714), но, судя по посвящению Александру Северу, эта надпись оставлена уже в III в. н.э.

Мы не можем сказать, какую роль в жизни римских фамилий играли эти <<семейные>> Фортуны. Возможно, эти богини воспринимались так же, как Фортуны-покровительницы других групп: женщин, девиц, всадников и всего римского народа. Однозначно отнести время возникновения этих культов к республиканской эпохе мы тоже не можем. К.~Латте полагает, что они возникают уже в императорское время\footcite[S. 182]{Latte1960}.

\subsection{Фортуна Туллиана}\label{FortunaTulliana}

Возможно, нижеследующую надпись следует также отнести к только что рассмотренной группе (как делает Р.~Петер\footcite[Sp. 1521]{Peter1890Fortuna}). Однако когномен этой Фортуны настолько красноречив, что мы выносим его отдельно и включаем в наше исследование, несмотря даже на то, что надпись однозначно оставлена в эпоху империи. Fortuna Tulliana известна нам из следующей надписи: CIL VI.8706 = ILS 3717: \textit{Ti. Claudius Aug.l. | Docilis | aeditus aedis | Fortunae Tullianae}.

Ничто не говорит о том, что храм Fortunae Tullianae появился обязательно в период Империи. Наоборот, сам когномен намекает на то, что культ может восходить к эпохе царского Рима. Л. Ричардсон полагает, что это может быть один из многочисленных храмов Фортуны, которые приписывают Сервию Туллию\footcite[P. 158]{Richardson1992}. Однако, как мы показали в п.~\ref{PlutarchiCritica}, все эти сведения о множестве культовых сооружений Фортуны, созданных царём Сервием, восходят к одному Плутарху, и нам не следует относиться к его сообщениям со слишком большим доверием.

Обратим внимание, что в надписи говорится о вольноотпущеннике Августа, смотрителе (\textit{aeditus = aeditumus}) храма Фортуны Туллианы. Если предположить, что когномен Tulliana связан не с какой-то фамилией Туллиев, а именно с царём Сервием, то тогда мы можем увидеть в этом тексте определённый смысл, а именно: как известно, римляне считали Сервия любимцем Фортуны, который, будучи рождён от рабыни, благодаря заступничеству богини сделался царём Рима (Plut. Quae. Rom. 106, Mor. 289~C). Быть может, вольноотпущенник чтил богиню, потому что считал обретённую свободу милостью Фортуны? И тогда Фортуна Туллиана "--- покровительница вольноотпущенников? Мы, всё же, ничего более не знаем об этом культе, и единственное упоминание этого когномена в надписи позволяет нам строить одни только предположения.

\section{Становление культа Фортуны в Риме}\label{Stanovlenie}

%Наш исторический обзор показывает, каково было разнообразие форм поклонения Фортуне в Риме к началу императорской эпохи. Подобная картина, разумеется, сложилась не сразу. Несмотря на отрывочность источников, мы всё же можем сложить себе представление об основных этапах развития культа Фортуны в Риме.

%Первые храмы Фортуны появились в Риме ещё в царскую эпоху. Нам достоверно известно, что в правление Сервия Туллия было создано два храма Фортуны: один "--- на Бычьем форуме, другой "--- святилище Форс Фортуны на берегу Тибра. 

%Вопрос этот теснейшим образом связан с проблемой возникновения и развития римской религии в целом и с проблемой эволюции религии вообще.

%Очевидна тесная связь римских представлений о Фортуне с лацийскими культами Анция и Пренесте, но также и своеобразие каждой из форм поклонения. Рассмотрение культов Лация, однако, выходит за рамки настоящей работы. 

%Г.~Виссова ограничивается тем, что причисляет Фортуну к ряду di novensides. Х. Розе видит в пренестинском культе Фортуны следы раннего греческого влияния. Он, а также С. Бейли считают Фортуну первоначально аграрным божеством, которое впоследствии, под влиянием эллинистического культа Тихи, стало почитаться как богиня удачи.

%К.~Латте полагает, что культ Фортуны проник в Рим из Пренесте и Анция, однако это не объясняет своеобразия римских и лацийских культов, а также не даёт ответа на вопрос, каким образом начали поклоняться Фортуне в самих Пренесте и Анции. А.~И.~Немировский говорит о ней как о древнейшем материнском божестве Италии. Вряд ли можно всерьёз относиться к его гипотезе о происхождении Фортуны и Портуна из одного (андрогинного?) божества. Ж. Шампо считает, что многообразие форм культа и представлений о Фортуне имеют гетерогенное происхождение; это, впрочем, не может считаться окончательным ответом.

%В целом, вопрос о происхождении культа Фортуны в Риме остаётся открытым. Скорее всего, располагая теми источниками, которые у нас в наличии, мы вряд ли сможем выйти за пределы разного рода гипотез, более или менее обоснованных.

На основании вышеизложенного мы можем достоверно отнести возникновение культа Фортуны в Риме к царской эпохе. Первые храмы Фортуны были основаны "--- или, по крайней мере, уже существовали "--- при Сервии Туллии. Нам известно, что при этом царе произошло официальное введение культа Фортуны в Риме, поскольку именно ему античная традиция приписывает основание первых храмов и праздников этой богини, включённых в фасты. Можем ли мы найти следы поклонения Фортуне в Риме в более отдалённую эпоху? Варрон называет Фортуну в числе божеств, которые римляне переняли у сабинян ещё при Ромуле\footnote{\mancite{Varr. De ling. Lat. V.74}: \textit{Feronia, Minerva, Novensidnses a Sabinis; paulo aliter ab eisdem dicimus haec: Palem, Vestam, Salutem, Fortunam, Fontem, Fidem}.}. Ничто, однако, не служит подтверждением этой версии; скорее всего, её можно объяснить, с одной стороны, сабинским происхождением Варрона\footnote{А.~И.~Немировский подчёркивает, что Варрон в своих произведениях выдвигал на первое место сабинские религиозные обычаи и приписывал сабинское происхождение общеиталийским божествам. См. \cite[С. 6]{Nemirovsky1964}.}, с другой "--- сложившимся в античной традиции обычаю приписывать все римские установления Ромулу. Как мы отмечали выше, археологические свидетельства показывают, что на территории храма на Бычьем форуме ещё до эпохи Сервия существовало некое святилище, возможно, восходящее к временам Ромула; однако мы ничего не можем сказать о том, кому оно было посвящено и что за культ там отправлялся. Храм на Бычьем форуме находился \textit{внутри} померия, на самой границе его. Храм Форс Фортуны был построен \textit{trans Tiberim}\footnote{См. свидетельства фастов на с. \pageref{Fastae}.}, вне померия. В целом, мы можем говорить о том, что культ Фортуны возникает на топографической периферии Города царской эпохи. Обратим внимание также на то, что эти древнейшие храмы были привязаны к Тибру, транспортной артерии Рима, связывающей его с соседними регионами. Храм Форс Фортуны, точное местоположение которого неизвестно, находился на берегу Тибра\footnote{Varr. De ling. Lat. VI.17: \textit{fanum Fortis Fortunae secundum Tiberim}; Dion. Hal. IV.27.7: \graecafn{>ep`i ta~is >hi'osi to~u Teb'erios}; Ovid. Fast. VI.776: \textit{in Tiberis ripa munera regis habet}.}; храм на Бычьем форуме располагался неподалёку от реки (рис. \ref{pic:ForiBoariSchema} и \ref{pic:Charta}), рядом с её излучиной, которая могла использоваться как естественная гавань для прибывающих в Рим купеческих судов "--- при этом не обязательно иметь в виду именно греческих купцов, как о том пишет Ж.~Шампо\footcite[P. 214]{Champeaux1982}. Таким образом, связь с водой может не только трактоваться ритуально или символически, но также и указывать на возможные пути проникновения культа Фортуны в Рим.

Были ли в царскую эпоху основаны иные культы Фортуны помимо храмов Форс Фортуны и того, что на Бычьем форуме, мы достоверно сказать не можем.

Перемещаясь в эпоху Республики, мы обретаем более точные сообщения источников. Для некоторых достаточно крупных храмов нам известны обстоятельства и даты их обетования и создания: это храмы Женской Фортуны (486 г. до н.э.), Форс Фортуны (консула Карвилия, 293 г. до н.э.), Фортуны Примигении Римского Народа (194 г. до н.э.), Фортуны Всаднической (173 г. до н.э.), а также, видимо, сюда же следует отнести храм Фортуны Сегодняшнего Дня за авторством Катула (после 101 г. до н.э.). Эти сведения дошли до нас постольку, поскольку перечисленные культы оказались так или иначе связаны с военно-политической историей Рима (речь идёт об обетовании, постройке храмов, предназначением культа).

Культ Фортуны занимал несколько периферийное положение в римской религии: Фортуна не имела своего фламина и не входила в число двенадцати капитолийских божеств. Многие крупные храмы Фортуны, о которых мы можем судить достоверно, располагались за пределами померия. Внутри померия находился храм Фортуны на Бычьем форуме и два храма Fortunae Publicae на Квиринале с каким-то третьим, нам неизвестным. Существовал также алтарь Fortunae Malae на Эсквилине (п. \ref{FortunaMala}), но его нельзя причислить к наиболее значимым культовым объектам. Возможно, культ какой-то Фортуны отправлялся на Капитолии, но свидетельства о нём крайне недостаточны, поэтому мы не можем достоверно назвать ни когномен, ни дату его основания.

Несмотря на указанную нами определённую <<периферийность>>, Фортуна высоко чтилась римлянами, о чём говорит не только её популярность, являющаяся чисто количественной характеристикой. Отличительной чертой культа Фортуны в Риме было то, что эта богиня считалась покровительницей различных общественных групп и объединений: женщин, мужчин, девиц, всадников, отдельных фамилий, наконец, всего римского народа. Посвятительные надписи Форс Фортуне оставляли ремесленные коллегии. Таким образом, культ Фортуны оказывался тесно встроен в римское общество на разных ступенях его иерархии. Такая картина возникла не сразу, она складывалась в течение всего республиканского периода.

Для эпохи Республики мы можем выделить две основные тенденции развития культа Фортуны. Во-первых, это количественный рост числа когноменов богини: появляются Фортуны, как кажется, на все случаи жизни. Во-вторых, всё более тесная интеграция этого изначально периферийного культа в римскую цивитас, вплоть до появления храма Фортуны Римского Народа, который однозначно можно отнести к государственной форме культа. Безусловно, за неимением достоверных данных мы не можем утверждать, что когномены Фортуны, рассмотренные нами в п.~\ref{PlutarchiCulti} и \ref{RespublicaeCulti} появились именно в эпоху Республики. Учтём, однако, что смена форм правления не оказывает столь кардинального влияния на развитие религии. Вполне можно допустить, что тенденции, наиболее полно проявившие себя в республиканский период, берут своё начало в царской эпохе, нельзя также полагать, что с установлением принципата они резким образом обрываются.

Результатом развития культа Фортуны в Риме к началу императорской эпохи было многообразие форм поклонения во множестве <<ипостасей>> и под множеством когноменов, встраивание культа в систему государственной военно-политической пропаганды, а также популярность его среди различных слоёв общества, что подтверждается наличием пяти праздников Фортуны в фастах (см. прил. \ref{appendix:festivales}).

% Гипотезы о происхождении к.Ф. в Риме

