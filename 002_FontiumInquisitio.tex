\chapter*{Обзор источников}
\addcontentsline{toc}{chapter}{Обзор источников}

\input{DeFontibus/000_Narrativi.tex}

\input{DeFontibus/010_Inscriptiones.tex}

\input{DeFontibus/020_Monetae.tex}

\input{DeFontibus/025_Archaeologia.tex}

\input{DeFontibus/028_Ars.tex}

\input{DeFontibus/030_Mythologia.tex}

\input{DeFontibus/040_Ethnographia.tex}

\input{DeFontibus/100_Conclusio_Fontium.tex}

% В целом сведения, которые мы можем почерпнуть из письменных источников, можно разделить на два вида: во-первых, это сведения фактического характера; во-вторых, это представления самих древних авторов о Фортуне, что позволяет нам охарактеризовать Фортуну как понятие.

% Говоря об эллинистическом влиянии на римскую религию и об отличиях собственно римской Фортуны от греческой Тюхе (что составляет отдельную исследовательскую проблему), необходимо постоянно помнить, что античные авторы, к сочинениям которых мы обращаемся, и были, собственно, агентами эллинистического влияния на римскую культуру. Латиноязычные авторы все получили греческое образование; о грекоязычных авторах не приходится говорить. Обращаясь конкретно к нашей теме мы видим, что как латиноязычные авторы на примере Пакувия и Плиния Старшего, так и грекоязычные на примере Плутарха и Дионисия Галикарнасского ставят знак равенства между латинским словом \graeca{Fortuna} и греческим словом \graeca{T'uqh}. Ни один из античных писателей не задавался прямо вопросом, который ставят современные исследователи: в чём отличие греческой богини \graeca{T'uqh} от римской богини \textit{Fortuna}\footnote{Да, Плутарх в сочинении <<Об удаче римлян>> говорил о специфических чертах почитания Удачи в Риме. Но он всё время называл римскую Фортуну греческим словом \graecafn{T'uqh} и не задумывался о том, что это могли бы быть две разные, хотя и в чём-то похожие, богини.}? В работах античных интеллектуалов мы не найдём ответа на этот вопрос; даже более: мы не сможем ответить на этот вопрос, опираясь только на письменные источники, оставленные образованными людьми древности. Здесь мы должны обращаться к источникам другого рода.

% Если говорить о представлении древних римлян о Фортуне

% Ценность трудов Плутарха "--- в том, что он акцентирует внимание на тех чертах, которые самим римлянам казались очевидными и которые они специально не выделяли. Так, греческий автор с удивлением говорит о том, под каким множеством имён Фортуна почиталась в Риме "--- а это типично римская черта, привлекающая внимание и современных исследователей.

