\chapter*{Список использованных источников и литературы}
\addcontentsline{toc}{chapter}{Список использованных источников и литературы}

\begin{flushleft}

\section*{Источники}
\addcontentsline{toc}{section}{Источники}

\begin{enumerate}

%\item Арнобий. Против язычников. СПб., 2008.

\item Валерий Максим. Достопамятные деяния и изречения. СПб, 2007.

\item Гай Светоний Транквилл. Жизнь двенадцати цезарей. М., 1993.

\item Дионисий Галикарнасский. Римские древности. Т. 1. Т. 2. М., 2005.

\item Корнелий Тацит. Сочинения в двух томах. Том I. <<Анналы. Малые произведения>>. М., 1993.

\item Марк Туллий Цицерон. Диалоги. М., 1994.

\item Плавт. Комедии. Т. 1. Т. 2. М., 1987.

\item Плутарх. Об удаче римлян // ВДИ. 1979. № 3. С. 233--251.

\item Плутарх. Римские вопросы // ВДИ. 1976. № 3. С. 230--252; № 4. С. 185--225.

\item Плутарх. Сравнительные жизнеописания в двух томах. М., 1994. Т. I.

\item Полонская~К.~П., Поняева~Л.~П. Хрестоматия по ранней римской литературе. М., 1984.

\item Публий Овидий Назон. Элегии и малые поэмы. М., 1973.

\item Тит Ливий. История Рима от основания города. В 3 тт. М., 2005.

\item Ad Lucilium Epistulae Morales. Vol. 1. Seneca. Camb., Harvard, L., 1917. 

\item Annales ab excessu divi Augusti. Cornelius Tacitus. Charles Dennis Fisher. Oxford, 1906.

\item De Natura Deorum. M. Tullius Cicero. Leipzig, 1917.

\item De Divinatione. M. Tullius Cicero. C. F. W. M\"{u}ller. Leipzig, 1915. 

\item Dionysii Halicarnasei Antiquitatum Romanarum quae supersunt, Vol II. Leipzig, 1898. Vol. III. Leipzig, 1891.

\item M. Terentii Varronis De lingua Latina quae supersunt. Lipsiae. 1910.

\item M. Tullius Cicero. De Legibus. Paris. 1959.

\item P. Ovidius Naso. Ovid's Fasti. Sir James George Frazer. L., Camb., 1933.

\item Plutarch's Moralia. Vol. 4. Camb., L., 1972.

\item Plutarch. Plutarch's Lives. with an English Translation by Bernadotte Perrin. Camb., L. 1916.

\item Sexti Pompeii Festi De verborum significatione libri XX. Vol. II. L., 1826.

\item The Loeb Classical Library. Livy. From the founding of the City. Vol. I--XIII. L., 1919--1951.

\item The Loeb Classical Library. Pliny. Natural History. Vol. I. Vol. III.  L., 1967. Vol. IX. L., 1951.

\item Valerius Maximus. Factorvm et Dictorvm Memorabilivm, Libri Novem. Karl Friedrich Kempf. Leipsig, 1888.

\item Vitruvius Pollio. On Architecture. F. Krohn. Lipsiae, 1912.

\end{enumerate}

\section*{Библиография}
\addcontentsline{toc}{section}{Библиография}

\begin{enumerate}

%\item ВДИ. 1976. No 3. С. 230. (*посмотреть, что это*)

%\item ВДИ, 1979, No 3. С. 235. (*посмотреть, что это*)

\item Античная литература. П/р А.~А.~Тахо-Годи. М., 1986.

\item Дератани~Н.~Ф., Нахов~И.~М., Полонская~К.~П., Чернявский~М.~П. История римской литературы. М., 1954.

\item Добролюбов~Н.~А. О Плавте и его значении для изучения римской жизни // Собр. соч. Т. 1. М.-Л., 1961. С. 328--343. % Посмотреть точные страницы!

\item Дюмезиль,~Ж. Верховные боги индоевропейцев. М., 1986.

\item История греческой литературы. Т. 3. М., 1960.

\item История римской литературы. Т. 1. М., 1959. Т. 2. М., 1962.

\item Кнабе~Г.~С. Корнелий Тацит. Время. Жизнь. Книги. М., 1981.

\item Лосев~А.~Ф. Плутарх. Очерк жизни и творчества // Плутарх. Сочинения. М., 1983. С. 5--44.

\item Лосев~А.~Ф. Античная литература. М., 1997.

\item Маяк~И.~Л. <<Римские древности>> Дионисия Галикарнасского как исторический источник // Дионисий Галикарнасский. Римские древности. Том 3. М., 2005. С. 243--269.

\item Маяк~И.~Л. Римляне ранней Республики. М., 1993.

\item Маяк~И.~Л. Римские древности по Авлу Геллию: история, право. М., 2012.

\item Немировский А. И. Идеология и культура раннего Рима. Воронеж, 1964.

\item Сморчков~А.~М. Храмовый обет (votum) в религиозно-политической практике республиканского Рима // Античный мир и археология. Вып. 14. Саратов, 2010. С. 107--121.

\item Сморчков~А.~М. Религия и власть в Римской республике. М., 2012.

\item Тэн~И. Тит Ливий. Критическое исследование. М., 1900.

\item Фрейберг Л.А. <<Моралии>> Плутарха // ВДИ, 1976, No 3. С. 219--229.

\item Шайд~Дж. Религия римлян. М., 2006.

\item Шайд~Дж. Миф, культ и реальность в «Фастах» Овидия // Шайд~Дж. Религия римлян. М., 2006. С. 223--241.

\item Штаерман~Е.~М. Социальные основы римской религии. М., 1987.

\item <<Энциклопедический словарь>>. Т. XXXVI. СПб, 1902.

\item Altheim,~F. A history of Roman religion. L., 1938.

\item Arya,~D.~A. The goddess Fortuna in Imperial Rome: cult, art, text. Austin. 2002.

\item Bailey,~C. Phases in the religion of ancient Rome. California, 1932.

\item Brill’s New Pauly. Ed. by H. Cancik and H. Schneider. Vol. 5. Leiden-Boston, 2004.

\item Dum\'{e}zil,~G. Servius et la fortune: essai sur la fonctíon sociale de louange et de bl\^{a}me et sur les \'{e}l\'{e}ments indo-europ\'{e}ens du cens romain. 1943.

\item Dum\'{e}zil,~G. La Religion romaine archa\"\i{}que, avec un appendice sur la religion des \'{E}trusques. 1966.

\item Encyclop\ae{}dia Britannica. Macropædia. Vol. 15. Chicago etc. 1978.

\item Ferguson, J. The religions of the Roman empire. Southampton, 1970.

\item Latte,~K. R\"{o}mische Religionsgeschichte. M\"{u}nchen, 1960.

\item Paulys Real-Encyclop\"{a}die der classischen Altertumwissenschaft. Hb 13. Stuttgart. 1910.

\item Rose,~H.~J. Greek and Roman religion. NY, 1959.

\item Walsh,~P. Livy. His historical aims and methods. Camb., 1974.

\item Warde Fowler, W. The Roman festivals of the period of the Republic. London, 1899.

\item Wissowa,~G. Religion und Kultus der R\"{o}mer. M\"{u}nchen, 1902.

\end{enumerate}


\end{flushleft}
