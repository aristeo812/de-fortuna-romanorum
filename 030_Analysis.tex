\chapter{Фортуна в контексте истории римской религии}

\section{Проблема происхождения и развития культа Фортуны}

% Проблема происхождения К.Ф.
Единого ответа на вопрос о происхождении культа Фортуны в Риме современная историческая наука пока не дала. Эту задачу нельзя разрешить без обращения к вопросу о происхождении римской религии в частности и проблеме эволюции религии вообще. К сожалению, мы располагаем крайне малым числом достоверных сведений о наиболее ранних этапах становления культа Фортуны в Риме, что оставляет большой простор для построения различных гипотез, часто противоречащих друг другу. Разнообразие мнений о возникновении культа Фортуны и его первоначальном характере весьма велико.

% Преллер
Л. Преллер указывает на возможные латинские и сабинские корни римского культа Фортуны\footcite[S. 179]{Preller1883}. Он предполагает, что первоначально Фортуна могла восприниматься как богиня доброй удачи (Positive Gl\"{u}cksg\"{o}ttin), а впоследствии "--- как божество безучастной судьбы (indifferentes Geschick)\footnote{\mancite{Ebd.}}. Вслед за Преллером, таких же взглядов на эволюцию представлений о Фортуне придерживается и Г.~Экстелл\footcite[Pp. 9--10]{Axtell1907}.

В монографии, посвящённой римским праздненствам, У.~Фаулер высказывает достаточно взвешенное мнение, с осторожностью предполагая, что Фортуна, с которой был связан древний оракул в Пренесте, как богиня-пророчица могла соотноситься с италийскими Карментами или североевропейскими Норнами, определяющими судьбу человека при его рождении. Таким образом, Фортуна, согласно его предположению, могла бы восприниматься как божество, покровительствующее матерям в деторождении (подобно Матери Матуте) и, через это, определяющее судьбу младенца\footcite[P. 167]{Fowler1899}. В другой своей работе, однако, Фаулер меняет точку зрения и полагает, что Фортуна была первоначально женским божеством, покровительницей плодородия и деторождения, а со временем переняла характер греческой богини Тихи и стала божеством удачи\footcite[P. 235]{Fowler1911}. Такой взгляд на эволюцию культа Фортуны и представлений о ней стал популярен в англоязычной литературе, и его придерживались С.~Бейли\footcite[P. 137]{Bailey1932}, Х.~Розе\footcite[P. 238]{Rose1959} и Дж.~Фергюсон\footcite[P. 85]{Ferguson1970}.

Г.~Виссова причисляет Фортуну к ряду di novensides\footcite[S. 206]{Wissowa1902}. Он соотносит наиболее ранние представления о Фортуне с божествами судьбы, греческими мойрами и италийскими парками\footcite[S. 213]{Wissowa1902}, что близко к первоначальной точке зрения У.~Фаулера, однако два исследователя, скорее всего, пришли к схожим выводам независимо друг от друга.

В. Отто высказывает мнение, что первоначально Фортуна могла быть или женской богиней, или местным божеством\footcite[Sp. 13--14]{FortunaOtto1910}. К.~Латте полагает, что культ Фортуны проник в Рим из Пренесте и Анция\footcite[S. 176]{Latte1960}, однако это не объясняет своеобразия римских и лацийских культов, а также не даёт ответа на вопрос, каким образом начали поклоняться Фортуне в самих Пренесте и Анции.

А.~И.~Немировский полагает Фортуну <<древнейшим материнским божеством Италии>>\footcite[С. 72]{Nemirovsky1964}, он придерживается мнения, что она одновременно была и богиней судьбы\footcite[С. 73]{Nemirovsky1964}, хотя и указывает, что <<в древнейшем слое легенды о Сервии Туллии Фортуна могла быть просто материнским божеством>>\footcite[С. 86]{Nemirovsky1964}. Вряд ли можно всерьёз относиться к его гипотезе о происхождении Фортуны и Портуна из одного (андрогинного?) божества\footcite[С. 578]{Nemirovsky1987}.

Ж. Шампо считает, что многообразие форм культа Фортуны имеет гетерогенное происхождение, и что необходимо различать представления о Фортуне как о богине-пророчице и верховной <<космической>> богине, с одной стороны, и её функции материнского божества "--- с другой; согласно её выводам, скорее всего, путей проникновения культа Фортуны в Рим было несколько\footcite[Pp. 471--479]{Champeaux1982}. Это, впрочем, не может считаться окончательным ответом.

В целом, вопрос о происхождении культа Фортуны в Риме остаётся открытым. Скорее всего, располагая теми источниками, которые у нас в наличии, мы вряд ли сможем выйти за пределы разного рода гипотез, среди которых, впрочем, мы можем выделить более и менее обоснованные.

% Близость материнским культам

Так, довольно очевидна связь архаической Фортуны с материнскими и, шире, женскими божествами. Об этом свидетельствуют наиболее ранние храмы Фортуны "--- на Бычьем форуме и Женской Фортуны, а также культы Фортуны-Девы и Мужской Фортуны. Праздник Фортуны на Бычьем форуме отмечался в Матралии (11 июня), а Мужской Фортуны "--- в Венералии (1 апреля). Культ Фортуны в Пренесте соединяет в себе черты поклонения богине-матери и богине-пророчице; врочем, рассмотрение этого культа выходит за рамки нашего исследования, к тому же, мы не можем точно датировать его возникновение\footcite[S. 504]{Kajanto1981}.

% Что происходило в начале

Вышеприведённые факты о культе Фортуны показывают, что у нас нет оснований считать её первоначально сельскохозяйственным божеством. В наиболее архаической форме культа, возникшего в Риме, мы видим прямую связь Фортуны с материнством и деторождением. Безусловно, мы можем говорить о том, что в интуитивном мифологическом сознании архаического человека существовала неразрывная связь плодородия почвы и творящей силы женщины\footcite[С. 307]{Eliade1999}, и на этом основании сделать заключение, что Фортуна некогда воспринималась как богиня плодородия вообще. Но все наши источники свидетельствуют о том, что когда Фортуна выступает на историческую сцену и в Риме появляются её первые храмы, прямой связи с сельскохозяйственными культами у неё нет, как нет её и после. О культе Фортуны до Сервия Туллия у нас нет никаких достоверных свидетельств, и всё, что мы можем "--- строить предположения по аналогии, основываясь на достижениях современного сравнительного религиоведения.

При сближении Фортуны с такими божествами судьбы как Парки и Карменты (Виссова, Фаулер), нам нужно с осторожностью относиться к предположению, что она могла почитаться как богиня, определяющая судьбу человека \textit{в момент рождения}. Известные нам факты не свидетельствуют о том, что фортуна человека в смысле его судьбы воспринималась римлянами как предначертанная от рождения. Фортуна, предсказавшая Гальбе его высокую судьбу, явилась к нему в день совершеннолетия. Сервий Туллий, насколько известно, был интересен ей как уже вполне созревшая личность. И наоборот, мы видим, что Фортуна оказывает влияние на судьбу в какой-то определённый момент, например, на поле битвы, или в конкретный день (Huiusce Diei), причём вмешивается как сила, способная внести решающие изменения. Предначертание в судьбе, обещанное Фортуной, является не неизбежностью фатума, но особой милостью богини, милостью, которая, в принципе, может смениться на немилость "--- и здесь Фортуна выступает не в качестве безликого проявления необходимости, но в своём антропоморфном облике, и ей не чужды человеческие эмоции. Столь широкое культовое почитание Фортуны в Риме было направлено, как можно полагать, на стяжание со стороны богини подобной благосклонности, обращающей течение дел к лучшему.


% Проблема эволюции К.Ф.: эллинистическое влияние

% То, что эллинистическое влияние шло, в первую очередь, через труды образованных авторов.
% Также через искусство, но оно было пропущено через римский фильтр, что мы показали в главе про символизм изображений.

% Развитие культа Фортуны в республиканску эпоху

О развитии культа Фортуны в республиканскую эпоху мы можем сделать более определённые выводы, однако в силу того, что о первоначальном характере этого культа мы имеем только более или менее гипотетические представления, то нам трудно заключить, насколько сильной была его трансформация. Как мы показали, в научной литературе, в основном, англоязычной, распространено мнение, что Фортуна под влиянием эллинистических представлений о Тихе превратилась из сельскохозяйственного или женского божества в богиню удачи и случая. Конечно, ещё Г. Буассье указывал, что, при сохранении внешних атрибутов культа, в представлении о римских богах могли происходить кардинальные изменения\footnote{<<Римские боги остались до такой степени неопределёнными, они так охотно поддаются всевозможным изменениям, что последние могут иногда произойти совершенно незаметно. С виду как будто ничто не изменилось: бог сохранял своё имя и свой внешний вид, но понятие о нём уже не то, что было прежде, и, оказывается, что под древним названием скрывается уже новый бог>> (\cite[С. 361]{Boissier1914}).}.

% Эллинистическое влияние

Однако предположение, что изначально сельскохозяйственное или женское божество, одно-единственное из множества, под некоторым влиянием сделалось богиней удачи и безразличного случая, представляется настолько фантастическим и из ряда вон выходящим, что требует особого обоснования, которого, впрочем, никто не даёт. Во-первых, сразу возникает вопрос: а почему такой трансформации подверглась именно Фортуна, а не другие женские божества? Почему не Церера, не Юнона, не Матерь Матута, не Теллус? Чем Фортуна так разительно отличалась от прочих богинь? Ответа на этот вопрос мы не находим.

Во-вторых, нам необходимо учитывать, что эллинистическое влияние, т.е. воздействие греческой культуры на римскую, было явлением комплексным, многогранным и развивавшимся закономерно. Если мы считаем, что под влиянием эллинистических представлений о Тихе культ Фортуны кардинальным образом изменился, мы должны также заключить, что такому же существенному влиянию вынуждены были подвергнуться и другие культы: Юпитер должен был бы принять облик Зевса, Юнона "--- Геры, Диана "--- Артемиды и т.д., но мы этого не видим. Мы, конечно, можем предположить, что образ и культ Тихи обладали каким-то особым гипнотическим воздействием на римлян, но тогда мы должны наблюдать это воздействие и во всех других областях. Мы бы видели, что римские скульпторы копируют с греческих оригиналов статуи одной Тихи, а прочие изображения ваяют в своём, оригинальном стиле; что римские литераторы перелагают на латынь не Одиссею и греческие комедии, а исключительно пассажи о Тихе; и, конечно же, мы бы не увидели ни одного римского философа-стоика: переняв у греческой философии одну-единственную идею о Тихе-Фортуне, латинские мыслители разрабатывали бы собственные, исконно римские учения. Но и этого мы не видим.

Наоборот, внимательное изучение тех изменений, которым подвергнулся культ Фортуны в эпоху Республики, при всём своеобразии этой богини, показывает, что в этой трансформации мы не можем найти чего-то из ряда вон выходящего.

% Появление военной функции

Так, в процессе развития культа Фортуны в республиканский период мы наблюдаем, как Фортуна начинает связываться с военным успехом римлян. Первое крупное святилище, посвящённое Женской Фортуне, было создано в память об отражении угрозы со стороны Кориолана. В дальнешем консул Карвилий употребляет часть военной добычи на возведение ещё одного храма Форс Фортуны (293 г. до н.э.), а римские полководцы обетуют на полях сражений храмы Фортуне Примигении (204 г. до н.э.), Всаднической (180 г. до н.э.) и Сегодняшнего Дня (101 г. до н.э.). Мы не должны находить ничего удивительного в том, что военному успеху римлян начинают покровительствовать божества, прежде не выполнявшие такой функции, ведь война имела огромное значение в жизни римской общины. Так, А.~И.~Немировский указывает, что со II в. до н.э. богиня Felicitas начинает восприниматься в смысле военного счастья, хотя первоначально не имела такого значения: в архаические времена прилагательное felix означало не <<счастливый>>, а <<благодатный>>\footcite[С. 153--154]{Nemirovsky1964}. Этот процесс мы можем признать закономерным.

% Огосударствление культа Фортуны

Также ничего удивительного мы не видим в том, что в период Республики произошло <<огосударствление>> культа Фортуны с появлением храма Фортуны Примигении Римского Народа (п.~\ref{FortunaPublicaPrimigenia}). Римляне имели обыкновение чтить богов покорённых народов, считая, что тем самым заслуживают их покровительство, при этом не допуская чужеземцев к отправлению культа Юпитера на Капитолии, за редким исключением\footcite[С. 315--316]{Boissier1914} "--- римские боги должны были покровительствовать одному только Риму. То, что столь популярная, чтимая и оттого могущественная богиня сделалась в конце концов также и покровительницей римской \textit{civitas}, также закономерно, и в эпоху Республики было только вопросом времени. Храм Фортуны Примигении был первым, обетованным этой богине на поле битвы (у нас нет данных, подтверждающих, что это не так), и время обетования (конец Второй Пунической войны) совпадает с тем, когда у слова \textit{felix} появляется значение военного счастья, в частности, того, которое сопутствовало Сципиону Африканскому Старшему\footcite[С. 153]{Nemirovsky1964}.

% Мы видим, что культовая практика со времён ранней Республики до императорского времени сохраняется в неизменности.


%То, что Фортуна, с одной стороны, соотносилась с материнскими божествами, а с другой "--- могла самым решительным образом определять судьбу человека, вовсе не означает, что она определяла судьбу именно в момент рождения.

%Ничто в мифах не говорит о том, что Фортуна определила судьбу Сервия Туллия именно в момент его рождения; наоборот, насколько нам известно, она интересовалась Сервием как уже созревшей личностью. Судьба Гальбы и Сеяна так же не была предопределена во время рождения, а немилость Фортуны к обоим деятелям была связана с их непочтительными действиями. Фортуна здесь выступает именно как сила, вносящая изменения в жизнь. При всей парадоксальности и противоречивости образа Фортуны, он не содержал в себе пары противоположностей <<предопределённость--непредсказуемость>>, ему была свойственна, в основном, непредсказуемость. Конечно, Фортуна, якобы, предсказала Гальбе высокую судьбу, но произошло это не в момент рождения.

% Тиха-Фортуна

Как мы показали в п.~\ref{Scriptores}, среди римлян, действительно, появляется представление о Фортуне как об олицетворении слепого случая и силы изменчивых обстоятельств. Это представление, действительно, является плодом влияния эллинистической культуры и образа богини Тихи. Однако оно находит отражение только в трудах античных писателей и не приводит к кардинальным изменениям в собственно культе Фортуны, который развивался, как мы показали, в русле развития всей римской религии, сохраняя своё самобытное и особенное положение. Таким образом, римские писатели, получившие образование в греческом духе (другого не было), а также собственно греки как раз и являлись агентами того эллинистического влияния, в результате которого возник образ \graeca{T'uqh}-\textit{Fortuna}. Этот образ возник закономерно в рамках восприятия римлянами греческой письменной культуры и греческих философских идей. Однако римский культ Фортуны и то представление об этой богине, которое было выражено в культе, не перевернулись вверх тормашками под влиянием греческой культуры. Оба процесса "--- перенятие эллинистической культуры и развитие римского культа "--- двигались параллельно в соответствии с собственными законами, оказывая друг на друга взаимное влияние, но не приводя к внезапным и необъяснимым изменениям. Непрерывный характер развития культа Фортуны в Риме и после эллинистического влияния подтверждается тем, что в позднереспубликанскую или раннеимперскую эпоху возникают культы Фортун отдельных фамилий (п.~\ref{FortunaeFamiliarum}): этот процесс находится в рамках выделенной нами тенденции к тому, чтобы Фортуна становилась покровительницей различных общественных групп.

Мы заключаем, таким образом, что культ Фортуны, с момента своего возникновения в Риме в позднюю царскую эпоху развивался в рамках общих тенденций развития римской религии и что он не являлся каким-то чужеродным элементом в её теле. Более того, римский культ Фортуны и римские представления об этой богине развивались в тесной связи с лацийскими и даже общеиталийскими тенденциями, хотя собственно в самом Риме культ обладал б\'{о}льшим разнообразием и яркими отличительными чертами. Отметим также, что мы не можем отыскать такое стороннее влияние, которому мы могли бы приписать определяющую роль в процессе формирования культа Фортуны в Риме. Это доказывает, что, несмотря на все те влияния, которые испытывал культ Фортуны со стороны, он сложился и развивался самобытно, на основе местных представлений и тенденций.

%\section{Аспекты образа Фортуны}

%Для удобства дальнейшего анализа выделим отдельные аспекты сложного и противоречивого образа Фортуны.

% Материнское божество

% Покровительница людей и общественных групп

% Фортуна-охранительница huius loci

% Непредсказуемость Фортуны (оракул в Пренесте)

% Военная функция

% Связь с торговлей

% Аграрная функция - плодородие

% Фортуна как богиня судьбы

% Отдельные абстрактные аспекты Фортуны (Bona, Mala, Salutaris, Viscata)

% Тиха-Фортуна

\section{Фортуна как обожествлённая абстракция}

Сравнивая римскую религию с религиями прочими, в первую очередь, греческой, мы видим, что римляне не стремились наделить своих богов яркой и запоминающейся индивидуальностью и сделать их образы как можно более очеловеченными. Это дало основание Гастону Буассье утверждать, что римские боги <<не имевшие ни истории, ни ясных образов, были не более как отвлечённые понятия>>\footcite[С. 45]{Boissier1914}. Римляне более других народов древности были склонны обожествлять отвлечённые понятия\footcite[С. 18]{Boissier1914}, такие, как Мужество (Virtus), Верность (Fides) и т.д. Попытаемся ответить на вопрос, можно ли отнести к этому ряду также и Фортуну.

Специально исследованиям обожествлённых абстракций посвящены две монографии, Г.~Экстелла\footfullcite{Axtell1907} и А.~Кларк\footfullcite{Clark2007}. Г. Экстелл, следуя классификации Варрона, относит к обожествлённым абстракциям, которые рассматривает в своей работе, во-первых, нравственные качества (virtutes), во-вторых, желаемые состояния (res expendendae)\footcite[P. 7]{Axtell1907}. Все эти божества являются умственными понятиями и не имеют материального выражения. К числу этих божеств он однозначно причисляет и Фортуну. А.~Кларк также относит Фортуну к числу обожествлённых абстракций, однако она не обращается к вопросу о происхождении культа Фортуны и особенностях представлений об этой богине у римлян.

% Фортуна у Экстелла и Кларк
% Рассматривая становление культов обожествлённых абстракций в Риме в хронологическом порядке, Экстелл на первое место ставит Фортуну, первые храмы которой возникли ещё при Сервии Туллии. Most probably she was a beneficient power of a good luck in the earliest stage. P. 9.

% Если Фортуна была изначально олицетворением только доброй удачи, то как возникли отрицательные коннотации её образа? Как появилась Fortuna mala? Если бы изначально сугубо положительные категории сознании римлян приобретали отрицательные значения, то тогда мы могли бы увидеть Felicitas Mala, Pietas Inpia, Virtus Ignava, Fides Infidelis и т.д., включая сюда, наверное, даже Bonus Eventus Malus.

% Фортуна, появившаяся в Риме, изначально была именно <<полноценной богиней>>, близкой к материнским божествам. Близость культов Фортуны на Бычьем форуме и Fortuna Muliebris к культу Матери Матуты показана нами выше. Представление о силе слепого случая, которое в античной литературе стало соотноситься с Фортуной в результате эллинистического влияния, носит амбивалентный характер. Для того, чтобы быть обожествлённой в качестве абстракции, это понятие в культовой практике должно было сопровождаться уточняющими атрибутами: так, Fortuna Bona мы можем соотнести с Fors Fortuna и Bonus Eventus, Fortuna Salutaris "--- с Salus. Эпиклезы Bona и Salutaris указывают на положительные аспекты многовалентного образа Фортуны. Эти отдельные аспекты уже могут считаться обожествлёнными абстракциями, подпадающими под расширенное и уточнённое нами определение Экстелла.

% Случай вряд ли можно представить себе как Salutaris, полезный для здоровья.
% Таким образом, даже в обожествлени отдельных аспектов Фортуны, выраженных множеством её когноменов, мы видим проявление именно римских представлений, а не образа \graeca{T'uqh}-\textit{Fortuna}, сложившимся под влиянием греческой культуры.

% Между различными аспектами Фортуны в представлениях римлян не было жёстких границ.

% Кларк: попыток исследовать генезис культа Фортуны в Риме она не предпринимает.

% Фортуна как понятие - её амбивалентность

% Фронтон и Ноний Марцелл - противопоставление Фортуны как богини и обожествлённых абстракций!
% У Плавта и на монете Калигулы Фортуна стоит в ряду обожествлённых абстракций
% Как так? Как толковать противоречивые свидетельства источников? Свидетельства источников противоречивы, но они отражают противоречивый образ богини в представлениях самих римлян.
% Как в самое раннее время, так и в позднюю эпоху Фортуна выступает в виде полноценной богини.

Г. Экстелл так описывает характеристики обожествлённых абстрактных божеств: <<\ldots{}scholarly and literary Romans felt strongly the difference between the deified abstracts and the other gods. They were to them, with certain exceptions, transparent projections of mental concepts without saga or personality, and were excluded from the rank of the chief gods, since their presence and potency were not so strongly felt. Their sex, which was in reality no more than the grammatical gender of their names, was often forgotten, and they were degraded mere personifications>>\footcite[P. 86]{Axtell1907}. Оговорка <<with certain exceptions>> относится, видимо, к Фортуне, про которую Экстелл пишет ниже, что она <<представляется почти столь же антропоморфной, как и любой из ``двенадцати богов''>>\footcite[P. 88]{Axtell1907}. Отметим, что, разобранные нами в п.~\ref{Myths} мифы о Фортуне позволяют нам утверждать, что слово <<почти>> в приведённой сентенции абсолютно лишнее. Про Фортуну были сложены популярные и красочные сказания, она обладала в мифах яркой индивидуальностью и отличалась от любого из <<двенадцати богов>> только тем, что не входила в их число, что свидетельствует лишь об определённой <<периферийности>> культа Фортуны, на которую мы указывали выше. Если, согласно Экстеллу, римляне и правда ясно видели разницу между <<полноценными>> богами и абстракциями, то римский мифологический образ этой богини недвусмысленно говорит о том, что отвлечённым понятием в их представлении она не была. Американский автор, относя Фортуну к категории божеств, для которых их пол "--- не более чем грамматический род имени, видимо, не учёл, насколько сильно заявляла о себе половая принадлежность этой богини в мифах: ведь она, желая узнать, каков царь Сервий как мужчина, влезала к нему в опочивальню прямо через окно.

Обрисованный нами яркий мифологический образ Фортуны и явственная антропоморфность её облика в представлении римлян являются веским аргументом против того, чтобы отнести её к числу обожествлённых абстракций.

Мы, однако, можем показать, что и само понятие, которое обозначается словом fortuna, не вписывается в множество тех категорий, каковые признавались римлянами за божества. Посмотрим на их (весьма неполный) список: римляне обожествляли такие сущности, как Concordia, Salus, Victoria, Ops, Spes, Fides, Libertas, Mens, Virtus, Pietas, Felicitas, Fecunditas etc. Все эти понятия являются положительными качествами в том смысле, что их присутствие оценивается позитивно в нравственном или практическом смысле. Случайность же может принести как хорошее, так и плохое, поэтому мы видим, что обожествлялась именно положительная сторона случая: Fors Fortuna, Bonus Eventus. Противоположные им качества (Discordia, Desperatio, Infidelitas, Inopia, Ignavia etc.) обладают отрицательным моральным смыслом. Римляне на практике не имели склонности обожествлять такие понятия. 

Итак, обожествляемые сущности не имели амбивалентного характера, они были однозначно положительными, обладая при этом отрицательными противоположностями, которые не являлись предметом культового почитания. Фортуна же несла в себе обе противоположности, она могла быть как хорошей (bona), так и дурной (mala). На это свойство богини Фортуны (в противоположность обожествлённым абстракциям) обращал внимание ещё бл. Августин, который писал: \textit{<<Зачем же существует богиня счастья (Felicitas)? У неё есть храм, жертвенник, и ей отправляется подобающий культ. \ldots{} Отчего, с одной стороны, есть Felicitas, а с другой "--- Фортуна? Ибо фортуна может быть и злой; счастье же не было бы счастьем, если бы было дурным>>}\footnote{Augustin. De Civ. Dei IV.18: \textit{Quid, quod et Felicitas dea est? Aedem accepit, aram meruit, sacra congrua persoluta sunt. \ldots{} An aliud est felicitas, aliud fortuna? Quia fortuna potest esse et mala; felicitas autem si mala fuerit, felicitas non erit}.}.

Конечно, мы можем предположить, вслед за Г. Экстеллом, что Фортуна изначально была олицетворением именно доброй удачи (<<was a beneficient power of a good luck in the earliest stage>>)\footcite[P. 9]{Axtell1907}, а её амбивалентный характер "--- следствие позднейшего эллинистического влияния представлений о Тихе. Однако здесь опять встаёт вопрос о закономерности такого влияния. Как именно возникли отрицательные коннотации образа Фортуны? Как появилась Fortuna Мala? Если бы изначально сугубо положительные категории в сознании римлян приобретали со временем отрицательные значения, то тогда мы могли бы увидеть Felicitas Mala, Pietas Inpia, Virtus Ignava, Fides Infidelis и т.д., включая сюда, наверное, даже Bonus Eventus Malus. Если же мы этого не видим, то тогда нам необходимо объяснить, чем же таким была примечательна Фортуна, что она столь сильно выделяется из ряда прочих обожествлённых абстракций, почему представление именно о ней сделалось амбивалентным, а о других понятиях "--- нет? Почему именно Фортуна в мифах выступает в роли грозной и карающей силы, почему никакое из благих абстрактных божеств, как кажется, не может причинить malum, в то время как на фреске, приведённой на рис.~\ref{pic:Cavemalum}, шуточная надпись предостерегает злоумышленника от зла, очевидно, со стороны изображённой там Фортуны? А так как объяснить это мы не можем, остаётся только предположить, что понятие фортуны было амбивалентным \textit{изначально}. Fortuna, очевидно, не является \textit{virtus}, двойственный же её характер не позволяет также отнести её к категории \textit{res expendendae}. Это является ещё одним основанием не считать Фортуну обожествлённой абстракцией.

Тем не менее, наши источники показывают, что Фортуна могла в определённых случаях соотноситься с обожествлёнными отвлечёнными понятиями. Так, Эргасил в комедии Плавта <<Пленники>> перечисляет Фортуну в ряду таких божеств, как Salus, Lux, Laetitia, Gaudium\footnote{\mancite{Plaut. Capt. 864}: \textit{idem ego sum Salus, Fortuna, Lux, Laetitia, Gaudium}.}, которые, безусловно, должны считаться абстрактными понятиями. Обратим также внимание на сестерций Калигулы, на котором изображены три его сестры с атрибутами женских божеств: Секуритас, Конкордии и Фортуны (RIC I, Cal. 33, рис.~\ref{pic:CaligulaeSorores}). Первые две богини также однозначно представляют собой абстрактные божества.

Существуют, однако, и противоположные свидетельства. Как мы указывали выше, Ноний Марцелл противопоставляет Фортуну и Форс, называя первую <<собственно богиней>> (\textit{dea ipsa}), а вторую "--- <<преходящим случаем>> (\textit{casus temporalis})\footnote{\mancite{Non. 425, 5}: \textit{Fors et Fortuna hoc distant. Fors est casus temporalis, Fortuna est dea ipsa}.}. Фронтон в письме к Марку Аврелию пишет: \textit{<<Кто же не знает, что разум (ratio) есть слово для обозначения человеческих суждений, Фортуна же богиня, и среди богинь выдающаяся? Фортуне всюду посвящены храмы и святилища, а Разуму (Ratio) нет нигде ни статуи, ни алтаря>>}\footnote{\mancite{Fronto Ad M. Caes. I.3.7, Nab. p. 8}: \textit{Quis autem ignorat rationem humani consilii vocabulum esse, Fortunam autem deam dearumque praecipuam? templa fana delubra passim Fortunae dicata, Rationi neс simulacrum neque aram usquam consecratam?}}. Разум (\textit{Ratio}), представляющий собой отвлечённое понятие, назван у Фронтона <<словом>> (\textit{vocabulum}), в то время как Фортуна "--- выдающаяся (\textit{praecipua}) богиня. В процитированном выше отрывке из <<Града Божьего>> Августин противопоставляет амбивалентную Фортуну однозначно положительному божеству счастья Felicitas.

% Фронтон в письме Марку Аврелию пишет: Фортуна же богиня, и среди богинь выдающаяся?

% Fronto Ad M. Caes. I.3.7, Nab. p. 5: Quis autem ignorat rationem humani consilii vocabulum esse, Fortunam autem deam dearumque praecipuam? templa fana delubra passim Fortunae dicata, Rationi neс simulacrum neque aram usquam consecratam ?

Итак, свидетельства источников противоречивы, и Фортуна в одних случаях может стоять в ряду абстрактных божеств, а в других "--- античные авторы противопоставляют её отвлечённым понятиям как <<полноценную>> богиню. Для объяснения этого, во-первых, подчеркнём, что, как мы уже показали, представление о Фортуне у римлян было многогранным и многоаспектным. \textit{Отдельные аспекты} образа Фортуны, такие, как Fortuna Bona или Fortuna Salutaris, \textit{можно рассматривать} как абстрактные понятия по аналогии с Fors Fortuna и Bonus Eventus, которые следует относить к категории \textit{res expetendae}. Когномены Фортуны уточняют её функцию и определяют сугубо положительную (Bona), отрицательную (Mala) или совершенно конкретную (Tutela Huius Loci) сторону многовалентного понятия Фортуны. Во-вторых, мы также показали выше, что, несмотря на многозначность слова fortuna и противоречивость его значений, античные авторы парадоксально легко и, кажется, часто незаметно для себя переходят от одного значения или оттенка к другому. Поэтому нам не следует ожидать, что в античных источниках мы \textit{всегда} будем видеть чёткое различение Фортуны <<вообще>> и отдельных аспектов образа этой богини.

Таким образом, Фортуна, взятая во всей полноте своего сложного и многоликого облика, является нам именно \textit{богиней}, и мы не можем утверждать, что под именем \textit{Fortuna} римляне обожествляли отвлечённое понятие. Отдельные аспекты её образа, однако, можно считать обожествлёнными абстракциями.

%(Цицерон осуждал греков за обожествление отрицательных качеств "--- вот откуда растут ноги у образа Тихи-Фортуны! Греки были не столь щепетильны в вопросах обожествления моральных категорий, как римляне)

%Немировский говорил об обожествлённых нравственных категориях. Тиха-Фортуна же, вообще говоря, категория безнравственная, или, вернее, вненравственная. Представляя собой олицетворение слепого случая, она несёт в себе не столько дуальность добра и зла, сколько безразличие к этим категориям. Римляне же, если говорить об обожествлённых абстракциях, поклонялись доброму случаю, положительной стороне непредсказуемости: у них были божества Fors Fortuna и Bonus Eventus.

%Возможно, определённое недоразумение в толкование, которое дают Фортуне современные учёные, вносит тот факт, что само слово <<фортуна>> проникло в современные европейские языки, однако обозначает в них нечто однозначно положительное: удачу, счастливый случай, счастье. Так, в английском языке, на котором не только писал, но и мыслил Г. Экстелл, существует, помимо слова \textit{fortune}, означающего удачу и счастье, такое довольно употребительное понятие как \textit{misfortune} "--- неудача, несчастье, беда. Однако, как мы можем видеть, римское представление о Фортуне включало в себя оба значения, фортуна "--- это одновременно и \textit{fortune}, и \textit{misfortune}.

%В вопросе же о том, относить или нет Фортуну к обожествлённым понятиям, очевидно, следует исходить не из современного, а из древнеримского понятия о ней. \textit{Fortuna} у римлян не обозначала такого же положительного понятия, как \textit{salus}, \textit{virtus}, \textit{felicitas}, \textit{mens} etc. Фортуна заключает в себе обе противоположности; она не может присутствовать или отсутствовать: она всегда есть, но может быть благой или дурной. Это представление о Фортуне, как мы показали, следует считать изначальным.

\section{Синтез образа Фортуны}

Проанализировав различные аспекты культовой практики и античных представлений о Фортуне, попытаемся охарактеризовать эту богиню в целом. Фортуна, как следует из вышеописанного, была многоликой и разносторонней богиней, но можем ли мы подвести всё это разнообразие под общий знаменатель?

Г. Экстелл указывает\footcite[P. 86]{Axtell1907}, что римляне вполне свободно использовали имена богов в метонимическом значении: так, например, \textit{sub Iove} означает <<под небом>>. Fortuna "--- не только имя богини, но и понятие, смысл которого кажется нам очевидным, поэтому возникает соблазн так же перенести значение по метонимии и, как, например, делает Р.~Петер, заявить про Фортуну: <<wie ihr Name sagt, die G\"{o}ttin des Zufalls>>\footcite[Sp. 1503]{Peter1890Fortuna}. Действительно, само слово <<фортуна>> проникло из латыни в современные языки, и может создаться впечатление, что, когда римляне говорят о Фортуне, то мы понимаем, о чём идёт речь, ибо имеется в виду одно и то же: не будем же мы прибегать к тавтологии и утверждать, что Фортуна была богиней фортуны. Но для нас сегодня <<фортуна>> "--- это удача, счастливый случай, успех, счастье, то есть нечто однозначно положительное. В настоящем же исследовании мы показали, что римская Фортуна обладала двойственным характером, она могла быть как хорошей (Fortuna Bona), так и плохой (Fortuna Mala). Более того, современная трактовка понятия <<фортуна>> не поможет нам объяснить, во-первых, разнообразие форм культа этой богини, и, во-вторых, то, что при такой изменчивости она всё же сохраняла свою идентичность и не растворялась в каком-нибудь другом божестве.

Так богиней чего же была Фортуна? Успеха, удачи, случая, счастья? Нам представляется, что последнее слово наилучшим образом может охарактеризовать Фортуну и свести воедино все её аспекты: как богиню-покровительницу и грозную силу, способную покарать; как олицетворение случая и материнское божество, помогающее при рождении ребёнка; и, в конце концов, как богиню-пророчицу, ведающую судьбой человека.

Действительно, обратим внимание на такой аспект человеческого счастья, как его краткость и недолговечность. Также ощущение счастья тесным образом связано с переживанием несчастья: наиболее глубоко переживает счастье как раз тот, кто испытал много невзгод. Счастье неразрывным образом диалектически связано со своей противоположностью, несчастьем, и обе эти противоположности мы находим включёнными в обрисованное нами древнеримское представление о богине Фортуне.

Удача может принести счастье, но слово <<удача>> в нашем понимании соотносится с совокупностью внешних обстоятельств. В этом отношении слово <<удача>> более применимо к греческой Тихе\footnote{Тиха, как и Фортуна, имела амбивалентный характер и могла означать как счастливый, так и несчастливый случай. К.~Циглер производит этимологию слова \graecafn{t'uqh} от глагола \graecafn{tugq'anw} (<<случаться>>, <<приключаться>>, <<выпадать на долю>>). См.: \cite[Sp. 1643--1644]{TycheZiegler1948}.}, а в случае с Фортуной мы хотели бы подчеркнуть более <<интровертный>> характер римской богини, соответствующий духу римской религии вообще "--- дух этот выражается в большей ориентированности римлян на внутреннюю сторону жизни, например, в том, что сфера нравственности и человеческих чувств наполнялась римлянами особым религиозным смыслом, о чём говорит обожествление ими таких моральных категорий, как Доблесть (Virtus), Верность (Fides), Почтение (Pietas), Счастье (Felicitas) и др.

Фортуна, как мы показали, соотносилась с материнским божествами: вряд ли можно рассматривать рождение ребёнка как просто удачу; но оно всегда является счастьем. Во время праздненства Мужской Фортуны женщины молились богине о счастливой семейной жизни (вернее, об одной из сторон семейной жизни). Девушки приносили свои одежды Фортуне-Деве при выходе замуж, а мужчины почитали Фортуну Барбату: возмужание и выход замуж есть счастье, однако юношей и девушек впереди ждёт трудная самостоятельная жизнь. Военный успех есть не просто победа, но счастье, добытое через превратности, тяготы и несчастья войны. Как богиня судьбы, Фортуна должна была олицетворять не столько высокую или низкую судьбу, не столько жизнь удачную или неудачную, сколько счастливую или несчастливую участь, долю.

Ничто не говорит о том, что почитание Фортуны как покровительницы отдельных общественных групп или конкретных мест имело целью снискание удачи; трудно представить себе, что означает удача для римских матрон или римских всадников вообще. Что такое удача или просто случай для отдельной фамилии? Но благополучие, процветание общественных групп и фамилий, то есть, в конечном итоге, их счастье, их счастливая судьба, которую они сообща разделяют "--- вот что можно просить у божества.

Краткость и недолговечность переживания счастья позволяют объяснить, почему Фортуна сблизилась с эллинистической богиней случая и непостоянства Тихой "--- ведь тот факт, что Тиха была ассоциирована именно с Фортуной, а не с каким-либо другим женским божеством, тоже требует своего объяснения.

То, что Фортуна являлась именно \textit{богиней}, отличает её, как мы показали, от божества счастья Felicitas. Будучи <<полноценной>> богиней, Фортуна несёт в себе \textit{амбивалетность сакрального}\footcite[С. 29--33]{Eliade1999}, заключающуюся в том, что всё, что относится к миру горнему, к уровню, превышающему человеческий, является, с одной стороны, благим, но в то же время и опасным: таковыми были в представлении древних все могущественные боги.

Итак, если мы хотим дать как можно более краткую и точную характеристику образа Фортуны, то для известного нам исторического периода от времени Сервия Туллия до начала Империи мы могли бы назвать её \textit{богиней счастья-и-несчастья}.

