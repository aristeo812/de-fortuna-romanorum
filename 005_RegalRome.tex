\chapter{Фортуна в царскую эпоху\label{RegalRome}}

% Самые ранние сведения о храмах в честь богини Фортуны античная традиция относит ко времени царя Сервия Туллия.

% Гипотезы о происхождении культа Фортуны
% Раскрыть этот пункт. Надо читать Вард Фаулера по этому поводу.

% Происхождение Фортуны от сабинян - версия Варрона

% Храмы Фортуны, которые от Сервия Туллия: разные версии.

%%%%%%%%%%%%%%%%%%%%%%%%%%%%%%%%%%%%%%%%%%%%%%%%%%%%%%%%%%%%%%%%%%%%%%%%%%%%%
%% Небольшое введение
%%%%%%%%%%%%%%%%%%%%%%%%%%%%%%%%%%%%%%%%%%%%%%%%%%%%%%%%%%%%%%%%%%%%%%%%%%%%%

% Дионисий Галикарнасский: построили два храма, на Бычьем форуме и храм Мужской Фортуны (см. тж. Овидия про это)
Античные авторы почти единодушно приписывают сооружение первых храмов Фортуны в Риме Сервию Туллию. Дионисий Галикарнасский (Dion Hal. IV.27.7) указывает, что этот царь построил два храма Фортуны\footnote{\graecafn{nao`us d'uo kataskeuas'amenos T'uqhs} (Dion. Hal. IV.27.7), "--- пишет Дионисий о Сервии.}, один "--- на Бычьем форуме\footnote{\graecafn{t`on m`en >en >agor~a| t~h| kaloum'enh| Boar'ia|}, Ibid.}, другой же "--- храм Мужской Фортуны на берегах Тибра\label{RipaTiberis}\footnote{\graecafn{t`on d> <'eteron >ep`i >hi'osi to~u Teb'erios, <'hn >andre'ian proshg'oreusen, <ws ka`i n~un <up`o <Rwma'iwn kale~itai}, Ibid.}.

%%%%%%%%%%%%%%%%%%%%%%%%%%%%%%%%%%%%%%%%%%%%%%%%%%%%%%%%%%%%%%%%%%%%%%%%%%%%%
%% Храм Фортуны на Бычьем Форуме
%%%%%%%%%%%%%%%%%%%%%%%%%%%%%%%%%%%%%%%%%%%%%%%%%%%%%%%%%%%%%%%%%%%%%%%%%%%%%

Первый из этих двух храмов, упомянутых Дионисием, однозначно приписывается античной традицией Сервию Туллию: это храм Фортуны рядом со святилищем Матери Матуты. Обратим внимание на эту богиню, ибо соседство с нею позволит нам пролить свет и на характер культа Фортуны. Храм Матери Матуты был также освящён Сервием Туллием, причём в тот же день, что и святилище Фортуны (Liv. V.19.6). Имя её происходит от слова maturus <<зрелый>>\footnote{Latte,~K. Op. cit. S. 88.}, и, добавив к этому, что Плутарх отождествляет её с Левкофеей, или Ино\footnote{См. Warde Fowler, W. Op. cit. P. 154; Latte, K. Op. cit. S. 97; А.~И.~Немировский также указывает, что такое отождествление характерно для римских антикваров конца Республики. См. Немировский~А.~И. Указ. соч. С. 75--76.}, (Plut. Camill. 5), мы можем нарисовать себе образ довольно архаического женского божества, связанного с материнством и деторождением, покровительницы матрон\footnote{Warde Fowler, W. Op. cit. P. 155.}. Два храма этих женских божеств, Матери Матуты и Фортуны, были построены весьма близко друг от друга, так что, например, у Ливия они нередко упоминаются вместе, когда ему нужно указать конкретное местоположение на карте древнего Рима (см. Liv. XXIV.47.15, XXV.7.6, XXXIII.27.4). Такое соотнесение двух богинь, которое мы находим в трудах античных авторов, вряд ли можно считать случайностью: их культы были безусловно близки друг другу\footnote{Шайд~Дж. Религия римлян. С. 161.}. Отметим, что согласно сообщению Овидия (Ovid. Fast. VI.621--622), в храме Фортуны на Бычьем форуме молились \textit{матроны}, что также характеризует этот культ как женский. Оба этих храма играли важную роль на протяжении всего Республиканского периода.

В святилище Фортуны находилась позолоченная деревянная статуя Сервия\footnote{Е.~М.~Штаерман видит в этом факте <<туманный намёк на обожествление Сервия Туллия>>. См. Штаерман~Е.~М. Указ. соч. С. 47--48.} (Dion. Hal. IV.40.7), голова которой была покрыта тогой (Ovid. Fast. VI.570--572, 623--624). Впрочем, Плиний Старший пишет, что в храме стояло изображение не самого Сервия, а всё же Фортуны, посвящённое богине царём и покрытое царской претекстой, которая сохранялась до смерти Сеяна, будучи на протяжении 560 лет удивительным образом не подвержена ни тлению, ни вреду от моли\footnote{\textit{Servi Tulli praetextae quibus signum Fortunae ab eo dicatae coopertum erat, duravere ad Seiani exitum, mirumque fuit neque diffluxisse eas neque teredinum iniurias sensisse annis quingentis sexaginta}, Plin. NH VIII.197.} (Plin. NH VIII.197). Надо думать, именно об этой тоге чуть ранее говорит Плиний, ссылаясь на Варрона, что-де её, тогу, изготовила Танаквиль\footnote{\textit{\ldots{}~durasse prodente se auctor est M.~Varro, factamque ab ea [Tanaquile "--- В.~Ж.] togam regiam undulatam in aede Fortunae, qua Ser. Tullius fuerat usus}, Ibid. 194.} (Ibid. 194). Это покрывало было настолько священным, что матроны, отправляющие службу в храме, не могли к нему прикасаться и должны были молиться на расстоянии\footnote{\textit{parcite, matronae, vetitas attingere vestes / (sollemni satis est voce movere preces)}, Ovid. Fast. VI.621--622.} (Ovid. Fast. VI.621--622). Противоречивые мнения древних авторов о том, чьё именно изображение стояло в храме, объясняются, быть может, тем, что за давностью лет покрытая тогой статуя Фортуны стала восприниматься как статуя Сервия. Впрочем, трудно найти неопровежимые аргументы как за, так и против этой гипотезы\footnote{Немировский~А.~И. Указ. соч. С. 72.}.

Сам храм вместе со святилищем Матери Матуты сгорел во время пожара 213 г. до н.э. (Liv. XXIV.47.15) и был восстановлен в следующем году (Liv. XXV.7.6), однако изображение Сервия (или Фортуны) чудесным образом уцелело (Ovid. Fast. VI.625, Val. Max. 1.8.11). Дионисий, который должен был видеть эту статую собственными глазами, подчёркивает, что хотя сам восстановленный храм и <<принадлежит искусству новых мастеров>>\footnote{\graecafn{t~hs kain~hs >esti t'eqnhs}, Dion. Hal. IV.40.7.} (Dion. Hal. IV.40.7), <<статуя, как и прежде, убеждает в том, что она является древним произведением>>\footnote{\graecafn{<h d> >eik'wn, <'oia pr'oteron >'hn, >arqaik`h t`hn kataskeu'hn}, Ibid.} (Ibid.). Он сообщает далее, что это изображение и в его время пользуется поклонением у римлян.

% Про то, какие мифы сообщает нам Овидий про эту статую
О том важном значении, которое придавалось храму Фортуны на Бычьем форуме, и о поклонении римлян перед покрытой статуей Сервия (или Фортуны) говорят и легенды, связанные с нею, каковые сообщает нам Овидий в своих <<Фастах>>. Он пишет, что неясно, почему статуя в храме покрыта тогой\footnote{\textit{\ldots{}causa latendi / discrepat, et dubium me quoque mentis habet}, Ovid. Fast. VI.571--572.} (Fast. VI.571--572), однако приводит несколько версий, объясняющих, отчего это так. Согласно первой из них (Ibid. 573--580), Фортуна вступала в любовную связь с Сервием, и тот скрыл лицо тогой от стыда (\textit{nunc pudet}, Ibid. 579). Богиня, пишет Овидий, проникала в свой дом через окошко\footnote{\textit{nocte domum parva solita est intrare fenestra}, Ibid. 577.} (Ibid. 577), и поэтому-то есть в Риме ворота, которые зовутся Fenestella (Ibid. 578). Плутарх также пишет о Сервии, что <<Удача с ним разделяла ложе, входя к нему в опочивальню через какое-то окно, которое сейчас называют воротами FENESTELLA>>\footnote{\graecafn{sune~inai doke~in a>ut~w| t`hn T'uqhn di'a tinos jur'idos kataba'inousan e>is t`o dwm'ation n~un Fen'estellan p'ulhn kalo~usin}, Plut. De Fort. Rom. 10, Mor. 322 E--F.} (Plut. De Fort. Rom. 10, Mor. 322 E--F). Обратим внимание, что здесь греческий автор несколько неточен в передаче фактов: воротами Фенестеллой звалось не окошко в опочивальне, а, согласно Овидию, которому мы должны в этом вопросе верить больше, действительно некие ворота в Риме. Плутарх затрагивает эту же тему и в трактате <<Quaestiones Romanae>>: в <<вопросе>> 36 он спрашивает, <<Почему одни из ворот Рима называются ``Окно'', а соседнее здание именуется ``Спальней Фортуны''?>>\footnote{\graecafn{di`a t'i p'ulhn m'ian jur'ida kalo~usi, t`hn g`ar fen'estran to~uto shma'inei, ka`i par> a>ut`hn <o kaloum'enos T'uqhs j'alam'os >esti}; Plut. Quae. Rom. 36, Mor. 273~B.} (Mor. 273~B). Одно из толкований, которые даёт греческий автор, звучит так: <<по преданию, царь Сервий, бывший счастливейшим, будто бы сочетался с Фортуной, входившей к нему через окно>>\footnote{\graecafn{S'erbios <o basile`us eutuq'estatos gen'omenos d'oxan >'esqe, t~h| T'uqh| sune~inai foit'wsh| di`a jur'idos pros a>ut'on}, Ibid. 273~B--C.} (Mor. 273~B--C).

Вернёмся, однако, к <<Фастам>> Овидия и к его толкованиям того факта, что голова священного изображения в храме Фортуны была покрыта тогой. Согласно второй версии, излагаемой римским поэтом (Ovid. Fast. VI.581--584), народ был так опечален смертью любимого им царя, что люди рыдали перед статуей, не переставая, пока она не оказалась покрыта тогой\footnote{\textit{donec eum positis occuluere togis}, Ovid. Fast. VI.574.}, "--- очевидно, произошло это чудесным образом "--- как это было принято, видимо, в античности у умирающих\footnote{Согласно Платону, лицо Сократа после смерти было закрыто (Plat. Phaed. 118~A); Цезарь так же, как сообщает Светоний, умирая, накинул тогу на голову (Suet. Div. Jul. 82.2).}. По третьей версии (Ibid. 611--621), Туллия, дочь царя, вероломно убив его, посмела даже войти в храм, освящённый Сервием, где стояло его же изображение, и вот тогда, сообщает Овидий, статуя заговорила:

\begin{verse}
\textit{dicitur hoc oculis opposuisse manum,\\
et vox audita est `voltus abscondite nostros,\\
ne natae videant ora nefanda meae.'}\footnote{Ovid. Fast.VI.614--616.}
\end{verse}

(Говорят, что он закрыл глаза рукой, / И прозвучал голос: очи скройте мои, / Чтобы они не видели ненавистного чела моей дочери.)

% Растолковать значение этих легенд, подчеркнуть значение их красочности, при необходимости сослаться на Розе
В этих сказаниях Фортуна предстаёт перед нами в олицетворённом и даже весьма очеловеченном виде, что довольно редко у римлян: не так много в римской мифологии можно найти историй о любовной связи богов со смертными, хотя с говорящими статуями нам ещё придётся столкнуться в будущем. Необычность и живость подобных легенд, ходивших в народе, обращает на себя внимание и раскрывает перед нами высокую значимость храма Фортуны на Бычьем форуме.


% Боже, Боже, помоги мне, пожалуйста, писать это! Я же смогу написать, правда? Я смогу написать!

%Святилища Фортуны и Матери Матуты были уничтожены в огне пожара в 213 г. до н.э. (Liv. XXIV.47.15) и восстановлены в следующем году (Liv. XXV.7.6). Интересно, что в этом храме находилось культовое изображение, которое сами древние называли статуей Сервия Туллия "--- впрочем, голова этой статуи была покрыта тогой, так что нельзя было даже разобрать пол изваянного существа. Описание этой стату мы находим у Дионисия 

%Дионисий пишет про Сервия (Dion. Hal. IV.40.7), что <<в храме богини Фортуны, который он сам построил, находилась его деревянная позолоченная статуя>> (\graeca{`en g`ar t~w| na~w| t~hs T'hqhs, <`on a>ut`os kateske'uasen, e>ik`wn a>uto~u keim'enh xul'inh kat'aqrusos >empr'hsews genom'enhs}). Об этом же есть многочисленные свидетельства и других авторов. Так, Овидий сообщает: \texit{sed superiniectis quis latet iste togis? / Servius est, hoc constat enim: sed causa latendi / discrepat, et dubium me quoque mentis habet}.


%%%%%%%%%%%%%%%%%%%%%%%%%%%%%%%%%%%%%%%%%%%%%%%%%%%%%%%%%%%%%%%%%%%%%%%%%%%%%
%% Храм Fortunae Virilis
%%%%%%%%%%%%%%%%%%%%%%%%%%%%%%%%%%%%%%%%%%%%%%%%%%%%%%%%%%%%%%%%%%%%%%%%%%%%%

% Сделать переход к этому месту

Как мы уже писали выше, помимо этого храма, построенного рядом со святилищем Матери Матуты, Дионисий Галикарнасский приписывает Сервию строительство храма Мужской Фортуны на берегах Тибра (Dion. Hal. IV.27.7). Г. Виссова довольно категорично утверждает, что Дионисий это делает ошибочно\footnote{Wissowa, G. Religion und Kultus der R\"{o}mer. M\"{u}nchen, 1902. S. 206.}. Постараемся выяснить, насколько это верно. К счастью, нам известно об этом храме и из других источников, что позволяет обрисовать его историю более полно. Святилище Мужской Фортуны (Fortunae Virilis) существовало ещё во времена поздней Республики и пользовалось тогда большим почтением у римлян, о чём нам сообщает Овидий. Он так описывает праздник Мужской Фортуны, приходившийся на 1 апреля:

\begin{verse}
\textit{discite nunc, quare Fortunae tura Virili\\
detis eo, calida qui locus umet aqua.\\
accipit ille locus posito velamine cunctas\\
et vitium nudi corporis omne videt;\\
ut tegat hoc celetque viros, Fortuna Virilis\\
praestat et hoc parvo ture rogata facit.}\footnote{Ovid. Fast. IV.145--150.}
\end{verse}

(Знайте теперь, почему Мужской Фортуне ладан / Подн\'{о}сите там, где тёплая влага течёт. / Принимает это место всех, сбросивших одежды, / И каждый изъян обнажённого тела видит. / Чтобы скрыть это от мужей, Фортуна Virilis, / умоленная при помощи кусочка ладана, это исправляет.)

Это сообщение недвусмысленно обрисовывает Мужскую Фортуну как типично женское божество, которое заведовало интимной стороной супружеских отношений.

% Плутарх: разобрать как следует. Те храмы, которые он приписывает С.Т. Те храмы, которые он приписывает Анку Марцию.
Упоминание этого "--- или подобного "--- храма мы находим и у Плутарха. Он, однако, единственный из древних авторов указывает, что <<первым соорудил святилище удачи Марций Анк>>\footnote{\graecafn{pr~wtos m`en g`ar <idr'usato T'uqhs <ier`on M'arkios >'Agkos}, Plut. De Fort. Rom. 5, Mor. 318 E.} (Plut. De Fort. Rom. 5, Mor. 318~E) и добавляет далее, что царь Анк <<дал Удаче имя Мужества>>\footnote{\graecafn{t~h| t'uqh| >andre'ian parwn'omasen}, Ibid. 318~F.} (Ibid. 318~F). В другом месте (Quae. Rom. 74, Mor. 281~E) Плутарх, однако, называет храм <<Фортуны Мужей>> (\graeca{T'uqhs \ldots >'arrenos \ldots <ier'on}) в числе тех, которые построил Сервий Туллий. Неясно, подразумевал ли греческий автор под этим святилищем и под тем, которому <<дано имя мужества>>, один и тот же храм или разные, также как неизвестным, видимо, останется то, откуда он взял сведения о возникновении культа Фортуны при Анке Марции. Однако упоминание царя Анка мы находим в трактате <<Об удаче римлян>>, а о создании Сервием Туллием храма \graeca{T'uqhs >'arrenos} Плутарх пишет в <<Римских вопросах>>, произведении более позднем, которое он создавал, очевидно, обладая большей эрудицией и прибегая к большему числу источников, о чём говорит хотя бы широта охвата тех вопросов, которые он там затрагивает. Из этих неясных и противоречивых фактов, сообщаемых нам греческим историком, мы можем заключить лишь, что он почерпнул из своих источников сведения о храме Фортуны, созданном ещё в царскую эпоху, который в греческом переводе назывался \graeca{T'uqhs >'arrenos <ier'on} или носил имя \graeca{>andre'ia}.

%К счастью, нам кое-что известно о ритуале, связанном с этим храмом. 

% Вставить сюда ссылки на литературу (Немировский и New Pauly) про то, что первые храмы Фортуны были посвящены женским божествам, что это были божества типично женские

Вероятно, святилище Мужской Фортуны было настолько же древним, как и храм Фортуны на Бычьем Форуме, скорее всего, его можно отнести к той же эпохе царского Рима или начала Республики, ибо эти два культа близки друг другу по своему архаическому облику, обе Фортуны схожи с женскими божествами плодородия, материнства и супружеских отношений. Древность этого святилища и, возможно, его меньшая значимость нежели храма Фортуны на Бычьем форуме, скорее всего, не позволили донести до позднереспубликанских времён более-менее внятные рассказы о его основании и происхождении: Овидий молчит об истории этого святилища, сообщения Плутарха туманны и противоречивы, да и Дионисий, видимо, ошибся с тем, что расположил этот храм на берегах Тибра.

%%%%%%%%%%%%%%%%%%%%%%%%%%%%%%%%%%%%%%%%%%%%%%%%%%%%%%%%%%%%%%%%%%%%%%%%%%%%%
%% Храм Fortis Fortunae
%%%%%%%%%%%%%%%%%%%%%%%%%%%%%%%%%%%%%%%%%%%%%%%%%%%%%%%%%%%%%%%%%%%%%%%%%%%%%
% --== Fanum Fortis Fortunae ==--
% Варрон: Сервий построил храм Fors Fortuna

Действительно, другие античные авторы указывают, что Сервий на берегах Тибра построил храм богини Счастливого Случая (fanum Fortis Fortunae). Так, Варрон пишет (De ling. Lat. VI.17): <<День [богини] Счастливого Случая провозглашён царём Сервием Туллием, который в месяце июне ей посвятил храм у Тибра за городом>>\footnote{\textit{Dies Fortis Fortunae appellatus ab Servio Tullio rege, quod is fanum Fortis Fortunae secundum Tiberim extra urbem Romam dedicavit Iunio mense}, Varr. De ling. Lat. VI.17.}. Тит Ливий также указывает (Liv. X.46.14), что Сервий Туллий освятил храм этой богини (согласно Ливию, консул 238 г. до н.э. Карвилий вложил деньги из военной добычи в строительство храма Счастливого Случая рядом с храмом той же богини, освящённым Сервием Туллием\footnote{\textit{aere aedem Fortis Fortunae de manubiis faciendam locavit prope aedem eius deae ab rege Ser. Tullio dedicatam}, Liv. X.46.14.}). Dies Fortis Fortunae, о котором пишет Варрон, приходится, согласно Овидию, на 24 июня (Ovid. Fast. VI.773--784). Овидий пишет о храме этой богини, что он расположен на берегу Тибра\footnote{\textit{in Tiberis ripa munera regis habet}, Ovid. Fast. VI.776.} (Ibid. 776), в восторженных тонах описывает народные гуляния, которые проходили в тот день и сопровождались попойкой:

\begin{verse}
\textit{pars pede, pars etiam celeri decurrite cumba,\\
nec pudeat potos inde redire domum.\\
ferte coronatae iuvenum convivia, lintres,\\
multaque per medias vina bibantur aquas.}\footnote{Ovid. Fast. VI.777--780.}
\end{verse}

(Кто пешком, кто даже в лодках, быстро сбирайтесь / Чтобы не стыдиться хмельными после вернуться домой. / Несите пиршества юношей, о украшенные цветами лодки, / И многие вина выпиваются на середине реки). Согласно римскому поэту, в гуляниях участвовал плебс (\textit{plebs colit hanc}, Ibid. 781) и даже рабы (\textit{convenit et servis}, Ibid. 783). Это поэтическое описание говорит нам о том, что праздник, связанный с <<ненадёжной богиней>> (\textit{dubia dea}, Ibid. 784) был поистине всенародным, т.е. не связанным с социальными полисными институтами, что, как мы уже отмечали, согласно Дж. Шайду было характерно для римской религии\footnote{См. с.~\pageref{ReligionAndPolice} и примеч.~\ref{ReligionAndPoliceFootnote}}. Культ же Fortis Fortunae был более широким, в нём могли принимать участие все, невзирая на своё социальное положение, вплоть до рабов. Эта черта достойна особого внимания, т.к. выделяет не только культ этого конкретного храма, но и богиню счастья и удачи в целом. Свой счастливый случай может быть даже и у раба, свидетельством чему была судьба основателя храма Fortis Fortunae царя Сервия. Мы, однако, не можем судить, были ли такие черты присущи культу этого святилища изначально, со времён царского Рима, или же описанные Овидием обычаи сложились позднее.

Храм этой богини был тесно связан с Фортуной, на что указывает не только её имя, но и тот факт, что в нём имелось изображение Фортуны (Согласно Ливию, XXVII.11.3: <<в Риме в приделе храма Счастливого Случая фигурка из венца Фортуны вдруг сама упала ей на руку>>\footnote{\textit{Romae intus in cella aedis Fortis Fortunae de capite signum quod in corona erat in manum sponte sua prolapsum}, Liv. XXVII.11.3.}). 

%%%%%%%%%%%%%%%%%%%%%%%%%%%%%%%%%%%%%%%%%%%%%%%%%%%%%%%%%%%%%%%%%%%%%%%%%%%%%
%% Разобрать то, что написано о храмах Фортуны у Плутарха
%%%%%%%%%%%%%%%%%%%%%%%%%%%%%%%%%%%%%%%%%%%%%%%%%%%%%%%%%%%%%%%%%%%%%%%%%%%%%

% Римские вопросы Плутарха
У Плутарха, однако, можно прочесть, что Сервий Туллий создал гораздо больше храмов богини удачи\label{PlutarchosDeServio}. Он пишет в трактате <<Об удаче римлян>> (De Fort. Rom. 10, Mor. 322~F), что царь Сервий построил храм Фортуны Первородной (Primigenia) на Капитолии\footnote{\graecafn{<idr'usato d> o>~un T'uqhs <ier`on >en m`en Kapetwl'iw| t`o t~hs Primigene'ias legom'enhs, <`o prwtog'onou tis >a`n <hrmhne'useie}, Plut. De Fort. Rom. 10, Mor. 322 F.} и Фортуны Послушной или Милостивой (Obsequentis)\footnote{\graecafn{ka`i t`o t~hs >Oyekou'entis, <`hn o<i m`en peij'hnion, o<i d`e meil'iqion e>'inai nom'izousi}, Ibid.}. В <<Римских вопросах>> он приписывает этому царю создание девяти храмов Фортуны (\graeca{T'uqhs}) в Риме (Plut. Quae. Rom. 74, Mor. 281~D--E): Малой Фортуны (\graeca{mikr~as T'uqhs}, для неё же он приводит транслитерированное латинское название "--- \graeca{br'ebem}, что должно означать \textit{Brevis}), Подательницы надежд (\graeca{e>u'elpidos}), Отвратительницы бед (\graeca{>apotropa'iou}), Милостивой (\graeca{meiliq'ias}), Первородной (\graeca{prwtogene'ias}), Фортуны Мужей (\graeca{>'arrenos}), Собственной (\graeca{>id'ias}), Оборачивающейся (\graeca{epistrefom'enhs}) и Фортуны-Девы (\graeca{parj'enou}). Мы видим, что в более позднем сочинении Плутарх расширил список, добавив к храмам Фортуны Первородной и Послушной ещё семь святилищ. Однако насколько мы можем доверять этому сообщению? Посмотрим, какие ещё сведения мы можем найти в других источниках.

Про \textit{алтарь} Фортуны "--- Подательницы надежд (\graeca{T'uqhs bwm`os E>u'elpidos}) пишет сам же Плутарх в сочинении <<Об удаче римлян>>, указывая, что он находился в Длинном переулке (\graeca{>en t~w| makr~w| stenwp~w|}) (Plut. De Fort. Rom. 10, Mor. 322 F). Каким образом соотносятся храм Фортуны и её алтарь с одним и тем же названием, о которых пишет Плутарх в разных местах, неясно. Из этого сообщения Плутарха мы можем только заключить, что в Риме, скорее всего, существовал храм или алтарь Фортуны с эпиклезой, которая переводилась на греческий как \graeca{e>u'elpis}.

% ἐν δὲ τῷ μακρῷ στενωπῷ Τύχης βωμὸς Εὐέλπιδος (Plut. De Fort. Rom. 10, Mor. 322 F)

Храм Фортуны Первородной был построен в Риме не на Капитолии, а на Квиринале, и не Сервием Туллием, а Публием Семпронием, который, будучи консулом, в 204 г. до н.э. и предводительствуя войсками в битве против Ганнибала в Кротонской области, обетовал этот храм (Liv. XXIX.36.8-9), а спустя 10 лет, в 194 г. до н.э., освятил его (Liv. XXXIV.36.8-9). Разумеется, в этом случае мы должны верить точному сообщению Ливия, а не туманному повестованию Плутарха. Подробнее об этом храме мы будем говорить в гл.~\ref{ChapterEarlyRepublic}.

Про Фортуну Собственную пишет сам Плутарх в сочинении <<Об удаче римлян>>, указывая, что её храм находился на Палатине\footnote{\graecafn{>id'ias T'uqhs <ier'on >estin >en Palat'iw|}, Plut. De Fort. Rom. 10, Mor. 322~F.} (Plut. De Fort. Rom. 10, Mor. 322~F). Однако из других источников ничего не известно о таком храме в самом центре Рима, и, скорее всего, следует признать, что это сообщение Плутарха ошибочно.

Фортуну Оборачивающуюся\label{FortunaRescipiens} упоминает Цицерон (De leg. II.28), объясняя такую её эпиклезу тем, что она оборачивается для того, чтобы оказать помощь: \textit{Rescipiens ad opem ferendam}. В трактате <<Об удаче римлян>> Плутарх пишет, что этот храм находился на Эсквилине\footnote{\graecafn{T'uqhs <ier'on >esti ka`i >en A>iskul'iais >epistrefom'enhs}, Ibid. 323~A.} (Plut. De Fort. Rom. 10, Mor. 323~A). О времени основания этого храма мы больше ничего не можем узнать из письменных источников.

Арнобий, христианский автор III--IV вв. н.э. пишет, что девушки приносили Девственной Фортуне свои одежды\footnote{\textit{puellarum togulas Fortunam defertis ad Virginalem}, Arnob. II.67.3.} (II.67.3). Следует полагать, что такой обряд выполнялся перед выходом замуж\footnote{Brill's New Pauly. Vol. 5. Leiden--Boston, 2004. P. 507.}. Согласно сообщению Плутарха, храм Фортуны-Девы находился <<неподалёку от источника, известного под названием Мускоса>>\footnote{\graecafn{par`a d`e t`hn Mousk~wsan kaloum'enhn kr'hnhn >'eti Parj'enou T'uqhs <ier'on >esti}, Plut. De Fort. Rom. 10, Mor. 322~F "--- 323~A.} (Plut. De Fort. Rom. 10, Mor. 322~F "--- 323~A). Обратим внимание на культ этой богини, о котором у нас есть так мало сведений, зато относящихся к эпохам, отстоящим друг от друга почти на тысячелетие: если даже сообщение Плутарха о том, что поклонение Фортуне Девственной учредил Сервий Туллий, и нельзя признать истинным, то мы можем, по крайней мере, судить, что за древностью культа этой богини имя основателя храма в её честь совершенно изгладилось из народной памяти, так что предание, скорее всего, приписало учреждение этого святилища царю Сервию как самому вероятному претенденту на эту роль\footnote{У. Фаулер указывает на существование устойчивой тенденции приписывать создание храмов Фортуны Сервию Туллию. См. Warde Fowler, W. Op. cit. P. 68.}. Обратим внимание и на то, что Fortuna Virginalis также заведует определённой стороной жизни женщины и, таким образом, близка к Фортуне на Бычьем Форуме и Фортуне Женской, храм которой был основан в самом начале Республики, о чём пойдёт речь в следующей главе. Итак, мы можем довольно грубо и с достаточной степенью осторожности отнести <<terminus, ante quem>> для основания культа Девственной Фортуны к эпохе царского Рима или ранней Республики (вероятно, не позднее IV в. до н.э.). А.~И.~Немировский почему-то отождествляет Фортуну-Деву с Фортуной на Бычьем форуме\footnote{Немировский Указ. соч. С. 73.}, однако, на наш взгляд, подобное утверждение недостаточно подтверждено фактами. То, что Танаквиль пожертвовала Фортуне тогу (Plin. NH.VIII.194) и что девушки, согласно Арнобию, жертвовали Фортуне-Деве свои одежды, вовсе не одно и то же: в первом случае Танаквиль поднесла богине царские одежды её воспитанника Сервия, во втором "--- девушки приносили \textit{свои собственные} одежды, что должно было символизировать смену социального статуса при переходе в замужество. При внешнем сходстве внутренняя символика ритуалов различна, и это не позволяет нам так легко поставить знак равенства между тем храмом, в который Танаквиль пожертвовала претексту Сервия (а это, вероятнее всего, было святилище in foro Boario), и храмом Фортуны-Девы.

%Про Мужскую Фортуну мы уже писали выше.


% Написать про храм Мужественной Удачи, сооружение которого Плутарх приписывает царю Анку Марцию. 

% Разобрать, что пишут про С.Т. древние авторы и как объясняют его связь с Фортуной

% Итоги разбора фактов
Итак, на основании вышеизложенного мы уже в состоянии сделать вывод о том, создание каких храмов Фортуны можно отнести к царской эпохе. Опираясь на античную традицию, мы можем уверенно утверждать, что храмы Фортуны на Бычьем форуме и Счастливого Случая на берегу Тибра были основаны Сервием Туллием. Несколько более противоречивы сообщения о храме Мужской Фортуны и Фортуны-Девы, но, судя по характеру этих культов, мы также можем отнести дату их основания к эпохе царей или ранней Республики. Об остальных храмах, создание которых Плутарх приписывает Сервию Туллию мы можем сказать что либо древнегреческий автор это делает ошибочно, либо же у нас нет каких-то других источников, по которым мы могли бы сложить определённое мнение. Доверять же в этом случае сообщениям Плутарха, не подтверждённым как-либо иначе, мы не можем.

% О характере культа Фортуны
Распутав, насколько это возможно, противоречивые факты касательно культа Фортуны в царскую эпоху, обратимся к вопросу о восприятии римлянами этой богини в то время, отдалённое не только от нас, но и от древних писателей, чьи сочинения проливают свет на эту нетривиальную проблему. Здесь мы сталкиваемся с почти непреодолимыми трудностями, ибо скудость фактов такова, что мы не можем построить единую гипотезу, которая бы учитывала все имеющиеся факты и не противоречила бы ни одному из них. Суть вопроса можно свести к следующему: являлась ли Фортуна изначально богиней случая и неопределённости, каковой её воспринимали в более позднее время, или нет? И если нет, то какова была её роль? В последнем случае возможны два варианта: или это было женское божество, или божество местное\footnote{Paulys Real-Encyclop\"{a}die der classischen Altertumwissenschaft. Hb 13. Stuttgart. 1910. S. 13--14.}. Дж. Фергюссон считает\footnote{Fergusson, J. The religions of the Roman Empire. Cornell, 1985. P. 85.}, что Фортуна изначально была божеством плодородия, привнесённым в Рим <<at an early date>>, но в результате раннего контакта с греками она стала идентифицироваться с Тюхе. Такого же мнения придерживаются Х.~Розе\footnote{Rose,~H.~J. Op. cit. P. 238.} и С.~Бейли\footnote{Bailey,~C. Op. cit. P. 137.}. А.~И.~Немировский полагает Фортуну <<древнейшим материнским божеством Италии>>\footnote{Немировский А.И. Указ соч.. С. 72.}, он придерживается мнения, что она одновременно была и богиней судьбы\footnote{Там же. С. 73.}, хотя и указывает, что <<в древнейшем слое легенды о Сервии Туллии Фортуна могла быть просто материнским божеством>>\footnote{Там же. С. 86.}. Ж.~Дюмезиль однозначно соотносит Фортуну с ведийским божеством случаности Бхаги\footnote{Дюмезиль Ж. Верховные боги индоевропейцев. С. 75, 80, 131--132.}, так что он придерживается мнения о том, что характер случая был изначально присущ этой богине. 

У.~Фаулер придерживается достаточно взвешенного мнения, с осторожностью предполагая, что Фортуна, с которой был связан древний оракул в Пренесте, как богиня-пророчица могла соотноситься с италийскими Карментами или североевропейскими Норнами, определяющими судьбу человека при его рождении. Таким образом, Фортуна, согласно его предположению, могла бы восприниматься как божество, покровительствующее матерям в деторождении (подобно Матери Матуте) и, через это, определяющее судьбу младенца\footnote{W. Warde Fowler. The Roman festivals of the period of Republic. London. 1899. P. 167.}. Такое толкование позволяет удачно объединить в одно целое противоречивые, на первый взгляд, стороны образа архаической Фортуны. Хотя эта точка зрения во многом совпадает с позицией А.~И.~Немировского, сравнительный материал, который привлекает У.~Фаулер для обоснования, делает подобное предположение более весомым.

Г.~Виссова придерживался схожей точки зрения, соотнося Фортуну с божествами судьбы, греческими мойрами и италийскими парками\footnote{Wissowa,~G. Op. cit. S. 213.}, что, согласно его мнению, говорило о связи этой богини с идеей судьбы. Он не ссылается на работу У.~Фаулера, и это говорит, видимо, о том, что оба автора пришли к похожим выводам независимо друг от друга.

Мы согласимся с мнением Дж.~Шайда о том, что римские божества сохраняли свою идентичность в ритуальной практике, пусть даже и приложенной к различным областям жизни\footnote{Шайд~Дж. Религия римлян. С. 160--161.}. Дж.~Шайд указывает, что <<плодородие вообще>> "--- слишком расплывчатое понятие\footnote{Там же.}, поэтому, даже связывая Фортуну с матрональными божествами и указывая на её функции богини плодородия и деторождения, мы должны также отмечать особенные черты культа этой богини и представления о ней, которые отличают её от прочих женских божеств. Эти черты проявляются в том, что Фортуна одновременно богиня случая, удачи и судьбы. %В этом отношении уязвима для критики позиция Х.~Розе, который полагал Фортуну изначально сельскохозяйственным божеством, а связь с идеей случайности и удачи, по его мнению, Фортуна приобрела впоследствии, т.к. сельскохозяйственная практика зависит от многих факторов, над которыми земледелец не властен\footnote{Rose,~H.~J. Op. cit. P. 238--239.}. Но тогда справедливо возникает вопрос: почему из всех божеств, имевших отношение к сельскому хозяйству, именно Фортуна приобрела таковой характер, и каким именно образом это произошло? Более логично предположить, следуя Дж. Шайду, что представление о Фортуне сохраняло свою идентичность, 

Другой вопрос заключается в том, автохтонным или заимствованным божеством была Фортуна\footnote{Дюмезиль Ж. Верховные боги индоевропейцев. С. 131.}. Г.~Виссова, однако, относит Фортуну к di novensidnses\footnote{Wissowa, G. Op. cit. S. 206}, указывая, что её не было в числе тех богов, почитание которых учредил Нума, и которые относятся к di indigetes, и с этим мнением, безусловно, трудно поспорить. Всё же отметим, что к эпохе Сервия Туллия можно отнести лишь исторически достоверное учреждение первых храмов Фортуны в Риме. Насколько в те отдалённые времена жители Рима могли почитать Фортуну частным образом, как это делал, например, в другую эпоху будущий император Гальба (см. Suet. Galb. 4.3, 18.2), неизвестно, так что вопрос о точном времени и конкретных путях заимствования Фортуны остаётся открытым.

К.~Латте указывает на то, что Фортуна была заимствована из Лация, доказательством чему служат два древних и почитаемых святилища этой богини в Пренесте и Анции\footnote{Latte, K. Op. cit. S. 176.}. Ж. Дюмезиль также пишет, что Фортуна изначально была латинским божеством\footnote{Дюмезиль Ж. Указ. соч. С. 132.}. Это, однако, расходится с мнением Варрона, который называл Фортуну в числе женских (sic!) божеств, которые были заимствованы у сабинян: согласно его мнению, таким образом появились в Риме богини Ферония, Минерва, Палес, Веста, Салюс, Фортуна, Фонс и Фидес\footnote{\textit{Feronia, Minerva, Novensidnses a Sabinis; paulo aliter ab eisdem dicimus haec: Palem, Vestam, Salutem, Fortunam, Fontem, Fidem}, Varr. De ling. Lat. V.74.} (De ling. Lat. V.74). Так как Варрон сам по происхождению был сабинянином, то, видимо, здесь он опирался на легенды и сказания, которые имели хождение на его родине и которые он слышал с детства\footnote{А.~И.~Немировский подчёркивает, что Варрон в своих произведениях выдвигал на первое место сабинские религиозные обычаи и приписывал сабинское происхождение общеиталийским божествам. См. Немировский А.~И. Указ. соч. С. 6.}. Однако это говорит лишь о том, что в разных областях существовали разные, противоречащие друг другу сказания о Фортуне, и что она была божеством не только общелатинским, но и общеиталийским. Таким образом, однозначного ответа на вопрос, откуда именно была заимствована эта богиня, видимо, нельзя дать в принципе, вместо этого наиболее продуктивный путь заключается в том, чтобы искать различные пути становления культа Фортуны в Риме\footnote{Brill’s New Pauly. Vol. 5. Leiden-Boston, 2004. P. 506.}, ибо их, скорее всего, было несколько.

%Сервий Туллий, как это было подмечено ещё в древности, особое внимание уделял богине Фортуне, 
%Рассмотрев сообщения античных авторов касательно культа Фортуны в царскую эпоху, подведём некоторые итоги. Мы видим, что Сервий Туллий сыграл ключевую роль в становлении культа Фортуны в Риме. Храмы, основанные в его эпоху, продолжали существовать на протяжении всего республиканского периода и даже во время Империи.

%В любом случае, деление на di indigetes и di novensidnses, введённое  говорить о <<заимствовании>> римлянами общелатинского (и даже общеиталийского) культа Фортуны, скорее, следует обозначать лишь исторически достоверную дату официального учреждения культа этой богини в Риме, которая относится ко времени правления царя Сервия Туллия.

В любом случае, появление этого культа на исторической сцене однозначно соотносится с именем Сервия Туллия и с его эпохой, имевшей решающее значение для римской истории. Значение Сервия Туллия для Фортуны и Фортуны для царя Сервия подмечали и древние авторы и давали этому определённое толкование. Нам следует подробнее остановиться на этом вопросе, чтобы по возможности пролить свет на восприятие Фортуны римлянами царской эпохи. Тот взлёт, который пережил Сервий, античные писатели объясняли покровительством богини счастья и удачи. Такую интерпретацию даёт Овидий: <<Плебс поклоняется ей [богине счастливого случая], ведь кто основал [храм], из плебса вознесён, из низов до [царского] скипетра дошёл>>\footnote{\textit{plebs colit hanc, quia qui posuit de plebe fuisse / fertur, et ex humili sceptra tulisse loco}, Ovid. Fast. VI.781--782.} (Fast. VI.781--782). Объяснение в таком же духе приводит и Плутарх: <<благодаря Фортуне Сервию, как говорят, рождённому от рабыни, суждено было стать знаменитым царем Рима>>\footnote{\graecafn{Serou'iw| kat`a t'uqhn, <'ws fasin, >ek jerapain'idos gen'omenw| basile~usai t~hs <R'wmhs >epifan~ws <up~hrxen}, Plut. Quae. Rom. 106, Mor. 289~B.} (Plut. Quae. Rom. 106, Mor. 289~B). Интересно, что Тит Ливий совсем не прибегает к подобного рода объяснениям, хотя и даёт религиозное толкование необыкновенной судьбе Сервия, а именно, пишет, что когда тот был ребёнком, его голова однажды горела огнём на глазах у многих, и это-то и послужило знамением, предвещавшим тому царский титул (Liv. I.39)\footnote{Овидий приводит эту же легенду (Fast. VI.635--636), объясняя такое чудо тем, что Сервий был зачат от огня (Ibid. 631--634). Это сказание было известно и Плутарху, ср. Plut. De Fort. Rom. 10, Mor. 323 A--D.}. Здесь следует сделать отсылку к тому официальному и формальному характеру религиозности Тита Ливия, о котором мы писали ранее (см. с.~\pageref{LiviusAndReligio}). Таким образом, умолчание Титом Ливием об этих интересных сказаниях касательно Сервия Туллия добавляет весомости мнению о том, что здесь как Овидий, так и Плутарх не дают собственного истолкования популярности царя Сервия и учреждённых им культов Фортуны и Счастливого Случая среди плебса, такого истолкования, которое могли бы дать люди книжной культуры, основываясь на тонкой игре смысла и аллюзиях на богатую античную литературу или опираясь на документы римских жреческих коллегий "--- наоборот, здесь, как мы можем судить, оба автора пересказывают легенды, имевшие хождение и рождённые в среде самого плебса, и именно к плебеям относится, в конечном итоге, <<\graeca{<'ws fasin}>> Плутарха.

Подобному истолкованию, чьи корни, согласно нашему предположению, лежат в народной среде, мы можем противопоставить объяснение Плутархом того факта, что царь Анк Марций, якобы, основав храм Фортуны, дал ей имя мужества (Plut. De Fort. Rom. 5, Mor. 318~F). Греческий автор толкует это так: мужеству, говорит он, <<в достижении победы более всего способствует Удача>>\footnote{\graecafn{<~h| [>andre'ia|} \textit{"--- В.~Ж.}\graecafn{] ple~iston e>is t`o nik~an t'uqhs m'etesti}, Plut. De Fort. Rom. 5, Mor. 318~F.} (Ibid.). Здесь подчёркивается роль Фортуны (\graeca{T'uqhs}) в военном успехе, к тому же, тут она соотносится с таким моральным качеством как мужество (\graeca{>andre'ia}), что, как мы уже отмечали (см. с.~\pageref{TycheAndEthos}), характерно вообще для этого раннего трактата Плутарха и не характерно для более позднего сочинения <<Об удаче>> (\graeca{Per`i T'uqhs}), в котором Плутарх противопоставляет Удачу и внутренний, нравственный и интеллектуальный мир человека. На основании вышеизложенного мы уже можем сделать вывод, что в царскую эпоху Фортуна не соотносилась римлянами напрямую с таким аспектом удачи как удача на поле битвы, что как богиня счастья и судьбы она была связана с рождением и материнством. В этом отношении объяснение, данное Плутархом, выбивается из того ряда фактов, каковой нам сообщают другие источники, также как и его мнение, будто первый храм Фортуны построил царь Анк Марций. Таким образом, здесь мы, скорее всего, видим интерпретацию, рождённую в голове человека, отягощённого книжным образованием и представлением о Фортуне или Тюхе более позднего времени, "--- будь это сам Плутарх или же автор одного из его источников.

%%%%%%%%%%%%%%%%%%%%%%%%%%%%%%%%%%%%%%%%%%%%%%%%%%%%%%%%%%%%%%%%%%%%%%%%%%%%%
%% Легенды, связанные с Сервием Туллием и его связью с Фортуной.
%% Сервий Туллий как религиозный устроитель - вслед за Нумой Помпилием!
%%%%%%%%%%%%%%%%я%%%%%%%%%%%%%%%%%%%%%%%%%%%%%%%%%%%%%%%%%%%%%%%%%%%%%%%%%%%%

%Обратим внимание также и на то, что Плутарх совершенно ничего не пишет про храм Фортуны на Бычьем форуме\footnote{Правда, Плутарх, как мы уже видели, пересказывает легенды, которые Овидий связывает с храмом на Бычьем форуме (Quae. Rom. 36, De Fort. Rom. 10, Mor. 322~D, ср. с Ovid. Fast. VI.575--578), однако греческий автор всё же не упоминает о самом храме и о том значении, которое он имел для римлян.}, строительство которого мы на основании вышеизложенного с уверенностью можем отнести ко времени Сервия Туллия и который, как мы уже видели, играл столь важную роль в жизни римской общины. С чем это может быть связано? Дело в том, что Плутарх должен был пользоваться, в первую очередь, источниками, написанными римскими интеллектуалами, а легенды, сложенные вокруг храма на Бычьем форуме, имели плебейский характер. Отметим, что и Тит Ливий, мимоходом упоминая этот храм, молчит о тех ярких сказаниях, которые нам сообщает Овидий. Характер этих сказаний контрастирует с официальным и формальным характером традиционного римского патрицианского культа: легенды о выдающихся деятелях царского Рима и времён Республики, о которых мы читаем, например, у Тита Ливия, так или иначе подчёркивают то, что человек сделал для res publica (будь то Муций Сцевола, Кориолан, Камилл или Валерий Корвин); человек в таких легендах не появляется просто потому, что необычна сама по себе и достойна восхищения судьба того, кто из сына рабыни сделался царём. Поэтому неудивительно, что наиболее подробно о разного рода сказаниях, сложенных вокруг храма на Бычьем форуме и священного изображения в нём, мы читаем в <<Фастах>> Овидия, экзегетические упражнения которого, согласно интерпретации Дж. Шайда, носят характер интеллектуальной игры и забавы\footnote{Шайд Дж. Миф, культ и реальность в <<Фастах>> Овидия // Шайд Дж. Религия римлян. М., 2006. С. 232.}.

%Правда, Плутарх пересказывает также легенды, несомненно, имевшие хождение в плебейской среде, причём связанные именно с храмом на Бычьем форуме, не упоминая, впрочем, само это святилище, легенды о том, что богиня Фортуна вступала в интимную связь с Сервием (Quae. Rom. 36), входя к нему через окно в храме (De Fort. Rom. 10, Mor. 322~D): об этом же говорит и Овидий в <<Фастах>> (Fast. VI.575--578).

% Вот это про Сервия Туллия
% Сервий Туллий был почитаем плебеями. Это аксиома. Так есть. На том и стою, и не могу иначе.
Возвращаясь к фигуре царя Сервия Туллия, подчеркнём, что его популярность среди плебеев, разумеется, не случайна. Реформа, проведённая им, открыла плебеям путь к политической жизни римской общины, стала предпосылкой для развернувшейся позднее борьбе с патрициями за равноправие. Нам хотелось бы подчеркнуть именно эту объективную роль Сервия Туллия в судьбе римского плебса, а не акцентировать внимание только на том факте, как это делают античные толкователи, что Сервий из раба сделался царём, и именно поэтому-то его фигура пользовалась такой популярностью у плебеев\footnote{Это же объяснение повторяет Е.~М.~Штаерман. См. Штаерман~Е.~М. Указ. соч. С. 76--77.}.

Отметим также, что помимо того, что царь Сервий прославился как государственный деятель "--- собственно, именно в его эпоху совершилось становление в Риме государства и классового общества\footnote{Маяк И.Л. Римляне ранней Республики. М., 1993. С. 15.} "--- он был также религиозным устроителем, и хотя эта его роль не особо подчёркивается как античным преданием, так и учёными нового времени, за исключением, разве что, Ж. Дюмезиля\footnote{См. с.~\pageref{DumezilFort}.}, мы показали, что культы, основанные в его эпоху, существовали на протяжении всего периода Республики и не утратили своё значение и во времена Империи. О важности и популярности этих культов говорят праздненства, о которых свидетельствовал Овидий, говорят легенды, сложенные, в первую очередь в плебейской среде, вокруг фигуры Сервия Туллия и его отношений с богиней Фортуной, вокруг древнего изображения, находившегося в святилище и пережившего пожар в храме. Далеко каждый храм "--- речь идёт не только о храмах именно Фортуны "--- имеет подобного рода <<легендарную историю>>.

% Это будет частью заключения
Итак, те сведения, которые нам сообщает античное предание о возникновении и ранних этапах становления культа Фортуны, о времени, относящемся к царской эпохе, хотя и довольно подробны, но противоречивы и обнаруживаются в мифологическом окружении. Тем не менее, мы можем указать, что храмы Фортуны на Бычьем форуме и Fortis Fortunae на берегу Тибра, согласно традиции, однозначно соотносятся с именем Сервия Туллия\footnote{Г. Виссова полагает, что именно эти два крупных храма и основал Сервий. См. Wissowa,~G. Op. cit. S. 206.}. С гораздо меньшей определённостью мы можем высказаться по поводу происхождения культов Fortunae Virilis и Fortunae Virginalis, хотя это, безусловно весьма древние культы. Г.~Виссова также приписывает Сервию учреждение <<einer Menge von kleinen Kapellen, in denen die G\"{o}ttin unter den verschiedensten Beinamen verehrt wurde>>\footnote{Wissowa, G. Op. cit. S. 206.}, ссылаясь при этом на (Plut. De Fort. Rom. 10)\footnote{Ibid.}, однако мы уже показали выше, что доверять греческому автору в этом вопросе не следует.

Тем не менее, на основании вышеизложенного мы можем заключить, что крупные храмы Фортуны, основанные во времена Царского Рима, не утратили своё значение и в эпоху Империи, более того, были высоко почитаемы, в первую очередь, в плебейской среде. Таким образом, в царский период совершилось становление культа Фортуны в Риме и наметились основные тенденции его дальнейшего развития. К сожалению, письменные источники, дошедшие до нас, оставляют много неясного в понимании этого ключевого для нашей темы периода, что заставляет исследователей чаще выдвигать предположения, чем делать однозначные выводы.


%%%%%%%%%%%%%%%%%%%%%%%%%%%%%%%%%%%%%%%%%%%%%%%%%%%%%%%%%%%%%%%%%%%%%%%%%%%%%%%%%%%%%%%%%%%%%%%%%%%%%%%%%%%%%%%%%%%%%%%%
%% --== Культ Фортуны в эпоху ранней Республики ==-- %%
%%%%%%%%%%%%%%%%%%%%%%%%%%%%%%%%%%%%%%%%%%%%%%%%%%%%%%%%%%%%%%%%%%%%%%%%%%%%%%%%%%%%%%%%%%%%%%%%%%%%%%%%%%%%%%%%%%%%%%%%

% Здесь нужно рассказать о храме Женской Фортуны на 6-м милевом камне Латинской дороги
% Обратить внимание на то, что храм раскопан. Найти ссылку на это дело.
%
% Companion for greek and roman historiography - посм. тудыть.

