\chapter{Античные представления о Фортуне}

\section{Фортуна у античных писателей}\label{Scriptores}

До сих пор мы общалались к сочинениям античных писателей, чтобы почерпнуть оттуда сведения о собственно культе Фортуны. Однако слово \textit{fortuna} появляется в письменных источниках в различном контексте. Оно употребляется в своём словарном значении, писатели говорят о Фортуне как о богине, философы "--- как о понятии, в исторических сочинениях мы можем встретить рассуждения о роли фортуны в истории. В настоящем исследовании мы сосредоточимся на реконструкции образа \textit{богини Фортуны}, какой возникает в трудах античных писателей; нам, однако, приходится затронуть и тему философского представления о Фортуне, но лишь в той степени, в которой эти представления связаны с религиозными материями.

Образ Фортуны возникает в сочинениях одних из самых ранних римских авторов. Так, о ней писал поэт Пакувий (кон. III "--- II вв. до н.э.):

\begin{verse}
\textit{Fortunam insanam esse et caecam et brutam perhibent philosophi,\\
saxoque instare in globoso praedicant volubilei,\\
quia quo id saxum inpulerit fors, eo cadere Fortunam autumant.\\
Insanam autem esse aiunt quia atrox incerta instabilisque sit;\\
caecam ob eam rem esse iterant quia nil cernat quo sese adplicet;\\
brutam quia dignum atque indignum nequeat internoscere.\\
Sunt autem alii philosophi qui contra Fortunam negant\\
esse ullam sed temeritate res regi omnes autumant.}\footcite[P. 318]{RemainsofLatin1935}
\end{verse}

Пакувий описывает Фортуну со слов неких философов: она изображается стоящей на круглом камне (\textit{in globoso}), в неустойчивом положении, и падает (\textit{eo cadere Fortunam}) в ту сторону, куда направит случай (\textit{quo id saxum inpulerit fors}). Фортуна наделена рядом отрицательных эпитетов: она безумна (\textit{insana}), потому что жестока (\textit{atrox}), ненадёжна (\textit{incerta}) и переменчива (\textit{instabilis}); слепа (\textit{caeca}), потому что не различает, кому посвящает себя (\textit{nil cernat quo sese adplicet}); глупа (\textit{bruta}), потому что не делает разницы между достойным и недостойным (\textit{dignum atque indignum}). 

Философы, на которых ссылается Пакувий, безусловно, должны быть греками и, разумеется, в оригинале вести речь не о римской Фортуне, а о греческой \graeca{T'uqh}. Примечательно здесь, что уже Пакувий ставит знак равенства между этими двумя богинями, не менее примечательно и то, что в его коротком стихотворении мы встречаемся с резко отрицательным образом Фортуны. И.~Кайянто указывает на то, что этот образ вполне соответствует представлению о Тихе в новоаттической комедии\footcite[S. 528--529]{Kajanto1981}.

О богине Фортуне мы можем услышать и из уст героев плавтовских комедий. Так, в <<Ослах>> герой пьесы говорит, что дозволено хвалить Фортуну\footnote{\mancite{Plaut. Asin. 718}: \textit{Licet laudem Fortunam}.}, а в <<Пленниках>> персонаж, в шутливой форме предлагая принести себе жертву, как богу, называет себя, в том числе, Фортуной, помещая её среди таких божеств, как Salus, Lux, Laetitia, Gaudium\footnote{\mancite{Plaut. Capt. 864}: \textit{idem ego sum Salus, Fortuna, Lux, Laetitia, Gaudium}.}. Перечисление римских божеств у героев Плавта, безусловно, необходимо отнести к типично римским реалиям, возникающим в этих сочинениях.

%Здесь мы встречаем Фортуну именно как богиню, и Плавт в этих отрывках рисует её как доброе, положительное божество. Здесь герои Плавта, чьи образы списаны с окружавших его людей, придерживаются мнения, противоположного тому, которое, со слов греческих философов, пересказывает Пакувий.

Но в произведениях Плавта мы можем найти и упоминание фортуны как понятия о судьбе или случае. Героиня <<Касины>> жалуется соседке на свою \textit{<<[плохую] судьбу в отношении мужа>>}\footnote{\textit{Nunc huc meas fortunas eo questum ad vicinam}, Plaut. Cas. 161.}, а персонаж комедии <<Канат>> сравнивает своего товарища со \textit{<<злой судьбой>>}\footnote{\textit{Malam fortunam in aedis te adduxi meas}, Plaut. Rud. 501.}. Здесь проявляется амбивалентный характер римского представления о Фортуне.

% Гораций

Образ богини Фортуны использует в своих <<Одах>> Гораций. Fortuna возникает у него как сила, способная произвести неожиданные изменения, как возвысить, так и низринуть\footnote{Hor. Carm. I.34.15--16: \textit{ Fortuna cum stridore acuto / sustulit, hic posuisse gaudet.}; Carm. III.29.49--52: \textit{Fortuna saevo laeta negotio et / \ldots{} / transmutat incertos honores, / nunc mihi nunc alii benigna}.}. Она <<играет в опасную игру>> (\textit{ludum insolentem ludere}, Carm. III.29.50). 35-я ода I книги Горация посвящена Фортуне Анцийской: \textit{O diva, gratum quae regis Antium} (Carm. I.35.1). В лацийском городе Анции находилось святилище Фортуны, в котором почитались две сестры-богини\footnote{См. \mancite{Suet. Cal. 57.3, Martial. V.1.3}.}, поэт, однако, обращается к божеству в единственном числе. В строках 1--4 и 9--12 описывается переменчивый характер Фортуны. Исследователи прослеживают влияние греческого представления о Тихе на образ Фортуны в этих строках, где она выступает в роли своевольной силы, вызывающей перемены и политические перевороты\footcite[P. 266]{Arya2002}. В дальнейших строках Фортуна, к которой обращается поэт, приобретает более величественный и торжественный вид. Амбивалентный характер богини подчёркивается её <<свитой>>: ей предшествует Необходимость (Necessitas, строки 17--20), её чтят Надежда и Верность (Spes et Fides, строки 21--24). Гораций молит Фортуну защитить Октавиана в предстоящем походе\footnote{\mancite{Hor. Carm. I.35.29--30}: \textit{serves iturum Caesarem in ultimos / orbis Britannos et iuvenum recens}.} и поразить арабов и массагетов, врагов Рима\footnote{\mancite{Ibid. 38--40}: \textit{o utinam nova / incude diffingas retusum in / Massagetas Arabasque ferrum}.}. Этот образ более соответствует тому духу, который находит выражение в римском культе Фортуны, где она считалась богиней-покровительницей и защитницей, а также спутницей военного счастья Рима. Упоминаемые в 18-й строке оды гвозди (\textit{clavi}), которые несёт в руке Necessitas, как считается\footcite[P. 268]{Arya2002}, могут быть аллюзией на культ Норции в Вольсиниях, где гвоздями отмечалось количество лет\footnote{\mancite{Liv. VII.3.7}: \textit{Volsiniis quoque clavos indices numeri annorum fixos in templo Nortiae, Etruscae deae}.}. Мы можем заключить, что в поэтическом образе Фортуны у Горация, несмотря на определённое греческое влияние, явственно проступают типично римские и, шире, италийские черты.


% Плутарх

Крайне важным источником для нашего исследования является трактат Плутарха <<Об удаче римлян>> (\graeca{Per`i t~hs <Rwma'iwn t'uqhs}). В этом произведении автор обращается к теме римской истории и пытается дать ответ на вопрос, какая из двух сил, удача или доблесть, ответственна за возвышение Рима. Тема эта не оригинальна: античные писатели обращались к антагонизму \graeca{>aret'h} и \graeca{t'uqh}, доблести и удачи как движущих сил истории со времён походов Александра\footcite[С. 165]{Fritz2007}. Это была популярная тема для риторических упражнений. Плутарх заявляет, что оба этих начала объединили свои усилия для того, чтобы Рим достиг своего могущества (De Fort. Rom. 2, Mor. 316~F "--- 317~A). Обосновывая свой тезис, он даёт краткий обзор римской истории от времён Ромула до начала империи, обращается ко множеству фактов и трактует их в историософском ключе.

Исследователь творчества Плутарха С.~Свейн приходит к выводу, что в религиозных и философских трактатах <<Моралий>> у Плутарха строго различаются те события, которые совершаются согласно воле бога или богов (\graeca{jeo~i}), и те, причиной которых является случай (\graeca{t'uqh}); в <<Жизнеописаниях>> словоупотребление Плутарха не столь строго\footcite[P. 273]{Swain1989a}. Неясно, правда, можно ли отнести сочинение об удаче римлян к числу <<серьёзных философских трактатов>>. С.~Свейн отмечает, что из всех <<Моралий>> только в этом сочинении \graeca{T'uqh} воспринимается как богиня\footcite[P. 506]{Swain1989}. Он соглашается с предположением Р.~Бэрроу\footcite[P. 126--127]{Barrow1967} о том, что во второй половине этого трактата значение слова \graeca{T'uqh} изменяется, возможно, и неосознанно для самого Плутарха: если в начале это слово можно трактовать как <<удачу>>, <<случай>>, то к концу оно приобретает смысл судьбы, предназначения.

Можно, правда, не согласиться с мнением Р.~Бэрроу и полагать, что слово \graeca{T'uqh} сохраняет своё значение на протяжении всего текста; однако бесспорно, что именно этим словом Плутарх обозначает римскую богиню Фортуну, о храмах которой он ведёт речь. \textit{<<У Удачи (\graeca{T'uqh}),} "--- пишет Плутарх об этой богине, "--- \textit{шаг возбужденный, движения необузданные, она полна надежды похвастаться. Опережая Доблесть, она все время недалеко от нее, но не думайте, будто она ``поднимает себя на легких крыльях'' или подходит, ступая осторожно и нерешительно, ``почти не касаясь земного круга'', а затем удаляется так же незаметно>>}\footnote{\mancite{Plut. De Fort. Rom. 4, Mor. 317~E}: \graecafn{t~hs d`e T'uqhs >ox`u m`en t`o k'inhma ka`i jras`u t`o fr'onhma ka`i meg'alauqos <h >elp'is, fj'anousa d`e t`hn >Aret`hn >egg'us >estin, o>u ptero~is >elafr'izousa ko'ufois <eaut`hn o>ud> >akr'wnuqon <up`er sfa'iras tin`os >'iqnos kaje~isa perisfal`hs ka`i >amf'ibolos pr'oseisin, e>~it> >'apeisin >aid'hs}.}. И далее, подчёркивая благосклонность Фортуны к Риму: \textit{<<Она пришла в Рим, чтобы остаться навеки>>}\footnote{\mancite{Ibid. 318~A}: \graecafn{e>is~hljen e>is <R'wmhn <ws meno~usa}.}.

%τῆς δὲ Τύχης ὀξὺ μὲν τὸ κίνημα καὶ θρασὺ τὸ φρόνημα καὶ μεγάλαυχος ἡ ἐλπίς, φθάνουσα δὲ τὴν Ἀρετὴν ἐγγύς ἐστιν, οὐ πτεροῖς ἐλαφρίζουσα κούφοις ἑαυτὴν οὐδ᾽ ἀκρώνυχον ὑπὲρ σφαίρας τινὸς ἴχνος καθεῖσα περισφαλὴς καὶ ἀμφίβολος πρόσεισιν, εἶτ᾽ ἄπεισιν ἀιδής

% εἰσῆλθεν εἰς Ῥώμην ὡς μενοῦσα

Плутарх верно подмечает ту особенность культа Фортуны в Риме, которая и сегодня вызывает у нас такое же удивление, как и у него: разнообразие и многочисленность форм поклонения, необычайно широкая популярность этой богини. Однако ему не приходит в голову, что латинское понятие \textit{Fortuna} может не совсем соответствовать греческому понятию \graeca{T'uqh}. Для Плутарха \graeca{T'uqh} = \textit{Fortuna}. Следуя этому представлению, он описывает \graeca{T'uqh} как благодетельницу, с одной стороны, различных исторических деятелей (царя Нумы, Суллы, Цезаря и др.), с другой же "--- самого города Рима. Наше исследование, однако, показывает, что римляне не воспринимали Фортуну как покровительницу \textit{города}: она считалась покровительницей различных общественных групп, вплоть до всего римского народа. Среди огромного многообразия когноменов этой богини мы не находим Фортуны Urbis Romae, зато у римлян была Fortuna Publica Populi Romani. Тиха же у греков считалась покровительницей как отдельных личностей, так и городов (\graeca{T'uqh p'olews}\footcite[Sp. 1345]{Tyche1915Wasser}). Таким образом, ставя знак равенства между понятиями \graeca{T'uqh} и \textit{Fortuna}, Плутарх переносит на римскую Фортуну некоторые характерные черты \textit{культа} Тихи, которыми римская богиня не обладала: римляне не воспринимали Фортуну в том смысле, в каком греки понимали \graeca{T'uqh p'olews}.

% Плиний Старший

Широко известна\footcites[С. 321]{ESBE1902}[P. 79]{Ferguson1970}[P. 134 ff]{Arya2002} пространная цитата из <<Естественной истории>> Плиния (Plin. NH II.22), в которой он даёт характеристику этой богине и популярности её культа:

\begin{quote}
\textit{По всему ведь миру, во всех местах и во всех землях во всех молитвах к одной Фортуне обращаются и одну её призывают, одну обвиняют, к ней одной всё сводят, об одной думают, об одной говорят и одну с бранью почитают, изменчивая~\ldots{} и многие считают её слепой, изменчивой, непостоянной, колеблющейся, покровительницей недостойных, ею всё оплачивается и ею берётся в долг, и во всех человеческих делах она одна записана на обеих этих страницах, и до того мы доходим, что полагаем её вместо бога, когда не уверены, к какому обратиться.}\footnote{Plin. NH II.22: \textit{Toto quippe mundo et omnibus locis omnibusque horis omnium vocibus Fortuna sola invocatur ac nominatur, una accusatur, res una agitur, una cogitatur, sola laudatur, sola arguitur et cum conviciis colitur, volubilis~\ldots que, a plerisque vero et caeca existimata, vaga, inconstans, incerta, varia indignorumque fautrix. huic omnia expensa, huic feruntur accepta, et in toto ratione mortalium sola utramque paginam facit, adeoque obnoxiae sumus sortis, ut prorsus ipsa pro deo sit qua deus probatur incertus}.}
\end{quote}

\textit{Toto mundo} Плиния "--- это, безусловно, вся Империя, и здесь автор <<Естественной истории>> должен подразумевать не просто римскую Фортуну, но эллинистический культ Тихи, который, действительно, имел широкое распространение по всей Римской державе\footcite[P. 85--87]{Ferguson1970}. Здесь мы опять видим отождествление Тихи и Фортуны. Эта богиня наделяется у Плиния рядом не слишком привлекательных черт: она слепа (\textit{caeca}), изменчива (\textit{vaga}), непостоянна (\textit{inconstans}), ненадёжна (\textit{incerta}), она покровительница недостойных (\textit{indignorum fautrix}). Этот образ во многом напоминает Фортуну Пакувия, однако если римский поэт рисовал портрет богини со слов неких философов, то в представлении Плиния именно таковой была Фортуна, которая являлась объектом почитания в культе. Популярность такой богини может показаться парадоксом; парадоксальным же образом, со слов Плиния, её и чтят "--- бранью (\textit{cum conviciis colitur}).

% Сенека

Современник Плиния Старшего, философ Сенека в <<Нравственных письмах Луцилию>> неоднократно обращается к понятию фортуны. Фортуна выступает у него как сила, производящая перемены\footnote{\mancite{Sen. Ep. XLIV.4}: \textit{Omnia ista longa varietas miscuit et sursum deorsum fortuna versavit}. }. Она может как возвысить человека\footnote{\mancite{Ibid. XXI.6}: \textit{Quoscumque in medium fortuna protulit}.}, так и опрокинуть вниз\footnote{\mancite{Ibid. VIII.4}: \textit{non vertit fortuna sed cernulat et allidit}.}. Фортуна ведёт с философом войну\footnote{\mancite{Ibid. LI.8}: \textit{Fortuna mecum bellum gerit'}.}, препятствует его начинаниям\footnote{\mancite{Ibid. XXIV.7}: \textit{`nihil' inquit `egisti, fortuna, omnibus conatibus meis obstando'}.}. Сенека считает идеалом \textit{<<высокий и здоровый дух,  презирающий фортуну>>}\footnote{\mancite{Ibid. IX.13}: \textit{animo sano et erecto et despiciente fortunam}.}, по его мнению, философу следует возвышаться над фортуной\footnote{\mancite{Ibid. XLI.2}: \textit{an potest aliquis supra fortunam nisi ab illo adiutus exsurgere?}; \mancite{Ibid. XLIV.6}: \textit{animus facit nobilem, cui ex quacumque condicione supra fortunam licet surgere}.} и уходить из-под её власти\footnote{\mancite{Ibid. XXXIX.3}: \textit{Sed felix qui ad meliora hunc impetum dedit: ponet se extra ius dicionemque fortunae}.}. Фортуна, согласно Сенеке, не имеет власти над нравами человека\footnote{\textit{Ibid. XXXVI.6}: \textit{In mores fortuna ius non habet}.}. Интересно его упоминание о \textit{viscata beneficia}  фортуны (Ep. VIII.3), которое, при всё отвлечённости образа фортуны у Сенеки, является, судя по всему, аллюзией на существовавший культ Fortuna Viscata (см. п. \ref{FortunaViscata}).

Образ Фортуны в изображении Сенеки, хотя и имеет некоторые общие черты с её образом у других писателей, всё же совершенно уникален. У Сенеки своё собственное, личное представление о фортуне как действующей силе, представление, преломленное через стоическую философию и непростую биографию самого философа. Испытав немало невзгод, пережив и падения, и взлёты, Сенека лелеял образ стоического мудреца, неподвластного внешним обстоятельствам (автаркия) и равнодушного к счатью и несчастью (атараксия). Фортуна предстаёт у него как олицетворение внешних обстоятельств, переменчивых и ненадёжных, оказывающих постоянное давление на душевное спокойствие мудреца. Сенека не рисует портрет олицетворённой фортуны, не наделяет её конкретными личными качествами, но ясно, что он мог бы вслед за Пакувием и Плинием назвать её слепой, непостоянной, изменчивой и т.п.

% Фронтон

% Фронтон в письме Марку Аврелию пишет: Фортуна же богиня, и среди богинь выдающаяся?

% Fronto Ad M. Caes. I.3.7, Nab. p. 5: Quis autem ignorat rationem humani consilii vocabulum esse, Fortunam autem deam dearumque praecipuam? templa fana delubra passim Fortunae dicata, Rationi neс simulacrum neque aram usquam consecratam ?

% Ювенал

% Выводы

Наш обзор показывает, что в эпоху поздней Республики и ранней Империи в античной литературе сложился довольно цельный образ Фортуны как олицетворения силы, вносящей внезапные изменения в жизнь. Фортуна в изображении писателей слепа, непостоянна, переменчива, ненадёжна; она стоит вне морали и оттого покровительствует как достойным, так и недостойным. В изображении писателей именно такая богиня является объектом культа Фортуны не только в самом Риме, но и по всей Римской державе. Древние авторы, как грекоязычные, так и латиноязычные, ставили знак равенства между словами \textit{fortuna} и \graeca{t'uqh} при всей их многозначности, а также довольно легко переходили в употреблении этих слов от одного смысла к другому: в частности, от \textit{fortuna} или \graeca{t'uqh} в значении отвлечённого понятия "--- к их аллегорическому олицетворению в искусстве, или же мифологическому образу, выраженному в конкретном культе.

Безусловно, сами античные авторы были прекрасно осведомлены о многозначности этих слов и при случае умело пользовались этим, как, например, Плутарх в трактате \graeca{Per`i T'uqhs}, где он пишет: \textit{<<Из-за удачи ли, благодаря ли (\graeca{>ek t'uqhs m`en ka`i di`a t'uqhn}) ей Аристид стойко переносил бедность, хотя мог стать владельцем большого богатства, а Сципион, захватив Карфаген, ничего не взял и даже не выбирал из добычи? Из-за удачи ли и благодаря ли ей (\graeca{>ek t'uqhs d`e ka`i di`a t'uqhn}) Филократ, получив деньги от Филиппа, ``стал тратить их на продажных женщин и рыбу'', и Ласфен с Эвфикратом погубили Олинф, ``измеряя благополучие чревом и постыднейшими деяниями''?>>}\footnote{\mancite{Plut. De Fort. 1, Mor. 97 C--D}: \graecafn{>ek t'uqhs m`en ka`i di`a t'uqhn >Ariste'dhs >enekart'erhse t~h| pen'ia|, poll~wn qrhm'atwn k'urios gen'esjai dun'amenos, ka`i Skip'iwn Karqhd'ona <el`wn o>ud`en o>'ut> >'elaben o>'ut> e>~ide t~wn laf'urwn, >ek t'uqhs d`e ka`i di`a t'uqhn Filokr'aths lab`wn qrus'ion par`a Fil'ippou <p'ornas ka`i >ixj~us >hg'oraze> ka`i Lasj'enhs ka`i E>ujukr'aths >ap'wlesan >'Olunjon, <t~h| gastr`i metro~untes ka`i to~is a>isq'istois t`hn e>udaimon'ian>}.}. В первом случае слово \graeca{t'uqh} имеет значение <<счастливый случай, успех, счастье>>, а во втором "--- <<несчастный случай, несчастье, беда>>. Здесь Плутарх обыгрывает амбивалентность понятия, выраженного словом \graeca{t'uqh}. Употребление Плутархом этого слова в различных значениях подчёркивает тонкую иронию процитированного отрывка.

%  ἐκ τύχης μὲν καὶ διὰ τύχην Ἀριστείδης ἐνεκαρτέρησε (97D) τῇ πενίᾳ, πολλῶν χρημάτων κύριος γενέσθαι δυνάμενος, καὶ Σκιπίων Καρχηδόνα ἑλὼν οὐδὲν οὔτ᾽ ἔλαβεν οὔτ᾽ εἶδε τῶν λαφύρων, ἐκ τύχης δὲ καὶ διὰ τύχην Φιλοκράτης λαβὼν χρυσίον παρὰ Φιλίππου ‘πόρνας καὶ ἰχθῦς ἠγόραζε’ καὶ Λασθένης καὶ Εὐθυκράτης ἀπώλεσαν Ὄλυνθον, ‘τῇ γαστρὶ μετροῦντες καὶ τοῖς αἰσχίστοις τὴν εὐδαιμονίαν’

То, что античные писатели так легко переходили от одного значения слов \textit{fortuna} и \graeca{t'uqh} к другому объясняется тем, что они не ставили перед собой цели использовать эти слова как термины, обозначающие конкретные, чётко очерченные понятия, и, в данном случае, слабо заботились о точности и строгости выражения мысли. Они запросто вели речь о том, что им было хорошо знакомо как по окружавшей их жизни, так и по книгам, в которые они были погружены.

Мы, однако, должны выделить, с одной стороны "--- представление рядовых римлян о богине Фортуне, которой они поклонялись в столь различных формах культа, о которой они слагали сказания, и, с другой стороны "--- то представление о фортуне как отвлечённом понятии, которым оперировали образованные античные авторы в своих сочинениях. Этому последнему представлению И. Кайянто даёт термин \graeca{T'uqh}-\textit{Fortuna}, указывая на то, что оно является результатом влияния эллинистических представлений на римскую культуру\footcite[P. 530--531]{Kajanto1981}. \graeca{T'uqh}-\textit{Fortuna} есть персонификация слепого случая, в то время как в культовой практике римлян мы видим совершенно иное представление о Фортуне.

Безусловно, каждый античный автор обладал собственными воззрениями на то, что такое \textit{fortuna} или \graeca{t'uqh}, однако многозначность этих терминов в античной традиции и нюансы их употребления у различных авторов представляют собой научную проблему, которая не может быть разрешена в рамках настоящего исследования.


\section{Символика изображений Фортуны}

Изображения Фортуны интересуют нас, в первую очередь, в свете того, какие религиозные идеи были в них выражены, какие культовые представления раскрываются в тех атрибутах, которыми художники наделяли богиню. Мы не будем обращать внимание на художественные особенности античных произведений, каковые представляют искусствоведческий интерес, но сосредоточимся на анализе символики этих изображений. Предметом нашего исследования, таким образом, является изобразительный \textit{тип} Фортуны.

Как было показано выше, первое изображение Фортуны появляется в Риме одновременно с учреждением её культа Сервием Туллием (см. п. \ref{FortunaInForoBoario}). Известно также, что в храме Женской Фортуны были установлены две статуи богини (см. п. \ref{FortunaMuliebris}). К сожалению, мы ничего не знаем о том, как выглядели эти изображения. В IV--II вв. до н.э. в Пренесте складывается самобытная традиция изображения этой богини, связанная с местным культом Фортуны Примигении, представленная женскими фигурками, у многих из которых на руках младенец\footcite[Pp. 40--43, Pl. V]{Champeaux1982}. Пренестинский культ, однако, развивался отличным от римского путём, и мы не рассматриваем его в настоящем исследовании.

Традиционные и широко известные изобразительные атрибуты Фортуны (рог изобилия, шар или колесо, кормило) появляются в Риме не ранее II в. до н.э. и складываются под влиянием эллинистической традиции изображения Тихи\footcite[P. 68--69]{Arya2002}. Примеры подобных изображений, характерных для эпохи конца Республики "--- начала Империи, приведены в прил. \ref{appendix:Pictura}.

% Написать здесь про то, когда и как сложилась изобразительная традиция Фортуны
% В этой традиции - эллинистическое влияние
% В лации существовала своя традиция - ссылка на Шампо.

% Авторы эпохи начала империи

% Плутарх и Фронтон
% Fortunas omnes cum pennis cum rotis cum gubernaculis reperias.

% Придя на Палатин и перейдя через Тибр, как кажется, положила крылья, выступила из сандалий, 
% ненадёжный и неустойчивый шар

% Plut. De Fort. Rom. 4, Mor. 317 E: τῷ δὲ Παλατίῳ προσερχομένη καὶ διαβαίνουσα τὸν Θύμβριν ὡς ἔοικεν ἔθηκε τὰς πτέρυγας, ἐξέβη τῶν πεδίλων, ἀπέλιπε τὴν ἄπιστον καὶ παλίμβολον σφαῖραν

Важную информацию предосталяют для нас труды античных писателей, где мы находим как толкование значения изобразительных символов Фортуны, так и примеры метафорического их упоминания.
У ряда авторов мы находим перечисление целого набора атрибутов богини. Плутарх в трактате <<Об удаче римлян>> пишет о том, как Тиха-Фортуна, \textit{<<подойдя к Палатину и пересекши Тибр, как кажется, положила крылья, выступила из сандалий и рассталась с ненадёжным и неустойчивым шаром>>}\footnote{\mancite{Plut. De Fort. Rom. 4, Mor. 317 E}: \graecafn{t~w| d`e Palat'iw| proserqom'enh ka`i diaba'inousa t`on J'ubrin <ws >'eoiken >'ejhke t`as pt'erugas, >ex'ebh t~wn ped'lwn, >ap'elipe t`hn >'apiston ka`i pal'imbolon sfa~iran}.}. Здесь Плутарх перечисляет те атрибуты, которые символизируют переменчивый и непостоянный характер богини: она лишается их, чтобы прочно утвердиться в Риме. Фронтон в письме к Марку Аврелию упоминает \textit{<<Фортун с крыльями, с колёсами, с кормилами>>}\footnote{\mancite{Fronto De orat. 5, Nab. p. 155}: \textit{Fortunas omnes cum pennis cum rotis cum gubernaculis reperias}.}.

% Дион Хрисостом

Дион Хрисостом в LXIII речи приводит толкование различным изобразительным атрибутам богини Тихи, которые были характерны также и для Фортуны:

\begin{quote}
\textit{Одни ведь поставили её на острие бритвы, другие "--- на шар, третьи дали ей в руки кормило. Однако лучшие художники наделили её рогом Амальфеи, переполненным высыпающимися плодами земли, "--- его Геракл отторгнул в битве у Ахелоя. Так, лезвие бритвы указывает на краткость счастья, шар "--- на лёгкость его перемены, ведь божественная сущность всегда пребывает в движении. Кормило обозначает, что удача управляет жизнью людей. Рог же Амальфеи указывает на одарение благами и процветание}\footnote{\mancite{Dio Chrys. LXIII.7}: \graecafn{o<i m`en g`ar >ep`i xuro~u >'esthsan a>ut'hn, o<i >ep`i sfa'iras, o<i d`e phd'alion >'edwkan krate~in: o<i d`e t`a kre'ittw gr'afontes t`o t~hs >Amalje'ias >'edosan k'eras pl~hres ka`i br'uon ta~is <'wrais, <`o >en m'aqh| <Hrakl~hs >Aqel'w|ou >ap'errhxen. t`o m`en o>~un xur`on t`o >ap'otomon t~hs e>utuq'ias mhn'uei: <h d`e sfa~ira <'oti e>'ukolos <h metabol`h a>ut~hs >estin: >en kin'hsei g`ar tugq'anei p'antote >`on t`o je~ion. t`o d`e phd'alion dhlo~i <'oti kubern~a| t`on t~wn >anjr'wpwn b'ion <h t'uqh. t`o d`e t~hs >Amalje'ias k'eras mhn'uei t`hn t~wn >agaj~wn d'osin te ka`i e>udaimon'ian}.}.
\end{quote}

% Dio Chrys. LXIII.7: οἱ μὲν γὰρ ἐπὶ ξυροῦ ἔστησαν αὐτήν, οἱ δὲ ἐπὶ σφαίρας, οἱ δὲ πηδάλιον ἔδωκαν κρατεῖν: οἱ δὲ τὰ κρείττω γράφοντες τὸ τῆς Ἀμαλθείας ἔδοσαν κέρας πλῆρες καὶ βρύον ταῖς ὥραις, ὃ ἐν μάχῃ Ἡρακλῆς Ἀχελῴου ἀπέρρηξεν. τὸ μὲν οὖν ξυρὸν τὸ ἀπότομον τῆς εὐτυχίας μηνύει: ἡ δὲ σφαῖρα ὅτι εὔκολος ἡ μεταβολὴ αὐτῆς ἐστιν: ἐν κινήσει γὰρ τυγχάνει πάντοτε ὂν τὸ θεῖον. τὸ δὲ πηδάλιον δηλοῖ ὅτι κυβερνᾷ τὸν τῶν ἀνθρώπων βίον ἡ τύχη. τὸ δὲ τῆς Ἀμαλθείας κέρας μηνύει τὴν τῶν ἀγαθῶν δόσιν τε καὶ εὐδαιμονίαν.

В следующей, второй речи об удаче, Дион Хрисостом продолжает толкование: \textit{<<В правой руке она держит кормило и, как можно сказать, правит судном. Отчего же это так? Может быть, согласно убеждению, что моряки более всего нуждаются в удаче? или оттого, что она правит нашей жизнью, как большим кораблём, и оберегает всех плывущих?>>}\footnote{\mancite{Dio Chrys. LXIV.5}: \graecafn{t~h| m`en dexi~a| xeir`i phd'alion kat'exei, ka'i, <ws >`an e>'ipoi tis, naut'illetai. t`i d`e >'ara to~uto  >~hn? p'oteron <ws m'alista t~wn ple'ontwn t~hs t'uqhs deom'enwn, >`h di'oti t`on b'ion <hm~wn <'ws tina meg'alhn na~un kubern~a| ka`i p'antas s'w|zei to`us >empl'eontas?}} И далее: \textit{<<В другой же руке богиня держит плоды, собранные и готовые к употреблению, указывая на изобилие благ, которые она предоставляет>>}\footnote{\mancite{Dio Chrys. LXIV.7}: \graecafn{t~h| d`e <et'era| t~wn qeir~wn <h je`oj karpo`us <eto'imous kat'eqei suneilegm'enous, mhn'usousa t`o pl~hjos t~wn >agaj~wn, <'aper a>ut`h d'idwsin}.}.

% Dio Chrys. LXIV.5: τῇ μὲν δεξιᾷ χειρὶ πηδάλιον κατέχει, καί, ὡς ἂν εἴποι τις, ναυτίλλεται. τί δὲ ἄρα τοῦτο ἦν; πότερον [p. 149] ὡς μάλιστα τῶν πλεόντων τῆς τύχης δεομένων, ἢ διότι τὸν βίον ἡμῶν ὥς τινα μεγάλην ναῦν κυβερνᾷ καὶ πάντας σῴζει τοὺς ἐμπλέοντας;

% Dio Chrys. LXIV.7: τῇ δὲ ἑτέρᾳ τῶν χειρῶν ἡ θεὸς καρποὺς ἑτοίμους κατέχει συνειλεγμένους, μηνύουσα τὸ πλῆθος τῶν ἀγαθῶν, ἅπερ αὐτὴ δίδωσιν

% Отдельные атрибуты

% Рог изобилия

Рассмотрим упоминания отдельных изобразительных атрибутов Фортуны, которые мы можем найти у различных авторов. Петроний в <<Сатириконе>> говорит о Фортуне с \textit{рогом изобилия}, нарисованной на стене дома Трималхиона\footnote{Petron. XXIX: \textit{Praesto erat Fortuna cornu abundanti copiosa}.}. Плутарх в трактате <<Об удаче римлян>> так описывает Тиху, отождествляемую им с римской Фортуной: \textit{<<В руках она держит многократно воспетый рог изобилия, но только переполнен он не спелыми плодами, а всем тем, что приносят земля и моря, реки, рудники и гавани. Щедро и во множестве изливает она всё это>>}\footnote{\mancite{Plut. De Fort. Rom. 4, Mor. 318 A--B}: \graecafn{t`o d> <umno'umenon >eke~ino to~u plo'utou k'eras >'exei di`a qeir'os, o>uk >op'wras >ae`i jallo'ushs mest'on, >all> <'osa f'erei p~asa g~h p~asa d`e j'alatta ka`i potamo`i ka`i m'etalla ka`i lim'enes, >'afjona ka`i <r'udhn >epiqeam'enh}.}. Арнобий упоминает Фортуну с рогом, наполненным яблоками, фигами и осенними плодами\footnote{\mancite{Arnob. VI.25.2}: \textit{Fortuna cum cornu pomis ficis aut frugibus autumnalibus pleno}.}.

Рог изобилия становится одним из основных изобразительных атрибутов Тихи и Фортуны начиная со II в. до н.э., хотя как символ одарения благами он был также принадлежностью многих других эллинистических и италийских божеств (Плутоса, Деметры, Исиды и др.; см. тж. рис. \ref{pic:CaligulaeSorores})\footcite[Pp. 77--78]{Arya2002}.

% Arnob. VI.25.2: Fortuna cum cornu pomis ficis aut frugibus autumnalibus pleno

%  τὸ δ᾽ ὑμνούμενον ἐκεῖνο τοῦ πλούτου κέρας ἔχει διὰ χειρός, οὐκ ὀπώρας ἀεὶ θαλλούσης μεστόν, ἀλλ᾽ ὅσα φέρει πᾶσα γῆ πᾶσα (B) δὲ θάλαττα καὶ ποταμοὶ καὶ μέταλλα καὶ λιμένες, ἄφθονα καὶ ῥύδην ἐπιχεαμένη. λαμπροὶ δὲ καὶ διαπρεπεῖς ἄνδρες οὐκ ὀλίγοι μετ᾽ αὐτῆς ὁρῶνται,

% Шар
% Ovid. Trist. V.8.7, II.3.56.

\textit{Шар} Фортуны, как следует из вышеприведённых свидетельств, означает переменчивый и непостоянный характер богини. Овидий в <<Скорбных элегиях>>\footnote{\mancite{Ovid. Trist. V.VIII.7--8}: \textit{nec metuis dubio Fortunae stantis in orbe / numen}.} и <<Письмах с Понта>>\footnote{\mancite{Ovid. Ex Ponto II.III.56}: \textit{stantis in orbe deae}.} обращается к поэтическому образу шара Фортуны.
Шар также символизирует \textit{orbis mundi} и является выражением идеи мирового господства. Такой символ появляется не ранее эпохи эллинизма\footcite[Pp. 80--84]{Arya2002}. Вероятно, в этом смысле надо толковать довольно распостраненое изображение Фортуны с кормилом, стоящим на шаре (см. рис. \ref{pic:FortunaMuliebris} и \ref{pic:Cavemalum}).

% Колесо

\textit{Колесо} Фортуны в более чистом виде, чем шар, отражает идею непостоянства и переменчивости, непрерывного движения. Метафору колеса Фортуны использует Цицерон в речи против Пизона, говоря, что тот не страшится поворотов судьбы\footnote{\mancite{Cic. Pis. 10}: \textit{ne tum quidem fortunae rotam pertimescebat}.}. Интересно, что с колесом Фортуны изображалась также Немезида "--- символизм, не вполне понятный нам сегодня\footcite[P. 85]{Arya2002}.

% Cicero Ad Pisonem 22 and Tibullus 1.5.70 directly reflect the fickle and
%haphazard nature of the Roman goddess.
% Тж. с этим атрибутом изображалась Немезида


% Кормило

\textit{Кормило} в руке Фортуны "--- довольно очевидное указание на связь богини с судоходством и торговлей\footcite[P. 77]{Arya2002}. Подчеркнём также отмеченное нами выше положение первых храмов Фортуны на берегу Тибра. Другое толкование, даваемое античными авторами, заключающееся в том, что Тиха или Фортуна управляет жизнью людей как кораблём (Dio Chrys. LXIII.7, LXIV.7), восходит ещё к эллинистическому времени: подобную сентенцию мы можем найти у Менандра\footnote{\graecafn{T'uqh kubern~a| p'anta}, \cite[P. 213, fr. III.a, 10]{FCG_IV_1841}.}. Это представление проникает в римскую литературу эпохи Республики: \textit{Fortuna gubernatrix} упоминается у Теренция\footnote{Terent. Eun. 1046: \textit{fortunam conlaudem quae gubernatrix fuit}.}, \textit{Fortuna gubernans} "--- у Лукреция (V.107). Корабельный руль "--- изобразительный атрибут, пожалуй, в наивысшей степени являющийся отличительным признаком именно Фортуны.

% In the West, Terence Eunuchus 1046 and Lucretius 5.107 are the earliest
%Latin writers to define Fortuna as gubernans (guiding) and gubernatrix
%(conductress), titles which, in fact, would become very common epithets of the
%goddess in the imperial period.246

% Головной убор - башенная корона или полос. Башенная корона не является характеризующим символом Фортуны, как в случае с Тихой.

\textit{Головной убор} Фортуны мог быть различным и, в целом, мы видим изображения, характерные и для других богинь. На монетах республиканской эпохи представлена голова Фортуны в диадеме (рис. \ref{pic:RRC440} и \ref{pic:RRC513}). Голову Фортуны могла увенчивать круглая корона "--- \graeca{p'olos} (рис. \ref{pic:Cavemalum} и \ref{pic:Fortuna2cent}). Эллинистическая Тиха (а также Кибела, в римском варианте Великая Мать) изображалась в башенной короне (\textit{corona muralis})\footcite[P. 5]{Arya2002}. Такое головное украшение характерно для изображений Фортуны, в основном, императорского периода\footcite[P. 76]{Arya2002}, однако не является её определяющим признаком. Так, Рим не знает аналога распространённому эллинистическому монетному типу, изображающему бюст или голову Тихи в башенной короне\footcite[Sp. 1370]{Tyche1915Wasser}. Мы можем связать это с тем, что, как было указано нами выше, римская Фортуна не воспринималась в качестве покровительницы городов наподобие \graeca{T'uqh p'olews}.

% Выводы. Изобразительные атрибуты и их толкование - также часть образа Тихи-Фортуны.

Приведённые факты свидетельствуют, что символы, которыми наделяли изображения Фортуны, с одной стороны, показывают её как богиню-благодетельницу, с другой же "--- отражают переменчивость и непостоянство её характера. Этот последний образ, как мы показали выше, возник в сочинениях античных писателей и получил в современной науке название \graeca{T'uqh}-\textit{Fortuna}. Описания и толкования атрибутов богини, принадлежащие авторам начала эпохи Империи не могут характеризовать \textit{исконно римское} представление о Фортуне, чьи корни восходят к царскому периоду, и отражают то влияние, которое оказал на Фортуну образ эллинистической Тихи.

Тем не менее, в практическом употреблении этих символов были расставлены определённо римские акценты. Помимо того, что \textit{corona muralis} не являлась характерным атрибутом Фортуны, в римских изображениях, как правило, эта богиня наделяется двумя типическими символами: рогом изобилия и кормилом, часто стоящим на шаре. Первый из этих символов рисует Фортуну как покровительницу, дарующую различные блага; второй же "--- говорит о могуществе Фортуны, о том, что она имеет власть управлять жизнью людей и всего мира (на что указывает шар). Таким образом, в римских изображениях Фортуны в меньшей степени подчёркивается идея непостоянной и переменчивой силы, вносящей внезапные изменения в жизнь.

\section{Мифы о Фортуне}\label{Myths}

Мифологические представления тесным образом связаны с культом и обрядовой практикой. Обращаясь к мифам о богах, мы можем составить себе более полное представление об образе тех высших сил, которым древние люди возносили молитвы и приносили жертвы. Римляне не создали столь яркой и сложной мифологии, подобной той, какую оставили нам греки, видимо, не испытывая в этом потребности. Тем не менее, самобытные римские мифы существовали, и мы не можем пройти мимо такого важного исторического свидетельства, как легенды и сказания о Фортуне.

Сервий Туллий, основавший, согласно античной традиции, культ Фортуны в Риме, как считалось, пользовался особым её покровительством. Плутарх писал, что многие из римлян придерживаются того мнения, что Сервий, рождённый от рабыни, сделался царём Рима благодаря заступничеству Фортуны\footnote{\mancite{Plut. Quae. Rom. 106, Mor. 289~C}: \graecafn{>~ar' <'oti Serou'iw| kat`a t'uqhn, <'ws fasin, >ek jerapain'idos genom'enw| basile~usai t~hs <R'wmhs >epifan~ws <up~hrxen; o<'utw g'ar o<i pollo`i <Rwma'iwn <upeil'hfasin}.}.

% ἆρ᾽ ὅτι Σερουίῳ κατὰ τύχην, ὥς φασιν, ἐκ θεραπαινίδος γενομένῳ βασιλεῦσαι τῆς Ῥώμης ἐπιφανῶς ὑπῆρξεν; οὕτω γάρ οἱ πολλοὶ Ῥωμαίων ὑπειλήφασιν. 

% (Plut. Mor. 289~C); этого, продолжает Плутарх, придерживаются многие римляне\footnote{\graecafn{o<'utw g'ar o<i pollo`i <Rwma'iwn <upeil'hfasin}, Plut. Quae. Rom. 106, Mor. 289~C.} (Ibid.).

% Про то, какие мифы сообщает нам Овидий про эту статую
Как мы отмечали в п.~\ref{FortunaInForoBoario}, до смерти Сеяна среди римлян считалось, что в храме Фортуны на Бычьем форуме находится статуя Сервия, покрытая тогой. Овидий в <<Фастах>> пишет, что неясно, по какой причине она покрыта\footnote{\mancite{Fast. VI.571--572}: \textit{\ldots{}causa latendi / discrepat, et dubium me quoque mentis habet}.}, однако приводит несколько версий, объясняющих, отчего это так. Согласно первой из них (Ovid. Fast. IV.573--580), Фортуна вступала в любовную связь с Сервием, и тот скрыл лицо тогой от стыда (Ibid. 579: \textit{nunc pudet}). Богиня, пишет Овидий, проникала в свой дом через окошко\footnote{\mancite{Ibid. 577}: \textit{nocte domum parva solita est intrare fenestra}.}, и поэтому-то есть в Риме ворота, которые зовутся Fenestella (Ibid. 578: \textit{unde Fenestellae nomina porta tenet}). Плутарх также пишет о Сервии, что \textit{<<Фортуна с ним разделяла ложе, входя к нему в опочивальню через какое-то окно, которое сейчас называют воротами FENESTELLA>>}\footnote{Plut. De Fort. Rom. 10, Mor. 322 E--F: \graecafn{sune~inai doke~in a>ut~w| t`hn T'uqhn di'a tinos jur'idos kataba'inousan e>is t`o dwm'ation n~un Fen'estellan p'ulhn kalo~usin}.}. К этим же сказаниям Плутарх обращается и в Римских вопросах: \textit{<<Почему одни из ворот Рима называются ``Окно'', а соседнее здание именуется ``Спальней Фортуны''?>>}\footnote{Plut. Quae. Rom. 36, Mor. 273~B: \graecafn{di`a t'i p'ulhn m'ian jur'ida kalo~usi, t`hn g`ar fen'estran to~uto shma'inei, ka`i par> a>ut`hn <o kaloum'enos T'uqhs j'alam'os >esti};}. Одно из толкований, которые даёт греческий автор, звучит так: \textit{<<по преданию, царь Сервий, бывший счастливейшим, будто бы сочетался с Фортуной, входившей к нему через окно>>}\footnote{Ibid. 273~B--C: \graecafn{S'erbios <o basile`us eutuq'estatos gen'omenos d'oxan >'esqe, t~h| T'uqh| sune~inai foit'wsh| di`a jur'idos pros a>ut'on}.}.

Вернёмся, однако, к <<Фастам>> Овидия и к его толкованиям того факта, что голова священного изображения в храме Фортуны была покрыта тогой. Согласно второй версии, излагаемой римским поэтом (Ovid. Fast. VI.581--584), народ был так опечален смертью любимого им царя, что люди рыдали перед статуей, не переставая, пока она не оказалась покрыта тогой\footnote{Ovid. Fast. VI.574: \textit{donec eum positis occuluere togis}.}, "--- очевидно, произошло это чудесным образом "--- как это было принято, видимо, в античности у умирающих\footnote{Согласно Платону, лицо Сократа после смерти было закрыто (Plat. Phaed. 118~A); Цезарь так же, как сообщает Светоний, умирая, накинул тогу на голову (Suet. Div. Jul. 82.2).}. По третьей версии (Ibid. 611--621), Туллия, дочь царя, вероломно убив его, посмела даже войти в храм, освящённый Сервием, где стояло его же изображение, и вот тогда, сообщает Овидий, статуя подала голос:  \textit{<<Говорят, что он закрыл глаза рукой, / И прозвучал голос: очи скройте мои, / Чтобы они не видели ненавистного чела моей дочери>>}\footnote{\mancite{Ovid. Fast.VI.614--616}: \textit{dicitur hoc oculis opposuisse manum, / et vox audita est `voltus abscondite nostros, / ne natae videant ora nefanda meae.'}}.

Говорить человеческим голосом, как передаёт нам античная традиция, умела также и статуя из храма Женской Фортуны (см. п. \ref{FortunaMuliebris}). Дионисий Галикарнасский, ссылаясь на \textit{<<записи верховных жрецов>>} (Dion. Hal. VIII.56.1: \graeca{<ws <ai <ierofant~wn peri'exousi graya'i}) "--- т.е., видимо, на те же tabulae pontificorum, к которым должны восходить его подробные рассказы об основании и освящении храма, "--- приводит легенду \textit{<<о явлении богини, случившемся в то время не один раз, а даже дважды>>}\footnote{Dion. Hal. VIII.56.1: \graecafn{t'o dhl~wsai t`hn genom'enhn >epif'aneian t~hs jeo~u kat> >eke~inon t`on qr'onon o>uq <'apax, >all`a ka`i d'is}.}. Плутарх (Plut. Coriol. 37) приводит ту же самую легенду, что и Дионисий (Dion. Hal. VIII.56.2), правда, без ссылок на какие-либо документы. В общих чертах их рассказы совпадают: сенат и народ на общественные деньги поставили одну статую богини в храм, женщины же, на собственные средства, ещё одну, и вот эта-то вторая статуя во время обряда заговорила человеческим голосом, повторив фразу дважды. Дионисий передаёт её слова так: \textit{<<По священному закону города вы, жёны, принесли меня в дар>>}\footnote{Dion. Hal. VIII.56.2--3: \graecafn{<os'iw| p'olews n'omw| guna~ikes gameta`i ded'wkat'e me}.}, Плутарх же так: \textit{<<Угоден богам, о жёны, ваш дар>>}\footnote{Plut. Coriol. 37.3: \graecafn{jeofile~i me jesm~w| guna~ikes ded'wkate}.}. 

% ‘θεοφιλεῖ με θεσμῷ γυναῖκες δεδώκατε (Plut. Coriol. 37.3)

%Далее он описывает случившееся со всеми возможными подробностями: <<когда сенат постановил оплатить все расходы на храм и статую за общественный счёт, а женщины соорудили другую статую на те средства, которые собрали, и обе они одновременно были принесены в дар в первый день освящения храма, одно из изваяний "--- то, которое установили женщины, "--- в присутствии многих произнесло отчётливо и громко фразу \ldots ``По священному закону города вы, жены, принесли меня в дар''>> (\graeca{}, Ibid. 2).

% τὸ δηλῶσαι τὴν γενομένην ἐπιφάνειαν τῆς θεοῦ κατ᾽ ἐκεῖνον τὸν χρόνον οὐχ ἅπαξ, ἀλλὰ καὶ δίς (Dion. Hal. VIII.56.1)

% ὅτι τῆς βουλῆς ψηφισαμένης ἐκ τοῦ δημοσίου πάσας ἐπιχορηγηθῆναι τὰς εἰς τὸν νεών τε καὶ τὸ ξόανον δαπάνας, ἕτερον δ᾽ ἄγαλμα κατασκευασαμένων τῶν γυναικῶν ἀφ᾽ ὧν αὐταὶ συνήνεγκαν [p. 210] χρημάτων, ἀνατεθέντων τ᾽ αὐτῶν ἀμφοτέρων ἅμα ἐν τῇ πρώτῃ τῆς ἀνιερώσεως ἡμέρᾳ, θάτερον τῶν ἀφιδρυμάτων, ὃ κατεσκευάσανθ᾽ αἱ γυναῖκες, ἐφθέγξατο πολλῶν παρουσῶν γλώττῃ Λατίνῃ φωνὴν εὐσύνετόν τε καὶ γεγωνόν: ἧς ἐστι φωνῆς ἐξερμηνευόμενος ὁ νοῦς εἰς τὴν Ἑλλάδα διάλεκτον τοιόσδε: ὁσίῳ (3)  πόλεως νόμῳ γυναῖκες γαμεταὶ δεδώκατέ με (Dion. Hal. VIII.56.2--3)

% Что пишет о говорящей статуе Валерий Максим
Валерий Максим сообщает, что богиня \textit{<<не единожды, но дважды>>} (Val. Max. I.8.4: \textit{non semel sed bis}) произнесла следующие слова: \textit{rite me, matronae, dedistis riteque dedicastis} (Ibid.) (\textit{<<Правильно, матроны, вы доверились мне и правильно посвятили себя мне>>}). При некоторых разногласиях между авторами в пересказе текста, все они одинаково передают смысл слов богини: Фортуна высказала матронам своё одобрение за то, что те поставили ей вторую статую за собственный счёт.

Валерий Максим передаёт ещё одну легенду (Val. Max. I.8.9): когда Л.~Лентул проплывал мимо побережья, то увидел дым от погребального костра, на котором сжигали Помпея Великого, но он этого не знал. В тот момент Лентулу явилась Фортуна, покрасневшая от стыда\footnote{\mancite{Val. Max. I.8.9}: \textit{Fortunae erubescendum rogum vidisset}.}, и тогда он спросил у товарищей: \textit{<<Как знать, не Помпея ли сжигают на этом костре?>>}; это явление богини, по словам Валерия Максима, было чудом\footnote{\mancite{Ibid.}: \textit{divinitus missae vocis miraculum est}.}.

Дион Кассий сообщает, что когда Сеян приносил жертвы перед статуей Фортуны, которую забрал из храма на Бычьем форуме, то богиня отвернулась (\graeca{apostref'omenon}) от него\footnote{\mancite{Dio Cass. LVIII.7.2--3}: \graecafn{T'uqhs t'e ti >'agalma, <`o >egeg'onei m'en, <'ws fasi, Toull'iou to~u basile'usant'os pote >en t~h| <R'wmh|, t'ote d`e <o Se"ian'os o>'ikoi te e>~iqe ka`i meg'alws >'hgallen, a>ut'os te j'uwn e>~iden apostref'omenon}.}.

Интересно сообщение Светония о том, как будущий римский император Гальба поклонялся Фортуне частным образом. Согласно римскому биографу, достигнув совершеннолетия (т.е. 1 января 14 г. н.э.), Гальба увидел во сне Фортуну\footnote{Suet. Galb. 4.3: \textit{Sumpta virili toga, somniavit Fortunam}.}. Она, со слов Светония, сообщала Гальбе, что устала ждать на пороге, и, если он не поторопится, то она достанется первому встречному\footnote{Ibid.: \textit{stare se ante fores defessam, et nisi ocius reciperetur, cuicumque obvio praedae futuram}.}. И действительно, у порога он нашёл медное изображение Фортуны длиной более локтя\footnote{Ibid.: \textit{simulacrum aeneum deae cubitali maius iuxta limen invenit}.}. Гальба отнес его в Тускул, где обычно проводил лето, и <<посвятил ему комнату в своем доме и с этих пор каждый месяц почитал его жертвами и каждый год — ночными празднествами>>\footnote{Ibid.: \textit{avexit et in parte aedium consecrato menstruis deinceps supplicationibus et pervigilio anniversario coluit}.}.

Далее Светоний сообщает, что Гальба, уже будучи императором, намеревался сделать своей Фортуне подношение в виде ожерелья, украшенного жемчугом и драгоценными камнями, однако, в итоге, посвятил его Капитолийской Венере\footnote{Suet. Galb. 18.2: \textit{Monile, margaritis gemmisque consertum, ad ornandam Fortunam suam Tusculanam ex omni gaza secreverat; id \ldots{} Capitolinae Veneri dedicavit}.}. На следующую ночь, пишет Светоний, Гальбе явилась во сне Фортуна, жалуясь, что ее лишили подарка, и грозясь, что теперь и она у него отнимет все, что дала\footnote{Ibid.: \textit{proxima nocte somniavit specie Fortunae querentis fraudatam se dono destinato, minantisque erepturam et ipsam quae dedisset}.}. Светоний повествует дальше, что Гальба пытался замолить этот свой грех перед Фортуной, но безуспешно, и это послужило одним из недобрых знамений, предрекавших его кончину (Galb. 18.2).

Эти легенды и сказания, о которых мы узнаём из сочинений античных писателей, имеют своим источником, очевидно, рассказы, передававшиеся в народной среде. Тесная связь этих мифов с римской историей и римскими историческими деятелями доказывает их самобытное, \textit{исконно римское} происхождение. Особенно важны для нас мифы, связанные с Сервием Туллием, фигурой, безусловно, необычайно популярной среди плебеев. То, что мифологическое сознание римлян связывало необыкновенный подъём, который испытал в своей судьбе Сервий, с покровительством Фортуны, показывает также, каким большим почитанием эта богиня пользовалась в плебейской среде.

Несмотря на довольно малое количество мифов о Фортуне, мы всё же в состоянии проследить некоторую трансформацию, которую её образ претерпевает со временем. В легендах о Сервии Туллии, восходящих к царским или, по крайней мере, раннереспубликанским временам, Фортуна представлена как благодетельница царя, а его трагическая гибель никак не связана с кознями или немилостью богини. А вот падение Сеяна и Гальбы миф напрямую объясняет тем, что Фортуна от них отвернулась, причём в случае с Сеяном "--- в буквальном смысле слова. В позднейших представлениях Фортуна является более грозным божеством, не только благодетельницей, но и карающей силой; она приобретает, следовательно, амбивалентный характер, что перекликается с тем её образом, который мы находим у древних писателей.

В легендах Фортуна предстаёт перед нами в антропоморфном, очеловеченном облике, более того, в облике подчёркнуто женском. Она не только вступает в любовную связь с Сервием Туллием, но также является рассерженной женщиной Гальбе, а от Сеяна она демонстративно отворачивается "--- жест обиженной фемины. Римляне, полагая, что голос богини обращается к смертным, вкладывают в её уста моральные суждения. Фортуна, явившаяся Л. Лентулу, безусловно, возмущена несправедливостью и вероломством, жертвой которых пал Помпей.

Рассмотренные нами мифы показывают, что римляне эпохи Республики и начала Империи воспринимали Фортуну как богиню, наделённую человеческим обликом и обладающую яркой индивидуальностью "--- насколько вообще римлянам было свойственно наделять богов индивидуальными чертами характера. В римских легендах Фортуна была способна испытывать эмоции и выражать своё нравственное отношение к происходящему среди смертных, что разительным образом контрастирует с портретом Фортуны античных писателей, учёных и мыслителей, изобразивших богиню, стоящую вне морали, покровительницу как достойных, так и недостойных.

%Обратим особое внимание на тот факт, что Фортуна произнесла не просто какие-то слова, а выразила своё одобрение действиям матрон: таким образом, в представлении римлян эта богиня являла смертным свой глас, чтобы высказывать моральные суждения.

%\section{Дуальность образа Фортуны}

%Наш обзор показывает, что портрет Фортуны у античных писателей-интеллектуалов весьма сильно отличается от мифологических представлений, порождённых народной средой, и того образа, что находит выражение в культовой практике.

% Образ богини в культе и изображениях

%Римляне поклонялись Фортуне как могущественной богине, покровительствовавшей людям, группам людей и конкретным местам (Fortuna Huius Loci). Указанная нами близость Фортуны к культам материнского божества и ларов/гениев, говорит о том, что она воспринималась как защитница и охранительница, проявляющая заботу, к которой можно взывать о помощи.

%В римских изображениях, как правило, Фортуна наделяется двумя характерными атрибутами: рогом изобилия и кормилом, часто стоящим на шаре. Первый из этих символов рисует Фортуну как покровительницу, дарующую различные блага; второй же "--- говорит о могуществе Фортуны, о том, что она имеет власть управлять жизнью людей и всего мира (на что указывает шар). Несмотря на то, что эти символы Фортуны римляне переняли у греков, в их практическом употреблении были расставлены характерно римские акценты.

%Могущество Фортуны находит выражение и в сказаниях, ведь её немилось, согласно мифам, погубила Сеяна и Гальбу, а доброе отношение "--- возвысило Сервия.

% Образ у писателей. Это плод эллинистического влияния.

%Фортуна же писателей представляет собой не столько <<полноценную>> богиню, сколько некую стихийную силу, слепую, бездумную и не разбирающую, что творит. Это божество не обладает полноценной личностью, и если можно говорить о Фортуне как об олицетворённом понятии, то эта характеристика в наибольшей степени подходит к обрисованному нами образу \graeca{T'uqh}-\textit{Fortuna}.

%Именно здесь мы наблюдаем в наиболее чистом виде эллинистическое влияние на представления о Фортуне.

% Выводы

%Мы должны заключить, что культ Фортуны развивался в республиканском Риме своим, самобытным путём, а представления античных интеллектуалов о Тихе-Фортуне жили отдельно от него и возникли под влиянием греческой культуры. На основании нашего исследования религиозных представлений о Фортуне, выраженных в мифах и культовой практике, мы имеем право не доверять Плутарху, Плинию и другим писателям, переносившим представление о Тихе-Фортуне на римский культ.

% хотя соотнесение подобных представлений у различных писателей (Плутарха, Плиния, Сенеки) с культовой практикой может смутить неискушённого читателя.


% Эти обожествляемые понятия являются положительными качествами в том смысле, что они могут присутствовать, а могут и отсутствовать. Противоположные им качества (перечислить) "--- также положительные в смысле возможности присутствия или отсутствия, однако они несут отрицательную моральную нагрузку, поэтому римлянами они не обожествлялись. (Цицерон осуждал греков за обожествление отрицательных качеств "--- вот откуда растут ноги у образа Тихи-Фортуны!)