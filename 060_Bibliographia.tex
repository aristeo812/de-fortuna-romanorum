% Источники и библиография
\chapter*{Источники и библиография}
\addcontentsline{toc}{chapter}{Источники и библиография}


% Это список источников
\nocite{Colini1938source,RRC_II,Degrassi1963,Degrassi1957,ILS_2_1,Richardson1992source,
Festus1826,AurVict1964trans,Apuleus2008trans,Arnobius2008trans,ValMax2007trans,Varro1963trans,
Suet1993trans,DionHal2005,Tertullian1994trans,Tacitus1993_I,Tacitus1993_II,SenEp1977trans,
Plautus1987,PlutLives1994,PlutDeFortRom1979,PlutQuaeRom1976_1,PlutQuaeRom1976_2,
Ovid1973trans,Livius2005,Cicero2000trans,Cicero1985trans,Cicero1962_I,Fronto1988,Fronto1919,
RemainsofLatin1935,PlutVitae1956,Pliny1938,Livius,RIC_I_1984,Moralia1974Teubner,
Moralia1971Teubner,DioVII_1955,DioChrys1951_V,RIC_III_1930,CIL_VI_Add_1912,AurVict1911,
VarroDeLL1910,TacHist1900,Vitruv1899,Suetonius1898,CIL_I_1893,HorCarm1883,Augustin1877,
CIL_VI_1_1876,Arnobius1875,Nonius1872,ValMax1865,OvidFast1846,ApulMet1842,TacAnn1829,
TertOpera1828}

%\nocite{*}

% Печатаем список источников по рубрикам

%\chapter*{Источники и литература}
%\addcontentsline{toc}{chapter}{Источники и литература}


\section*{Список источников}
\phantomsection
\addcontentsline{toc}{section}{Список источников}

\printbibliography[keyword=originales,heading=subsubbibliography,title={Публикации трудов античных писателей}]

\printbibliography[keyword=translationes,heading=subsubbibliography,title={Русские переводы античных писателей}]

\printbibliography[keyword=inscriptiones,heading=subsubbibliography,title={Публикации надписей}]

\printbibliography[keyword=coinage,heading=subsubbibliography,title={Публикации нумизматических источников}]

\printbibliography[keyword=archaeologia,heading=subsubbibliography,title={Археологические источники}]

% Общая библиография

\pagebreak
\printbibliography[notkeyword=source,heading=mysubbibintoc]


% Список сокращений

\pagebreak
\section*{Список сокращений}
\phantomsection
\addcontentsline{toc}{section}{Список сокращений}

ВДИ "--- Вестник древней истории.

CIL "--- Corpus Inscriptionum Latinarum.

ILLRP "--- Inscriptiones Latinae Liberae Rei Publicae.

ILS "--- Inscriptiones Latinae Selectae.

RIC "--- Roman Imperial Coinage.

RRC "--- Roman Republican Coinage.

