\chapter{Фортуна в эпоху конца Республики "--- начала Империи}

% Короткое введение
Начиная с I в. до н.э. резко возрастает количество письменных источников по римской истории, дошедших до нас. Соответственно, возрастает и число свидетельств о храмах, алтарях и культе Фортуны в Риме. Отметим, что, воссоздавая историю культа в предыдущие эпохи, мы, в основном, опирались на произведения, относящиеся к этому же периоду. Однако в этой главе мы отвлечёмся от исследования ретроспективных штудий античных писателей и будем рассматривать только свидетельства авторов о современных им храмах и алтарях Фортуны, отмечая сообщения о возведении новых культовых сооружений, относящихся к этому периоду. Безусловно, все эти места почитания имели собственную историю, однако мы вынуждены расписаться в том, что для большинства из них она нам неизвестна. Некоторые из таких храмов мы уже рассмотрели в нашем повествовании выше: мы, по возможности, будем избегать повторов, отсылая читателя к материалам предыдущих глав.
% Написать дальше про то, что проявляется многовалентность образа Фортуны, в частности, в том, что начинают поклоняться Злой Судьбе, хотя число храмов Доброй Фортуны значительно превышает число храмов недоброй. Но деление Фортуны на добрую-недобрую недостаточно хорошо отражает многообразие проявлений этой богини.


Одно из самых ранних письменных свидетельств этой эпохи принадлежит Цицерону. В трактате <<О законах>> он противопоставляет культы дурных и благих божеств, указывая, что <<обожествлять следует доблести, а не пороки>>\footnote{\textit{Virtutes enim, non vitia consecrari decet}, Cic. De leg. II.28.} (Cic. De leg. II.28). К числу последних, говоря о Фортуне, он относит алтарь Злой Судьбы на Эсквилине (\textit{Esquiliis Malae Fortunae}, Ibid.). Об этом же алтаре Цицерон пишет и в произведении <<О природе богов>>, перечисляя его среди других <<дурных>> божеств: <<Ведь на Палатинском холме есть храм Лихорадки, а возле храмов в честь ларов там находится храм, посвященный Орбоне, и на Эсквилине мы видим алтарь Злой Судьбе>>\footnote{\textit{Febris enim fanum in Palatio et Orbonae ad aedem Larum et aram Malae Fortunae Exquiliis consecratam videmus}, Cic. De Nat. Deor. III.63.} (Cic. De Nat. Deor. III.63). Этих же божеств в том же порядке, почти дословно цитируя Цицерона, перечисляет Плиний Старший в своей <<Естественной истории>>\footnote{\textit{Ideoque etiam publice Febris fanum in Palatio dicatum est, Orbonae ad aedem Larum, ara et Malae Fortunae Esquiliis}, Plin. NH II.16.} (Plin. NH II.16). Отметим, что выражение \textit{mala fortuna} встречается ещё у Плавта: \textit{Malam fortunam in aedis te adduxi meas} (Plaut. Rud. 501), однако неизвестно, было ли это отвлечённое понятие злой судьбы обожествлено в те времена или нет.

Цицерон далее в произведении <<О законах>> противопоставляет Дурной Фортуне благие <<ипостаси>> этой богини: обожествлять следует, пишет он, или Фортуну Нынешнего дня, <<ибо она обладает силой во все дни>>\footnote{\textit{Fortunaque sit [consecrata "--- В.~Ж.] vel Huiusce diei — nam valet in omnis dies}, Cic. De leg. II.28.} (Cic. De leg. II.28), или Оборачивающуюся, дабы оказать помощь\footnote{\textit{vel Respiciens ad opem ferendam}, Ibid.} (Ibid.), или Случай, в котором наиболее выражаются неопределённые обстоятельства\footnote{\textit{vel Fors in quo incerti casus significantur magis}, Ibid.} (Ibid.), или  Перворожденную, спутницу от рождения\footnote{\textit{vel Primigenia a gignendo comes}, Ibid.} (Ibid.). О Фортуне Оборачивающейся и Примигении речь шла выше (см. с.~\pageref{FortunaRescipiens} и \pageref{FortunaPrimigenia}).

% Ибо надлежит обожествлять доблести, а не пороки. И древний алтарь Горячки на Палатине, как и алтарь Злой Судьбы на Эсквилине и все ненавистное в этом же роде должно быть удалено. А если надо придумывать имена, то лучше выбрать имя Вики Поты [оно происходит от слов «побеждать» и «овладевать»] и Статы [останавливающей отступающие войска], прозвания Статора и Непобедимого Юпитера и названия желательных качеств — Здоровья, Чести, Благоденствия, Победы66, так как ожидание всего хорошего укрепляет дух; ведь Калатин67 с полным к тому основанием обожествил Надежду. Что касается Фортуны, то пусть это будет либо Фортуна Нынешнего дня (ибо она обладает силой во все дни), либо Оглядывающаяся назад, дабы оказать помощь, либо Судьба, более выражающая случайность, либо Перворожденная, спутница наша со времени рождения на свет68,

% Храмы и алтари Фортуны, согласно сообщениям авторов:

% Что пишет Цицерон: он противопоставляет культы дурных божеств культам благих богов.


% Fortuna Huiusque Diei
Фортуну Сегодняшнего Дня\footnote{См. Wissowa, G. Op. cit. S. 211. Latte, K. Op. cit. S. 179. Altheim, F. A history of Roman religion. P. 190.} (Fortuna Huiusque Diei) упоминает также Плиний Старший, указывая, что Павел Эмилий посвятил храм Минерве рядом со святилищем Фортуны Сегодняшнего Дня\footnote{\textit{\ldots{}Minervam, quam Romae Paulus Aemilius ad aedem fortunae Huiusce Diei dicavit}, Plin. NH XXXIV.54.} (NH XXXIV.54). Плутарх сообщает, что во время битвы с кимврами при Верцеллах 30 июля 101 г. до н.э. проконсул Катул, воздев руки, молился и давал обеты Фортуне Сегодняшнего Дня\footnote{\graecafn{e>'uxato d`e ka`i K'atlos <omo'iws >anasq'wn t`as qe~iras kajier'wsein t`hn t'uqhn t~hs <hm'eras >eke'inhs}, Plut. Marius 26.2.} (Plut. Marius 26.2). Отметим, что здесь Фортуна также связывается с идеей военной удачи, однако Плутарх не указывает прямо, как это делает Тит Ливий, что Катул обетовал богине постройку храма. Скорее всего, обета построить храм всё же не было: ведь, если верить процитированному выше отрывку из Плиния, это святилище уже существовало во времена Эмилия Павла, что позволяет положить <<terminus, ante quem>> для даты его возведения не позднее 1-й половины II в. до н.э. Е.~М.~Штаерман относит этот храм к числу тех, которые были построены в довольно широкий промежуток времени <<II "--- нач. I в. до н.э.>>\footnote{Штаерман Е.М. Социальные основы религии древнего Рима. М., 1987. С. 112.}. У. Фаулер предполагает, что 30 июля, в день битвы при Верцеллах, в храме Fortunae Huiusque Diei совершались жертвоприношения\footnote{Warde Fowler, W. Op. cit. P. 165.}. Он считает, что этот храм был основан не ранее битвы при Пидне\footnote{Ibid.}.

% εὔξατο δὲ καὶ Κάτλος ὁμοίως ἀνασχὼν τὰς χεῖρας καθιερώσειν τὴν τύχην τῆς ἡμέρας ἐκείνης (Plut. Marius 26.2)

% Fortuna Rescipiens
%Цицерон пишет также и о Фортуне Оборачивающейся (Fortuna Rescipiens), указывая, что она оборачивается, чтобы оказать помощь (\textit{}).

% Как-то надо осветить то, что Фортуна бывает плохой и хорошей

% Fortuna Mala
%Так, Цицерон в двух местах упоминает об алтаре Злой Судьбы на Эсквилине: в (De leg. II.28) и в (De Nat. Deor. III.63). Последнее произведение дословно цитирует Плиний Старший: \textit{Febris fanum in Palatio dicatum est, Orbonae ad aedem Larum et ara Malae Fortunae Esquiliis} (Plin. NH II.16). Отметим, что выражение \textit{mala fortuna} встречается ещё у Плавта: \textit{Malam fortunam in aedis te adduxi meas} (Plaut. Rud. 501), однако неизвестно, было ли это отвлечённое понятие дурной судьбы обожествлено в те времена или нет. % Что-нибудь ещё добавить про то, что там Плавт пишет


% Изваяния доброй Фортуны
Интересно, что, противопоставляя алтарю Fortunae Malae храмы благих Фортун, Цицерон умалчивает об изваяниях Доброй Фортуны на Капитолии, о которых пишет Плиний, приписывая их авторство, в числе других скульптур, Праксителю\footnote{\textit{Romae Praxitelis opera sunt \ldots Bonae Fortunae simulacra in Capitolio}, Plin. NH XXXVI.23.} (Plin. NH XXXVI.23). Видимо, Плиний здесь пишет о копиях греческих статуй богини Тюхе. К сожалению, об этих изображениях нам больше ничего неизвестно, в том числе, окружались ли они культовым почитанием, и если да, то в какой форме.

% Что пишет Плутарх
Плутарх в трактате <<Об удаче Римлян>> пишет о неком храме Фортуны Сильной (\graeca{>isqur'a}), Доблестной (\graeca{>aristeutik'a}) или Мужественной (\graeca{>andre'ia}): <<Удачу, [которая почитается] у реки, называют словом FORTIN, что означает сильная, доблестная или мужественная, как имеющую силу побеждать всё[, что угодно]. Её храм построили в садах, завещанных Цезарем народу>>\footnote{\graecafn{t`hn d`e pr`os t~w| potam~w| T'uqhn ``f'ortin'' in kalo~usin <'oper >est`in >isqur`an >`h >aristeutik`hn >`h >andre'ian, <ws t`o nikhtik`on <ap'antwn kr'atos >'eqousan. ka`i t'on ge na`on a>ut~hs >en to~is <up`o Ka'isaros t~w| d'hmw| kataliefje~isi k'hpois >w|kod'omhsan}, Plut. De Fort. Rom. 5, Mor. 319~A--B.} (Plut. De Fort. Rom. 5, Mor. 319~A--B). И действительно, из сообщения Тацита мы знаем, что в 16 г. н.э. был освящён храм Фортуны на берегу Тибра, в садах, завещанных\footnote{О завещании Цезаря см.: Suet. Div. Jul. 83.2, Plut. Brut.20, Dio Cass. XLIV.35.3.} Цезарем народу\footnote{\textit{aedes Fortis Fortunae Tiberim iuxta in hortis, quos Caesar dictator populo Romano legaverat \ldots dicantur}, Tac. Ann. II.41.} (Tac. Ann. II.41). Однако нам неизвестно из других источников, под каким конкретно именем почитали Фортуну в этом храме. Сообщение же Плутарха наводит нас на мысль, что он мог спутать этот храм с другим, который также, как известно, находился у реки (и мог бы называться \graeca{<`h t~w| potam~w| T'uqh}), а именно, с храмом Fortis Fortunae, каковых было даже два, построенных рядом друг с другом (один "--- царём Сервием Туллием, см. Varr. De Ling. Lat. VI.17, Ovid. Fast. VI.771--784, другой "--- консулом Карвилием, Liv. X.46.14). На ошибочность толкования, данного Плутархом, нас наталкивает тот факт, что нам неизвестно святилище фортуны под названием \textit{Fortuna Fortis} (т.е. Сильная Фортуна), на которое почти однозначно, не допуская разночтений, указывает греческий автор. С другой стороны, словосочетание \textit{Fors Fortuna} имеет такой же генетив (\textit{Fortis Fortunae}), поэтому если человек, плохо знакомый с реалиями римской топографии встретит такое выражение в тексте\footnote{Напр. Ovid. Fast. VI.773: \textit{quam cito venerunt Fortunae Fortis honores!} "--- здесь римский поэт использует инверсию, помещая определяемое слово после определения, и из-за упомянутой идентичности словоформ разглядеть эту фигуру речи может быть затруднительно.}, то он может истолковать его неверно.

% Витрувий и "Три Фортуны"
% Подумать, как изменить конец этого абзаца
Таким образом, согласно рассмотренным выше источникам, на берегу Тибра находились три храма Фортуны\footnote{Warde Fowler, W. Op. cit. P. 162.}: это два храма Fortis Fortunae, один построенный Сервием Туллием, во-вторых, другой возведённый консулом Карвилием, а третьим было святилище Фортуны в садах Цезаря, освящённое, согласно Тациту, в 16 г. н.э. Интересно указание Витрувия на то, что у Коллинских ворот находились сразу три храма Фортуны\footnote{Wissowa, G. Op. cit. S. 211; Arya, D.A. Op. cit. P. 187, 257.}, поэтому само место носило название <<Три Фортуны>>: <<Экземпляр этого [стиля, о котором шла речь у Витрувия "--- В.~Ж.] будет у трёх Фортун, из трёх тот, что ближе к Коллинским воротам>>\footnote{\textit{Huius autem exemplar erit ad tres Fortunas ex tribus quod est proxime portam Collinam}, Vitruv. 3.2.2.} (Vitruv. 3.2.2). Однако Витрувий подразумевал под этими <<Тремя Фортунами>>, видимо, другие храмы, т.к. локализация их у Коллинских ворот не позволяет соотнести их с теми святилищами, которые находились на берегу Тибра. Если это так, то мы можем заключить, что в Риме существовало по крайней мере два места, где целых три храма Фортуны стояли рядом друг с другом. Это служит ещё одним доказательством множественности храмов этой богини в Риме и популярности её культа.



%  τὴν δὲ πρὸς τῷ ποταμῷ Τύχην ‘φόρτιν’ ιν (319 B) καλοῦσιν ὅπερ ἐστὶν ἰσχυρὰν ἢ ἀριστευτικὴν ἢ ἀνδρείαν, ὡς τὸ νικητικὸν ἁπάντων κράτος ἔχουσαν. καὶ τόν γε ναὸν αὐτῆς ἐν τοῖς ὑπὸ Καίσαρος τῷ δήμῳ καταλειφθεῖσι κήποις ᾠκοδόμησαν (Plut. De Fort Rom. 5, Mor. 319 A--B)

Существовало в Риме и святилище Фортуны-<<Птицеловки>> (греч. \graeca{>ixeutr'ia}), название которой Плутарх толкует следующим образом: <<Есть на Палатине \ldots храм Фортуны-``Птицеловки'' "--- название хотя и смешное, но иносказанием наводящее на размышление о природе судьбы: она как бы издали заманивает и цепко удерживает всё, что к ней прикоснулось>>\footnote{\graecafn{\ldots{}>estin >en Palat'iw|, ka`i t`o t~hs >ixeutr'ias [T'uqhs <ier'on}\textit{ "--- В.~Ж.}\graecafn{], e>i ka`i gelo~ion, >all> >'eqon >ek metafor~as >anaje'wrhsin, o<'ion <elko'ushs t`a p'orrw ka`i krato'ushs sumprosisq'omena}, Plut. De Fort. Rom. 10, Mor. 322~F.} (Plut. De Fort. Rom. 10, Mor. 322~F). В <<Римских вопросах>> он также говорит об этом храме, упоминая его латинское название (Viscata) и давая схожее объяснение такой эпиклезе: <<\ldots{}есть даже святилище Фортуны-``Птицеловки'' (Viscata — как называют ее [римляне]); эта Фортуна как бы издали уловляет нас и крепко держит в путах обстоятельств>>\footnote{\graecafn{T'uqhs >ixeutr'ias <ier'on >estin, <`hn bisk~atan >onom'azousin, <ws p'orrwjen <hm~wn <aliskom'enwn <up> a>ut~hs ka`i prosisqom'enwn to~is pr'agmasin}, Plut. Quae. Rom. 74, Mor. 281~E.} (Plut. Quae. Rom. 74, Mor. 281~E). Упоминание Плутархом этого храма и толкование, данное им, помогает нам лучше понять оксюморон Сенеки из его VIII письма Луцилию; там он, ведя речь о дарах фортуны (\textit{munera fortunae}, Sen. Ep. VIII.3), предлагает за лучшее считать их всё же её кознями (\textit{insidiae}, Ibid.): <<Кто из вас хочет прожить наиболее безопасную жизнь, пусть, насколько возможно, избегает этих вымазанных птичьим клеем благодеяний>>\footnote{\textit{Quisquis vestrum tutam agere vitam volet, quantum plurimum potest ista viscata beneficia devitet}, Sen. Ep. VIII.3.} (Ibid.). Сенека в <<письмах>> рассматривает фортуну как отвлечённое понятие, связанное с внешними по отношению к человеку обстоятельствами, однако его \textit{viscata beneficia} есть, скорее всего, аллюзия на действительно существовавший культ Фортуны-<<Птицеловки>>, аллюзия, хорошо понятная как ему, так и его адресату, риторический приём, вполне достойный его мастерства. Храм Фортуне-кознодейке, прозванной Птицеловкой, можно поставить в один ряд с алтарём Fortunae Malae, о котором шла речь выше, как место, где культовым почитанием окружались отрицательные качества этой богини.

% ἐστιν ἐν Παλατίῳ, καὶ τὸ τῆς ἰξευτρίας, εἰ καὶ γελοῖον, ἀλλ᾽ ἔχον ἐκ μεταφορᾶς ἀναθεώρησιν, οἷον ἑλκούσης τὰ πόρρω καὶ κρατούσης συμπροσισχόμενα (Plut. De Fort. Rom. 10, 322~F)

% Τύχης ἰξευτρίας ἱερόν ἐστιν, ἣν βισκᾶταν ὀνομάζουσιν, ὡς πόρρωθεν ἡμῶν ἁλισκομένων ὑπ᾽ αὐτῆς καὶ προσισχομένων τοῖς πράγμασιν (Plut. Quae. Rom. 74, Mor. 281~E)

Впрочем, подобных мест, видимо, было не слишком много "--- если считать, что наши письменные источники более-менее пропорционально отражают действительно существовавшую ситуацию с храмами и алтарями Фортуны "--- ибо количество упоминаний благих <<ипостасей>> Фортуны значительно превышает количество её дурных проявлений в культовой практике, о которых нам известно. Выше мы уже перечисляли список из девяти храмов, сооружение которых Плутарх приписывает Сервию Туллию (см. с.~\pageref{PlutarchosDeServio} и далее), и все они, так или иначе, отражают положительные качества богини. Конечно, Плутарх, как мы выяснили, допускает ошибки в деталях повествования, и мы не можем ему доверять в том, что он приписывает сооружение всех этих храмов царю Сервию, и даже в том, что он локализует определённый храм в конкретном месте, ибо в топографии Рима греческий философ разбирался не очень хорошо, но, в целом, его сообщение о существовании в Риме данных храмов (или, быть может, алтарей) мы можем признать достоверным, ибо в своём изложении он опирался на латинские источники, о чём свидетельствует отсылка его к оригинальным латинским названиям римских топонимов (Plut. Mor. 281~D--E, 319~A, 322~E--F).

% Основание храма F. Redux

% Частный культ Фортуны Гальбы
Интересно сообщение Светония о том, что будущий римский император Гальба поклонялся Фортуне частным образом. Согласно римскому биографу, достигнув совершеннолетия (т.е. 1 января 14 г. н.э.), Гальба увидел во сне Фортуну\footnote{\textit{Sumpta virili toga, somniavit Fortunam}, Suet. Galb. 4.3.} (Suet. Galb. 4.3). Она, со слов Светония, сообщала Гальбе, что устала ждать на пороге, и, если он не поторопится, то она достанется первому встречному\footnote{\textit{stare se ante fores defessam, et nisi ocius reciperetur, cuicumque obvio praedae futuram}, Ibid.} (Ibid.). И действительно, у порога он нашёл медное изображение Фортуны длиной более локтя\footnote{\textit{simulacrum aeneum deae cubitali maius iuxta limen invenit}, Ibid.} (Ibid.). Гальба отнес его в Тускул (<<на своей груди>> "--- \textit{gremio suo}, Ibid.), где обычно проводил лето, и <<посвятил ему комнату в своем доме и с этих пор каждый месяц почитал его жертвами и каждый год — ночными празднествами>>\footnote{\textit{avexit et in parte aedium consecrato menstruis deinceps supplicationibus et pervigilio anniversario coluit}, Ibid.} (Ibid.).

Далее Светоний сообщает, что Гальба, уже будучи императором, намеревался сделать своей Фортуне подношение в виде ожерелья, украшенного жемчугом и драгоценными камнями, однако, в итоге, посвятил его Капитолийской Венере\footnote{\textit{Monile, margaritis gemmisque consertum, ad ornandam Fortunam suam Tusculanam ex omni gaza secreverat; id \ldots{} Capitolinae Veneri dedicavit}, Suet. Galb. 18.2.} (Suet. Galb. 18.2). На следующую ночь, пишет Светоний, Гальбе явилась во сне Фортуна, жалуясь, что ее лишили подарка, и грозясь, что теперь и она у него отнимет все, что дала\footnote{\textit{proxima nocte somniavit specie Fortunae querentis fraudatam se dono destinato, minantisque erepturam et ipsam quae dedisset}, Ibid.} (Ibid.). Светоний повествует дальше, что Гальба пытался замолить этот свой грех перед Фортуной, но безуспешно, и это послужило одним из недобрых знамений, предрекавших его кончину (Ibid.).

Таким образом, Светоний, опираясь на некое предание, которое он пересказывает, связывает высокую судьбу Гальбы с покровительством Фортуны, а его гибель "--- с тем, что богиня счастливого случая и удачи от него отвернулась. В этом представлении личная Фортуна Гальбы обладает даже большей силой, чем Венера Капитолийская, про которую сам император подумал, что она более достойна (\textit{dignius}, Ibid.) подношения; тем не менее, согласно той точке зрения, которую озвучивает Светоний, именно немилость Фортуны погубила Гальбу, в то время как другие боги не смогли ему помочь. Трудно сказать, каков именно был источник этой легенды, возможно, здесь Светоний пересказывает народную молву; вряд ли он мог почерпнуть эти сведения из каких-либо официальных архивов. Если верно первое, то тогда мы можем судить о том большом значении, каковое придавали Фортуне римляне, о том, насколько могущественной и всесильной богиней считали её.

Об этом же говорит и то широкое распространение, которое получил культ Фортуны к середине "--- второй половине I в. н.э. Плутарх, современник эпохи, пишет про <<святилища Удачи, знаменитые и древние, и к тому же разбросанные по наиболее видным местам Города>>\footnote{\graecafn{t'a ge t~hs T'uqhs <ier`a lampr`a ka`i palai'a, ka`i <omo~u ti to~is pr'wtios katamemeigm'ena t~hs p'olews jemel'iois g'egone}, Plut. De Fort. Rom. 5, 318~E.} (Plut. De Fort. Rom. 5, 318~E). На основании всего вышеизложенного мы можем заключить, что здесь Плутарх не грешит против истины и что храмов и алтарей Фортуны разного рода в Риме насчитывалось множество. 

% Цитата Плиния Старшего и её анализ. Распространённость культа Ф. То, что в Риме - отражение того, что по всей Империи.
Подобные оценки Фортуне давали и другие современники эпохи. Широко известна\footnote{См. <<Энциклопедический словарь>>. Т. XXXVI. СПб, 1902. С. 321; Ferguson,~J. Op. cit. P. 79; Arya,~D.~A. Op. cit. P. 134 ff.} пространная цитата из <<Естественной истории>> Плиния (Plin. NH II.22), в которой он даёт характеристику этой богине и популярности её культа:

\begin{quote}
\textit{По всему ведь миру, во всех местах и во всех землях во всех молитвах к одной Фортуне обращаются и одну её призывают, одну обвиняют, к ней одной всё сводят, об одной думают, об одной говорят и одну с бранью почитают, изменчивая~\ldots{} и многие считают её слепой, изменчивой, непостоянной, колеблющейся, покровительницей недостойных, ею всё оплачивается и ею берётся в долг, и во всех человеческих делах она одна записана на обеих этих страницах, и до того мы доходим, что полагаем её вместо бога, когда не уверены, к какому обратиться.}\footnote{\textit{Toto quippe mundo et omnibus locis omnibusque horis omnium vocibus Fortuna sola invocatur ac nominatur, una accusatur, res una agitur, una cogitatur, sola laudatur, sola arguitur et cum conviciis colitur, volubilis~\ldots que, a plerisque vero et caeca existimata, vaga, inconstans, incerta, varia indignorumque fautrix. huic omnia expensa, huic feruntur accepta, et in toto ratione mortalium sola utramque paginam facit, adeoque obnoxiae sumus sortis, ut prorsus ipsa pro deo sit qua deus probatur incertus.} (Plin. NH II.22).}
\end{quote}

\textit{Toto mundo} Плиния "--- это, безусловно, вся Империя, и здесь автор <<Естественной истории>> должен подразумевать не просто римскую Фортуну, но эллинистический культ Тюхе, который, действительно, имел широкое распространение по всей Римской державе\footnote{Ferguson,~J. Op. cit. P. 85--87.}. Однако Рим, как столица, был отражением всей Империи, центром которой он являлся. Для настоящего исследования, впрочем, гораздо важнее та оценка, которую даёт Плиний качествам Фортуны, полагая её противоречивой, изменчивой, непостоянной и несправедливой богиней, но в то же время такой, к помощи которой часто взывают и которая пользуется крайне большой популярностью. Эти качества нашли своё выражение в разнообразных формах культовой практики, во множестве храмов, о которых мы писали выше и в которых поклонялись Фортуне, известной под различными когноменами. %Примечательно, что Плиний априори приписывает богине удачи те черты, которые мы вывели на основании анализа элементов её культа.

Мы вышли за пределы хронологических рамок Республиканского периода и вступили в эпоху ранней Империи. Д.~Арья указывает, что этот рубеж служит границей нового этапа в развитии культа Фортуны, почитание этой богини обретают новые формы в связи с тем, что она начинает ассоциироваться с единоличной властью принцепса\footnote{Arya,~D.~A. Op. cit. P. VIII--IX.}. Обращение к этому предмету уже выходит за рамки нашего исследования. Тем не менее, нам удалось показать, что в I в. н.э. продолжаются те тенденции развития культа Фортуны, которые мы можем проследить ещё со времён эпохи царей, а именно, широкую популярность культа Фортуны, в первую очередь, в низших слоях римского общества, среди плебеев и даже рабов.

Мы можем говорить о многогранных и даже противоречивых вариантах культовой практики. Деление проявлений божества на <<дурные>> и <<благие>>, что, как мы видели, делает Цицерон, в случае с Фортуной недостаточно хорошо отражает разнообразие форм её культа. Мы можем указать те пары противоположностей, между которыми существовал культ Фортуны, те широкие границы, в которые он был заключён: так, существовали публичные и частные культы Фортуны; этой богине поклонялись все, от императора до раба, но имели место и строгие формы поклонения, в которых отправлять обряды могли только добродетельные матроны; Фортуна была покровительницей женщин, мужчин, социальных групп (Всадническая Фортуна), отдельных личностей (Сервий Туллий и император Гальба), всего римского народа\footnote{См. Latte,~K. Op. cit. S. 178; Штаерман~Е.~М. Указ. соч. С. 137.}, но она же могла быть дурной и лукавой богиней. Мы позволим себе присоединиться к мнению К.~Латте, который видел в этом многообразии форм поклонения, в существовании множества отдельных Фортун, почитаемых под различными когноменами, выражение склонности римлян дифференцировать различные проявления высших сил\footnote{Latte,~K. Op. cit. S. 182.}. 

%Множественность храмов и алтарей Фортуны в Риме, а также уникальные сведения о частном культе богини удачи говорят о том, что её значимость выходила далеко за пределы официального государственного культа. Мы наблюдаем к I в н.э. картину, о которой писал К.~Латте\footnote{Latte,~K. Op. cit. S. 182.}: склонность римлян различать проявление всех высших сил привела к возникновению множества отдельных Фортун, почитаемых под различными когноменами.

% (далее про то, что начинается новый этап в развитии культа Фортуны, однако в этот же период, в начале империи происходит максимальное развитие и расцвет тех тенденций, которые продолжались на протяжении Республики).

% (*дописать окончание*)
%  τά γε τῆς Τύχης ἱερὰ λαμπρὰ καὶ παλαιά, καὶ ὁμοῦ τι τοῖς πρώτοις καταμεμειγμένα τῆς πόλεως θεμελίοις γέγονε

% Даже императоры иногда обращаются к Фортуне (по Светонию)

