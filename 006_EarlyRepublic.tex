\chapter{Фортуна в эпоху ранней и средней Республики\label{ChapterEarlyRepublic}}

% Храм Fortunae Muliebris

Выше мы писали о том, что в царское время Фортуна, помимо своего аспекта счастья и удачи, несла на себе функции богини материнства, деторождения и плодородия. Её культ в раннюю республиканскую эпоху во многом сохраняет тот облик, каковой он имел прежде: первым крупным храмом Фортуны, построенным после изгнания царей, был храм Фортуны Женской (Fortunae Muliebris). Согласно римской легенде, это было сделано, чтобы вознаградить женщин, которые отговорили Гая Марция Кориолана вести вольсков на Рим: <<в память о случившемся, "--- пишет Тит Ливий, "--- был воздвигнут и освящён храм Женской Фортуны>>\footnote{\textit{monumento quoque quod esset, templum Fortunae muliebri aedificatum dedicatumque est}, Liv. II.40.11--12.} (Liv. II.40.11--12). 

% Рассказ Дионисия Галикарнасского об этом событии
Дионисий Галикарнасский приводит гораздо более подробный рассказ об этом событии. Он пишет, что женщины стали <<просить, чтобы сенат дозволил им возвести храм Женской Фортуны на том месте, где они обратились с просьбами за свой город, и ежегодно всем вместе приносить ей жертвы в тот день, когда они прекратили войну>>\footnote{\graecafn{>axio~un, d> >epitr'eyai sf'isi t'hn boul`hn >ep`i t'uqhs gunaik~wn <idr'usasjai <ier'on, >en <~w| t`as per`i t~hs p'olews >epoi'hsanto lit`as qwr'iw|, jus'ias te kaj> <'ekaston >'etos a>ut~h| sunio~usas >epitele~in >en <~h| p'olemon >'elusan <hm'era|}, Dion. Hal. VIII.55.3.} (Dion. Hal. VIII.55.3). И далее: <<сенат и народ постановили посвятить богине участок, купленный на общественные средства, и из них же соорудить на нём храм и алтарь, как укажут жрецы, и жертвенных животных доставлять за общественный счёт, а начинать священные обряды той, кого сами женщины назначать исполнять эти обряды>>\footnote{\graecafn{<h boul`h ka`i <o d~hmos >ap`o t~wn koin~wn >eyef'isanto qrhm'atwn t'emen'os t> >wnhj`en kajeirwj~hnai t~h| je~w|, ka`i >en a>ut~w| ne`wn ka`i bwm'on, <ws >`an o<i <ieromn'hmones >exhg~wntai, suntelesj~hnai, jus'ias te pros'agesjai dhmotele~is katarxom'enhs t~wn <ier~wn gunaik'os, <`hn >`an >apode'ixwsin a>uta`i leitourg`on <ier~wn}, Ibid.} (Ibid.). 

% Что пишет Плутарх об этом?
В сочинении <<Об удаче римлян>> Плутарх приводит такой же вариант легенды: <<Храм Женской Удачи они построили ещё до Камилла, после того как благодаря женщинам повернули прочь Марция Кориолана, который вёл на город вольсков>>\footnote{\graecafn{t`o d`e t~hs Gunaike'ias T'uqhs kateskeu'asanto pr`o Kam'illou <'ote M'arkion Kori'olanon >ep'agonta t~h| p'olei O>uolo'uskous >apestr'eyanto di`a t~wn gunaik~wn}, Plut. De Fort. Rom. 5, Mor. 318~F.} (Plut. De Fort. Rom. 5, Mor. 318~F).
% τὸ δὲ τῆς Γυναικείας Τύχης κατεσκευάσαντο πρὸ Καμίλλου ὅτε Μάρκιον Κοριόλανον ἐπάγοντα τῇ πόλει Οὐολούσκους ἀπεστρέψαντο διὰ τῶν γυναικῶν
% Биография Кориолана
Рассказывая о том же событии в биографии Гая Марция Кориолана, Плутарх пишет, что женщины <<попросили [у сената "--- В.~Ж.] только дозволения соорудить храм Женской Удачи с тем, чтобы средства на постройку собрали они сами, а расходы по совершению обрядов и всех прочих действий, каких требует культ богов, приняло на себя государство>> (%*вставить оригинальную цитату*
Plut. Coriol. 37). 

Дионисий Галикарнасский в своём повествовании далее указывает, что <<впервые тогда женщинами была избрана жрица>>\footnote{\graecafn{<i'ereia <up`o t~wn gunaik~wn >apede'iqjh t'ote pr~wton}, Dion. Hal. VIII.55.4.} (Dion. Hal. VIII.55.4), а также называет её имя: это была Валерия (\graeca{O>ualer'ia}, Ibid.), та самая, которая, согласно тому же Дионисию, сподвигла мать и жену Кориолана отправиться к нему с посольством (см. Ibid. VIII.39--44). Первые жертвоприношения, сообщает далее античный историк, были принесены женщинами за народ, и <<Валерия начала священные обряды на алтаре, установленном на священном участке, ещё до того как были воздвигнуты храм и статуя>>\footnote{\graecafn{jus'ian d`e pr'wthn a<i guna~ikes >'equsan <up`er to~u d'hmou katarqum'enhs t~wn <ier~wn t~hs O>ualer'ias >ep`i to~u kataskeuasj'entos >en t~w| tem'enei bwmo~u, pr`in >`h t`on ne`wn ka`i t`o x'oanon >anastaj~hnai}, Ibid.} (Ibid. VIII.55.4). Скрупулёзность Дионисия в изложении фактов доходит до того, что он любезно сообщает нам дату этого первого жертвоприношения: его совершили в декабрьские календы года, следовавшего после похода Кориолана на Рим\footnote{\graecafn{mhn`i Dekembr'iw| to~u kat'opin >eniauto~u, t~h| n'ea| sel'hnh|, <`hn \ldots <Rwma~ioi kal'andas kalo~usin}, Ibid.} (т.е. в 487 г. до н.э.) (Ibid.).

Автор <<Римских древностей>> не обходит стороной и, пожалуй, более важную дату окончания постройки и освящения храма Женской Фортуны, указывая, что произошло это событие <<приблизительно в седьмой день, ведя счёт по луне, месяца Квинтилия: этот день у римлян является предшествующим дню квинтильских нон>>\footnote{\graecafn{ne`ws sunetel'esjh te kai kajier'wjh Kointil'ion mh'os <ebd'omh| m'alista kat`a sel'hnhn: a<'uth kat`a <Rwma'ious >est`in <o prohgoum'enh t~wn Kointil'iwn nwn~wn <hm'era}, Ibid. 4--5.} (Ibid. 4--5). В то время шёл год консульства Прокула Вергиния, который и освятил храм\footnote{\graecafn{<o d`e kajier'wsas a>ut`on >~hn Pr'oklos O<uerginios, <'ateros t~wn <up'atwn}, Ibid. 5.} (Ibid. 5). Таким образом, сведения, передаваемые Дионисием, позволяют точно датировать основание храма Fortunae Muliebris: согласно Ливию (Liv. II.41.1), Прокул Вергиний был консулом вместе со Спурием Кассием в 486 г. до н.э., т.е. через два года после похода Кориолана на Рим (Liv. II.39.1 и далее), что соответствует и рассказу Дионисия (на основании вышеизложенного, через год после войны были совершены первые жертвоприношения, а ещё через год "--- уже освящён храм).

% καταρχομένης τῶν ἱερῶν γυναικός, ἣν ἂν ἀποδείξωσιν αὐταὶ λειτουργὸν τῶν ἱερῶν. (Dion. Hal. VIII.55.3)
% ἱέρεια μὲν ὑπὸ τῶν γυναικῶν ἀπεδείχθη τότε πρῶτον (Dion. Hal. VIII.55.4)
% θυσίαν δὲ πρώτην αἱ γυναῖκες ἔθυσαν ὑπὲρ τοῦ δήμου καταρχομένης τῶν ἱερῶν τῆς Οὐαλερίας ἐπὶ τοῦ κατασκευασθέντος ἐν τῷ τεμένει βωμοῦ, πρὶν ἢ τὸν νεὼν καὶ τὸ ξόανον ἀνασταθῆναι (Dion. Hal. VIII.55.4)
% μηνὶ Δεκεμβρίῳ τοῦ κατόπιν ἐνιαυτοῦ, τῇ νέᾳ σελήνῃ, ἣν Ἕλληνες μὲν νουμηνίαν, Ῥωμαῖοι [p. 209] δὲ καλάνδας καλοῦσιν: αὕτη γὰρ ἦν ἡ λύσασα τὸν πόλεμον ἡμέρα (Dion. Hal. VIII.55.4)
%  νεὼς συνετελέσθη τε καὶ καθιερώθη Κοιντιλίου μηνὸς ἑβδόμῃ μάλιστα κατὰ σελήνην: (5) αὕτη δὲ κατὰ Ῥωμαίους ἐστὶν ὁ προηγουμένη τῶν Κοιντιλίων νωνῶν ἡμέρα. ὁ δὲ καθιερώσας αὐτὸν ἦν Πρόκλος Οὑεργίνιος ἅτερος τῶν ὑπάτων.

% Разобрать рассказ о чуде, произошедшем со статуей


% Написать про то, что сообщение о храме Женской Фортуны хорошо задокументировано

Итак, храм Женской Фортуны был освящён 6 июля 486 г. до н.э., и нам повезло, что мы можем с такой высокой точностью датировать столь отдалённое от нас событие, тем более, что, как отмечает И.Л.~Маяк, <<копилка источников истории ранней Республики беднее не только по сравнению с поздней Республикой и Империей, но и с более ранними, царским и доцарским, периодами>>\footnote{Маяк И.Л. Римляне ранней Республики. М., 1993. С. 4.}. В данном случае нас выручает подробный и обстоятельный рассказ Дионисия Галикарнасского. Наличие множества подробностей строгого фактического характера (имена консула и жрицы, детали отправления обрядов и обстоятельства юридического характера, вроде тех, за чей счёт возводился храм и ставилась статуя (см. Dion. Hal. VIII.56.2), а также точные до дня даты событий) говорит о том, что здесь Дионисий прибегал к документальным источникам. Чуть далее в своём повествовании о чуде, произошедшем в храме Женской Фортуны, он, действительно, ссылается на записи верховных жрецов (\graeca{<ws a<i <ierofant~wn peri'equosi grafa'i}, Ibid. 56.1). Таким образом, степень достоверности событий в пересказе Дионисия определяется степенью достоверности его источников, которыми, судя по всему, являются в данном случае tabulae pontificum. Отсылка же к материалам подобного рода может только прибавить весомости сообщению нашего историка; нам представляется уместным присоединиться к мнению П.~Уолша, считавшего традицию, опирающуюся на официальные документы римских коллегий, вполне достоверной (в смысле правильной передачи первоначальных сведений) вследствие той сакральности, каковую придавали ей сами римляне\footnote{\label{WalshRegistersQuote}Процитируем П. Уолша: <<\ldots history-writing in Latin was in origin official and religious, and this character remains impressed upon it even in its maturity. For the first historical records, the \textit{tabulae pontificum} were contrast to that of the Greeks \ldots the Roman enshrined his written traditions, believing it important to perpetuate them whether true or not \ldots The attitude of veneration towards Roman traditions is reflected in the importance attached to the old registers>>. Walsh, P. Livy. His historical aims and methods. Camb., 1974. P. 31.}.

% ὡς αἱ τῶν ἱεροφαντῶν περιέχουσι γραφαί (Dion. Hal. VIII.56.1)

% Подумать, куда ткнуть этот абзац
% Отметить, что Плутарх сообщает некоторые подробности, в частности, то, о чём именно просили женщины, каковых нет у Дионисия. Это, видимо, связано с тем, какими именно источниками пользовался Плутарх
Отметим, что, несмотря на то, что Дионисий Галикарнасский в своём изложении, как мы видим, скрупулёзно приводит даже самые мелкие факты\footnote{И. Тэн остроумно критиковал эту черту сочинения Дионисия: <<Греческий ритор с мелочной подробностью объясняет учреждения, войны, переговоры; вы можете следить за каждым шагом его действующих лиц; он имеет полные планы всех сражений; ни один военный совет не обойдётся без того, чтобы он не перечислил вам все высказанные мнения. Нет такого переворота, в побудительные причины которого он не проник, ни одного события, "--- происхождение которого он бы не раскрыл>> и т.д. Тэн И. Тит Ливий. Критическое исследование. М., 1900. С. 55. Впрочем, в нашем случае мы вынуждены воздать должное греческому ритору, ибо его сведения, как мы выяснили, имеют документальное происхождение.}, сведений о том, за чей счёт женщины просили построить храм и проводить жертвоприношения, о чём пишет Плутарх в биографии Кориолана (Plut. Coriol. 37), у него нет. Вряд ли Плутарх мог выдумать такого рода подробность; нам представляется более вероятным, что эти сведения также имеют документальное происхождение. Это говорит в пользу того, что Плутарх, скорее всего, опирался либо на те же архивы, что и Дионисий (нам известно, что греческий философ бывал в Риме, и, быть может, имел возможность работать с документами жреческих коллегий), либо на некоторый не дошедший до нас источник, в котором использовались те же материалы. Но сравнивая сообщения двух авторов, мы вынуждены признать, что Дионисий передаёт нам гораздо более ценные факты, чем Плутарх.

% О легенде про говорящие статуи
Однако in tabulis pontificum мы могли бы найти сведения несколько более фантастические, нежели сообщение об основании и освящении такого-то храма в такой-то день. А именно, ссылаясь на эти записи, Дионисий Галикарнасский (Dion. Hal. VIII.56.1, см. выше) приводит легенду <<о явлении богини, случившемся в то время не один раз, а даже дважды>>\footnote{\graecafn{t'o dhl~wsai t`hn genom'enhn >epif'aneian t~hs jeo~u kat> >eke~inon t`on qr'onon o>uq <'apax, >all`a ka`i d'is}, Dion. Hal. VIII.56.1.} (Ibid.). Дионисий (Dion. Hal. VIII.56.2) и Плутарх (Plut. Coriol. 37) пересказывают дальнейшие события примерно одинаково: сенат и народ на общественные деньги поставили одну статую богини в храм, женщины же, на собственные средства, ещё одну, и вот эта-то вторая статуя во время установки заговорила человеческим голосом. Дионисий передаёт её слова так: <<По священному закону города вы, жёны, принесли меня в дар>>\footnote{\graecafn{<os'iw| p'olews n'omw| guna~ikes gameta`i ded'wkat'e me}, Dion. Hal. VIII.56.2--3.} (Dion. Hal. VIII.56.2--3), Плутарх же так: <<Угоден богам, о жёны, ваш дар>>\footnote{\graecafn{jeofile~i me jesm~w| guna~ikes ded'wkate}, Plut. Coriol. 37.3.} (Plut. Coriol. 37.3). 

% ‘θεοφιλεῖ με θεσμῷ γυναῖκες δεδώκατε (Plut. Coriol. 37.3)

%Далее он описывает случившееся со всеми возможными подробностями: <<когда сенат постановил оплатить все расходы на храм и статую за общественный счёт, а женщины соорудили другую статую на те средства, которые собрали, и обе они одновременно были принесены в дар в первый день освящения храма, одно из изваяний "--- то, которое установили женщины, "--- в присутствии многих произнесло отчётливо и громко фразу \ldots ``По священному закону города вы, жены, принесли меня в дар''>> (\graeca{}, Ibid. 2).

% τὸ δηλῶσαι τὴν γενομένην ἐπιφάνειαν τῆς θεοῦ κατ᾽ ἐκεῖνον τὸν χρόνον οὐχ ἅπαξ, ἀλλὰ καὶ δίς (Dion. Hal. VIII.56.1)

% ὅτι τῆς βουλῆς ψηφισαμένης ἐκ τοῦ δημοσίου πάσας ἐπιχορηγηθῆναι τὰς εἰς τὸν νεών τε καὶ τὸ ξόανον δαπάνας, ἕτερον δ᾽ ἄγαλμα κατασκευασαμένων τῶν γυναικῶν ἀφ᾽ ὧν αὐταὶ συνήνεγκαν [p. 210] χρημάτων, ἀνατεθέντων τ᾽ αὐτῶν ἀμφοτέρων ἅμα ἐν τῇ πρώτῃ τῆς ἀνιερώσεως ἡμέρᾳ, θάτερον τῶν ἀφιδρυμάτων, ὃ κατεσκευάσανθ᾽ αἱ γυναῖκες, ἐφθέγξατο πολλῶν παρουσῶν γλώττῃ Λατίνῃ φωνὴν εὐσύνετόν τε καὶ γεγωνόν: ἧς ἐστι φωνῆς ἐξερμηνευόμενος ὁ νοῦς εἰς τὴν Ἑλλάδα διάλεκτον τοιόσδε: ὁσίῳ (3)  πόλεως νόμῳ γυναῖκες γαμεταὶ δεδώκατέ με (Dion. Hal. VIII.56.2--3)

% Что пишет о говорящей статуе Валерий Максим
Об этом же событии мы можем прочесть и у Валерия Максима: он сообщает точное местоположение храма и священного изображения "--- у четвёртого милевого камня Латинской дороги\footnote{\textit{Fortunae etiam Muliebris simulacrum, quod est Latina via ad quartum miliarium, eo tempore cum aede sua consecratum}, Val. Max. 1.8.4.} (Val. Max. 1.8.4). Согласно Валерию Максиму, богиня дважды (\textit{non semel sed bis}, Ibid.) произнесла следующие слова: \textit{rite me, matronae, dedistis riteque dedicastis} (Ibid.) (<<Правильно, матроны, вы доверились мне и правильно посвятили себя мне>>). Отметим, что Плутарх, в отличие от Дионисия и Валерия Максима, считает неправдоподобным то обстоятельство, что статуя могла заговорить дважды: <<Утверждают, "--- пишет он, "--- будто эти слова раздались и во второй раз: нас хотят убедить в том, что похоже на небылицу и звучит весьма неубедительно>>\footnote{\graecafn{ta'uthn ka`i d`is gen'esjai t`hn fwn`hn mujologo~usin, >agen'htois <'onoia ka`i qalep`a peisj~hnai pe'ijontes <hm~as}, Plut. Coriol. 38.1.} (Plut. Coriol. 38.1). Таким образом, все авторы единодушно утверждают, что богиня, согласно традиции, обратилась к женщинам именно дважды.

Обратим внимание на то, что Валерий Максим пишет об изображении (\textit{simulacrum}), стоявшем отдельно от храма (\textit{aedes}). Об этой статуе богини, что находилась у милевого столба Латинской дороги, рассказывает и Фест, прибавляя, что прикасаться к нему могла лишь та женщина, которая была единожды в браке\footnote{\textit{via Latina ad milliarium illi (?) Fortunae muliebris, nefas est attingi, nisi ab ea, quae semel nupsit}, Fest. 242~M.}. Это подтверждается сообщением Тертуллиана, который в трактате <<О единобрачии>> отмечает, что <<увенчивать венком Женскую Фортуну, также как и Матерь Матуту, может лишь та, кто была лишь один раз замужем>>\footnote{\textit{Fortunae Muliebri coronam non imponit nisi univiris sicut Matri Matutae}, Tert. De Monogam. XVII.4.} (Tert. De Monogam. XVII.4). Эти отрывочные свидетельства представляют для нас высокую ценность, потому что доказывают, что поклонение Женской Фортуне продолжалось и в I в. до н.э., во времена Веррия Флакка, и во II в. н.э., во время жизни Тертуллиана, причём с сохранением традиционного обряда. Обратим также внимание на близость ритуалов Женской Фортуны и Матери Матуты, которую подчёркивает христианский автор. Итак, если из сочинений Дионисия Галикарнасского, Плутарха и Валерия Максима мы узнаём, что ещё первоначально Женской Фортуне поклонялись матроны, то эти поздние свидетельства открывают для нас также и детали ритуала, который, вследствие крайней консервативности римлян в религиозной области\footnote{Х.~Розе таким образом отмечает эту черту римской религиозности: <<It follows that in dealing with Roman religion, the department in which that conservative people were most conservative, we can quite easily find, almost on the surface as it were, remnants of a very early simple type of thought>>. Rose, H. Op. cit. P. 158.}, мы можем полагать сохранявшимся в неизменности с раннереспубликанских времён.

% Fortunae etiam Muliebris simulacrum, quod est Latina via ad quartum miliarium, eo tempore cum aede sua consecratum, quo Coriolanum ab excidio urbis maternae preces reppulerunt, non semel sed bis locutum constitit ~ prius his verbis: 'rite me, matronae, dedistis riteque dedicastis'

% Статуя Женской Фортуны была установлена близ четвёртого милевого камня по Латинской дороге, её почитали так же, как посвящённый ей храм. Когда римские матроны своими молитвами убедили Кориолана вернуться и не разрушать Город, богиня, говорят, не единожды, но дважды произнесла следующие исторические слова: «Правильно, матроны, вы доверились мне и правильно посвятили себя мне».

%Далее, однако, он вообще пускается в рассжудение о том, что 

% ταύτην καὶ δὶς γενέσθαι τὴν φωνὴν μυθολογοῦσιν, ἀγενήτοις ὅμοια καὶ χαλεπὰ [p. 212] πεισθῆναι πείθοντες ἡμᾶς (Plut. Coriol. 38.1)

% Написать про ритуалы, связанные с этим изображением

Наличие таких подробных и точных сведений о этом храме разительно контрастирует с характером тех сообщений о культе Фортуны в царскую эпоху, каковыми мы располагаем. В последнем случае, как мы уже писали выше, нам приходится выдвигать различные гипотезы, более или менее обоснованные косвенными данными, строить свои предположения на основании весьма зыбком и зачастую расписываться в отсутствии в источниках сведений, необходимых для однозначного заключения. Так что мы многим обязаны Дионисию Галикарнасскому, который потрудился привести в своём произведении подробные выдержки из документов жреческих коллегий. 

% Далее написать про что-нибудь ещё

% Ἀληθῶς ἀνέστη

% Храм Fortis Fortunae, возведённый Карвилием.

Далее сведения о возведении новых храмов Фортуны приобретают отрывочный характер. В наших источниках следует лакуна до начала III в. до н.э., когда Ливий сообщает, что консул Карвилий после победы над самнитами и этрусками (см. Liv. X.39 и далее) употребил военную добычу на возведение храма богини Счастливого Случая (Fortis Fortunae) рядом с храмом той же богини, освящённым Сервием Туллием\footnote{\textit{reliquo aere aedem Fortis Fortunae de manubiis faciendam locavit prope aedem eius deae ab rege Ser. Tullio dedicatam}, Liv. X.46.14.} (Ibid. 46.14). Консульство Спурия Карвилия приходилось на 293 г. до н.э.

% Храм Fortunae Primigeniae

Основание следующего храма\label{FortunaPrimigenia} Фортуны в Риме источники, известные нам, относят только к концу III "--- началу II вв. до н.э. Согласно Титу Ливию, консул 204 г. до н.э. Публий Семпроний, вступив в битву с Ганнибалом в Кротонской области, <<дал обет построить храм Фортуне Примигении, если он в этот день разобьет врага>>\footnote{\textit{consul principio pugnae aedem Fortunae Primigeniae vovit si eo die hostes fudisset}, Liv. XXIX.36.8.} (Liv. XXIX.36.8). Обет консула осуществился, пишет дальше римский историк\footnote{\textit{composque eius voti fuit}, Ibid.} (Ibid.). Храм был построен и освящён спустя 10 лет, в 194 г. до н.э.: <<храм Фортуне Примигении на Квиринальском холме посвятил Квинт Марций Ралла, дуумвир, нарочно для того назначенный; а обет дал консул Публий Семпроний Соф за десять лет пред тем, во время Пунической войны; он же, будучи уже цензором, сдал подряд на строительство этого храма>>\footnote{\textit{aedem Fortunae Primigeniae in colle Quirinali dedicavit Q. Marcius Ralla, duumvir ad id ipsum creatus: voverat eam decem annis ante Punico bello P. Sempronius Sophus consul, locaverat idem censor}, Liv. XXXIV.53.5--6.} (Liv. XXXIV.53.5--6).

% Про Культ Примигении в Пренесте
Культ Фортуны Примигении был перенесён в Рим из Пренесты, где находилось древнее святилище и оракул этой богини (Cic. De div. II.85--86, Liv. XXIII.19.18). Рассмотрение пренестинского храма, находившегося за пределами Рима, выходит за рамки нашего исследования, однако мы ненадолго задержим на нём своё внимание. Несмотря на то, что точная дата основания храма в Пренесте неизвестна, исследователи сходятся в том, что это был древний культ, в котором, возможно, видны следы раннего влияния греческого культа Тюхе\footnote{Wissowa, G. Op. cit. S. 211; Latte, K. Op. cit. S. 176; Rose, H. Op. cit. P. 171.}. Пренестинская Фортуна была изображена в виде матери, держащей на руках младенцев Юпитера и Юнону (Cic. De div. II.85), что говорит о ней как о материнском божестве\footnote{Wissowa, G. Op. cit. S. 209--211; Warde Fowler, W. Op. cit. P. 165--166, 223--224; Altheim, F. A history of Roman religion. P. 268; Немировский~А.~И. Указ. соч. С. 73.}. Г. Виссова указывает, что, перенесённая в Рим, эта богиня сделалась заведовать благополучием Римского государства\footnote{Wissowa, G. Op. cit. S. 210.}. Е.~М.~Штаерман полагает, что появлению культа Фортуны Примигении в Риме способствовала популярность гадания по жребиям в Пренесте\footnote{Штаерман~Е.~М. Указ. соч. С. 128.}. Пренестинский оракул действительно пользовался широкой популярностью среди всех слоёв населения, так, Светоний указывает, что к нему ежегодно обращался даже император Домициан (Suet. Dom. 15.2).

% Различные толкования, согласно античным авторам, термина Primigenia
Ещё в античности писатели пытались давать толкования названию Primigenia, что должно переводиться на русский как Перворожденная. Цицерон объясняет это так, что Фортуна Примигения сопутствует человеку с его рождения\footnote{\textit{[Fortuna] Primigenia a gignendo comes}, Cic. De leg. II.28.} (Cic. De leg. II.28). Плутарх (Quae. Rom. 106) более многословен и даёт целых три толкования, ни одно из которых не перекликается с цицероновым. Согласно первому, римляне поклоняются Фортуне Примигении в память о Сервии Туллии, рождённом от рабыни и сделавшимся, благодаря заступничеству богини, царём Рима (Plut. Mor. 289~C); этого, продолжает Плутарх, придерживаются многие римляне\footnote{\graecafn{o<'utw g'ar o<i pollo`i <Rwma'iwn <upeil'hfasin}, Plut. Quae. Rom. 106, Mor. 289~C.} (Ibid.). Согласно второму толкованию, Фортуна дала начало и рождение Риму\footnote{\graecafn{t~hs <R'wmhs <h t'uqh par'esqe t`hn >arq`hn ka`i t`hn g'enesin}, Ibid.} (Ibid.). Третье объяснение Плутарха таково: <<Фортуна — начало всех вещей, и благодаря ей природа возникает из случайных соединений>>\footnote{\graecafn{t`hn t'uqhn p'antwn o>'usan >arq`hn ka`i t`hn f'usin >ek to~u kat`a t'uqhn sunistam'enhn}, Ibid.} (Ibid.). Отметим, что первое из этих объяснений есть, действительно, передача сказаний, имевших хождение среди римлян и, как мы показали в гл.~\ref{RegalRome}, скорее всего, par excellence в плебейской среде, то последнее толкование больше похоже на плод отвлечённых философских размышлений.

%  οὕτω γάρ οἱ πολλοὶ Ῥωμαίων ὑπειλήφασιν (Plut. Quae. Rom. 106, Mor. 289 C)
% τῆς Ῥώμης ἡ τύχη παρέσχε τὴν ἀρχὴν καὶ τὴν γένεσιν (Plut. Quae. Rom. 106, Mor. 289 C)
% τὴν τύχην πάντων οὖσαν ἀρχὴν καὶ τὴν φύσιν ἐκ τοῦ κατὰ τύχην συνισταμένην

% Храм Fortunae Equestris

Через довольно короткое время после основания храма Фортуны Примигении мы встречаемся с ещё одним обетованным храмом, на этот раз Фортуне Всаднической\footnote{Г.~Виссова связывал её с гением всаднического сословия. См. Wissowa, G. Op. cit. S. 211--212.}. В 180 г. до н.э. проконсул Испании Квинт Фульвий Флакк во время битвы с кельтиберами в Манлиевом урочище (см. Liv. XL.39 и далее) дал обет богине "--- правда, в отличие от Публия Семпрония, обратившегося к высшим силам перед сражением, сделал он это ближе к концу битвы, когда исход её был уже ясен: <<\ldots{}все кельтиберы обратились в бегство, а римский полководец, глядя на опрокинутого врага, обетовал храм Всаднической Фортуне и игры Юпитеру Всеблагому Величайшему>>\footnote{\textit{Celtiberi omnes in fugam effunduntur, et imperator Romanus aversos hostes contemplatus aedem Fortunae equestri Iovique optimo maximo ludos vovit}, Liv. XL.40.10.} (Liv. XL.40.10). Римский историк отмечает далее, что <<деньги на это были собраны для него [консула Флакка "--- В.~Ж.] испанцами>>\footnote{\textit{vovisse \ldots{} aedem equestri Fortunae sese facturum: in eam rem sibi pecuniam collatam esse ab Hispanis}, Ibid. 44.9.} (Ibid. 44.9), к тому же, <<было решено избрать дуумвиров, чтобы они сдали подряд на строительство храма>>\footnote{\textit{ut duumviri ad aedem locandam crearentur}, Ibid. 10.} (Ibid. 10).

% Что нехорошего сделал Квинт Фульвий Флакк:
% Версия Ливия
Античная традиция передаёт нам, что при строительстве этого храма Фульвий Флакк совершил святотатство\footnote{Штаерман~Е.~М. Социальные основы религии древнего Рима. М., 1987. С. 126--127.}, разорив храм Юноны Лацинийской в Бруттии: он снял мраморные плиты, покрывавшие крышу того храма, чтобы украсить ими свою постройку (Liv. XLII.3.1--4). Ливий описывает, что такое деяние вызвало возмущение сената, и Флакку воспретили украшать свой храм этими плитами, которые уже были доставлены в Рим (Ibid. 5--11). Валерий Максим передаёт ту же историю несколько иначе и с нравоучительным оттенком; согласно его версии, Флакк был наказан высшими силами: <<когда он узнал, что из двух его сыновей, сражавшихся в Иллирии, один погиб, а другой был тяжело ранен, он в отчаянии испустил дух>>\footnote{\textit{per summam aegritudinem animi expiravit, cum ex duobus filiis in Illyrico militantibus alterum decessisse, alterum graviter audisset adfectum}, Val. Max. 1.1.20.} (Val. Max. 1.1.20). Сенат, согласно Валерию, потрясённый этим событием, распорядился <<отправить плиты обратно в Локры>>\footnote{\textit{senatus tegulas Locros reportandas curavit}, Ibid.} (Ibid.). Валерий почему-то отправляет мраморные плиты в Локры, тогда как, согласно Ливию, храм Юноны Лацинийской находился в шести милях от Кротона (Liv. XXIV.3.3) "--- оба города, действительно, находятся в регионе под названием Бруттий. Ливий более точен, чем Валерий, он указывает, что именно Флакк освятил обетованный им храм Всаднической Фортуны в 173 г. до н.э.\footnote{\textit{Fulvius aedem Fortunae equestris, quam proconsul in Hispania dimicans cum Celtiberorum legionibus voverat, annis sex post, quam voverat, dedicavit}, Liv. XLII.10.5.} (Liv. XLII.10.5). Сообщение Валерия о том, что Флакк умер, не пережив известий о несчастии со своими сыновьями, подтверждается и Ливием. Римский историк пишет, что тому <<сообщили, что из двух сыновей его, служивших в Иллирии, один умер, а другой опасно и тяжело болен>>\footnote{\textit{ex duobus filiis eius, qui tum in Illyrico militabant, nuntiatum alterum <mortuum, alterum> gravi et periculoso morbo aegrum esse}, Liv. XLII.28.11.} (Liv. XLII.28.11). После этого <<рабы, вошедшие утром в спальню хозяина, нашли его висящим в петле>>\footnote{\textit{mane ingressi cubiculum servi laqueo dependentem invenere}, Ibid. 12.} (Ibid. 12). Далее, согласно Ливию, в народе возник слух, что так Флакка постигла кара богини Юноны за осквернение храма\footnote{\textit{vulgo Iunonis Laciniae iram ob spoliatum templum alienasse mentem ferebant}, Ibid. 13.} (Ibid. 13). Ливий, таким образом, указывает и источник той легенды, которую в качестве нравоучительного примера приводит в своём сборнике Валерий Максим.

% Версия Валерия Максима

% Сообщение про отсутствие храма Всаднической Фортуны в Риме за авторством Тацита

Дальнейшая история храма Всаднической Фортуны также ставит перед нами любопытные вопросы. Тацит сообщает нам, что уже в 22 г. н.э. в Риме не могли найти этот храм: когда римские всадники захотели принести Всаднической Фортуне дар за выздоровление Юлии Августы (о её болезни см. Tac. Ann. III.64), они не смогли отыскать в Риме святилище с таким названием, хотя храмов Фортуны в Городе насчитывалось множество\footnote{\textit{Incessit dein religio quonam in templo locandum foret donum quod pro valetudine Augustae equites Romani voverant equestri Fortunae: nam etsi delubra eius deae multa in urbe, nullum tamen tali cognomento erat}, Tac. Ann. III.71.} (Tac. Ann. III.71). В конце концов, выяснилось, что храм с таким именем находится в Анции, туда-то и направили дар (Ibid.).

% И сообщение об этом храме Витрувия!
Получается, что храма, основанного в Риме в 1-й половине II в. до н.э. уже не существовало в 1-й половине I в. н.э.? Этому, однако, противоречит сообщение Витрувия, который явно свидетельствует о том, что в Риме в его время был храм Всаднической Фортуны: говоря о \textit{систилосе}, т.е. способе построения колоннады, при котором расстояние между колоннами в два раза больше их диаметра, он приводит в пример именно такой храм, указывая, что находится он рядом с Каменным театром: <<Как, например, (храм) Фортуны Всаднической у Каменного театра и другие, построенные таким же образом>>\footnote{\textit{quemadmodum est Fortunae Equestris ad theatrum lapideum reliquaeque, quae eisdem rationibus sunt conpositae}, Vitruv. 3.2.2.} (Vitruv. 3.3.2).

% Исследовать, почему Тацит противоречит Витрувию, и кому из них доверять стоит больше
Сравнивая сообщения двух авторов, отметим в первую очередь, что Тацит пишет о событии, отстоявшем от него почти на столетие (учитывая время написания <<Анналов>>), таким образом, он в своём изложении мог опираться только на документальные (или иного рода) источники, каковые были в его распоряжении. Витрувий, с другой стороны, приводит здание храма Всаднической Фортуны в качестве иллюстрации к своему изложению, иллюстрации, которая, судя по всему, была хорошо понятна и знакома как ему, так и его читательской аудитории. Тем не менее, нельзя сбрасывать со счетов вероятность того, что Витрувий мог ошибиться, приняв за храм Fortunae Equestris некий другой храм Фортуны или даже какой-нибудь другой богини. Но существует также возможность и того, что сообщения обоих авторов истинны, а ошибку совершили те, кто в 22 г. н.э. искали и не смогли найти реально существовавший храм Fortunae Equestris в Риме. Решить однозначно, кому из авторов стоит доверять больше, на основании только письменных источников невозможно. В разрешении этого противоречия могло бы помочь привлечение археологических данных, но, к сожалению, нам неизвестно, чтобы поблизости от Каменного театра или где-нибудь ещё в Риме был найден храм Всаднической Фортуны\footnote{Arya, D.A. Op. cit. P. 200.}.

% В этом храме служат матроны

Примечательно, что Витрувий, критикуя далее недостатки описываемого им стиля \textit{systilos}, пишет: <<Ведь матери семейств, когда поднимаются по ступеням, чтобы сделать приношение, не могут пройти между колоннами все вместе, но только по очереди>>\footnote{\textit{Matres enim familiarum cum ad supplicationem gradibus ascendunt, non possunt per intercolumnia amplexae adire, nisi ordines fecerint}, Vitruv. 3.3.3.} (Vitruv. 3.3.3). Это говорит нам о том, что в храме Fortunae Equestris (или, быть может, другой Фортуны) религиозные обряды отправляли \textit{матроны}, также как и в храме Fortunae Muliebris. Нас не должен удивлять тот факт, что женскому божеству служат именно матроны\footnote{Впрочем, W. Warde Fowler принимает эту черту множества культов Фортуны (он называет Fortuna Equestris, Muliebris, Virilis) за <<perhaps, the most striking fact>>. См. Warde Fowler, W. Op. cit. P. 167--168.}, однако важно, что мы приходим к такому выводу не на основании догадок или косвенных данных, а располагаем на то прямым указанием источника.

% Заключение главы: связь Ф. с военной удачей. Это связано с характером наших источников, с Титом Ливием. 
% А также отметить, что здесь мы опираемся, в итоге, на документальные источники, что гут.
% Приведённые выше факты уже позволяют нам сделать определённые выводы касательно культа Фортуны в Республиканскую эпоху.

До сих пор при обращении к истории культа Фортуны и представления о ней мы пользовались произведениями авторов, живших намного позже описываемых ими событий. Однако с рубежа III и II вв. до н.э. мы уже располагаем свидетельствами современников. Так, о Фортуне писал поэт Пакувий:

\begin{verse}
\textit{Fortunam insanam esse et caecam et brutam perhibent philosophi,\\
saxoque instare in globoso praedicant volubilei,\\
quia quo id saxum inpulerit fors, eo cadere Fortunam autuniant.\\
Insanam autem esse aiunt quia atrox incerta instabilisque sit;\\
caecam ob eam rem esse iterant quia nil cernat quo sese adplicet;\\
brutam quia dignum atque indignum nequeat internoscere.\\
Sunt autem alii philosophi qui contra Fortunam negant\\
esse ullam sed temeritate res regi omnes autumant.\\
Id magis verisimile esse usus reapse experiundo edocet\\
velut Orestes modo fuit rex, fiictust mendieus modo.}\footnote{Полонская~К.~П., Поняева~Л.~П. Хрестоматия по ранней римской литературе. М., 1984. С. 58.}
\end{verse}

Приведём перевод В.~И.~Модестова: 

\begin{verse}
\textit{Судьба глупа, безумна, "--- думают философы\\
Она, по их словам, стоит на камне на катящемся.\\
Куда случайно пущен он, туда судьба направится.\\
Безумна "--- ибо жестока, обманчива и ветрена;\\
Слепа же "--- потому что без разбора льнёт ко всякому;\\
Глупа "--- затем, что доброго не отличит от низкого.\\
Но есть философы, которые судьбы не признают.\\
Которым кажется, что случай лишь делами ведает;\\
И это ближе к истине, опыт учит этому:\\
Хотя бы вот "--- Орест: недавно царь, теперь стал нищим он.}\footnote{Там же. С. 59.}
\end{verse}

Мы видим, что Пакувий здесь наделяет Фортуну, со слов неких философов, весьма отрицательными эпитетами: она \textit{insana}, \textit{atrox}, \textit{incerta}, \textit{instabilis}, \textit{caeca}. Философы, на которых ссылается Пакувий, безусловно, должны быть греками и, разумеется, в оригинале вести речь не о римской Фортуне, а о греческой \graeca{T'uqh}. Примечательно здесь, что уже Пакувий ставит знак равенства между этими двумя богинями, не менее примечательно и то, что в его коротком стихотворении мы встречаемся с резко отрицательным образом Фортуны. %Обратим внимание на пример Ореста, который приводит поэт в последней строчке: он, лишённый царства из-за Фортуны, является полной противоположностью римскому царю Сервию Туллию, который сделался царём благодаря покровительству богини удачи.

Встречаем мы упоминание этой богини и в произведениях современника Пакувия, римского комедиографа Плавта. Так, в <<Ослах>> герой пьесы говорит, что дозволено хвалить Фортуну\footnote{\textit{Licet laudem Fortunam}, Plaut. Asin. 718.}, а в <<Пленниках>> персонаж, в шутливой форме предлагая принести себе жертву, как богу, называет себя, в том числе, Фортуной, перечисляя ряд безусловно положительных божеств\footnote{\textit{idem ego sum Salus, Fortuna, Lux, Laetitia, Gaudium}, Plaut. Capt. 864.}. Здесь мы встречаем Фортуну именно как богиню, и Плавт в этих отрывках рисует её как доброе, положительное божество. Здесь герои Плавта, чьи образы списаны с окружавших его людей, придерживаются мнения, противоположного тому, которое, со слов греческих философов, пересказывает Пакувий.

Но в произведениях Плавта мы можем найти и упоминание фортуны как понятия о судьбе или случае. Героиня <<Касины>> жалуется соседке на свою <<[плохую] судьбу в отношении мужа>>\footnote{\textit{Nunc huc meas fortunas eo questum ad vicinam}, Plaut. Cas. 161.}, а персонаж комедии <<Канат>> сравнивает своего товарища со <<злой судьбой>>\footnote{\textit{Malam fortunam in aedis te adduxi meas}, Plaut. Rud. 501.}. Здесь мы видим уже неоднозначное отношение к фортуне.

Нам неизвестно, воспринималась ли Фортуна изначально, со времён царского Рима, только как благая богиня, или же отношение к ней, как и впоследствии, было двойственным: за отсутствием прямых свидетельств мы можем только строить предположения. Укажем лишь на то, что, античные авторы приписывали блистательный <<карьерный взлёт>> Сервия Туллия покровительству Фортуны, и в то же время не соотносили его трагическую кончину с кознями непостоянной богини, которая, как легко можно предположить, в итоге отвернулась от своего любимца. Мы, однако, остережёмся делать далеко идущие выводы из того, что ни один из известных нам авторов не считает Сервия жертвой непостоянства Фортуны. Отметим лишь, что, когда на рубеже III и II вв. до н.э. появляются свидетельства современников о Фортуне, то мы видим уже двойственное отношение к этой богине, которая может быть как благой, так и дурной.

Подведём некоторые итоги. Первое, на что стоит обратить внимание в развитии культа Фортуны времени Республики, это резко изменившийся характер источников по сравнению с царской эпохой. Выше мы уже подробно разбирали сообщение Дионисия Галикарнасского о строительстве и освящении храма Женской Фортуны. Обратившись теперь к процитированным выше свидетельствам Тита Ливия, отметим, что он также приводит множество деталей документального характера. Ливий точно указывает имя консула Спурия Карвилия и источник его средств на возведение храма Fortis Fortunae (Liv. X.46.14), описывает подробности сооружения храма Фортуны Примигении (Liv. XXXIV.53.5--6), в рассказе о сооружении храма Fortunae Equestris указывает, откуда брались деньги (Liv. XL.44.9) и то, какие были для этого назначены должностные лица (Ibid. 10). Такого рода подробности должны восходить, в конечном итоге, к архивам римских коллегий, в первую очередь, tabulae pontificum. Разумеется, сообщения Дионисия и Ливия о сооружении храмов Фортуны не покрывают всей истории развития культа, и о множестве известных нам храмов мы не знаем, когда и как они были основаны, но мы можем утверждать, что то, что нам известно о возведении храмов эпохи начала и середины Республики, нам известно точно, чего совершенно нельзя сказать про те храмы, основание которых традиция относит к царской эпохе.

% Liv. X.46.14 - храм Фортис Фортунай
% Liv. XXXIV.53.5–6 - детали строительства храма Фортуны Примигении
% Liv. XL.44.9 - кто собирал деньги на строительство храма Всаднической Фортуны, это испанцы
% Liv. XL.44.10 - об избрании дуумвиров для строительства этого храма

Нельзя также не заметить, что из четырёх храмов Фортуны, сведения об основании которых дошли до нас, три напрямую связаны с военным успехом. Храм Fortis Fortunae консул Карвилий построил на трофейные средства, святилища же Fortunae Primigeniae и Fortunae Equestris были обетованы на полях сражений. Да и поводом к возведению храма Fortunae Muliebris (которая, как мы выяснили, сохраняет черты женской богини, связанной с замужеством) послужило отвращение от Рима военной угрозы. Отметим эту новую черту, которую мы не наблюдали в эпоху царского Рима "--- если не считать того, что Плутарх приписывал царю Анку Марцию мнение о важности удачи в военном успехе (Plut. De Fort. Rom. 5, Mor. 318~F), что, как мы уже объяснили выше, является, судя по всему, позднейшим толкованием. Но значит ли это, что в республиканское время Фортуна стала ассоциироваться, в первую очередь, с военными удачами или неудачами? Разумеется, здесь мы сталкиваемся только с одной из сторон многогранного и противоречивого образа богини Фортуны. Односторонность той картины, каковую мы нарисовали в этой главе, связана с характером источников, проливающих свет на историю культа Фортуны. Действительно, темой труда Тита Ливия является политическая и военная история Рима, и всё, что он сообщает о культе Фортуны, так или иначе связано с событиями из жизни rei publicae. Именно с этим связана однобокость тех сведений, что мы черпаем из его <<Римской истории>>. Ниже мы встретимся со множеством храмов и алтарей Фортуны, о которых свидетельствуют авторы конца Республики "--- начала Империи, и которые, таким образом, должны были быть основаны не позднее того времени, но, увы, мы не располагаем точными сведениями об их истории.

Продолжая разговор о характере источников, относящихся к этому периоду, подчеркнём важность того, что мы уже располагаем письменными свидетельствами современников эпохи. Правда, те сведения, которые мы можем почерпнуть из поэзии Пакувия и Плавта, ничего не дают нам в плане изучения культа Фортуны, зато проливают свет на представление римлян того времени о ней. Мы уже можем, с некоторой долей осторожности, судить о неоднозначном восприятии этой переменчивой богини даже среди низших слоёв общества, с которых Плавт должен был списывать героев своих комедий. Стихотворение же Пакувия, приведённое выше, свидетельствует о том, что римская интеллектуальная элита того времени уже начала перенимать эллинистическое представление о \graeca{T'uqh}, каковая богиня сливалась в их понимании с римской Фортуной.

% 1) Появляется связь Ф. с военной удачей

% 2) Характер источников.

