\chapter*{Обзор использованной литературы}
\addcontentsline{toc}{chapter}{Обзор использованной литературы}

Римская религия "--- благодатная почва для исследователя. Число даже общих трудов по ней весьма велико. Специальные работы, посвящённые Фортуне, представляют собой, по большей части, статьи о ней в энциклопедиях. В отечественной литературе римская религия и Фортуна в частности освещены значительно хуже, чем в зарубежной. Мы разделяем наш обзор на зарубежных и отечественных авторов и стараемся придерживаться хронологического порядка рассмотрения, делая из него незначительные исключения.

\section*{Зарубежная литература}
\phantomsection
\addcontentsline{toc}{section}{Зарубежная литература}

% Hartung

Наиболее ранняя работа по римской религии, к которой мы можем обратиться "--- монография И.~Гартунга, вышедшая в свет в 1836 г. И.~Гартунг сближал Фортуну с божествами судьбы "--- Фатумом и Парками. Из них о первой он говорил как о подвижной и изменчивой, о других же "--- как о неизменных и неподвижных. Они подступали к человеку в момент его рождения и сопровождали его на протяжении всей жизни. Ссылаясь на Макробия (Sat. I.19), Гартунг говорит о четырёх силах, определяющих жизнь человека: Гении, Случае, Влечении и Необходимости\footcite[S. 233]{Hartung1836}. Из этих четырёх сил Фортуна представляла Случай. Таким образом, Фортуна для Гартунга "--- не более чем олицетворение случая. Это довольно односторонняя трактовка образа Фортуны, однако автора извиняет то, что он одним из первых подходил к написанию общего труда по римской религии, не располагая данными эпиграфических, нумизматических источников и не имея возможности воспользоваться достижениями современной археологии. Эта работа была значимым событием для своей эпохи. Гартунг показывает эрудированность и глубокое знание текстов античных авторов. Давая далее обзор некоторых культов Фортуны, он отмечает, что римляне приносили жертвы и подношения Фортуне в дни, обозначавшие важные этапы их жизни (напр., девушки подносили тогу Фортуне Деве при выходе замуж)\footcite[S. 235]{Hartung1836}.


% W. Becker

В 1843 г. В.~Беккер выпустил работу, посвящённую римским древностям\footcite{Becker1843I}. Изложение в ней построено по географическому принципу. Автор описывает как сохранившиеся, так и не дошедшие до нас памятники древнеримской архитектуры, воспроизводя их историю по данным письменных источников. В то время храмы Фортуны на Бычьем форуме и Фортуны Сегодняшнего Дня, известные сегодня по археологическим данным, ещё не были найдены; ему также не были доступны множественные эпиграфические данные, хотя он и обращается к некоторым из важнейших надписей, в частности, к фастам. Однако блестящее знакомство с массой письменных источников и римской топографией позволило ему описать историю множества храмов Фортуны: храм на Бычьем форуме, храм Форс Фортуны, храмы Фортуны Всаднической, Примигении, Римского Народа.

% Fortuna in Smith's Dictionary

В словаре греко-римской биографии и мифологии У.~Смита Фортуне посвящена небольшая статья. В ней Фортуна интерпретируется как богиня случая и доброй удачи <<both in Greece and Italy>>, в Риме же, подчёркивает автор, её воспринимали как надёжную богиню удачи, успеха и всякого рода процветания\footcite[P. 180]{FortunaSmith1867}. Автор, очевидно, совмещает в одном представлении греческую богиню Тиху и римскую Фортуну, а оценка, данная им последней, не только чрезмерно упрощена, как у Гартунга, но и попросту неверна. Впрочем, трудно ожидать чего-то серьёзного от американской публикации второй половины XIX в.

% Г. Буассье

Монография Г.~Буассье <<Римская религия от времён Августа до Антонинов>>, вышедшая в 1874 г., стала определённым этапом в изучении римской религии. В этой работе Буассье выходит за хронологические рамки, определённые названием монографии. Включая в рассмотрение период поздней Республики, когда сложились предпосылки образования империи, он характеризует также римскую религию в целом. Он рассматривает эволюцию религии в широком социальном, культурном, философском, идеологическом контексте. Такой подход позволяет автору показывать неразрывную связь развития римской религии и римского общества. Нам представляется несколько несправедливой критика Н.~А.~Машкина, который, признавая, что труды Г.~Буассье <<имеют определённое значение>>, указывал, что у него <<не устанавливается связи идеологических течений с социальными отношениями>>\footcite[С. 350]{Mashkin1949}. Г. Буассье касается в своём труде не религиозно-идеологических аспектов социальных отношений, а социальных аспектов религиозно-философских представлений древних римлян. Он подробно останавливается на значении религии в государственной пропаганде, на её роли в жизни различных общественных групп и слоёв: высшего общества, женщин, рабов. Мы можем утверждать, что положения Г. Буассье, высказанные им в этой работе, не утратили актуальности и по сей день.

Говоря о поклонении римлян Фортуне, Буассье полагал, что римляне искали у этой богини, в первую очередь, защиты от неожиданных ударов с её стороны, т.е. от ударов судьбы. Римляне, указывал он, <<хорошо понимали, что одна минута может разрушить самые искусные планы, и что окончательный успех дела зависит не от человека>>\footcite[С. 204]{Boissier1914}. Таким образом, Буассье отмечал только непостоянный характер Фортуны, богини удачи и случая.


% Preller

Начиная с первого тома Корпуса латинских надписей Т.~Моммзена, опубликованного в 1863 г., историки древнего Рима стали получать доступ ко всё более расширяющемуся объёму эпиграфических источников. Этот материал позволил исследователям в значительной степени дополнить и уточнить знания о древности. Публикации в области римской религии, основанные также на новом материале, не  заставили себя ждать: монография Л. Преллера <<Римская религия>> остаётся во многом актуальным научным исследованием до сих пор. Приводя исторический обзор культа Фортуны в Риме, Преллер утверждает, что в древности Фортуна воспринималась как богиня доброй удачи (positive Gl\"{u}cksg\"{o}ttin), впоследствии "--- как божество безразличной судьбы (indefferentes Geschick)\footcite[S. 179]{Preller1883}. Как божество судьбы, Фортуна, по Преллеру, была близка Паркам, определяющим судьбу человека в момент его рождения\footcite[S. 193]{Preller1883}.

% Peter in Roschers Lexikon

Обстоятельная статья Р.~Петера о Фортуне помещена в <<Мифологическом лексиконе>> Рошера. В статье подробно рассматриваются изобразительные атрибуты Фортуны (пожалуй, до Д.~Арьи, опубликовавшего свою диссертацию в нач. XXI в., мало кто обращался к этой теме) и связанные с ними представления времён античности, даётся подробное описание множества культов, связанных с этой богиней, от Сервия Туллия до эпохи Империи, исследуется связь Фортуны с культами других божеств. В своей интерпретации этой богини, однако, Петер недалеко ушёл от И.~Гартунга и говорит, что, исходя из имени, это богиня случая (<<wie ihr Name sagt, die G\"{o}ttin des Zufalls>>)\footfullcite[Sp. 1503]{Peter1890Fortuna}.

% A. Aust

Работа Э.~Ауста, посвящённая римским храмам времён Республики\footcite{Aust1889}, представляет собой справочник по храмам божеств и связанным с ними праздненствам. Справочник организован в хронологическом порядке, автор приводит известные в его время данные по письменным и эпиграфическим источникам для каждого храма, а также даёт ссылки на современную ему литературу. В этой работе рассматриваются те культы Фортуны, про которые нам достоверно известно, что с ними были связаны храмы, построенные в республиканское время.

% Carter

Дж. Картер в обстоятельной работе, посвящённой проблемам когноменов римских богов, даёт краткий обзор храмов Фортуны в Риме. Он относит Фортуну к di Italici\footcite[P. 29--30]{Carter1898} и приводит список некоторых её когноменов, которые, впрочем, не все сугубо римские, но также и италийские\footcite[P. 61--62]{Carter1898}.

%В зарубежной, в основном англоязычной, историографии (Бейли, Розе) существует тенденция относить первоначальную Фортуну к исключительно <<сельскохозяйственным божествам>>, а характер богини удачи, каковой она имела, приписывать дальнейшему эллинистическому влиянию.

%Называть Фортуну <<первоначально сельскохозяйственным божеством>> "--- значит оставаться на уровне общих слов и выражений. Никакое последующее эллинистическое влияние не может объяснить, отчего Фортуна так сильно выделилась из среды прочих аграрных богов и богинь, что утратила всякую связь с сельскохозяйственным культом и приобрела совершенно особенные, только ей свойственные черты. Почему именно Фортуна, а, скажем, не Церера и не Теллус? Это точно такие же <<аграрные божества>>. Чтобы ответить на этот вопрос, нам необходимо будет предположить, что ещё в то время, когда культ Фортуны выступает на историческую сцену и получает освещение в наших источниках, эта богиня в представлении римлян уже обладала определёнными качествами, отличавшими её от других богинь, каковые качества и получили своё выражение в дальнейшем развитии культа\footnote{Отметим, что, пытаясь объяснить сущность эллинистического влияния на культ Фортуны, Х.~Розе указывает, что земледельца заставляли обращаться к Фортуне те случайности, независящие от воли человека, которым подвержен труд на земле (см. \cite[P. 238.]{Rose1959}). Таким образом, Розе, говоря о Фортуне как об <<agricultural deity>> как бы исподволь признаёт, что она также была и богиней случая и удачи, т.е. обладала теми <<определёнными качествами>>, отличавшими её от других богинь, о которых мы и говорим. Слабое место в рассуждениях Розе "--- отсутствие явной связи между культом Фортуны исторического периода и сельским хозяйством; эту связь мы можем проследить только для богини Fors Fortuna.}.

%Другая тенденция в историографии (Акстелл, Дюмезиль и др.) "--- считать Фортуну обожествлённым абстрактным понятием par excellence на протяжении всей истории её культа в Риме. Эта точка зрения не учитывает тот факт, что начиная с появления культа Фортуны в Риме, со времён царя Сервия Туллия, Фортуна сразу получает как антропоморфный облик в изображениях, так и вполне ясно очерченную индивидуальность в мифах и сказаниях, отчего затруднительно становится рассматривать саму богиню только как олицетворённую абстракцию. Таким образом, исследователи, отказываясь рассматривать исторические источники во всей их полноте, уходят от проблемы сложности, противоречивости и даже парадоксальности образа богини Фортуны в древнем Риме. Подобные односторонние точки зрения можно встретить и в литературе, вышедшей уже после публикации фундаментального труда Ж.~Шампо\footcite{Champeaux1982}, в котором даётся подробное освещение многосторонности и противоречивости образа архаической Фортуны: так, М.~Липка полагает, что появление изображения Фортуны при царе Сервии "--- свидетельство очень раннего наделения антропоморфным обликом абстрактных божеств\footnote{\cite[P. 89.]{Lipka2009}.  Подобная оценка Липки тем более странная, что он указал работу Ж. Шампо в библиографии (с. 197) и использует ссылки на неё по всей книге.}.

%Итак, в научной литературе получили освещение вопросы происхождения культа Фортуны в Риме и в Лации (сюда же относим вопрос о первоначальном характере этой богини в представлениях римлян и довольно частную проблему этимологии имени Фортуны), история создания храмов Фортуны в Риме, проблема внешнего влияния на культ (этрусского, эллинистического), история становления и символика изобразительных атрибутов богини; довольно подробно рассмотрен вопрос о понятии \textit{fortuna} в произведениях античных авторов.

%Однако если в реконструкции событийной истории среди авторов нет принципиальных разногласий, то в оценке сущности богини Фортуны, особенно в архаическое время, на ранних этапах становления её культа, можно встретить диаметрально противоположные мнения. Здесь надо отказаться от ригористических суждений.

%Убеждение, что Фортуна "--- однозначно обожествлённое отвлечённое понятие, должно быть подвергнуто сомнению, и, хотя его нельзя совершенно отбросить, его необходимо должным образом скорректировать. То же самое следует сказать и о противоположной точке зрения, согласно которой Фортуна изначально "--- аграрное божество, а категория удачи или случая, связанная с понятием \textit{fortuna} "--- наследие эллинистического влияния. Обращаясь к ранней истории культа Фортуны в Риме, нужно отказаться от однозначных суждений и признать, что, с одной стороны, состояние источников не даёт нам оснований для заключения подобного рода выводов, а, с другой стороны, и объективная ситуация, судя по всему, такова, что Фортуна была божеством сложным и парадоксальным, поэтому мы едва ли сможем построить её полный и непротиворечивый образ. Вслед за Ж.~Шампо мы должны стремиться к тому, чтобы как следует выявить каждую грань её сложного образа. Помочь в этом нам может только сопоставление различных видов источников во всей их полноте. Отметим, что недостатком рассмотренной выше историографии является отказ от рассмотрения мифов, связанных с Фортуной, как исторического источника: привлечение мифологического материала в изучаемой области оказывается весьма продуктивно.

% Энман

А. Энман в монографии, посвящённой римским царям, обращается к легендам об основании культа Фортуны в Риме. Он полагает Сервия Туллия <<мифологической личностью, близкой к Фортуне и ее культу>>\footcite[С. 225]{Enman1896} и пытается определить <<первоначальное значение любимца Фортуны>>\footcite[С. 228]{Enman1896}. Энман приходит к выводу, что Сервий Туллий, если следовать из выведенной им этимологии имени этого царя, олицетворял хранителя гадательных жребиев Фортуны, а его мать, Окрисия "--- разбирательницу жребиев. Обе эти фигуры он признавал вымышленными. Объясняя легенду о горящей голове младенца Сервия, он обращался к обрядам dies lustricus, предполагая, что в них <<поклонение Фортуне и гадание посредством жребиев занимали особенно видное место>>\footcite[С. 243]{Enman1896}. Царь Сервий становится у него <<легендарной царской личностью, придуманной для объяснения происхождения культа богини счастья и священной обстановки культа>>\footcite{Enman1896}.

Из всего множества сведений о Сервии Туллии, дошедших до нас, Энман рассматривает только цикл легенд, связанных с Фортуной, и уже из-за этого его выводы являются крайне односторонними. Более того, его умозаключения основываются на таком количестве явных и неявных допущений и предположений, что можно поневоле задаться вопросом: а не являются ли сведения и толкования античных писателей, взятые как есть, более заслуживающими доверия, чем рассуждения Энмана? Преуспев в критике, он, по нашему мнению, не смог построить убедительных положительных заключений. Отметим, однако, что Энман обратил внимание на роль гадания по sortes и подчеркнул, что оно, в отличие от ауспицины и гаруспицины, не только не имело официального статуса, но было распространено по преимуществу в плебейской среде\footcite[С. 229]{Enman1896}. С этим его заключением мы соглашаемся.

% Warde Fowler

В 1899 г. английский исследователь У.~Фаулер опубликовал работу, посвящённую римским праздненствам эпохи Республики\footcite{Fowler1899}. Автор организовал изложение материала по аналогии с римскими фастами, по месяцам года и дням, когда проводились праздненства, посвящённые определённым божествам. В связи с таким построением своей работы, У. Фаулер обращается, в первую очередь, к истории важнейших храмов Фортуны, с которыми были связаны праздненства, внесённые в фасты: Фортуны на Бычьем форуме\footcite[P. 156--157]{Fowler1899}, Мужской Фортуны\footcite[P. 68]{Fowler1899}, Fortis Fortunae\footcite[P. 161--163]{Fowler1899}, Фортуны Примигении в Пренесте\footcite[P. 72, 165--168]{Fowler1899}. Конкретная дата, связанная с каким-либо божеством, служит для У.~Фаулера поводом к рассмотрению проблемы в более широком ключе. Так, он затрагивает важные темы происхождения богини Фортуны и, в частности, этимологии её имени\footcite[P. 163--171]{Fowler1899}. Согласно У. Фаулеру, общепринятой гипотезой являлось сведение имени Фортуны к глаголу \textit{ferre}, что позволяет толковать функцию богини как \textit{dea, quae fert}\footcite[P. 163--164]{Fowler1899} (т.е. богиня, приносящая различные блага, в том числе помогающая деторождению).

В другой своей работе, основанной на его лекциях по римской религии, прочитанных в Оксфордском университете, Фаулер пишет о Фортуне как первоначально женском божестве, покровительнице плодородия и деторождения, которая со временем переняла характер греческой богини Тихи и стала божеством удачи\footcite[P. 235]{Fowler1911}.

% Wissowa

Монография Г.~Виссовы <<Religion und Kultus der R\"{o}mer>> до сих пор остаётся одной из самых фундаментальных и актуальных работ по римской религии. Исследование Виссовы разделено на три части, посвящённые, соответственно, историческому обзору, отдельным божествам и формам поклонения. Из них наибольший объём занимает вторая часть. Г.~Виссова в своём подходе к изучению римской религии использовал концепцию Варрона и делил римские божества на di indigetes (тех, почитание которых было установлено царём Нумой, и которые, поэтому, могут считаться исконными римскими богами) и di novensidenses, тех, чьи культы были заимствованы позднее. Указывая, что почитание Фортуны официально учредил Сервий Туллий, Г.~Виссова относит эту богиню ко второй категории\footcite[S. 206]{Wissowa1902}. Он даёт обстоятельный обзор истории культа Фортуны начиная со времён Сервия Туллия и до эпохи Империи. Вслед за Л.~Преллером, Г.~Виссова подчёркивает первоначальную связь Фортуны с такими божествами судьбы как как италийские парки и греческие мойры\footcite[S. 213]{Wissowa1902}. Интересно его сравнение представления древних о Фортуне с представлением о fatum или fata\footcite[S. 213--214]{Wissowa1902}.

% H. Axtell

В 1907 г. американец Г.~Экстелл опубликовал работу, посвящённую обожествлению абстрактных идей у римлян (\fullcite{Axtell1907}). Определяя предмет исследования, Экстелл даёт позитивное определение обожествлённым абстракциям. Он рассматривает только такие обожествлённые абстракции, которые связаны с умственными понятиями (e.g. Annona, Pecunia) и исключает из рассмотрения, например, Tranquilitas, поскольку та имеет конкретное материальное выражение: спокойное море\footcite[P. 7]{Axtell1907}. Образы обожествлённых понятий у римлян, согласно Экстеллу, были неясными представлениями, лишёнными индивидуальности и мифологического окружения, и даже их пол был не более чем грамматическим родом имени\footcite[P. 86]{Axtell1907}.

Включая Фортуну в число обожествлённых абстракций, Экстелл пишет, что первоначально она имела характер богини-покровительницы, дарующей удачу, а впоследствии, под влиянием эллинистических представлений, стала олицетворением случая\footcite[P. 9--10]{Axtell1907}. Тем не менее, ему приходится признать, что Фортуна весьма отличалась от прочих божеств, олицетворяющих отвлечённые понятия: он отмечает, что она была почти столь же антропоморфна, как и любой из двенадцати главных богов римлян\footcite[P. 88]{Axtell1907}. Заметим, что здесь можно было бы отказаться от слова <<почти>>.

% Otto

Наиболее подробное исследование культа Фортуны мы находим в энциклопедии Pauly-Wissowa\footcite[Sp. 12--42]{FortunaOtto1910}. Статья разделена на две части: в первой рассматриваются общие проблемы изучения культа Фортуны, во второй "--- собственно культ в разнообразных своих проявлениях, поклонение Фортуне под различными именами (Beinamen). Автор статьи подчёркивает характер случайности, присущий Фортуне изначально\footcite[Sp. 13]{FortunaOtto1910}. Рассмотрению подвергаются различные стороны многогранного и противоречивого характера богини: её связь с эллинистической \graeca{T'uqh} и греческой мифологией, матрональный характер её культа, общеиталийские аспекты поклонения Фортуне, её роль как богини-предстказательницы\footcite[Sp. 14--15]{FortunaOtto1910}.

% J. Frazer

Один из самых фундаментальных трудов, посвящённых религии и мифологии "--- <<Золотая ветвь>> Дж. Фрэзера. Для того, чтобы кратко охарактеризовать это объёмное произведение, приведём слова А.~Гольденвейзера, процитированные М.~Элиаде: <<Незначительно в теоретическом отношении, непревзойдённо в отношении собранного материала по первобытной религии>>\footcite[С. ]{Eliade1999}. Римскую богиню Фортуну Фрэзер упоминает вскользь. Говоря о римских легендах, связанных с Фортуной и царём Сервием Туллием, он признаёт, что они слишком неясны, чтобы дать представление о характере этой богини\footcite[P. 193]{Frazer1911II}. Он проводит параллели между сказаниями о связи царя Сервия с Фортуной и царя Нумы с нимфой Эгерией\footcite[P. 272]{Frazer1911II}. Фрэзер также обратил внимание на знаменательную дату праздника богини Fors Fortuna, день летнего солнцестояния, полагая это праздненство противоположным сатурналиям\footnote{\mancite{Ibid.}}.

% Bailey

С.~Бейли в монографии, посвящённой этапам развития римской религии, указывает, что Фортуна первоначально была божеством плодородия и, в частности, деторождения, а как олицетворение случайности была переосмыслена под влиянием культа Тихи в городах Великой Греции\footcite[P. 137]{Bailey1932}. Он отмечает связь Фортуны с эллинистической Исидой, каковая прослеживается во времена Империи\footcite[P. 255, 258]{Bailey1932}. Интересно, что Бейли соотносит Фортуну и \graeca{T'uqh} с таким понятием как fatum\footcite[P. 234, 255]{Bailey1932}.

% F. Altheim

Ф.~Альтхайм в обширной монографии, посвящённой римской религии\footcite{Altheim1938}, касается лишь некоторых аспектов культа Фортуны. Так, он рассматривает влияние эллинистических представлений о богине Тихи на культ Фортуны\footcite[P. 190]{Altheim1938}, рассматривает частные случаи почитания Fortunae Huiusque Diei\footnote{Ibid.} и Fortunae Reducis\footcite[P. 268]{Altheim1938}.

% Dumezil

К истории культа Фортуны обращался и знаменитый религиовед Ж.~Дюмезиль. Вопросу возникновения этого культа в Риме посвящена его статья <<Сервий Туллий и Фортуна>>\footcite{Dumezil1943}. К сожалению, эта работа оказалась для нас недоступной, поэтому мы сообщаем её содержание в пересказе А.~И.~Немировского\footcite[С. 85--86]{Nemirovsky1964}. Дюмезиль исходил из своей концепции первоначального деления индоевропейского общества на три общественных группы (жрецов, воинов и земледельцев), у каждой из которых были свои боги-покровители\footfullcite{Dumezil1986}. Экстраполируя это положение на римскую историческую действительность, он указывает в качестве такой троицы богов-покровителей Юпитера, Марса и Квирина. Устранение этого архаического деления на три группы античная традиция приписывает Сервию Туллию, который также сыграл решающую роль в утверждении культа Фортуны в Риме. Дюмезиль полагает, что мероприятия Сервия в сакральной области полностью соответствуют его социальным преобразованиям. Фортуна "--- не просто материнское божество, но богиня, олицетворяющая богатство как основной принцип огранизации общества.

% H. Rose

Х.~Розе полагает, что изначально культ Фортуны имел сельскохозяйственный характер (<<there are some fairly definite indications that she was originally an agricultural deity>>\footcite[P. 238]{Rose1959}, пишет он). С осторожностью он указывает на связь культов Фортуны и Матери Матуты. По его мнению, характер богини случая и удачи она приобрела позднее, в силу того, что в сельском хозяйстве многое зависит от обстоятельств, над которыми земледелец не властен\footnote{Ibid.}. Далее, полагает Х.~Розе, Фортуна со временем приобрела множественные проявления (Мужская Фортуна, Женская и т.д.). Также он говорит о существовании знаменитых лацийских святилищ: оракула в Пренесте и храма в Анции\footcite[P. 238--239]{Rose1959}.

% K. Latte

В 1960 г. К.~Латте издал фундаментальный труд по римской религии "--- в нём, по утверждению самого автора, на основании новых данных должны были быть пересмотрены положения Г.~Виссовы, со времени публикации работы которого прошло уже более полувека\footcite[S. VII]{Latte1960}. Отдельная глава в этой монографии посвящена богине Фортуне. Автор указывает на общелатинское происхождение культа Фортуны\footcite[S. 178]{Latte1960}. Привлекая широкий фактологический материал, он описывает историю развития культа Фортуны в Риме от времён Сервия Туллия до эпохи Империи. Отметим, что в отношении богини Фортуны труд К.~Латте носит менее концептуальный характер по сравнению с работой Г.~Виссовы. Сосредоточиваясь на подробном изложении фактов, К.~Латте оставляет в стороне те проблемы интерпретации Фортуны, которые затрагивает в своём труде Г.~Виссова.

% J. Ferguson

В работе Дж. Фергюсона о религиях Римской империи отдельная глава посвящена обзору богини Тихи. Таковой выбор темы не случаен: в эпоху Империи, выходя за пределы Рима и даже Италии, мы наблюдаем, как размываются границы представления о божествах, и Фортуна начинает идентифицироваться с Тихой, так, что сами древние авторы зачастую не разделяют их. Дж.~Фергюссон указывает, что если Плутарх пишет о Тихе, подразумевая римскую Фортуну, то Плиний Старший "--- наоборот, повествуя о Фортуне, подразумевает общеимперский культ Тихи\footcite[P. 79]{Ferguson1970}. В его работе большое внимание уделяется представлению древних авторов об этих двух соотносящихся божествах. Опираясь на множество нарративных источников, в том числе на философские и художественные произведения, Фергюсон рисует образ противоречивой, ненадёжной и непостоянной богини случая и удачи, отношение к которой в эпоху античности было двойственным. Отметим, что в то время как многие другие авторы уделяли большое внимание ранним этапам становления культа Фортуны в Риме, Фергюсон показывает, что и в эпоху поздней Империи, Домината, роль культа Фортуны-Тихи была высока\footcite[P. 86--87]{Ferguson1970}. 

% Dumezil

В работе, посвящённой архаической римской религии\footcite{Dumezil1974}, Ж.~Дюмезиль не выражает ясной точки зрения на происхождения культа Фортуны в Риме. Представление о том, что этот культ проник в Рим из Лация, отмечает Дюмезиль, основывается только на факте существования в Лации таких древних и более знаменитых, чем римские, святилищ богини, как Пренестинское и Анцийское, из которых первое, безусловно оказало влияние на собственно римский культ. Однако Дюмезиль однозначно относит первоначальную Фортуну к категории обожествлённых абстракций (она, по его мнению, представляла не более чем олицетворение удачи и счастливого случая) и пишет, что красочные легенды, связанные с этой богиней, возникли позднее, с развитием культа\footcite[P. 424.]{Dumezil1974}.

В своей работе, посвящённой индоевропейской религии в целом, вышедшей в 1977 г., Дюмезиль сопоставляет Фортуну с ведийским божеством удачи и случая Бхаги\footcite[С. 75, 80, 131--132]{Dumezil1986}.

Отметим, что особенности исследовательского подхода Дюмезиля заключаются в том, что, используя методы сравнительной мифологии, он рассматривает в римской религии, в первую очередь, общее, что связывает её с индоевропейской культурой и религией в целом. Он задаётся целью выявить концепции, лежащие в основе общеиндоевропейских представлений о религии, поэтому к интересущей нас проблеме он подходит с противоположной стороны, ибо нас интересует par excellence то особенное, что составляет отличительные черты культа Фортуны в Риме. А.~И.~Немировский критиковал подход Ж.~Дюмезиля, указывая, в частности, что <<в древнем Риме не существовало особой жреческой касты, которую Ж.~Дюмезиль постулирует у всех индоевропейских народов>>\footcite[С. 177]{Nemirovsky1964}.

%Отметим, что особенности исследовательского подхода Дюмезиля заключаются в том, что, используя методы сравнительной мифологии, он рассматривает в римской религии, в первую очередь, общее, что связывает её с индоевропейской культурой и религией в целом. Он задаётся целью выявить концепции, лежащие в основе общеиндоевропейских представлений о религии, поэтому к интересущей нас проблеме он подходит с противоположной стороны, ибо нас интересует par excellence то особенное, что составляет отличительные черты культа Фортуны в Риме. А.~И.~Немировский критиковал подход Ж.~Дюмезиля, указывая, в частности, что <<в древнем Риме не существовало особой жреческой касты, которую Ж.~Дюмезиль постулирует у всех индоевропейских народов>>\footnote{Немировский~А.~И. Указ. соч. С. 177.}.

%Таким образом, Ж.~Дюмезиль, хотя и не называет Фортуну неопределённым <<сельскохозяйственным божеством>>, всё же отказывается признавать, что римляне от начала придавали ей антропоморфный облик и наделяли её определёнными личностными качествами; с другой стороны, он считает, что ей был изначально присущ характер богини удачи и счастья.

% Encyclopaedia Britannica

В <<Encyclop\ae{}dia Britannica>> в статье, посвящённой римской религии\footcite[P. 1063]{Britannica15_1978}, сообщается, что Фортуна, чей культ был основан в Риме в конце царского периода, была первоначально <<a farming deity>>, божеством плодородия, впоследствии же стала идентифицироваться с эллинистической богиней Тихой, покровительницей городов и олицетворением удачи.

% Кайянто

Статья И. Кайянто в ANRW "--- одна из немногих работ, специально посвящённая Фортуне. Статья состоит из двух частей: в первой автор рассматривает историю развития культа Фортуны в Риме, во второй "--- даёт обзор особенностям использования слова \textit{fortuna} у различных римских писателей. Хронологические и географические рамки исторического исследования Кайянто чрезвычайно широки: он начинает свой обзор с возникновения культа Фортуны в Риме в царское время и завершает его поклонением Фортуне в римских провинциях императорского времени. Это не идёт на пользу сочинению, поскольку размывается сам предмет исследования. В работе такого ограниченного формата нет возможности охватить все основные проблемы, которые возникают при охвате столь обширного материала.

Тем не менее, И. Кайянто затронул ряд немаловажных вопросов, касающихся происхождения культа Фортуны в Риме\footcite[S. 504]{Kajanto1981}. Он утверждает, что по своему первоначальному характеру Фортуна была богиней-покровительницей, приносящей удачу, защитницей людей и местностей, подобно гению. В литературе же, отмечает Кайянто, слово \textit{fortuna} принимает ряд значений, зачастую противоречивых. Греческое понимание \graeca{T'uqh} как слепого случая наложилось на типично римские представления о Фортуне\footcite{Kajanto1981}.

% H. H. Scullard

В работе Г. Скалларда, представляющей собой справочник по римским праздненствам\footcite{Scullard1981}, организованном в календарном порядке, рассматриваются те культы Фортуны, которые были занесены в фасты и которые были связаны с определёнными календарными праздниками. Это культы Мужской Фортуны, Форс Фортуны, Фортуны Римского Народа и Примигении, а также праздник Матралий, приходившийся на день освящения храма Фортуны на Бычьем форуме. Во вступлении, посвящённом обзору общерелигиозных вопросов, связанных с праздненствами, Скаллард высказывает мнение, что Фортуна первоначально была сельскохозяйственным божеством, привнесённым в Рим из Антия\footcite[P. 19]{Scullard1981}.

% Champeaux

Самое подробное исследование, посвящённое Фортуне "--- две монографии французской исследовательницы Ж.~Шампо, из которых первая посвящена истории римских культов Фортуны в царскую и республиканскую эпоху\footcite{Champeaux1982}, вторая "--- трансформации представления о Фортуне в эпоху от расцвета Республики до смерти Цезаря\footcite{Champeaux1987}. Признавая разносторонность и многогранность образа Фортуны, Ж.~Шампо критикует исследователей, сводивших Фортуну к олицетворению слепого случая, к материнскому или аграрному божеству\footcite[P. VIII--XIV]{Champeaux1982}. Отказываясь приводить противоречивый образ Фортуны к общему знаменателю, она сосредоточивается на как можно более детальном описании каждого из аспектов культа этой богини, чему и посвящено её объёмное и детальное исследование. Ж. Шампо категорически отказывается признавать Фортуну обожествлённым понятием в римской религии\footcite[P. XXI]{Champeaux1982}.

% K. Mustakalio

К. Мустакалио в своей статье <<Some Aspects of the Story of Coriolanus and the Women Behind the Cult of Fortuna Muliebris>>\footcite{Mustakallio1990} обращается к истории женского посольства к Кориолану, ведущему армию вольсков на Рим, и связанному с этим основанию храма Женской Фортуны. Автор оспоривает гиперкритические положения некоторых исследователей и обосновывает достоверность тех подробных сведений, которые сообщает нам античная традиция об истории возникновения культа Женской Фортуны в Риме.

% Richardson

<<A New Topographical Dictionary of Ancient Rome>> Л. Ричардсона\footfullcite{Richardson1992} "--- пожалуй, наиболее полная и актуальная на сегодняшний день работа, посвящённая римским древностям. В этом издании учтены большинство последних археологических открытий в области исследования культа Фортуны, приведены карты, археологические планы и реконструкции. Некоторые иллюстрации из этой работы размещены в приложении к настоящему исследованию. Ричардсон поместил в свой словарь не только дошедшие до нас и уже изученные римские древности, но и те, о существовании которых мы знаем только из письменных источников или надписей, местоположение которых неизвестно. Недостатком этой работы можно назвать отсутствие подробной карты Рима, на которую были бы нанесены указанные в ней памятники, а также организация справочника в алфавитном порядке, из-за чего в нём трудно найти информацию о памятнике, если не знать его точного названия или местоположения. 

% Adam Ziolkowsky

Монография А. Жолковского <<The Temples of Mid-Republican Rome>>\footcite{Ziolkowski1992} посвящена храмовому строительству в Риме в период 396--219 гг. до н.э. Автор обращается к вопросам обетования, постройки, дедикации храмов, их топографического положения. В рассмотрение автора попадают храм Форс Фортуны на берегу Тибра, построенный консулом Карвилием в 393 г. до н.э. рядом с храмом той же богини, освящённом Сервием Туллием, а также храмы Фортуны Римского Народа, точная дата основания которых нам неизвестна.

% J. Shayd

В работе Дж.~Шайда о римской религии, вышедшей в 1997 г., Фортуне уделено крайне мало внимания. Можно отметить, что автор включает культ Женской Фортуны в число матрональных культов\footcite[С. 137]{Shayd2006} и отмечает, что участвовать в поклонении богине могли также и рабы\footcite[С. 160]{Shayd2006}.

% Eric M. Orlin

В работе Э. Орлина <<Temples, Religion and Politics in the Roman Republic>> проблема храмового строительства рассматривается через призму политической истории Римской республики. Автор подходит с позиции противопоставления коллективизма Римской республики, выраженного сенатом, и индивидуализма отдельных личностей, занимающих магистратские должности. Э. Орлин подчёркивает, что политический строй Римской республики с её срочными и коллегиальными магистратурами не позволял амбициозным личностям в полной мере реализовать свои устремления; существовали почти непреодолимые препятствия тому, чтобы отдельная личность приобрела слишком большое влияние\footcite[P. 2]{Orlin2002}. При таких обстоятельствах решающую роль в соревновании аристократического честолюбия, согласно автору, играли обетование, строительство и посвящение храмов, т.к. решение о возведении храма конкретному божеству не требовало одобрения сената и народа и было исключительно личным делом\footcite[P. 4--5]{Orlin2002}. Э. Орлин акцентирует внимание на обстоятельствах обетования, локации и дедикации храмов, на юридическую сторону посвящения культовых сооружений. В связи с этим автор включает в своё рассмотрение только те храмы, для которых мы располагаем подобного рода сведениями из письменных источников. Из интересующих нас храмов это храм Женской Фортуны, храм Форс Фортуны, возведённый в 393 г. до н.э. Карвилием, храмы Фортуны Примигении и Всаднической.

% D. A. Arya

Д.~Арья создал довольно пространную диссертацию, посвящённую богине Фортуне в эпоху Римской империи. Большое внимание в ней он уделяет привлечению изобразительного материала, описанию и анализу памятников искусства. Довольно широко, хотя, на наш взгляд, несколько бессистемно, он привлекает и материал нарративных источников. Хотя темой его исследования заявлен культ Фортуны в эпоху Империи, хронологические рамки работы несколько шире, и Д.~Арья захватывает в своём обзоре также и период поздней Республики. Говоря об актуальности выбранной темы, автор подчёркивает, что в эпоху Империи Фортуна приобретает новую значимость (новую иконографию, культовые ассоциации и разновидности культа, связанные с императорской властью)\footcite[P. V, VIII--IX]{Arya2002}. Несмотря на заявленную тему, Д.~Арья не рассматривает в своей работе период империи целиком, сосредоточивась на изучении эпохи Августа и сделав в конце работы некоторый экскурс вплоть до времени Севера "--- в то время как работа Дж.~Фергюсона, о которой шла речь несколько выше, показывает, что эпоха Домината также предоставляет весьма любопытный материал для интересующей нас темы.

% Brill's New Pauly

В энциклопедии <<Brill's New Pauly>> богине Фортуне посвящена отдельная статья\footcite[P. 505--509]{FortunaBrill2004}. В ней даётся краткий обзор истории культа в Риме и за его пределами, в Лации. Отмечается, что крайне трудно выдвинуть единую версию происхождения этой богини и её функций\footcite[P. 505]{FortunaBrill2004}. Отдельный параграф посвящён представлению о богине Фортуне как в письменных источниках, так и в изобразительном искусстве\footcite[P. 508]{FortunaBrill2004}.

В последние десятилетия в историографии ведётся поиск новых подходов и путей интерпретации уже известных фактов. Эта тенденция связана с кризисом не только истории, но и гуманитарных дисциплин вообще в эпоху постмодерна. Две следующих работы представляют собой результат такого поиска новых подходов к истории римской религии.


Первая из них "--- монография А.~Кларк <<Divine Qualities: Cult and Community in Republican Rome>>, вышедшая в 2007 г. Насколько мы можем понять, автор в исследовании пытается определить, что же такое <<римляне>> и <<римский народ>>, который был так склонен обожествлять абстрактные понятия, и какую роль культы таких божеств играли в жизни римлян. Категория <<римскости>> (<<roman-ness>>), согласно ей, недостаточно точно определена в науке\footcite[P. 5--6]{Clark2007}. Никаких более ясных целей и задач, стоящих перед исследователем, из путаного и многословного введения мы понять не можем. А.~Кларк не даёт точное определение divine qualities, подобно тому, как Экстелл привёл определение deified abstractions (но оба имеют в виду одно и то же). В этой монографии автор обращается к довольно большому числу фактов о Фортуне и однозначно относит её к числу divine qualities. Однако важных вопросов происхождения культа Фортуны, интерпретации её образа и того, насколько и в каком качестве её можно отнести к обожествлённым абстракциям, в этой работе не ставится.

М. Липка в монографии <<Roman Gods (A Conceptual Approach)>> ставит целью определить понятие <<бога>> в представлении древних римлян, основываясь на определённых концептах (места, времени, ритуала и т.д.)\footcite[P. 8]{Lipka2009}. Наряду с другими божествами, Липка обращается также к фактам о культе Фортуны для обоснования или иллюстрации своих тезисов.

% В трудах этих двух авторов не высказывается ни новых, ни старых идей о культе Фортуны в Риме. Не затрагивают они также проблемы возникновения этого культа, его трансформации на протяжении эпохи Республики и в период Империи. 

В обеих монографиях авторы оперируют с уже известными фактами о Фортуне. Они не пытаются дать своё видение образа этого божества или обратиться к проблеме возникновения и развития культа Фортуны в Риме. Обе работы представляют мало интереса для рассматриваемой темы.

\section*{Отечественная литература}
\phantomsection
\addcontentsline{toc}{section}{Отечественная литература}

% Брокгауз и Ефрон

Первая по времени публикация о Фортуне в отечественной историографии "--- статья в энциклопедическом словаре Брокгауза и Ефрона. Она основана, по большей части, на данных Р.~Петера и Г.~Виссовы. В этой статье отмечается, что Фортуны культ был одним из самых древних заимстствованных италийских культов\footcite[С. 320--321]{ESBE1902}. Указывается вероятный путь заимствования: из Лация, где находились древние святилища в Анции и Пренесте. В статье обрисовывается история становления культа, отмечается, что <<вскоре он дифференцировался на сотни и тысячи отдельных культов>>\footcite[С. 320--321]{ESBE1902}, характеризуются разные стороны характера Фортуны: покровительство определённым лицам, группам, учреждениям, дням, временам года и т.п. Не остаются без внимания также изобразительные атрибуты, которыми древние наделяли богиню счастья: рог изобилия, шар, руль. Подчёркивается слияние Фортуны с другими божествами, в частности, с эллинистической Исидой.

% Немировский

Работа А.~И.~Немировского <<Идеология и культура раннего Рима>>\footfullcite{Nemirovsky1964} посвящена первоначальным этапам становления римской религии. Автор обращается к истокам культа Фортуны, относящимся ко времени царя Сервия Туллия. Он полагает Фортуну изначально материнским божеством\footcite[С. 62]{Nemirovsky1964}, подробно останавливаясь на её качествах женской богини. Но он считает, что Фортуна одновременно являлась и божеством судьбы\footcite[С. 74]{Nemirovsky1964}, указывая на существование оракулов, связанных с этой богиней. В круг внимания автора попадают святилища Фортуны, основанные в царское время и в эпоху ранней Республики: святилище на Бычьем форуме, храм Женской Фортуны и Фортуны Примигении в Пренесте. А.~И.~Немировский подробно останавливается на рассмотрении богатого мифологического материала, относящегося к интересующей его ранней эпохе\footcite[С. 73--74]{Nemirovsky1964}. Обращается он и к вопросу об этимологии имени Фортуны и, со ссылкой на М.~Маркони, сводит её к слову \textit{fortis [for c tis]} в значении <<сильная>>, <<богатая>>, <<плодовитая>>, а также <<беременная>>\footcite[С. 72]{Nemirovsky1964}.

% Штаерман

В своём труде, посвящённом социальным аспектам римской религии, Е.~М.~Штаерман обращается к богатому фактическому материалу, связанному с культом Фортуны на протяжении всей римской истории "--- от царских времён до эпохи Империи\footcite{Shtaerman1987}. В связи с особенностями построения этой работы Е.~М.~Штаерман, в ней не выделяются отдельные главы, посвящённые конкретным божествам, в том числе и Фортуне. Это не позволило автору обратиться к каким-либо общим концепциям, каковые сложились в современной науке по поводу представления об этой богине или об истории её культа, тем не менее, ценность этой работы заключается в том, что Е.~М.~Штаерман вписала историю культа Фортуны в общую картину развития религиозных институтов римского общества.

% Мифы народов мира

В энциклопедии <<Мифы народов мира>> о Фортуне пишет А.~И.~Немировский в статье <<Италийская мифология>> и Е.~М.~Штаерман в статье <<Фортуна>>. А.~И.~Немировский называет Фортуну древнейшим материнским божеством Италии и высказывает предположение, что первоначально её образ, вероятно, был слит с образом бога Портуна, а впоследствии женская и мужская ипостаси этого божества разделились: <<Портунус (мужская половина) "--- покровитель мужской доли, а Фортуна (женская половина) "--- покровительница женской доли. Позднее в италийских мифах Портунус и Фортуна отделились друг от друга>>\footcite[С. 578]{Nemirovsky1987}. Впоследствии же Фортуна стала почитаться как богиня судьбы и покровительница гаданий. Такую необычную гипотезу мы не встречали более нигде. А.~И.~Немировский обосновывает её близостью звучания слов \textit{Fortuna} и \textit{Portunus}, а также наличием похожих слов в других языках (\graeca{p'oros} в древнегреческом и <<пора>> в русском). На наш взгляд, подобные этимологии, основанные на близости звучания слов в различных языках, не являются доказательными.

Е.~М.~Штаерман в статье <<Фортуна>> описывает эту богиню как первоначально покровительницу урожая, материнства и женщин, которая <<в классическое время>> стала идентифицироваться с греческой Тихой и почиталась как богиня счастья, случая и удачи\footcite[С. 571]{Shtaerman1988}. Е.~М.~Штаерман также даёт краткий обзор наиболее значительных культов, связанных с Фортуной.


\section*{Выводы}
\phantomsection
\addcontentsline{toc}{section}{Выводы}

%Основные проблемы, связанные с Фортуной и её культом, которые получили освещение в исторической науке, следующие: проблема возникновения и первоначального характера культа Фортуны в Риме; интерпретация образа Фортуны в римском и лацийском культе; трансформация культа Фортуны на протяжении Республики и в эпоху Империи; образ Фортуны в произведениях различных античных авторов.

Несмотря на широкое освещение темы Фортуны в литературе, среди авторов существуют значительные разногласия по основным проблемам, связанным с образом и культом этой богини. Если до второй половины XIX в. исследователи простодушно считали её богиней удачи или человеческой судьбы, то впоследствии стала очевидна сложность её образа и его трансформация с течением времени. Фортуну стали называть богиней судьбы, близкой к Паркам; материнским божеством, покровительницей женщин и деторождения; сельскохозяйственным божеством, дарующим плодородие "--- таковы основные интерпретации её первоначального образа, не считая более экзотических. Разнообразие форм поклонения Фортуне в Риме было столь велико, что современная историческая наука не смогла дать единый и непротиворечивый её образ, а Ж.~Шампо высказывает сомнения, что это вообще возможно. Существуют также разногласия в вопросе о том, можно ли причислять Фортуну к числу обожествлённых понятий (таких, как Ops, Virtus, Felicitas etc.), а если можно, то в какой мере, или, вернее, в каком из её многочисленных аспектов. В отечественной историографии эти вопросы почти не поднимались и решены неубедительно.

В целом, в мировой исторической науке признаётся противоречивость и многовалентность образа Фортуны, выраженного в культе и известных представлениях древних римлян. Нет согласия среди авторов также по некоторым частным вопросам, связанным с конкретными проявлениями культа. Всё это оставляет простор для дальнейших интерпретаций и поиска ответов на загадки, связанные с культом Фортуны в Риме.

% Штаерман: богатый фактический материал с опорой на многочисленные источники. К сожалению, то, что касается культа Фортуны, она не рассматривает концепции, имеющие место быть в историографии.


%К истории культа Фортуны обращался и знаменитый религиовед Ж.~Дюмезиль\label{DumezilFort}. Вопросу возникновения этого культа в Риме посвящена его статья <<Сервий Туллий и Фортуна>>\footnote{Dum\'{e}zil,~G. Servius et la fortune: essai sur la fonctíon sociale de louange et de bl\^{a}me et sur les \'{e}l\'{e}ments indo-europ\'{e}ens du cens romain. 1943.}. Воззрения, изложеннные в этой статье, он повторял и на страницах своей работы <<Архаическая римская религия>>\footnote{Dum\'{e}zil,~G. La Religion romaine archa\"\i{}que, avec un appendice sur la religion des \'{E}trusques. 1966.}. К сожалению, эти работы недоступны для нас, поэтому мы сообщаем их содержание в пересказе А.~И.~Немировского\footnote{Немировский~А.~И. Указ. соч. С. 85--86.}. Дюмезиль исходил из своей концепции первоначального деления индоевропейского общества на три общественных группы (жрецов, воинов и земледельцев), у каждой из которых были свои боги-покровители\footnote{См. Дюмезиль,~Ж. Верховные боги индоевропейцев. М., 1986.}. Экстраполируя это положение на римскую историческую действительность, он указывает в качестве такой троицы богов-покровителей Юпитера, Марса и Квирина. Устранение этого архаического деления на три группы античная традиция приписывает Сервию Туллию, который также сыграл решающую роль в утверждении культа Фортуны в Риме. Дюмезиль полагает, что мероприятия Сервия в сакральной области полностью соответствуют его социальным преобразованиям. Фортуна "--- не просто материнское божество, но богиня, олицетворяющая богатство как основной принцип огранизации общества.

%\nocite{*}
%\printbibliography


