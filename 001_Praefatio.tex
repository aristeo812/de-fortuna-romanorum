\setlength{\epigraphrule}{0pt}
\chapter*{Введение}
\epigraphhead{\epigraph{Я люблю того, кто стыдится, когда игральная кость выпадает ему на счастье, и кто тогда спрашивает: неужели я игрок-обманщик?}{\textit{Ницше}}}
\addcontentsline{toc}{chapter}{Введение}


%\setlength{\epigraphrule}{0pt}
%\begin{epigraphs}
%\qitem{Где взять слова, какие выбрать струны,\\
%чтоб о Фортуне в песне рассказать,\\
%о том, как все зависим от Фортуны\\
%\\
%и, на себе неся её печать,\\
%вдруг начинаем видеть в мрачном цвете\\
%то, что привыкли розовым считать?}{Никколо Макиавелли}
%\epigraph{Для того чтобы создать большую книгу, надо выбрать большую тему}{Герман Мелвилл}
%\end{epigraphs}

%Цель нашего исследования "--- проследить историю почитания богини Фортуны в Риме в процессе возникновения и развития её культа на протяжении от эпохи царей до начала Империи (I в. н.э.). 

% Хотя нас от Макиавелли отделяют добрых полтысячелетия, нам понятны его слова про фортуну.

% Про зарождение к.ф. на берегах тибра в отдалённую от нас эпоху царского Рима, того Рима, который тогда был всего лишь одним из множества античных городов, далеко не самым выдающимся, и, казалось, ничто не предвещало того, что он распространит своё влияние далеко за пределы Италии, и римская культура станет понятной множеству народов. Частью этой культуры является и образ богини Фортуны.
% Именно в царскую эпоху на берегах Тибра возникают первые храмы этой богини, чей образ, пережив тысячелетия, всё ещё узнаваем нами.

% Мы намереваемся показать, что развитие культа Фортуны было неразрывно связано с развитием римского государства и общества.

Фортуна "--- одна из наиболее любопытных богинь римского пантеона. Она не стояла в ряду величайших божеств, таких, как Юпитер или Юнона, и, тем не менее, пользовалась огромной популярностью среди римлян. Широчайшее разнообразие форм поклонения Фортуне делает её образ парадоксальным. В истории развития культа Фортуны остаётся немало белых пятен. Ниже будет показано, что среди исследователей нет единого мнения касательно проблем происхождения Фортуны, возникновения её культа в Риме, характера его развития, а также по ряду частных вопросов, связанных с ним. Никто из отечественных исследователей специально не занимался этими вопросами, и мы попытаемся восполнить этот пробел.

Итак, выбрав предметом исследования богиню Фортуну и её римский культ, мы ставим перед собой цель проследить историю развития этого культа и древеримских представлений о Фортуне на протяжении от поздней царской эпохи, когда первые храмы Фортуны возникают в Риме, до начала Империи (I в. н.э.). Географические рамки исследования культа Фортуны ограничиваются городом Римом и его ближайшими окрестностями. Однако говоря об образе Фортуны в целом, мы должны учитывать, что собственно римское представление об этой богине было тесно связано с общеиталийским, поэтому в некоторых случаях мы привлекаем эпиграфический и изобразительный материал, происходящий не из Рима. Также нам нельзя обойтись без обращения к популярным лацийским культам Фортуны в Пренесте и Анции, хотя они не являются предметом нашего исследования.

Обозначенная цель ставит перед нами три задачи: во-первых, раскрыть историю возникновения и развития культа Фортуны в Риме в указанный период; во-вторых, охарактеризовать античные представления о Фортуне и проследить их эволюцию "--- решение этих вопросов позволит нам обрисовать как можно более полный и многогранный образ этой богини. И, наконец, в-третьих, нам необходимо выявить закономерности развития культа Фортуны в Риме.

Первая глава работы посвящена обзору культа Фортуны: мы обращаемся к истории создания храмов и святилищ Фортуны, прослеживаем появление её различных когноменов, выявляем тенденции развития культа. Насколько возможно, повествование построено в хронологическом порядке, однако история возникновения многих эпиклез Фортуны нам неизвестна. Во второй главе мы обращаемся к образу богини Фортуны, который находит отражение в сочинениях античных писателей, в произведениях искусства, а также в мифах. Третья глава посвящена проблемам возникновения и развития культа Фортуны в Риме и вопросу о месте этой богини в римском пантеоне.

Приложения к исследованию нумеруются буквами латинского алфавита. В приложении A помещены списки когноменов Фортуны, её календарных праздников, а также карта-схема известных храмов Фортуны в Риме. В приложении B приведены археологические материалы (фотографии и схемы), используемые нами в исследовании. В приложении C "--- фотографии римских монет, а в приложении D "--- изобразительные свидетельства. Иллюстрации нумеруются арабскими цифрами внутри каждого раздела приложений: так, рис.~B.2 "--- рисунок за номером 2 в приложении B.


%Само исследование разбито на три главы, в которых изложение материала организовано в хронологическом порядке. В первой главе мы рассматриваем начальный этап становления культа Фортуны в Риме, приходящийся на конец эпохи царей. Это ключевой период для настоящего исследования, в который были заложены основные тенденции развития культа Фортуны в последующее время, а храмы, время основания которых относится к этой эпохе, существовали и почитались и в эпоху Империи, и даже, по некоторым отрывочным сведениям, вплоть до Домината. Во второй главе рассматривается период от начала Республики до II в. до н.э. Здесь мы прослеживаем историю сооружения некоторых храмов Фортуны, о которых сообщают наши источники, а также обращаемся к самым ранним свидетельствам современников о Фортуне, каковые дошли до нас. Третья глава охватывает период I в. до н.э. "--- I в. н.э., в который происходит наивысшее развитие тенденций, характерных для предыдущих этапов, и культ Фортуны приобретает широчайшую популярность во всех слоях римского общества: богиню удачи почитали во множестве мест под множеством имён.

